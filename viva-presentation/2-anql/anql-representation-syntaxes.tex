
\begin{frame}{How to represent context information in RDF?}

  \begin{itemize}
  \item RDF triples \\
    \begin{tabular}{ll}
      \begin{minipage}{0.5\paperwidth}
        \VerbatimInput[frame=single,fontsize=\scriptsize]{queries/address.ttl}
      \end{minipage}
      &
      \only<2->{
        \begin{minipage}{0.5\paperwidth}
          \alert{ Not enough!}
      \end{minipage}
    }
    \end{tabular}
    
    \item<3-> Domain vocabulary/ontology \\
      \begin{tabular}{ll}
        \begin{minipage}{0.5\paperwidth}
          \VerbatimInput[commandchars=\\\{\},fontsize=\scriptsize,frame=single]{queries/domain-vocab.ttl}
        \end{minipage}
        &
        \begin{minipage}{0.4\paperwidth}
          \only<6->{\alert{No defined \\semantics!}}
        \end{minipage}
      \end{tabular}

    \item<4-> Reification \\
      \begin{tabular}{ll}
        \begin{minipage}{0.5\paperwidth}
          \VerbatimInput[commandchars=\\\{\},fontsize=\scriptsize,frame=single]{queries/reification.ttl}
        \end{minipage}
        &
        \begin{minipage}{0.4\paperwidth}
          \only<6->{\alert{No defined \\semantics!}}
        \end{minipage}
      \end{tabular}

    \item<5-> Named Graphs

  \end{itemize}
\end{frame}

%%% Local Variables:
%%% mode: latex
%%% mode: flyspell
%%% mode: reftex
%%% TeX-master: "../presentation"
%%% End:
