
\subsection*{}


\begin{frame}{Data Integration: Pizza Delivery Company} %on the Web

  \hspace*{-1em}
  \begin{tikzpicture}

    \begin{scope}[visible on=<+->]
      \node (rdb) at (5,-2) {\pgfuseimage{rdb}} ;
      \node at (rdb.north east) {\pgfuseimage{rdb-table}}; 
      \node[box=green,visible on=<.->] at (rdb.north west) {RDB};
      \node[inner sep=0,left=.1em of rdb.west] (rdb-name1) {Clients};
      \node[inner sep=0,below=.1em of rdb-name1.south] (rdb-name) {/ Orders};
      
      \begin{pgfonlayer}{background}
        \node[box=gray, fit=(rdb) (rdb-name)] (rdbms) {} ;
      \end{pgfonlayer}

      \begin{scope}[visible on=<+->]
        \node[boxFull=green,text width=3cm] at (rdb.south west) (rdb-order){
          {\small
            \begin{tabular}{c|c}
              person & address\\
              \hline
              Nuno & Galway \\
            \end{tabular}
          }
        };
      \end{scope}
      

    \end{scope}


    \begin{scope}[visible on=<+->]
      \node at (-4,-2) (excel) {\pgfuseimage{excel}};

      \node[inner sep=0] at (excel.south) (excel-name) {Deliveries};
      
      \begin{pgfonlayer}{background}
        \node[box=gray, visible on =<.->, fit=(excel) (excel-name)] {} ;
      \end{pgfonlayer}

      \node at (excel.north east) {\scalebox{.7}{\pgfuseimage{xml}}};
      \node[box=green,visible on=<.->] at (excel.north west) {XML};

      \node[visible on=<+->] at (excel.south east) {\pgfuseimage{deliveries}};
    \end{scope}

    
    \begin{scope}[visible on=<+->]
      \node (eat) at (-4,3){\pgfuseimage{eat}};
      \node[inner sep=0,below=.1em of eat.south] (eat-name1) {Clients};
      \node[inner sep=0,below=.1em of eat-name1.south] (eat-name) {/ Orders};

      \begin{pgfonlayer}{background}
        \node[box=gray, visible on =<.->, fit=(eat) (eat-name)] {} ;
      \end{pgfonlayer}

      \node[] at (eat.north east) {\scalebox{.7}{\pgfuseimage{xml}}};
      \node[boxFull=green] at (eat.north west) {XML};

      
      \node[boxFull=green,text width=4.2cm,anchor=north,visible on=<+->] at (eat.south east) (eat-order){
        \VerbatimInput[firstline=2,lastline=5]{queries/justeat.xml}
      };

    \end{scope}


    \node[visible on=<+>,anchor=north east] at (4,3) (dbpedia)  {\pgfuseimage{lod}};

    \begin{scope}[visible on=<+->]
      \node at (5,3) (dbpedia)  {\pgfuseimage{dbpedia}};

      \begin{pgfonlayer}{background}
        \node[box=gray, visible on =<.->, fit=(dbpedia)] {} ;
      \end{pgfonlayer}

      \node at (dbpedia.north east) {\scalebox{.7}{\pgfuseimage{rdf}}};
      \node[box=green,visible on=<.->] at (dbpedia.north west) {RDF};

    \end{scope}




    \begin{scope}[visible on=<+->]
      \node[boxFull=red,text width=7cm,visible on=<.>] at (1,1) (di)  {How many pizzas were delivered to Nuno's home during his PhD?};
      
      \path[-latex] (eat.south east) edge [color=xquery,line width=5pt] (di); 
      \path[-latex] (rdb.north west) edge [color=xquery,line width=5pt] (di); 
      \path[-latex] (excel.north east) edge [color=xquery,line width=5pt] (di); 
      \path[-latex,visible on=<.>] (dbpedia.south west) edge [color=xquery,line width=5pt] (di);
      
    \end{scope}
    
    \begin{scope}[visible on=<+->]
      \path[latex-latex] (dbpedia.south west) edge [color=xquery,line width=5pt] (di);
      \node at (di)  {\pgfuseimage{rdf}};
    \end{scope}

    
    \begin{scope}[visible on=<+->]
      \node[boxFull=red,anchor=north] at (eat-order.south){2012};
      \node[boxFull=red,anchor=south] at (rdb-order.north){From 2008 to 2012};
    \end{scope}


  \end{tikzpicture}

\end{frame}

\begin{frame}{Using Query Languages for Data Integration}

  \begin{tikzpicture}[scale=.9,transform shape]
    \matrix[ampersand replacement=\&, column sep = 4em,row sep=.5em ] {

      \node[visible on=<2->,text width=4.7cm] (sql) {
        \begin{block}{SQL}
          \VerbatimInput{queries/example.sql}
        \end{block}
      }; \&
      \node[right= of sql] (rdb) {\pgfscaleimage{.9}{rdb}}; 
      \\

      \node[visible on=<2->,text width=4.7cm] (sparql) {
        \begin{block}{SPARQL}
          \VerbatimInput[fontsize=\scriptsize]{queries/example.sparql}
        \end{block}
      };  \& 
      \node[right= of sparql] (rdf) {\pgfscaleimage{.9}{rdf}};
       \\

      \node[visible on=<2->,text width=4.7cm] (xquery) {
        \begin{block}{XQuery}
          \VerbatimInput{queries/example.xquery}
        \end{block}
      };  \& 
      \node[right= of xquery] (xml) {\pgfscaleimage{.9}{xml}};
       \\
    };
    \path[visible on=<2->,latex-] ([xshift=.5em]sql.east) edge [color=sql,line width=5pt] (rdb); 
    \path[visible on=<2->,latex-] ([xshift=.5em]sparql.east) edge [color=sparql,line width=5pt] (rdf); 
    \path[visible on=<2->,latex-] ([xshift=.5em]xquery.east) edge [color=xquery,line width=5pt] (xml); 

    \node[boxFull=green,visible on=<3->,anchor=south east] at (sql.south east) {relation};
    \node[boxFull=green,visible on=<3->,anchor=south east] at (sparql.south east) {solution sequence};
    \node[boxFull=green,visible on=<3->,anchor=south east] at (xquery.south east) {sequence of items};


    \draw[box=blue,visible on=<4->] let \p1 = (xquery.south west) in let \p2 = (sql.north west) in (\x1-5em,\y1) rectangle (\x1-3em,\y2) node[rotate=90,midway] (med) {Mediator / Data Warehouse};

    \foreach \point in {sql,sparql,xquery}
    { \path[visible on=<4->,-latex] ([xshift=1.1em]med.east |- \point) edge [color=orange,line width=5pt] (\point);  }

    \node[visible on=<5->] at (med.south west) {\pgfuseimage{rdf}};

    \node[boxFull=red,text width=6cm,visible on=<6->,left=6em of sparql,anchor=south,rotate=90] {How many pizzas were delivered to Nuno's home
      during his PhD?};


  \end{tikzpicture}
  
\end{frame}



\begin{frame}{How to represent context information in RDF?}

  \begin{itemize}
  \item RDF triples \\
    \begin{tabular}{ll}
      \begin{minipage}{0.5\paperwidth}
        \VerbatimInput[frame=single,fontsize=\scriptsize]{queries/address.ttl}
      \end{minipage}
      &
      \only<2->{
        \begin{minipage}{0.5\paperwidth}
          \alert{ Not enough!}
      \end{minipage}
    }
    \end{tabular}
    
    \item<3-> Domain vocabulary/ontology \\
      \begin{tabular}{ll}
        \begin{minipage}{0.5\paperwidth}
          \VerbatimInput[commandchars=\\\{\},fontsize=\scriptsize,frame=single]{queries/domain-vocab.ttl}
        \end{minipage}
        &
        \begin{minipage}{0.4\paperwidth}
          \only<6->{\alert{No defined \\semantics!}}
        \end{minipage}
      \end{tabular}

    \item<4-> Reification \\
      \begin{tabular}{ll}
        \begin{minipage}{0.5\paperwidth}
          \VerbatimInput[commandchars=\\\{\},fontsize=\scriptsize,frame=single]{queries/reification.ttl}
        \end{minipage}
        &
        \begin{minipage}{0.4\paperwidth}
          \only<6->{\alert{No defined \\semantics!}}
        \end{minipage}
      \end{tabular}

    \item<5-> Named Graphs

  \end{itemize}
\end{frame}

%%% Local Variables:
%%% mode: latex
%%% mode: flyspell
%%% mode: reftex
%%% TeX-master: "../presentation"
%%% End:








\begin{frame}{Hypothesis and Research Questions}

  \begin{block}{Hypothesis}%
    Efficient data integration over heterogenous data sources can be achieved by
    \begin{itemize}
    \item<2-> a combined query language that accesses heterogenous data in its original sources
    \item<3-> optimisations for efficient query evaluation for this language
    \item<4-> an RDF-based format with support for context information
    \end{itemize}
  \end{block}


  \begin{block}<5->{Research Questions}%
    \begin{itemize}
    \item<5-> How to design a query language that bridges the different formats?
    \item<6-> Can we reuse existing optimisations from other query languages?
    \item<7-> How to adapt RDF and SPARQL for context information?
    \end{itemize}
  \end{block}


\end{frame}





%%% Local Variables:
%%% mode: latex
%%% mode: flyspell
%%% mode: reftex
%%% TeX-master: "presentation"
%%% End:
