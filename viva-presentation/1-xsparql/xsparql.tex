\subsection*{Overview}

\begin{frame}{XSPARQL}

    \begin{columns}
      \begin{column}{0.25\linewidth}
          \pgfuseimage{XSPARQLogo}
      \end{column}
      \begin{column}{0.65\linewidth}
        \begin{block}{}
          \begin{itemize}
          \item<+-> \alert<.>{Transformation language} between RDB, XML, and RDF
          \item<+-> \alert<.>{Syntactic} extension of XQuery
          \item<+-> \alert<.>{Semantics} based on XQuery's semantics
          \end{itemize}
        \end{block}
      \end{column}
    \end{columns}
    
    \vspace{10pt}

    \begin{block}<+->{Why based on XQuery? }
      \begin{itemize}
      \item<.-> Expressive language
      \item<.-> Use as scripting language
      \item<+-> Arbitrary Nesting of expressions
      \end{itemize}
    \end{block}

\end{frame}


\subsection{Syntax \& Semantics}
\begin{frame}{Same Language for each Format}

  \begin{center}
  \begin{tikzpicture}

    \node[visible on=<4>,text width=5.5cm] (xsparql) {
      \begin{block}{XSPARQL}
        \VerbatimInput{queries/example.sql.xsparql}
      \end{block}
    };

    \node[visible on=<5->,text width=5.5cm] (xsparql) {
      \begin{block}{XSPARQL}
        \VerbatimInput{queries/example.sparql.xsparql}
      \end{block}
    };

    \node[visible on=<2-3>,text width=5.5cm] (xsparql) {
      \begin{block}{\only<2>{XQuery}\only<3>{XSPARQL}}
        \VerbatimInput{queries/example.xquery}
      \end{block}
    };


    \node[box=green,visible on=<2>,text width=3cm,above =of xsparql.west] {
      {\small
        \begin{tabular}{l}
          \FOR \emph{var} \IN \emph{\XSPARQLFLWORExpr}\\
          \LET \emph{var} := \emph{\XSPARQLFLWORExpr}\\
          \WHERE \emph{\XSPARQLFLWORExpr}\\
          \ORDERBY \emph{\XSPARQLFLWORExpr}\\
          \RETURN \emph{\XSPARQLFLWORExpr}
        \end{tabular}}
    };

    \node[box=green,visible on=<4>,text width=3.5cm,above =of xsparql.west] {
      {\small
        \begin{tabular}{l}
          \FOR \emph{SelectSpec} \\
          \FROM \emph{RelationList} \\
          \WHERE \emph{WhereSpecList }\\ 
          \RETURN \emph{\XSPARQLFLWORExpr}
        \end{tabular}}
    };

    \node[box=green,visible on=<5>,text width=3.5cm,above =of xsparql.west] {
      {\small
        \begin{tabular}{l}
          \FOR \emph{varlist} \\
          \FROM~DatasetClause \\
          \WHERE \{\emph{ pattern } \}\\
          \RETURN \emph{\XSPARQLFLWORExpr}
        \end{tabular}}
    };


    \matrix[right =4em of xsparql,ampersand replacement=\&, column sep = 5em, row sep = 0, matrix of nodes] (matrix){
      
      \pgfuseimage{rdb} \\
      \pgfuseimage{rdf} \\
      \pgfuseimage{xml} \\ 
    };

    \path[visible on=<2->,latex-latex] ([xshift=.5em]xsparql.south east) edge [color=xsparql,line width=5pt] (matrix-3-1); 
    \path[visible on=<4->,-latex] ([xshift=.5em]xsparql.north east) edge [color=xsparql,line width=5pt] (matrix-1-1); 
    \path[visible on=<5->,-latex] ([xshift=.5em]xsparql.east) edge [color=xsparql,line width=5pt] (matrix-2-1); 

  \end{tikzpicture}
\end{center}
\end{frame}


\begin{frame}{Creating RDF with XSPARQL}
  
  \begin{tikzpicture}
    \node [text width=\linewidth] {
      \begin{block}<+->{Convert online orders into RDF}%
        \begin{tikzpicture}
        \node  [text width=\linewidth,]{
          \vspace{-1em}\VerbatimInput{queries/example_lifting.xsparql}
        };
        
        \begin{pgfonlayer}{background}
          \draw[box,visible on=<+>] (-0.51\linewidth,-0.6em) rectangle (0.45\linewidth,-1.5em);
        \end{pgfonlayer}

      \end{tikzpicture}

    \end{block}
    };

    \node[boxFull=green,text width=3cm,visible on=<.>] at (10.5em,-3em) {\CONSTRUCT{} clause generates RDF};
    \node[sqballoonFull,visible on=<+>,anchor=west, callout absolute pointer={(4.7em,-2.5em)},text width=4cm] at (2em,1em) {Arbitrary XSPARQL expressions in subject, predicate, and object};
  \end{tikzpicture}
  

\vspace{-5pt}

  \begin{block}<+->{Query result}
    \VerbatimInput{queries/example_lifting.xsparql.ttl}
  \end{block}

\end{frame}



\begin{frame}{Integration Query Example}
  
  \begin{block}{``Display pizza deliveries in Google Maps using KML''}
    \begin{tikzpicture}
      \node[text width=\linewidth]{\VerbatimInput{queries/example-integration.xsparql}};

      \begin{pgfonlayer}{background}
        \draw[box,visible on=<2>] (-0.51\linewidth,4.5em) rectangle (0.45\linewidth,3.5em);
        \draw[box,visible on=<3>] (-0.51\linewidth,3.6em) rectangle (0.45\linewidth,2em);
        \draw[box,visible on=<4>] (-0.51\linewidth,-1.9em) rectangle (0.45\linewidth,0em);
        \draw[box,visible on=<5>] (-0.51\linewidth,-1.8em) rectangle (0.45\linewidth,-4.9em);
      \end{pgfonlayer}
      \node[visible on=<2>] at (.35\linewidth,2.5em) {\scalebox{.7}{\pgfuseimage{xml}}};
      \node[visible on=<2>] at (.4\linewidth,3.5em) {\pgfuseimage{eat}};
      \node[visible on=<3>] at (.4\linewidth,2.2em) {\scalebox{.7}{\pgfuseimage{rdb}}};
      \node[visible on=<4>] at (.4\linewidth,0) {\scalebox{.7}{\pgfuseimage{rdf}}};
      \node[visible on=<5>] at (.35\linewidth,-2em) {\pgfuseimage{map-pizza}};

    
    \end{tikzpicture}
  \end{block}

  \begin{block}<6>{More involved XSPARQL queries: RDB2RDF}
    \begin{itemize}
    \item Direct Mapping: $\sim$130 LOC
    \item R2RML: $\sim$290 LOC
    \end{itemize}
  \end{block}

\end{frame}





\begin{frame}{Language Semantics}

  \begin{block}{}
    \begin{itemize}
    \item<1-> Extension of the XQuery Evaluation Semantics 
    \item<2-> \FOR{} add variables and values to the dynamic environment
    \item<3-> Reuse semantics of original languages for the new expressions
    \end{itemize}
  \end{block}


  \begin{tikzpicture}
    \node[text width=.95\linewidth]{
      \begin{block}{}
        \begin{tikzpicture}
          \node[text width=\linewidth]{\VerbatimInput{queries/example-integration.xsparql}};
          
          \begin{pgfonlayer}{background}
            \draw[box,visible on=<2>] (-0.51\linewidth,4.1em) rectangle (0.45\linewidth,3.2em);
          \end{pgfonlayer}

          \begin{pgfonlayer}{background}
            \draw[box,visible on=<3-5>] (-0.51\linewidth,3.2em) rectangle (0.45\linewidth,1.7em);
          \end{pgfonlayer}
          \node[boxFull=green,visible on=<4>,text width=3.5cm] at (0.35\linewidth,1em) {{\small  dynEnv.varValue  $\Rightarrow$ relation\\~ \\\begin{tabular}{l}person\\\hline "nuno"\end{tabular}}};
          \node[boxFull=green,visible on=<5>,text width=3cm] at (0.4\linewidth,3em) {{\small  eval(SQL SELECT) $\Join$ dynEnv.varValue}};

          \begin{pgfonlayer}{background}
            \draw[box,visible on=<6-7>] (-0.51\linewidth,0em) rectangle (0.45\linewidth,-1.6em);
          \end{pgfonlayer}
          \node[boxFull=green,visible on=<6>,text width=3.8cm] at (0.35\linewidth,-3.5em) {{\small dynEnv.varValue $\Rightarrow$ SPARQL sol. seq.\\\{\{~person~$\Rightarrow$~"nuno"~\}\}}};
          \node[boxFull=green,visible on=<7>,text width=3.8cm] at (0.35\linewidth,-3.5em) {{\small eval(SPARQL SELECT) solutions compatible dynEnv.varValue}};

        \end{tikzpicture}
      \end{block}
    };

    \node[visible on=<2>,anchor=east] at (-.45\linewidth,2.5em) {\scalebox{.4}{\pgfuseimage{xml}}};
    \node[visible on=<3-5>,anchor=east] at (-.44\linewidth,2em) {\scalebox{.4}{\pgfuseimage{rdb}}};
    \node[visible on=<6-7>,anchor=east] at (-.42\linewidth,0) {\scalebox{.4}{\pgfuseimage{rdf}}};

    \node[boxFull=beamer@lightblendedgreen,visible on=<8->,text width=\linewidth] {{\tiny
      \begin{prooftree}
        \def\ScoreOverhang{1pt}%
        \def\extraVskip{1pt}%
        \alwaysNoLine%
        % 
        \AxiomC{$\dyn.\ecomp{globalPosition} = \seq{ \envElem{Pos}{1}, \cdots, \envElem{Pos}{j} } $}
        % 
        \UnaryInfC{$\dynEnv{\funcCall{fs{:}dataset}{\DatasetClause} \Rightarrow \envElem{Dataset}{}}$}
        % 
        \UnaryInfC{$\begin{array}{r@{\hspace{-1pt}}l}%
            \dyn &~\proofs \funcCall{fs{:}sparql}{%
              \begin{array}{l}%
                \envElem{Dataset}{}, \SparqlWhereClause,\\
                \SolutionModifier
              \end{array}
            } \Rightarrow \sm{1}{}, \dots, \sm{m}{}
          \end{array}
          $}
        % 
        \UnaryInfC{$\begin{array}{r@{\hspace{-1pt}}l}%
            \dyn & \envExtend{globalPosition}{\seq{ \envElem{Pos}{1}, \cdots, \envElem{Pos}{j}, 1 }} ~\envExtend{activeDataset}{\grammarRule{Dataset}}\\
            &\envExtend{varValue}{%
              \begin{array}{l}%
                % \envElem{Variable}{pos} \Rightarrow 1;\\
                \envElem{Var}{1} \Rightarrow \funcCall{fs{:}value}{\sm{1}{},\envElem{Var}{1}};\\
                \dotsc;\\
                \envElem{Var}{n} \Rightarrow \funcCall{fs{:}value}{\sm{1}{},\envElem{Var}{n}}
              \end{array}%
            }~\proofs \grammarRule{ExprSingle} \Rightarrow
            \envElem{Value}{1}
          \end{array}$}
        % 
        \UnaryInfC{$\vdots$}
        % 
        \UnaryInfC{$\begin{array}{r@{\hspace{-1pt}}l}
            \dyn & \envExtend{globalPosition}{\seq{ \envElem{Pos}{1}, \cdots, \envElem{Pos}{j}, m }} ~\envExtend{activeDataset}{\grammarRule{Dataset}}\\
            &\envExtend{varValue}{\begin{array}{l}
                % \envElem{Variable}{pos} \Rightarrow n;\\
                \envElem{Var}{1} \Rightarrow \funcCall{fs{:}value}{\sm{m}{},\envElem{Var}{1}}; \\
                \dotsc; \\
                \envElem{Var}{n} \Rightarrow \funcCall{fs{:}value}{\sm{m}{},\envElem{Var}{n}} 
              \end{array}%
            }~\proofs \grammarRule{ExprSingle} \Rightarrow \envElem{Value}{m}
          \end{array}$}
        % 
        \singleLine
        % 
        \UnaryInfC{$\dynEnv{\begin{array}{l}
              \FOR~\envElem{\var{Var}}{1}\dotsb\envElem{\var{Var}}{n}~\DatasetClause\\
              \SparqlWhereClause~\SolutionModifier\\
              \RETURN~\grammarRule{ExprSingle}
            \end{array}} \Rightarrow \envElem{Value}{1}, \dots, \envElem{Value}{m}$}
      \end{prooftree}%
    }};

  \node[boxFull=red,visible on=<9->,text width=\linewidth] at (0,3.5em){{\tiny
      \vspace{-2em}%
      \begin{prooftree}%
        \def\ScoreOverhang{1pt}%
        \def\extraVskip{1pt}%
        \alwaysNoLine%
        % 
        \AxiomC{$\begin{array}{r@{\hspace{-1pt}}l}%
            \dyn &~\proofs \funcCall{fs{:}sparql}{%
              \begin{array}{l}%
                \envElem{Dataset}{}, \SparqlWhereClause,\\
                \SolutionModifier
              \end{array}
            } \Rightarrow \sm{1}{}, \dots, \sm{m}{}
          \end{array}
          $}
      \end{prooftree}%
    }};

  \end{tikzpicture}
\end{frame}



%%% Local Variables:
%%% mode: latex
%%% mode: flyspell
%%% mode: reftex
%%% TeX-master: "../presentation"
%%% End:



\begin{frame}{Language Implementation}

  \begin{block}{}
    \begin{itemize}
    \item<1-> Reuse components (SPARQL engine, Relational Database)
    \item<2-> Implemented XSPARQL by rewriting to XQuery
    \item<3-> Semantics implemented by substitution of bound variables
    \end{itemize}
  \end{block}

  \begin{block}<4->{Bound Variable Substitution}
    \begin{itemize}
    \item<4-> \alert{Bound} variables are replaced by their \alert{value} at runtime
    \item<5-> Implemented in the generated XQuery
    \item<6-> Pushing the variable bindings into the respective query engine
    \end{itemize}
  \end{block}

  \begin{block}<4->{}
    \begin{tikzpicture}
      \node[text width=\linewidth]{\vspace*{-1em}\VerbatimInput[lastline=3]{queries/example-integration.xsparql}};

      \begin{pgfonlayer}{background}
        \draw[box,visible on=<4>] (-13em,.5em) rectangle (-9.5em,1.5em);
        \draw[box,visible on=<4>] (-4em,0em) rectangle (-8em,-1em);

        \draw[box,visible on=<5-6>] (-0.51\linewidth,-1em) rectangle (0.45\linewidth,.5em);

      \end{pgfonlayer}

      \node[boxFull=green,visible on=<5-6>,text width=4.7cm,anchor=west] at (1em,-1em) {\vspace*{-1em}\VerbatimInput[firstline=2,lastline=4]{queries/variable-replacement.xq}};

    
    \end{tikzpicture}
  \end{block}


\end{frame}


\subsection{Implementation}


\begin{frame}{Implementation}

  \begin{center}
    
  \begin{tikzpicture}[
    terminal/.style={ rectangle,minimum size=6mm,rounded corners=3mm,text centered,draw=black!50,fill=gray!20},%
    rdb/.style={cylinder, shape border rotate=90, draw,minimum width=1.5cm,minimum height=1cm, shape aspect=.35}%
    ]
    \tikzstyle{every node}=[font=\footnotesize]


    % XSPARQL engine
    \begin{scope}[node distance = 20pt]
      \node[terminal,text width=1.7cm] (rewriter) {Query Rewriter}; %
      \node[right=of rewriter,doc=1cm] (rewritenXQ) {XQuery query};%
      \node[right=of rewritenXQ, terminal,text width=1.7cm] (EXQuery) {Enhanced XQuery engine}; %
    \end{scope}


    % XSPARQL query
    \node[above=2.5em of rewriter, doc=1.4cm] (XS-Q) {XSPARQL query};%

    \coordinate[below=1.5em of EXQuery.south] (input-separator);
    % input sources
    \begin{scope}[node distance = 7pt]
      \documentSet{below=2.5em of input-separator}{xml}{XML / JSON};
      \documentSet{right=of xml}{rdf}{RDF};
      \node[left=of xml, rdb]  (rdb)  {RDB};%
    \end{scope}
  
    \draw[-latex] (input-separator) -- (EXQuery.south);%
    \draw[-latex] (input-separator) -- (xml.north) node [midway,inner sep=0,fill=white] {XQuery};%
    \draw[-latex] (input-separator) -| (rdb.north) node [near end,inner sep=0,fill=white] {SQL};%
    \draw[-latex] (input-separator) -| (rdf.north) node [near end, inner sep=0,fill=white] {SPARQL};%

    % outputs
    \begin{scope}[node distance = 2pt]
      \node[above=2em of EXQuery, doc=1cm] (output) {XML~/ RDF};%
    \end{scope}
  
    % inner diagram arrows
    \draw[-latex] (rewriter.east) -- (rewritenXQ.west);%
    \draw[-latex] (rewritenXQ.east) -- (EXQuery.west);%
    
    % outer box
    \node[draw=black,dotted,thick,inner sep=10pt,rectangle,fit=(rewriter) (EXQuery)] (XSPARQL) {} ;%
    \node[inner sep=0,fill=white] at (XSPARQL.north) {XSPARQL};

    % outer diagram arrows
    \draw[-latex] (XS-Q.south) -- (rewriter.north);%

    % output lines
    \draw[-latex] (EXQuery.north) -- (output.south) ;%

  \end{tikzpicture}

%%% Local Variables:
%%% mode: latex
%%% mode: flyspell
%%% mode: reftex
%%% TeX-master: "../thesis"
%%% End:


  \end{center}
  
\end{frame}



\subsection{Evaluation}

\begin{frame}{XSPARQL Evaluation}

  \vspace{-5pt}

  \begin{center}{}
    \begin{tikzpicture}
    \matrix<+->[ampersand replacement=\&, column sep = 4em, matrix of nodes,row sep=2em,inner sep=.1em] (xmark){
      $XMark$ \&       |[visible on=<+->]|$XMark_{RDF}$ \&  |[visible on=<+->]|     $XMark_{RDB}$ \\
      \scalebox{.65}{\pgfuseimage{xml}} \& |[visible on=<.(-1)->]|\scalebox{.65}{\pgfuseimage{rdf}} \& |[visible on=<.->]|\scalebox{.65}{\pgfuseimage{rdb}} \\
    };

    \begin{pgfonlayer}{background}
      \node[box=gray,visible on=<.(-2)->,fit={(xmark-1-1) (xmark-2-1)}] {} ;%
      \node[box=gray,visible on=<.(-1)->,fit={(xmark-1-2) (xmark-2-2)}] {} ;%
      \node[box=gray,visible on=<.->,fit={(xmark-1-3) (xmark-2-3)}] {} ;%
    \end{pgfonlayer}

    \path[visible on=<.(-2)->,-latex] (xmark-1-1) edge [color=green,line width=5pt] (xmark-2-1); 
    \path[visible on=<.(-1)->,-latex] (xmark-1-2) edge [color=green,line width=5pt] (xmark-2-2); 
    \path[visible on=<.->,-latex] (xmark-1-3) edge [color=green,line width=5pt] (xmark-2-3); 

    \path[visible on=<.(-1)-.>,-latex,dashed] (xmark-1-1) edge [color=orange,line width=3pt] (xmark-1-2); 
    \path[visible on=<.(-1)-.>,-latex,dashed] (xmark-2-1) edge [color=orange,line width=3pt] (xmark-2-2); 

    \path[visible on=<.>,-latex,dashed] (xmark-1-1) edge [color=orange,line width=3pt,bend left=17] (xmark-1-3); 
    \path[visible on=<.>,-latex,dashed] (xmark-2-1) edge [color=orange,line width=3pt,bend right=27] (xmark-2-3); 

    \end{tikzpicture}
  \end{center}

  \vspace{-10pt}

  \begin{block}<+->{Experimental Results}
    \begin{itemize}
    \item Compared to native XQuery (XMark)
      \begin{description}
      \item[RDB and RDF:] same order of magnitude for most queries
      \end{description}
    \item<+-> Except on RDF nested queries (self joining data)
      \begin{itemize}
      \item[] several orders of magnitude slower
      \item[] due to the number of calls to the SPARQL engine
      \end{itemize}
    \end{itemize}
  \end{block}

\end{frame}


\begin{frame}{Rewriting techniques for Nested Queries}

  \begin{block}<+->{{\small $Q_8$: ``List the names of persons and the number of items they bought.''}}
    \begin{tikzpicture}
      \node[text width=\linewidth] {
        \VerbatimInput{queries/query08.xsparql}
      };

      \begin{pgfonlayer}{background}
        \draw[box,visible on=<+-.(6)>] (-0.51\linewidth,4em) rectangle (0em,2em);
      \end{pgfonlayer}

      \node[boxFull=green,visible on=<.-.(3)>,text width=3.8cm,anchor=west] at (2em,5em) 
      {Returns $\left(\begin{array}{c} {\texttt{\scriptsize nuno}},\\ {\texttt{\scriptsize axel}},\\ \vdots\end{array} \right)$};
          

      \begin{pgfonlayer}{background}
        \draw[box,visible on=<+->] (-13em,.5em) rectangle (0em,-1.2em);
      \end{pgfonlayer}

      \node[boxFull=green,visible on=<.-.(2)>,text width=5cm,inner sep=0] at (0.25\linewidth,-1em) (rewrite) {\VerbatimInput{queries/nested-inner-rewriting.xq}};

      \node[boxFull=orange,visible on=<+>,text width=1cm,anchor=south west] at (10em,-2.4em) {{\scriptsize\tt nuno}};
      \node[boxFull=orange,visible on=<.>,text width=1cm,anchor=south west] at (7em,6em) {{\scriptsize\tt nuno}};

      \begin{scope}[visible on=<+>]
        \node[boxFull=orange,text width=1cm,anchor=south west] at (10em,-2.5em) {{\scriptsize\tt axel}};
        \node[boxFull=orange,text width=1cm,anchor=south west] at (7em,4.5em) {{\scriptsize\tt axel}};
      \end{scope}
    
          

      \path[visible on=<+>,-latex,dashed] (0em,0em) edge [color=orange,line width=3pt,bend right] (0em,4.5em); 

      \node[boxFull=green,visible on=<.-.(1)>,text width=5.5cm,anchor=south east] at (0em,4.5em) {\VerbatimInput{queries/nested-inner-rewriting-opt.xq}};
      \draw[draw=none,fill=beamer@lightblendedgreen,visible on=<+->] (-13.1em,.6em) rectangle (.1em,-1.3em) node[midway] {Join in XQuery};

      \node[boxFull=green,visible on=<+>,text width=4cm,anchor=north west] at (.1em,4em) {Nested Loop rewriting};
      \node[visible on=<.->,text width=5.5cm,anchor=south east] at (0em,4.5em) {\VerbatimInput{queries/nested-inner-rewriting-opt.xq}};

      % other optimisations
      \node[boxFull=green,visible on=<+>,text width=5cm,anchor=north west] at (.1em,0em) {Applied to related language: SPARQL2XQuery~\cite{GroppeGroppeLinnemann:2008aa}};

      \node[boxFull=green,visible on=<+>,text width=5cm,anchor=north west] at (.1em,0em) {Other optimisations: \\ Join in SPARQL};
      

    \end{tikzpicture}
  \end{block}

\end{frame}


\def\plotsDir{1-xsparql/plots/}
%
\begin{frame}{Evaluation of different rewritings - $XMark_{RDF} Q_8$}
  \pgfplotsset{
    tick label style={font=\small},
    label style={font=\small},
    legend style={font=\footnotesize}
  }
  \begin{figure}
    \setlength{\abovecaptionskip}{0pt}
    \setlength{\belowcaptionskip}{0pt}
      \import{\plotsDir}{optimisations-paper-query08}
    \vspace{-5pt}
  \end{figure}

  \vspace{-5pt}
  \begin{block}{}
    \begin{itemize}
    \item Optimised rewritings show promising results
    \item Best Results: SPARQL-based 
    \item Results included in~\cite{BischofDeckerKrennwallner:2012aa}
    \end{itemize}
  \end{block}
\end{frame}


\begin{frame}{Evaluation of different rewritings - $XMark_{RDB} Q_8$}
  \pgfplotsset{
    tick label style={font=\small},
    label style={font=\small},
    legend style={font=\footnotesize}
  }
  {
    \begin{figure}
      \setlength{\abovecaptionskip}{0pt}
      \setlength{\belowcaptionskip}{0pt}
      \import{\plotsDir}{optimisations-paper-query08-rdb}
      \vspace{-5pt}
    \end{figure}
  }
  \vspace{-5pt}
  \begin{block}{}
    \begin{itemize}
    \item Not so effective for RDB, requires different optimisations
    \item Due to schema representation?
    \item Additional Results for this thesis
    \end{itemize}
  \end{block}
\end{frame}



%%% Local Variables:
%%% mode: latex
%%% mode: flyspell
%%% mode: reftex
%%% TeX-master: "../presentation"
%%% End:
