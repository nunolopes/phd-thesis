\section{Extensible Markup Language (XML)}
\label{sec:xml}


As we have highlighted in \cref{cha:introduction}, with the growing success of the \ac{WWW}, where data exposed as
\ac{HTML} is often extracted from relational databases, the need to query Web Data in a structured way and thus consider
the Web as a \emph{global database} increased~\cite{SilberschatzKorthSudarshan:2005aa,AbiteboulBunemanSuciu:1999aa}.
%
Also powered by several data integration projects, research began to focus on the representation and querying of
\sd data following a \emph{graph} or \emph{tree} structure.
%
\Sd data models were devised as the required formats for representing data available on the Web and as a
representation-independent way to transfer data between different database management
systems~\cite{Abiteboul:1997aa,Buneman:1997aa}.

The \acf{XML}~\cite{BrayPaoliSperberg-Mcqueen:2008aa} is a \sd representation format and, with the support of the
\ac{W3C}, it has become the \emph{de facto} standard for data exchange on the
Web~\cite{Suciu:1998aa,AbiteboulBunemanSuciu:1999aa}.  XML is a subset of the Standard Generalized Markup Language
(SGML) ISO standard~\cite{ISO:1986:SGML} and is designed to be compatible with SGML and \ac{HTML}.
%
\ac{XML} represents data in a tree-like format that, when compared to the relational format, is a more flexible data
representation format and is also considered easier to read and write for both humans and machines.


\ac{XML} has also brought forward a new class of databases: \emph{XML databases}.  Although currently most databases
provide easy creation of \ac{XML} data, for example by exporting the data they contain as an \ac{XML} document, \ac{XML}
databases refer to a database management system that manage collections of \ac{XML}
data~\cite{KatzChamberlinKay:2003aa}.  Even though the data may be internally represented in another format, access and
manipulation is based on \ac{XML} formats and languages.


\cref{fig:bands-xml} contains the representation of the \usecase data from \cref{ex:usecase-data} in \ac{XML}.  This
document starts by representing a user and the top bands they listen to, where each band includes information regarding
its members and albums, and for each album, the songs listened to by the user.
%
As per~\citet{BrayPaoliSperberg-Mcqueen:2008aa}, the \acl{XML} describes what are called \emph{\ac{XML} documents},
which are composed primarily of \ac{XML} \emph{elements}.
%
In turn, \ac{XML} elements consist of a \emph{start-tag}, the \emph{element content}, and an \emph{end-tag}. 
%
\begin{data}[t]
  \centering
  \lstinputlisting[language=XML,basicstyle=\small\ttfamily]{0-data+queries/usecaseData.xml}
  \caption{Bands in XML (\fname{bands.xml})}
\label{fig:bands-xml}
\end{data}
%
Consider the following \ac{XML} element: 
%
\begin{lstlisting}[language=XML,basicstyle=\small\ttfamily,frame=none,numbers=none]
<song>Wishmaster</song>
\end{lstlisting}
%
Start- and end-tags are indicated by \stringValue{<song>} and \stringValue{</song>}, respectively, where
\stringValue{song} is called the \emph{element name}, and the element content may consist of text (any string of
characters), other (nested) \ac{XML} elements, CDATA sections, processing instructions or comments.
%
CDATA sections can be used to include text that contains markup characters (such as the start- and end-delimiters) and
processing instructions contain data that is to be sent to the application consuming the document.
%
Comments can be present anywhere in the document and, similar to any programming language, can be ignored by the
\ac{XML} processor.
%
Furthermore, \ac{XML} elements may contain \emph{attributes} enclosed in their start-tags, in this case the
\stringValue{album} element has the \stringValue{name} attribute with value \stringValue{Wishmaster}:
%
\begin{lstlisting}[language=XML,basicstyle=\small\ttfamily,frame=none,numbers=none]
<album name="Wishmaster">
\end{lstlisting}
%
In \ac{XML} text, elements, CDATA, processing instructions, comments, and attributes are collectively referred to as
\emph{\ac{XML} nodes}.

\subsection{XML Namespaces}
\label{sec:xml-namespaces}

\ac{XML} provides a way to disambiguate entities such as element and attribute names by using \ac{XML}
namespaces~\cite{BrayHollanderLayman:2009aa}, where each \ac{XML} namespace is identified by a URI
reference~\cite{Berners-LeeFieldingMasinter:2005aa}.
%
\ac{XML} allows, by means of reserved attributes, to associate partial \acp{URI} with a \emph{prefix name} and/or to
declare a \emph{default namespace}.
%
\emph{Qualified names} (or QNames) provide a convenient form of naming element and attribute names in \ac{XML} and can
be composed of prefixed or unprefixed names.  Prefixed names make use of the previously declared prefixes and are
combined with the \emph{local part} to specify the \ac{URI} reference.
%
For unprefixed names, if a default namespace declaration exists it is taken as the namespace value, otherwise there will
be no namespace value.
%
For example, including the ``\verb+xmlns+'' attribute in an \ac{XML} element declares the default namespace to be used
within that element:
%
\begin{lstlisting}[language=XML,basicstyle=\small\ttfamily,frame=none,numbers=none]
<user xmlns="http://example.org/bands/">
\end{lstlisting}
%
while \acp{URI} can be associated with a prefix in the following manner:
%
\begin{lstlisting}[language=XML,basicstyle=\small\ttfamily,frame=none,numbers=none]
<members xmlns:foaf="http://xmlns.com/foaf/0.1/">
\end{lstlisting} 
%
\ac{XML} namespaces are scoped to the element in which they are declared, including any child elements.


\subsection{XML Validation}
\label{sec:xml-validation}


The \ac{XML} \ac{W3C} specification~\cite{BrayPaoliSperberg-Mcqueen:2008aa} defines two levels of conformance for
\ac{XML} documents: \emph{well-formed} documents and \emph{valid} documents.  \emph{Well-formedness} constraints
primarily ensure that the \ac{XML} document follows syntactic specifications, such as (to name but a few):
\begin{inparaenum}[(i)]
\item they must contain at least one element;
\item a distinct element, called the \emph{root}, is not included in the content of any other element;
\item for all non-root elements, its start- and end-tags must be included within the content of the same element,
  \ie~opening and closing tags must not overlap; and
\item attribute names must be unique within the same element.
\end{inparaenum}

On the other hand, \emph{valid} documents rely on a \emph{schema} that, similar to relational databases, specifies the
structure of a particular class of \ac{XML} documents. Such schemas can be specified using two different formats:
\ac{DTD} or \ac{XML} Schema, both of which are detailed below.\footnote{There are other schema languages for \ac{XML},
  such as the Relax NG language, but for the scope of this thesis we will focus on \ac{W3C} specifications.}
%
In \cref{cha:xsparql} we will define \ac{XML} Schema datatypes for representing \ac{RDF} concepts and thus
incorporating them into XQuery.

\subsubsection{Document Type Definition}
\label{sec:dtd}


\begin{figure}[t]
  \centering
  \lstinputlisting[language=XML,basicstyle=\small\ttfamily]{0-data+queries/usecaseData.dtd}
  \caption{\ac{DTD} definition for the bands XML data}
\label{fig:bands-xml-dtd}
\end{figure}

\ac{DTD} specifications are mostly referenced here for historical reasons, since \ac{XML} Schema is more widely used (as
detailed in the next section).
%
\ac{DTD} specifications consist of \emph{markup declarations}, such as \emph{element type}, \emph{attribute list},
\emph{entity} or \emph{notation} declarations.  
% 
Element type declarations are defined using the \character{ELEMENT} keyword, for instance:
%
\begin{lstlisting}[language=XML,basicstyle=\normalfont\ttfamily,frame=none,numbers=none]
<!ELEMENT album (song*)>
\end{lstlisting}
%
specifies an \stringValue{album} element that is constituted by any number of \stringValue{song} elements. The
\stringValue{album} element is required to have an attribute named \stringValue{name} by the following attribute list
declaration:
%
\begin{lstlisting}[language=XML,basicstyle=\normalfont\ttfamily,frame=none,numbers=none]
<!ATTLIST album name CDATA #REQUIRED>
\end{lstlisting}
%
The complete \ac{DTD} definition for the \usecase \ac{XML} structure is presented in \cref{fig:bands-xml-dtd}.  An
attribute declared as \texttt{CDATA} indicates that its value must be a sequence of characters and/or \ac{XML} markup.
%
On the other hand, \texttt{PCDATA} (meaning ``parsed character data'') indicates that only one text element, and no
other nodes are allowed in the content.
%
Adding this \ac{DTD} definition to the \ac{XML} document from \cref{fig:bands-xml} would ensure that any
validating \emph{\ac{XML} processor} checks the structure of the \ac{XML} data against the provided schema definition.
%


\subsubsection{XML Schema}
\label{sec:xml-schema}

While \acp{DTD} are included in the \ac{W3C} \ac{XML} specification and therefore are widely available, there are some
drawbacks to their use, most noticeably the lack of namespace support.
%
To overcome such drawbacks, the \ac{W3C} has defined the \ac{XML} Schema specification, composed of two parts:
%
\begin{inparaenum}[(i)]
\item an \ac{XML}-based syntax for validating \ac{XML} documents~\cite{ThompsonBeechMaloney:2004aa}; and
\item a specification of \ac{XML} datatypes~\cite{BironMalhotra:2004aa}.
\end{inparaenum}


\begin{figure}[p]
  \centering
  \lstinputlisting[language=XML]{0-data+queries/usecaseData.xsd}
  \caption{XML Schema definition for Bands XML data (partial)}
\label{fig:bands-xml-xsd}
\end{figure}
%
The \ac{XML} Schema definition of the \usecase \ac{XML} data is presented in \cref{fig:bands-xml-xsd}, which has the
same effect as the \ac{DTD} presented in \cref{fig:bands-xml-dtd}: validating the \ac{XML} document from
\cref{fig:bands-xml}.
%
In \ac{XML} Schema, \ac{XML} elements and attributes are declared using an \ac{XML} element named \stringValue{element}
and \stringValue{attribute}, respectively, contained in the \stringValue{\url{http://www.w3.org/2001/XMLSchema}}
namespace.
%
For example, the \stringValue{album}, along with its \stringValue{name} attribute and \stringValue{song} elements, are
defined in lines 41--48 of \cref{fig:bands-xml-xsd}.


The specification of datatypes in \ac{XML}~\cite{ThompsonBeechMaloney:2004aa} introduces a datatype system that is also
used by other \ac{W3C} specifications, such as the \ac{RDF} specification~\cite{ManolaMiller:2004aa}.
%
A datatype is defined by
%
\begin{inparaenum}[(a)]
\item the \emph{value space}: a set of values for a datatype; 
\item the \emph{lexical space}: a set of valid character strings for the datatype; and
\item a \emph{lexical-to-value mapping} linking elements of these two sets.
\end{inparaenum}
%
A datatype is identified by a \ac{URI} and a \emph{datatype map} associates the \ac{URI} with the specific datatype.
%
The defined datatype system allows for the creation of user-defined datatypes, where such datatypes are \emph{derived}
from existing datatypes (called the \emph{base type}) by restricting or extending its value space and lexical space.


The formalisation of \ac{XML} Schema was proposed by~\citet{SimeonWadler:2003aa}, where the authors also describe a more
human readable notation for both \ac{XML} elements and \ac{XML} schema types.  This notation was later adopted by the
XQuery semantics specification~\cite{DraperFankhauserFernandez:2010aa}.
%
Following this notation, the \ac{XML} element
%
\lstinline[basicstyle=\small\ttfamily]{<song>Wishmaster</song>}
%
is represented as
%
\lstinline[basicstyle=\small\ttfamily]+element song { "Wishmaster" }+.
%
The \stringValue{song} and \stringValue{album} elements from the~\ac{XML} Schema in \cref{fig:bands-xml-xsd} can
be represented in the shorthand notation as:
%
\begin{lstlisting}[language=XML,basicstyle=\small\ttfamily,frame=none,numbers=none]
define element song of type xs:string
define element album of type albumType
define type albumType {
  element song*,
  attribute name of type xs:string  }
\end{lstlisting}
%
After performing validation with the presented \ac{XML} Schema, the \ac{XML} element is represented as:
\lstinline[basicstyle=\small\ttfamily]+element song of type xs:string { "Wishmaster" }+.
%
In \cref{cha:xsparql} we specify the types introduced by the XSPARQL language following this notation.


\subsection{XML Abstract Representations}
\label{sec:xml-data-models}

The \ac{W3C} specifications are defined over abstract representations of \ac{XML} documents, with the objective of
omitting the concrete syntax of \ac{XML} documents, namely the \ac{Infoset} and the \ac{XDM}.
%
The \ac{Infoset} provides the basic definitions for describing well-formed \ac{XML} documents, with the purpose of
serving as a reference for other \ac{XML} specifications.
%
On the other hand, the more complex \ac{XDM} is meant to act as a data model for the XPath, XSLT and XQuery languages:
describing their input documents and the values for expressions.  These query languages will be the focus of
\cref{sec:prelim-xquery}.
%
The \ac{Infoset} describes only the basic information contained in an \ac{XML} document, while the XQuery and XPath Data
Model is used mostly for the \ac{XML} Query languages (described in \cref{sec:prelim-xquery}).

\subsubsection{XML Infoset}
\label{sec:xml-infoset}

The \acl{Infoset}, described by~\citet{CowanTobin:2004aa}, provides definitions referring to a well-formed \ac{XML} document
and as such, any well-formed \ac{XML} document, although not necessarily a valid document, will have an \ac{Infoset}.
%
The \ac{Infoset} consists of a set of \emph{information items}, where each information item describes a part of the \ac{XML}
document by means of \emph{properties} that refer to other information items.
%

An \ac{Infoset} contains exactly one document information item that, directly or indirectly, refers to all of the other
information items in the set.  Other information items are used to represent \ac{XML} nodes such as elements,
attributes, processing instructions, or comments.


\subsubsection{XQuery 1.0 and XPath 2.0 Data Model (XDM)}
\label{sec:xpath-data-model}
%
The XPath, XSLT, and XQuery languages use the \emph{\acf{XDM}}~\cite{FernandezMalhotraMarsh:2010aa} for describing their
input \ac{XML} documents.  \ac{XDM} is based on the \ac{XML} \ac{Infoset} and extends it with support for:
%
\begin{inparaenum}[(i)]
\item \ac{XML} Schema types~\cite{ThompsonBeechMaloney:2004aa,BironMalhotra:2004aa};
\item typed atomic values; 
\item collections of documents and complex values; and
\item ordered, heterogeneous sequences.
\end{inparaenum}

The basic element of the data model is called an \emph{item}, which has a type (either an \ac{XML} \emph{node} or an
atomic type) and an associated value.
%
Types in the data model are uniquely represented using (expanded) QNames %, in this case after expansion,
and the pre-defined atomic types include the built-in types presented in~\citet{BironMalhotra:2004aa}, extended with the
following additional types:
%
\begin{inparaenum}[(i)]
\item \qname{xs}{anyAtomicType};
\item \qname{xs}{untyped};
\item \qname{xs}{untypedAtomic};
\item \qname{xs}{dayTimeDuration}; and
\item \qname{xs}{yearMonthDuration}.
\end{inparaenum}


The data model defines seven types of \ac{XML} nodes, namely: document, element, text, attribute, namespace, processing
instruction, and comment.  Each item that represents an \ac{XML} node is associated with the corresponding type and for
every type of node, it is possible to compute a string value.
%
In the data model, each \ac{XML} node has a unique identity, which is equal only to itself.  On the other hand, all
instances of the same atomic values are considered equal, \ie~they do not have a unique identity.
%
XDM defines a total order among all available nodes, called \emph{document order}, which consists of the order that
nodes appear in the original document.

In comparison to the \ac{Infoset}, another addition of the \ac{XDM} is the support for sequences.  A \emph{sequence}
consists of any number of items (\ac{XML} nodes and/or atomic values) and are represented as a
comma-separated~\character{,} list of items, delimited by the~\character{(} and~\character{)} characters.
%
Each item is considered equal to a singleton sequence containing that item and furthermore sequences are not allowed to
include any other sequences and are thus always considered as a flattened sequence.
%
For example, the sequence \inlineExpr{\lstinline{(1,(<a/>), "3")}} is translated to \inlineExpr{\lstinline{(1,<a/>, "3")}}.


%%% Local Variables:
%%% mode: latex
%%% mode: flyspell
%%% mode: reftex
%%% TeX-master: "../thesis"
%%% End:
