\subsubsection*{Running example}
\label{sec:example}

In this thesis we will use examples from the music domain, where we are interested in representing \emph{persons},
\emph{bands}, \emph{albums}, and \emph{songs}.
%
In our simplified model, persons can be members of bands and can listen to specific songs.  Bands release albums, which
in turn include songs.
%
For presentation purposes and conciseness of examples, we will use a reduced set of entities, included in
\cref{ex:usecase-data}.

We chose to use the music domain, as opposed to more enterprise oriented examples, due to the availability of
information.
%
We note that data required for this \usecase is available in the \ac{WWW}, for instance, information regarding bands can
be found in Wikipedia (\url{http://www.wikipedia.org/}) or MusicBrainz~\cite{Swartz:2002aa}, while personalised
information about the songs individuals listen to can be found in Last.fm (\url{http://last.fm/}).


\paragraph*{Wikipedia/DBpedia.} The widely known online encyclopaedia Wikipedia relies on user contributions for its
contents.
%
DBpedia~\cite{BizerLehmannKobilarov:2009aa} consists of a partial export of the information from Wikipedia into the RDF
format, accessible using standard query languages such as SPARQL.
%
For our running example, we are interested in extracting information regarding artists, bands, and albums and we often
use DBpedia \acp{URI} as identifiers for entities.


\paragraph*{Last.fm.} The online Web service Last.fm allows users to submit the songs and artists they listen to.
%
These songs are aggregated in order to create a user profile containing the top artists, lists of songs from each user,
and provide personalised recommendations of new artists.
%
The data presented in this thesis was extracted from this author's Last.fm website.\footnote{Last.fm user profile
  available at~\url{http://last.fm/user/jacktrades/}, retrieved on 2012/04/10.}
%
The data retrievable via the Last.fm API contains information such as the five most played bands from a user profile
and, for each band, the most played tracks by the user and the albums they are included in.

\begin{example}[\Usecase data]
  % 
  \label{ex:usecase-data}
  % 
  This example presents the data we are using in the examples of this thesis:\footnote{For the sake of conciseness of
    examples, we restrict the presented data to one band, two members and one album of the band.}
  % 
  \begin{description}[nosep,labelindent=!]
  \item[persons:] Marco Hietala, Tarja Turunen
  \item[bands:] Nightwish
  \item[albums:] Wishmaster
  \item[songs:] FantasMic, Wishmaster%, Deep Silent Complete
  \end{description}
  %
  In this thesis, we are interested in representing this data in four different formats:
  % 
  \begin{inparaenum}[(i)]
  \item a relational database,
  \item XML,
  \item RDF, and
  \item JSON.
  \end{inparaenum}
  % 
  The representation of this data in each specific syntax is accompanied by the description of each data model
  throughout this chapter.
  %
\end{example}



%%% Local Variables:
%%% mode: latex
%%% mode: flyspell
%%% mode: reftex
%%% TeX-master: "../thesis"
%%% End:
