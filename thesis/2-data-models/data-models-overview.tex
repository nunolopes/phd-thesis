
\section{Comparison of the Data Models}
\label{sec:overview-data-models}
%
\begin{table} [t]
\centering
\caption{Feature overview of data models}
\label{fig:data-models-overview}
\begin{tabular}{rcccc}
\toprule
                         & RDB        &   \acs{XML}     & \acs{JSON}     & \acs{RDF} \\
\midrule
\multicolumn{1}{l}{\bf Model:}  & & & & \\
logical structure         & relations  &  tree           &  tree          & graph    \\
ordered                   & \no        &  \yes           &  \kindof       & \no      \\
schema validation         & \yes       &  \yes           &  \no           & \no      \\
inference                 & \no        &  \no            &  \no           & \yes     \\
\tabularnewline
\multicolumn{1}{l}{\bf Languages:}  & & & & \\
query                     & \yes       &  \yes           &  \no           & \yes     \\
data manipulation         & \yes       &  \yes            &  \no           & \no      \\
schema manipulation       & \yes       &  \no            &  \no           & \no      \\
\bottomrule
\end{tabular}
\end{table}
%
We present a brief comparison of the data models in \cref{fig:data-models-overview}, focusing on features of the
data model and existing query languages.\footnote{In this table we are using the term RDB as a shorthand for the
  relational model.}
%
The features of the data model we considered were the logical structure it uses to represent data, whether the model is
an intrinsically ordered data model, and if it provides means, possibly external, of performing schema validation.
%
Inference capabilities allow to deduce new data based on existing one by specifying structural and representational
properties.
%
Regarding the languages, we represent the existence of languages for querying data contained in the respective model, as
well as manipulation languages for both data and schema.  Such languages can be used, for example, for inserting and
updating data or changing the representation structure of data.
%


\paragraph*{The Relational Model.}
%
As we have discussed in \cref{cha:introduction}, the relational model is used to store information for many
software applications.  A relational database consists of a set of relations (also commonly known as tables), and data
for each table is called a record.
%
As the data model overview presented in \cref{fig:data-models-overview} shows, the relational model is the most
mature, having stable query and manipulation languages.
%

\paragraph*{XML.}
With the uptake of the \ac{WWW}, more flexible data models were introduced, such as \ac{XML}.
%
Also Web pages are described using the \ac{HTML} language, a syntax that although similar to \ac{XML} is mostly focused
on rendering the contents in a \emph{web browser}.
%
\ac{XML} is more concerned with describing the data it contains while rendering \acs{XSLT} transformation language.
% 
\ac{XML} is a tree-based, ordered data representation format that imposes no restrictions on its element and attribute
names, nor on the nesting structure.
%
The \ac{XML} query and transformation languages were presented in \cref{sec:prelim-xquery}, mostly focusing on the
XQuery language.  A recommendation for an XQuery Update language was also presented
by~\citet{RobieChamberlinDyck:2011aa}.

\paragraph*{JSON.}
%
\cref{sec:json} presented another tree-based representation format, \ac{JSON}, that has recently gained traction and
uptake on the Web due to its easy integration with the JavaScript language, which is supported by all modern web
browsers.
%
\acs{JSON} is mostly regarded as an interchange format, notably lacking the specifications of any type of query language
and schema validation.  The \ac{JSON} data model is also tree-based and it distinguishes different structures
(\emph{objects} and \emph{arrays}), where objects consist of unordered sets, while arrays represent an ordered sequence
of elements.


\paragraph*{RDF.}
%
The advance of the traditional, human-readable Web into a machine-readable \emph{Semantic
  Web}~\cite{LeeHendlerLassila:2001aa} introduces a new data model: \ac{RDF}.
%
\ac{RDF} is a graph-based data model and, as discussed in \cref{sec:model-integr-data}, is suitable for representing
integrated data.
%
One main difference between \ac{RDF} and the other data models relates to its capabilities for deducing new data, based
on a specialised vocabulary called \acl{RDFS}.  \ac{RDFS}, as opposed to XML Schema, does not behave as a form of data
validation but rather as a form of deducing new data.
%
Although the new SPARQL~1.1 query language includes the specification of an update
language~\cite{GearonPassantPolleres:2012aa}, this is still not a finalised \ac{W3C} standard so we chose to omit it
from \cref{fig:data-models-overview}.
%
Possible forms of validating \ac{RDF} data, even though no recommendation exists, may involve
\begin{enumerate*}[label=(\roman*)]
\item using the SPARQL query language (presented in \cref{sec:sparql-preliminaries}) for determining if any triples
  do not match the constraints;
\item SPIN~\cite{KnublauchHendlerIdehen:2011aa} is a vocabulary that also allows to specify constraints for \ac{RDF}
  data; or
\item by using extensions of \ac{OWL} towards integrity constraints, \eg~\citet{TaoSirinBao:2010aa}.
\end{enumerate*}
%
In this thesis we focus primarily on the \ac{RDF} data model; however, several other graph-based database models exist
and a survey is presented by~\citet{AnglesGutierrez:2008ab}.
%



%%% Local Variables:
%%% mode: latex
%%% mode: flyspell
%%% mode: reftex
%%% TeX-master: "../thesis"
%%% End:
