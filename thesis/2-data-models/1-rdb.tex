

\section{Relational Model}
\label{sec:relational-model}

Due to the ever-growing need to store information, database systems were one of the most researched software systems and
have evolved from the use of the filesystem to store the data into the currently ubiquitous relational database
management systems~\cite{AbiteboulHullVianu:1995aa}.
%
Initially, the simple use of filesystems to store data did not enforce any structure on the data, where each file could
have its own internal structure. 
% 
One major turning point in the evolution of database systems was the separation of the \emph{logical} definition of the
data from its \emph{physical} representation (known as the \emph{data independence principle}).  Thus, the task of
managing the physical representation is left up to the \emph{database management system} and is usually hidden from the
database user.
%
This separation also led to the development of several logical data models that allowed data to be described
independently of their physical representation.
% 
The logical data models can be composed primarily of a Data Definition Language (DDL) and a Data Manipulation Language
(DML).  The DDL specifies the structure used to represent data while the DML specifies methods to access and update
data.
%
The hierarchical and network data models were the first logical models to be introduced, where the former used a tree
structure for representing its data and the latter a graph structure.
%
However, according to~\citet{AbiteboulHullVianu:1995aa}, major issues with these logical models were:
\begin{inparaenum}[(i)]
\item they were still closely related to the physical representation model; and
\item their DML were limited, focusing mostly on navigating the physical representation.
\end{inparaenum}

The introduction of the relational model by~\citet{Codd:1970aa}, with its strong theoretical foundations, propelled
database management systems forward, allowing for advances in efficient query translation methods (from the relational
logical model into the physical representation model) and query optimisation techniques.
%
In the relational model, data is represented primarily using named \emph{relations}, where each \emph{relational tuple}
(or record) consists of several typed and named \emph{attributes}.  A commonly used alternative representation for
relational data depicts each relation as a table, where the attributes are the columns of this table, and each
relational tuple is represented as a row in the table.
%
Next we present a definition of the relational model, based on~\citet{AbiteboulHullVianu:1995aa}, that relies on the
pairwise disjoint and countably infinite sets \AR for relation names, \AAs for attribute names and \AD for the domain of
values that the attributes can hold.
%
An element~$d \in \AD$ is called a \emph{constant} and for an attribute~$a \in \AAs$ we represent the domain of~$a$
as~$\dom{a}$.
%
Furthermore, a total order is assumed between the elements of \AAs: this is a necessary feature to later allow us to
specify relational instances in a similar fashion to logic programming~\cite{Lloyd:1987aa}.
%
\begin{definition}[Relation and database schema]
  \label{def:database-schema}
  A \emph{relation schema} is represented as~$r[U]$, where~$r \in \AR$ is a relation name and~$U \subset \AAs$ is a set
  of attribute names, called the \emph{sort} of~$r$ and denoted by~$\funcCall{sort}{r}$.  The \emph{arity} of~$r$
  consists of its number of attributes:~$|\funcCall{sort}{r}|$.
  % 
  In turn, a \emph{database schema}~$S$ is a non-empty and finite set of relation schemas.
\end{definition}



\begin{example}[Relational Schema]
  %
  \label{ex:bands-schema}
  %
  A possible schema for a relational database that stores information relevant to our \usecase
  is~$S = \set{\relationName{person}, \relationName{band}, \relationName{album}, \relationName{song}}$, where
  %
  \begin{align*}
    \funcCall{sort}{\relationName{person}} &= \set{\mathit{personId}, \mathit{personName}, \mathit{bandId}}\\
    \funcCall{sort}{\relationName{band}} &= \set{\mathit{bandId}, \mathit{bandName}}\\
    \funcCall{sort}{\relationName{album}} &= \set{\mathit{albumId}, \mathit{albumName}, \mathit{bandId}}\\
    \funcCall{sort}{\relationName{song}} &= \set{\mathit{songId}, \mathit{songName}, \mathit{albumId}}%\\
  \end{align*}%
  %
\end{example}



Other features of the relational model include primary and foreign keys.
%
Intuitively, a primary key consists of a set of attributes that uniquely identify the tuples of a relation.  For
example, in our database schema we assume an artificially generated number that uniquely identifies each person or band
($\mathit{personId}$ and~$\mathit{bandId}$, respectively).
%
Foreign keys are used to specify dependencies between attributes of two different relations: the connected attributes
must have the same value in both relations.  This can be seen in \cref{ex:bands-schema}, where the same attribute
names are used in different relations to specify the foreign keys, \eg~$\mathit{bandId}$ in the
relations~\relationName{person} and~\relationName{band}.

%
Furthermore, the \NULL value is assumed to belong to all domains and, unless otherwise specified by means of
constraints, can be used in place of any valid value for an attribute of a relation. The intended meaning of \NULL
values is to represent missing or unknown information.
%
However, since \NULL values greatly complicate the definition of the algebra operations (presented in
\cref{sec:sql-semantics}), we will, for the most part, ignore \NULL values in the presented definitions.



\subsection*{Database Instances}
\label{sec:database-instances}


\citet{AbiteboulHullVianu:1995aa} present different perspectives for representing relational tuples \ie~\emph{instances}
of relational schemas, the \emph{conventional} and \emph{logic programming} perspectives.
%
The so-called \emph{conventional} perspective on relational databases (used later in \cref{sec:sql-semantics})
represents tuples as functions, where a tuple~$t$ over a finite set of attributes~$U$ consists of a function~$u$ with
domain~$U$.
%
The sort of~$u$ is~$U$ and the value of~$u$ of an attribute~$a \in U$ is denoted~\funcCall{u}{a}.
%
Extending this notion to a set of attributes~$V \subseteq U$, we say that~$u[V] = u|_{V}$ denotes the restriction of the
function~$u$ to~$V$, \ie~$u[V]$ denotes a new tuple~$v$ over~$V$ such that~$\funcCall{v}{a} = \funcCall{u}{a}$ for each
attribute~$a \in V$.


%
An alternate view focuses on the \emph{logic programming} perspective, under which a relational tuple can be viewed as a
\emph{fact}.  For a relation name~$r$ with arity~$n$, a fact is an expression~$\fact{r}{a_{1}, \dots, a_{n}}$, where
each~$a_{i} \in \AD$ is a constant.  Facts can also be represented as~$\fact{r}{u}$, where~$u = \tuple{a_{1}, \dots,
  a_{n}}$.
%
According to this representation, a \emph{relation instance} over a relation schema~$r$ is a finite set of facts
over~$r$ and a \emph{database instance} over a database schema~$S$ is the union of all relation instances over~$r$, for
each relation schema~$r \in S$.
%
Since relations are represented as sets, the standard set operations of intersection, union and difference ($\cap,
\cup$, and~$-$, respectively) can be applied and relations can be compared using the~$\subset, \subseteq, =$, and~$\neq$
operators.
%
\footnote{We note that although the relational model is formally described using a set based semantics, it is common for
  database systems to use multi-sets for representing the data and the results of \acs{SQL} queries.}
%
\cref{ex:relational-data} represents a database instance following the logic programming perspective.
%
\begin{example}[Database Instance]
  %
  \label{ex:relational-data}
  % 
  The database instance containing the \usecase data from \cref{ex:usecase-data}, over the database schema
  presented in \cref{ex:bands-schema}, is as follows:
  %
  \pagebreak[2]
  \begin{align*}
    \{\ 
        & \fact{person}{1, \atom{Marco Hietala}, 1}, \fact{person}{2, \atom{Tarja Turunen}, 1}, \\
        & \fact{band}{1, \atom{Nightwish}}, \\
        & \fact{album}{1, \atom{Wishmaster}, 1}, \\
        & \fact{song}{1,\atom{FantasMic}, 1}, \fact{song}{2, \atom{Wishmaster}, 1} 
        \}
  \end{align*}
\end{example}
%
Both views on relational data are equivalent and are used for different formalisations of the relational model and query
languages (as presented in \cref{sec:querying-rdb}).









%%% Local Variables:
%%% mode: latex
%%% mode: flyspell
%%% mode: reftex
%%% TeX-master: "../thesis"
%%% End:
