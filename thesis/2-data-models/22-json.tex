
\section{JavaScript Object Notation (JSON)}
\label{sec:json}


\ac{JSON} is defined as a ``lightweight data-interchange format'' is another tree-based model designed as an alternative
to \ac{XML} for data transmission between different applications.
%
Although it originates from the JavaScript language, its format is language independent and thus can be used by several
programming languages.

The main structure of \ac{JSON} is called an \emph{object}, enclosed between~\character{\textbraceleft}
and~\character{\textbraceright}, and consists of an unordered sequence of \emph{name-value} pairs, separated
by~\character{:}. 
%
In such structures, the \emph{name} is restricted to be a string while \emph{value} may be one of
\begin{inparaenum}[(a)]
\item string;
\item number;
\item object;
\item array;
\item boolean (true or false); or
\item null.
\end{inparaenum}
%
\emph{Arrays} consist of an ordered list of values and are enclosed between~\character{[} and~\character{]}.
%
The \ac{JSON} representation of our \usecase data is presented in \cref{fig:bands-json}.
%
\begin{data}[t]
  \centering
  \lstinputlisting[basicstyle=\small\ttfamily]{0-data+queries/usecaseData.json}
  \caption{Bands in JSON (\fname{bands.json})}
\label{fig:bands-json}
\end{data}
%
This simple and unambiguous structure, coupled with the fact that \ac{JSON} is natively recognised and imported by
JavaScript, made \ac{JSON} extremely popular on the Web.  A comparative study of the uptake of \ac{XML} and JSON was
presented by~\citet{Musser:2011aa}.\footnote{The presentation is available
  at~\url{http://www.slideshare.net/jmusser/j-musser-semtechjun2011/}, retrieved on 2012/04/10.}


Although \ac{JSON} and \ac{XML} serve very similar purposes, commonly presented advantages for using \ac{JSON} over
\ac{XML} are:
%
\begin{inparaenum}[(i)]
\item \ac{JSON} documents are (usually) smaller; and
\item an external schema is not required to unambiguously represent the content.
\end{inparaenum}
%
On the other hand, one of the biggest disadvantages of \ac{JSON} is its lack of support for namespaces: whereas in
\ac{XML} it is possible to distinguish attribute and element names by giving them different namespace prefixes, this is
not possible in \ac{JSON}.


Due to the similarities between the \ac{JSON} and \ac{XML} formats, the question of translating between them has arisen.
Since \ac{XML} is arguably the more expressive language, being more flexible in its format, converting from \ac{XML} to
\ac{JSON} poses some problems:
%
\begin{description}
\item[Namespaces.]  Since \ac{JSON} does not natively support namespaces, a non-trivial issue is how to represent
  \ac{XML} namespaces in such a fashion that can be translated back into \ac{XML};
\item[Attributes.]  Similar to namespaces, \ac{JSON} has no equivalent for \ac{XML} element attributes and similar
  representational questions arise for \ac{XML} attributes;
\item[Mixed Content.]  Since the contents of \ac{XML} elements can consist of text values arbitrarily mixed with other
  elements, an accurate representation of this mixed content in \ac{JSON}, although possible, would yield a very verbose
  representation.
\end{description}
%
On the other hand, converting from \ac{JSON} to \ac{XML} is a straightforward task, relying solely on using predefined
element names to represent \ac{JSON} objects and arrays.  This straightforward conversion will be used in
\cref{cha:xsparql} for the inclusion of \ac{JSON} data into our proposed transformation language.



%%% Local Variables:
%%% mode: latex
%%% mode: flyspell
%%% mode: reftex
%%% TeX-master: "../thesis"
%%% End:
