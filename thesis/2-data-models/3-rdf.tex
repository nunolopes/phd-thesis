

\section{Resource Description Framework (RDF)}
\label{sec:rdf}

In the attempts to transform the Web into a global database, another model, the~\acl{RDF}, was proposed as the data model
for representing machine readable data, also known as \emph{Semantic Web} data.  The \ac{RDF} model allows for the
specification of \emph{statements} about \emph{Web Resources}~\cite{ManolaMiller:2004aa}.
%
However, this general notion of resource may refer not only to virtual entities (that can only be found on the Web) but
also any physical entity that can be \emph{identified} on the Web.
%
Such resources are identified by a \ac{URI}, generally indicating where the resource is located, or a \emph{blank node},
which plays the role of an anonymous resource and allows for the modelling of incomplete or unknown data.  In the
following, we identify blank nodes by using the prefix \character{\_:} followed by a string, called the \emph{blank node
  label}.  Blank nodes are scoped to the document they appear in, \ie~two blank nodes from different documents, even if
they have the same label, must be considered different.
%
Furthermore, \ac{RDF} \emph{literals} can be used to specify string- or datatype-based values for properties.
%
The atomic \emph{statements} of the \ac{RDF} data model are called \emph{RDF triples} consisting of \emph{subject},
\emph{predicate} and \emph{object}, and intuitively state that
%
the subject is connected to the object by the predicate relation.
%
Since triples can also be viewed as part of a labelled directed graph, where subjects and objects correspond to nodes
and predicates to edges of the graph, we refer to a set of such \ac{RDF} triples as an \emph{RDF graph}.
%


For the definitions of the \ac{RDF} syntax, we rely on the the pairwise disjoint alphabets~$\AU$,~$\AB$, and~$\AL$
denoting \emph{URI references}, \emph{blank nodes} and \emph{literals}, respectively.\footnote{We assume~$\AU, \AB$,
  and~$\AL$ fixed, and for presentation purposes we will denote unions of these sets by concatenating their names.}  We
call the elements in~$\AUBL$ \emph{terms}.
%
\begin{definition}[RDF Triple]
  \label{def:rdf-triple}
  An \emph{RDF triple} is~$\tau = \triplee{s}{p}{o} \in \AUBL \times \AU \times \AUBL$, where~$s$ is called the
  \emph{subject},~$p$ the \emph{predicate}, and~$o$ the \emph{object}.
\end{definition}
%
\nd Strictly speaking, according to the \ac{RDF} specification~\cite{Hayes:2004aa} literals are not allowed to be the
subject of \ac{RDF} triples however, as commonly adopted in other
works~\cite{MunozPerezGutierrez:2007aa,PrudhommeauxSeaborne:2008aa,CarrollBizerHayes:2005aa}, this definition considers
a \emph{generalised RDF Triple}, that allows literals for the subject positions.


\begin{definition}[RDF Graph]
  \label{def:rdf-graph}
  Following the definition of an \ac{RDF} triple, an \emph{RDF graph}~$G$ consists of a set of triples.
  % 
  The \universe of~$G$,~$\fc{\universe}{G}$, is the set of elements in~$\AUBL$ that occur in the triples of~$G$ and the
  \emph{vocabulary} of~$G$,~$\fc{\voc}{G}$, is~$\fc{\universe}{G} \cap \AUL$.
  % 
  Furthermore, we say that~$G$ is \emph{ground} \iff~$\fc{\universe}{G} = \fc{\voc}{G}$, \ie~$G$ does not contain blank nodes.
\end{definition}

When combining different \ac{RDF} graphs some care must be taken to ensure the local scope of blank nodes:
%
\begin{definition}[RDF merge]
  \label{def:rdf-merge}
  Let~$S$ be a set of RDF graphs.  The \emph{RDF merge} of~$S$ consists of the set-theoretical union of all the graphs
  in~$S$ after blank nodes have been \emph{standardised apart}: if any two graphs contain the same blank node label, all
  occurrences of these labels within the same graph are replaced by a new blank node label that is not present in any of
  the other graphs.
\end{definition}
%
\noindent This disambiguation of blank node labels is meant to keep any blank nodes between different graphs distinct, thus
maintaining the scope of blank nodes to the graph they occur in.
%

Similar to \ac{XML} namespaces, \acp{URI} can be abbreviated by using a namespace prefix.
%
For example, the URI \stringValue{\qname{foaf}{Person}} from the widely used \ac{FOAF} ontology, consists of the prefix
\stringValue{\term{foaf}}, which is associated with the URI \stringValue{\url{http://xmlns.com/foaf/0.1/}}, and the
local part \stringValue{\term{Person}}.  The complete URI represented by \stringValue{\qname{foaf}{Person}} is thus
\stringValue{\url{http://xmlns.com/foaf/0.1/Person}}.
%
\qname{rdf}{type} predicates can be used to specify that an \ac{RDF} resource is an \emph{instance} of a \emph{class};
for example the triple:
%
\begin{equation}
\label{eq:rdf-triple}
\triplee{\qname{dbpedia}{Marco\mathunderscore{}Hietala}}{\qname{rdf}{type}}{\qname{foaf}{Person}}
\end{equation}
% 
intuitively specifies that the resource \qname{dbpedia}{Marco\mathunderscore{}Hietala} is used to identify a person.

\ac{RDF} literals can be further classified as \emph{plain}, in which case they can optionally contain a \emph{language
  tag}, or \emph{typed} literals.
%
Typed literals include a \ac{URI} that refers to their datatype, usually one of the \ac{XML} Schema built-in datatypes
or the newly defined \ac{RDF} datatype \qname{rdf}{XMLLiteral} (used to indicate the literal contains \ac{XML} data).
%
The specific syntax of literals is presented in the next section.


Another \ac{RDF} feature, although not so commonly used, is \emph{reification}, can be used to represent
meta-information about an \ac{RDF} triple, \eg~provenance information.  Any \ac{RDF} statement can be reified by
representing it as four distinct \ac{RDF} triples with a common subject.  Although it is possible to use a \ac{URI} for
the subject of reified triples, as presented in \cref{ex:reified-triple}, it is common to use a blank node.
%
Reification is later used in \cref{cha:anql} as one possible serialisation for Annotated RDFS graphs.
%
\begin{example}[Reified \ac{RDF} statement]
  %
  \label{ex:reified-triple}
  The \ac{RDF} statement~\eqref{eq:rdf-triple} can be reified as the following triples:
  %
  \vspace{-\abovedisplayskip}
  %
  \begin{align*}
    &\triplee{\bnode{r1}}{\qname{rdf}{type}}{\qname{rdf}{Statement}}\\
    &\triplee{\bnode{r1}}{\qname{rdf}{subject}}{\qname{dbpedia}{Marco\mathunderscore{}Hietala}}\\
    &\triplee{\bnode{r1}}{\qname{rdf}{predicate}}{\qname{rdf}{type}}\\
    &\triplee{\bnode{r1}}{\qname{rdf}{object}}{\qname{foaf}{Person}}
  \end{align*}
  % 
\end{example}


Yet another feature of \ac{RDF} are \emph{collections}, which allow to state that a group of resources are members of
the collection.  This is represented in \ac{RDF} using a list structure following a predefined vocabulary:
\qname{rdf}{List} states the type of the resource and the \qname{rdf}{first} and \qname{rdf}{rest} properties are used
to represent the list.  This list must be terminated by \qname{rdf}{nil}.  Collections are used in
\cref{sec:bf-extended-bgp}.

\medskip

Next, \cref{sec:representation-syntax} presents how \ac{RDF} can be serialised in order to be stored or
exchanged, focusing on the RDF/XML and Turtle syntaxes and \cref{sec:rdf-semantics} presents the semantics of
\ac{RDF}.  Finally, \cref{sec:rdf-schema} focuses on the inferencing capabilities of \ac{RDF} by describing
\acl{RDFS}.

\subsection{Representation Syntaxes}
\label{sec:representation-syntax}

Although the \ac{RDF} specification states that the normative syntax for writing \ac{RDF} graphs is
RDF/XML~\cite{BeckettMcBride:2004aa}, this syntax is not favoured among practitioners and there have been proposals to
support other serialisation formats and move away from \ac{XML} based representations~\cite{Beckett:2010aa}.
% 
Other well known syntaxes for \ac{RDF} are Turtle~\cite{BeckettBerners-Lee:2008aa} and RDFa~\cite{AdidaBirbeck:2008aa},
where Turtle consists of a specialised syntax for RDF and RDFa defines a mechanism to incorporate \ac{RDF} statements into
(X)HTML webpages.
%
In the following, we briefly highlight the constructs of the RDF/XML and Turtle syntaxes.


\subsubsection{RDF/XML}
\label{sec:rdfxml}

Although RDF/XML is the normative syntax for \ac{RDF}, this serialisation is very flexible, and the same \ac{RDF} graph can
be serialised in numerous different ways.
%
As we will point out in \cref{cha:xsparql}, this lack of a canonical RDF/XML serialisation is one of the major
roadblocks to process \ac{RDF} data using \ac{XML} tools.
%


%
The RDF/XML syntax uses \ac{XML} elements to represent \ac{RDF} subjects, predicates and objects:
\stringValue{\qname{rdf}{Description}} elements are used to represent nodes (subjects and objects) of the \ac{RDF}
graph, where the \stringValue{\qname{rdf}{about}} attribute specifies the URI of the node.
%
In turn, predicates are represented as \ac{XML} elements where the name of the element corresponds to the \ac{URI}
(represented as an \ac{XML} QName) of the predicate.
%
A possible RDF/XML serialisation of the \ac{RDF} graph from our running example is presented in
\cref{fig:bands-rdfxml}.
%
\begin{data}[p]
  \centering
  \lstinputlisting[language=XML,basicstyle=\small\ttfamily]{0-data+queries/usecaseData.rdf}
  \caption{Bands in RDF/XML}
  \label{fig:bands-rdfxml}
\end{data}
%
\begin{data}[t]
  \centering
  \lstinputlisting{0-data+queries/usecaseData-abbrev.rdf}
  \caption{Bands in abbreviated RDF/XML}
\label{fig:bands-rdfxml-abbrev}
\end{data}


The RDF/XML serialisation allows the use of several abbreviations and 
%
an abbreviated serialisation of the \ac{RDF} graph in \cref{fig:bands-rdfxml} is presented in
\cref{fig:bands-rdfxml-abbrev}. 
%
One of the possible abbreviations is, if an object node does not contain any other predicates, to omit the
\stringValue{\qname{rdf}{Description}} element and specify the URI reference of the object in the
\stringValue{\qname{rdf}{resource}} attribute of the predicate element node (for example in lines~13 and~14). 
%
Another abbreviation is, in case the subject contains an \stringValue{\qname{rdf}{type}} predicate, to replace the
\stringValue{\qname{rdf}{Description}} element name with the type of the subject (for example lines~9, 12, and~16).
%
Also, several predicates about the same subject can be nested in the same \ac{XML} element (as presented in
lines~$19$--$24$ of \cref{fig:bands-rdfxml-abbrev}).
%


Blank nodes (anonymous resources) can be given a label using the \stringValue{\qname{rdf}{nodeID}} attribute and can
later be referred to (from within the same document).
%
Literals can be specified as the text content of a property \ac{XML} element (\eg~line~$17$ of
\cref{fig:bands-rdfxml-abbrev}) or alternatively as the value of an attribute where the attribute name is the
corresponding property \ac{URI} (as in line~$12$ of \cref{fig:bands-rdfxml-abbrev}).
%
Language tags are specified as the value of the \stringValue{\qname{xml}{lang}} attribute, while typed literals use the
\stringValue{\qname{xml}{datatype}} attribute.




\subsubsection{Turtle}
\label{sec:turtle}
%

Stemming from its \ac{XML} roots and, even with all the proposed abbreviations, the RDF/XML syntax is still very verbose
and neither easy to read nor write for humans.
%
To overcome these problems, the Turtle syntax~\cite{BeckettBerners-Lee:2008aa} aims to be a compact representation for
\ac{RDF} graphs that is easier to read and write for users and includes abbreviations for common \ac{RDF} patterns.
%
Turtle is based on N-Triples, a simple syntax introduced for the \ac{RDF} test cases~\cite{GrantBeckett:2004aa} that
represents one triple per line.  Furthermore, Turtle incorporates features from Notation 3~\cite{Berners-Lee:2005aa},
most notably:
%
\begin{inparaenum}[(i)]
\item namespace declarations,
\item shortcuts for commonly used \ac{RDF} patterns, and
\item a syntax for anonymous blank nodes.
\end{inparaenum}

%
\begin{data}[t]
  \centering
  \lstinputlisting{0-data+queries/usecaseData.ttl}
  \caption{Bands in Turtle (\fname{bands.ttl})}
\label{fig:bands-turtle}
\end{data}
%

The Turtle \ac{RDF} representation of the \usecase data from \cref{ex:usecase-data} is presented in
\cref{fig:bands-turtle}.  In the Turtle syntax, \term{\texttt{@}prefix} declarations can be used to abbreviate common
URIs (similar to \ac{XML} namespaces and QNames) and URIs must be enclosed between the \character{<} and \character{>}
characters.
%
Literals are surrounded by double-quotes, as in \literal{Nightwish}, and may include a suffix to specify the language
tag following the \character{@} separator character, for example \lang{Marco\ Hietala}{en}, or a datatype \ac{URI} after
the \character{\^{}\^{}} separator as in \dt{5}{\uril{http://www.w3.org/2001/XMLSchema\#integer}}.
%
Blank nodes are prefixed with~\character{\_:} \eg~\bnode{song506} where the blank node label is \term{song506}.  A
shortcut for unnamed blank nodes is provided by using the \character{[]} notation.  Another useful shortcut is the
\character{\term{a}} keyword (line~7 of \cref{fig:bands-turtle}), which represents the URI
\uril{http://www.w3.org/1999/02/22-rdf-syntax-ns\#type}, also commonly abbreviated as \qname{rdf}{type}.


Furthermore the \character{;} and \character{,} symbols can be used to create new triples without repeating the subject
or subject and predicate, respectively.  For example, the triples from lines 7--10 of \cref{fig:bands-turtle} can
be written as:
\begin{lstlisting}[frame=none,numbers=none]
dbpedia:Nightwish a mo:MusicGroup ;
                  foaf:name "Nightwish" ; 
                  foaf:member dbpedia:Marco_Hietala, dbpedia:Tarja_Turunen .
\end{lstlisting}
%
Commonly used datatypes can also be abbreviated: for instance \term{5} is equivalent to
\dt{5}{\uril{http://www.w3.org/2001/XMLSchema\#integer}}, while \term{5.0} corresponds to
\dt{5.0}{\uril{http://www.w3.org/2001/XMLSchema\#decimal}}.
%
Turtle also provides abbreviations for \ac{RDF} collections by listing a space-separated sequence of \ac{RDF} terms
enclosed by \lit{(} and \lit{)}.


\subsection{Semantics}
\label{sec:rdf-semantics}

The semantics of \ac{RDF} is specified using a model theory as per~\citet{Hayes:2004aa}, which is a common form of specifying
semantics, for example for first-order logic.
%
Model theoretic semantics of formal languages assign any expression in the language to an element of a possible
``world'' -- called an \emph{interpretation} -- and also specify the necessary conditions for an interpretation to be
considered valid -- called a \emph{model}.  The notion of \emph{entailment} between two expressions,~$A$ entails~$B$,
can then be defined as any interpretation that is a model of~$A$ must also be a model of~$B$.
%
Based on this semantics, it is possible to define what are the \emph{entailed} consequences of the interpretation and
what are valid \emph{inference rules}. 


In the case of \ac{RDF}, language expressions are considered as being the terms in the universe of the graph and also the
individual triples, \ie~each term in the vocabulary is assigned to an interpretation element, where plain literals are
interpreted as themselves and blank nodes are interpreted as existential variables (scoped to the \ac{RDF} graph in which
they occur).


The \ac{RDF} semantics~\cite{Hayes:2004aa} specifies different types of interpretation, and hence of entailment, ranging
from the so-called \emph{simple} interpretation to the more complex \emph{RDFS} and \emph{datatype} interpretations.
%
Simple interpretations consider only the vocabulary of the triples present in the graph while other types of
interpretations, namely \ac{RDF} and \ac{RDFS}-interpretations, consider predefined vocabularies and a set of \ac{RDF}
triples that any interpretation must satisfy by default: the so-called \emph{axiomatic} triples.


\ac{RDF} interpretations consider the terms defined in the \url{http://www.w3.org/1999/02/22-rdf-syntax-ns#} namespace
(commonly abbreviated with the prefix \term{rdf}).
%
For instance, \ac{RDF} interpretations impose conditions that identify a subset of the interpretation resources as being
\emph{properties} (interpretation resources of type \term{rdf{:}Property}) and introduce the new datatype
\term{rdf{:}XMLLiteral} to represent well-formed \ac{XML} literals. % (c.f. \cref{sec:xml}).



\ac{RDFS}-interpretations consider further vocabulary in the \term{rdfs} namespace
(\url{http://www.w3.org/2000/01/rdf-schema#}) that define further conditions on top of \ac{RDF} interpretations and
introduce the notion of a \emph{class}.  A class is itself a resource that denotes a common set of resources, which are
called \emph{instances} of the class and all have the class as the value for their \qname{rdf}{type} property.
%
Informally, the \ac{RDFS} vocabulary states the following:
%
\begin{inparaenum}[(i)]
\item $\triple{p, \term{rdfs{:}subPropertyOf}, q}$ means that any resources related by property~$p$ are also related by
  property~$q$;
\item $\triple{c, \term{rdfs{:}subClassOf}, d}$ means that any instance of class~$c$ is also an instance of class~$d$;
\item $\triple{a, \term{rdf{:}type}, c}$ means that~$a$ is an instance of~$c$;
\item $\triple{p, \term{rdfs{:}domain}, c}$ means that the \emph{domain} of property~$p$ is~$c$, \ie~any resource that
  is the subject of a triple with predicate~$p$ is an instance of~$c$; and
\item $\triple{p, \term{rdfs{:}range}, c}$ means that the \emph{range} of property~$p$ is~$c$, \ie~any resource that is
  an object of a triple with predicate~$p$ is an instance of~$c$.
\end{inparaenum}
%

Further extending \ac{RDFS}-interpretations, a D-interpretation provides an (admittedly minimal) support for \ac{XSD}
datatypes~\cite{BironMalhotra:2004aa} extended with the \term{rdf{:}XMLLiteral} datatype.
%
Another \ac{W3C} specification that provides a more expressive inference system than \ac{RDFS} is the \ac{OWL}, now in
its second version~\cite{HitzlerKrotzschParsia:2009aa}.  The \ac{OWL} language introduces new concepts such as the
distinction between object and datatype properties, class disjointness assertions, and assertions of equality between
individuals, among others.
%
It is noteworthy that D-interpretations, \ac{OWL}, and the \term{rdf{:}XMLLiteral} datatype may introduce
inconsistencies in \ac{RDF}.
%
However, in this thesis, we are mostly interested in \ac{RDFS} inferences and we do not consider D-interpretations,
\ac{OWL} constructs, nor the typing of \term{rdf{:}XMLLiteral} and thus we avoid any inconsistencies in \ac{RDF}.  
%


Although in the \ac{RDF} Semantics specification~\cite{Hayes:2004aa} the semantic conditions for each interpretation are
detailed separately, in this thesis we follow the formalism defined
by~\citet{MunozPerezGutierrez:2007aa,MunozPerezGutierrez:2009aa} and provide a single notion of interpretation that
covers Simple, \ac{RDF}, and \ac{RDFS}-interpretations. 
%
Intuitively, the interpretation of an \ac{RDF} triple~$\triple{s,p,o}$ is true if~$s$,~$p$ and~$o$ belong to the
interpretation vocabulary,~$p$ is a property and the pair~$(s,o)$ belongs to the extension of the property~$p$.  An
interpretation assigns the value true to an \ac{RDF} graph if it assigns the value true to all of its triples.
%

%
Additionally, in order to assign a truth value for a graph containing blank nodes, an interpretation must rely on a
mapping from the set of blank nodes present in the graph to terms in the graph.  This mapping of blank nodes ensures
that all occurrences of the same blank node are mapped to the same interpretation element and, since this mapping is not
an integral part of the interpretation, it also ensures that blank nodes have no visibility outside the graph.
%
Based on~\citet{MunozPerezGutierrez:2007aa}, we define the notion of \emph{map}:
%
\begin{definition}[Map]
  A \emph{map} is a function~$\theta : \AUBL \to \AUBL$ preserving URIs and literals, \ie~$\theta(t) = t$, for all~$t \in
  \AUL$. 
  % 
  Given a graph~$G$, we define~$\theta(G) = \set{ \triple{\theta(s), \theta(p), \theta(o)} \mid \triple{s, p, o} \in G }$. We speak of
  a map~$\theta$ from~$G_{1}$ to~$G_{2}$, and write~$\theta : G_{1} \to G_{2}$, if~$\theta$ is such that~$\theta(G_{1}) \subseteq
  G_{2}$.
  %
  Furthermore, we say that a map~$\theta$ is a \emph{grounding} of a graph~$G$, iff~$\theta(G)$ is a ground graph.
\end{definition}
%

We next present the definition of interpretation according to~\citet{MunozPerezGutierrez:2007aa}:
%
\begin{definition}[Interpretation,~\citet{MunozPerezGutierrez:2007aa}]
  \label{def:interpretation}
  % 
  An \emph{interpretation}~$\I$ over a vocabulary~$V$ is a tuple~$\I =\tuple{\Delta_{R}, \Delta_{P}, \Delta_{C},
    \Delta_{L}, \intP{\cdot}, \intC{\cdot}, \int{\cdot}}$, where~$\Delta_{R}, \Delta_{P}$,~$\Delta_{C}, \Delta_{L}$ are
  the interpretation domains of~$\I$, which are finite non-empty sets, and~$\intP{\cdot}, \intC{\cdot}, \int{\cdot}$ are
  the interpretation functions of~$\I$. They have to satisfy:
  % 
  \begin{enumerate}[noitemsep]
  \item $\Delta_{R}$ are the resources (the domain or universe of~$\I$);
  \item $\Delta_{P}$ are property names (not necessarily disjoint from~$\Delta_{R}$);
  \item $\Delta_{C} \subseteq \Delta_{R}$ are the classes;
  \item $\Delta_{L} \subseteq \Delta_{R}$ are the literal values (containing~$\AL \cap V$);
  \item $\intP{\cdot}$ is a function~$\intP{\cdot}\colon \Delta_{P} \to 2^{\Delta_{R} \times \Delta_{R}}$;
  \item $\intC{\cdot}$ is a function~$\intC{\cdot}\colon \Delta_{C} \to 2^{\Delta_{R}}$;
  \item $\int{\cdot}$ maps each~$t \in \AUL \cap V$ into a value~$\int{t} \in \Delta_{R} \cup \Delta_{P}$ such
    that~$\int{\cdot}$ is the identity for plain literals and assigns an element in~$\Delta_{R}$ to each element
    in~$\AL$.
  \end{enumerate}
\end{definition}



\subsection{RDF Schema}
\label{sec:rdf-schema}

As briefly presented in the previous section, \ac{RDFS} is a vocabulary that allows for the description of relations
between \ac{RDF} resources.
%
For this thesis, we will rely on a fragment of \ac{RDFS}, called~$\rhodf$, presented
by~\citet{MunozPerezGutierrez:2007aa}, that covers essential features of \ac{RDFS}.
%
$\rhodf$ consists of the following subset of the \ac{RDFS} vocabulary:~$\set{ \term{rdfs{:}subPropertyOf},
  \term{rdfs{:}subClassOf}, \term{rdf{:}type}, \term{rdfs{:}domain}, \term{rdfs{:}range}}$.  In the following, for
readability purposes, we are using the following abbreviations:~$\spp$ for \term{rdfs{:}subPropertyOf},~$\subclass$ for
\term{rdfs{:}subClassOf},~$\typeR$ for \term{rdf{:}type},~$\domR$ for \term{rdfs{:}domain}, and~$\range$ for
\term{rdfs{:}range}.


Based on the definition of interpretation (\cref{def:interpretation}), we can define the concept of
\emph{model} of an \ac{RDF} graph:
%
\begin{definition}[Model~\cite{MunozPerezGutierrez:2007aa}]
  % 
  \label{def:rdf-model}
  %
  An interpretation~$\I$ is a \emph{model} of a ground graph~$G$, denoted~$\I \models G$, \iff~$\I$ is an interpretation
  over the vocabulary~$\rhodf \cup \universe(G)$ that satisfies the following conditions:
  % 
  \begin{description}\label{condRDF}
  \item[Simple:] \
    \begin{enumerate}
    \item for each~$\triple{s, p, o} \in G$,~$\intA{p} \in\Delta_{P}$ and~$(\intA{s}, \intA{o}) \in \intP{\intA{p}}$;
    \end{enumerate}
  \item[Subproperty:] \
    \begin{enumerate}
    \item $\intP{\intA{ \spp}}$ is transitive over~$\Delta_{P}$;
    \item if~$(p, q) \in \intP{\intA{ \spp}}$ then~$p, q \in \Delta_{P}$ and~$\intP{p} \subseteq \intP{q}$;
    \end{enumerate}
  \item[Subclass:] \
    \begin{enumerate}
    \item $\intP{\intA{\subclass}}$ is transitive over~$\Delta_{C}$;
    \item if~$(c, d) \in \intP{\intA{\subclass}}$ then~$c, d \in \Delta_{C}$ and~$\intC{c} \subseteq \intC{d}$;
    \end{enumerate}
  \item[Typing I:] \
    \begin{enumerate}
    \item $x \in \intC{c}$ \iff~$(x,c) \in \intP{\intA{\typeR}}$;
    \item if~$(p, c) \in \intP{\intA{\domR}}$ and~$(x, y) \in \intP{p}$ then~$x \in \intC{c}$;
    \item if~$(p, c) \in \intP{\intA{\range}}$ and~$(x, y) \in \intP{p}$ then~$y \in \intC{c}$;
    \end{enumerate}
  \item[Typing II:] \
    \begin{enumerate}
    \item For each~$\eee \in \rhodf$,~$\intA{\eee} \in \Delta_{P}$
    \item if~$(p, c) \in \intP{\intA{\domR}}$ then~$p \in \Delta_{P}$ and~$c \in \Delta_{C}$
    \item if~$(p, c) \in \intP{\intA{\range}}$ then~$p \in \Delta_{P}$ and~$c \in \Delta_{C}$
    \item if~$(x, c) \in \intP{\intA{\typeR}}$ then~$c \in \Delta_{C}$
    \end{enumerate}
  \end{description}
\end{definition}

\nd Entailment among ground graphs~$G$ and~$H$ behaves as per the model-theoretic semantics: any interpretation that is
a model of~$G$ is also a model of~$H$.
%
In the case where~$G$ and~$H$ may contain blank nodes,~$G \models H$ \iff for any grounding~$G'$ of~$G$ there is a
grounding~$H'$ of~$H$ such that~$G' \models H'$.
%

In~\citet{MunozPerezGutierrez:2007aa}, the authors define two variants of the semantics: the default one includes
reflexivity of~$\intP{\int{ \spp}}$ and~$\intC{\int{\subclass}}$ over~$\Delta_{P}$ and~$\Delta_{C}$, respectively, but
herein we are only considering the alternative semantics presented in~\citet[Definition 4]{MunozPerezGutierrez:2007aa},
which omits this requirement.
% 
As a consequence, inferences such as~$G \models \triple{a, \subclass, a}$ are not supported.  However, the drawback of
this is minimal since such inferences do not add expressive power and are thus of marginal interest.
%


\subsubsection*{Deductive System}
% 
In what follows, we present the sound and complete deductive system from~\citet{MunozPerezGutierrez:2007aa}.
%
The system is arranged in groups of rules that capture the semantic conditions of models. In every rule,~$A,B,C,X$,
and~$Y$ are meta-variables representing elements in~$\AUBL$ and~$D,E$ represent elements in~$\AUL$.  
%
The rules are as follows:
% 
\begin{description}
\item[1. Simple:] ~ \\[0.5em]
  \begin{tabular}{llll}
    $(a)$ & $\frac{G}{G'}$ for a map~$\theta:G' \to G$ & $(b)$ & $\frac{G}{G'}$ for~$G' \subseteq G$
  \end{tabular}
\item[2. Subproperty:] ~ \\[0.5em]
  \begin{tabular}{llll}
    $(a)$ & $\frac{\triple{A,  \spp, B},  \triple{B,  \spp, C}}{\triple{A,  \spp, C}}$ & $(b)$ & $\frac{\triple{D,  \spp, E},  \triple{X, D, Y}}{\triple{X, E, Y}}$
  \end{tabular}
\item[3. Subclass:] ~ \\[0.5em]
  \begin{tabular}{llll}
    $(a)$ & $\frac{\triple{A, \subclass, B},  \triple{B, \subclass, C}}{\triple{A, \subclass, C}}$ & $(b)$ & $\frac{\triple{A, \subclass, B},  \triple{X, \typeR, A}}{\triple{X, \typeR, B}}$
  \end{tabular}
\item[4. Typing:] ~ \\[0.5em]
  \begin{tabular}{llll}
    $(a)$ & $\frac{\triple{D, \domR, B},  \triple{X, D, Y}}{\triple{X, \typeR, B}}$ & $(b)$ & $\frac{\triple{D, \range, B},  \triple{X, D, Y}}{\triple{Y, \typeR, B}}$
  \end{tabular}
\item[5. Implicit Typing:] ~ \\[0.5em]
  \begin{tabular}{llll}
    $(a)$ & $\frac{\triple{A, \domR, B},  \triple{D,  \spp, A}, \triple{X, D, Y}}{\triple{X, \typeR, B}}$  &
    $(b)$ & $\frac{\triple{A, \range, B},  \triple{D,  \spp, A}, \triple{X, D, Y}}{\triple{Y, \typeR, B}}$
  \end{tabular}
\end{description}
%
The deductive system presented by~\citet{MunozPerezGutierrez:2007aa} includes 7 rules, where the missing rules (rules
6-7) handle reflexivity.  
%
Furthermore, as noted in~\citet{MunozPerezGutierrez:2007aa}, the ``Implicit Typing'' rules are a necessary addition to
the rules presented in~\citet{Hayes:2004aa} for complete \ac{RDFS} entailment.  These represent the case when variable~$A$
in~$\triple{D, \spp, A}$ and~$\triple{A, \domR, B}$ or~$\triple{A, \range, B}$, is a property implicitly represented by
a blank node.

We denote with~$\set{\tau_{1}, \ldots, \tau_{n}} \vdash_{\mathsf{RDFS}} \tau$ that the consequence~$\tau$ is obtained from
the premise~$\tau_{1}, \ldots, \tau_{n}$ by applying one of the inference rules 2--5 above.
% 


%%% Local Variables:
%%% mode: latex
%%% mode: flyspell
%%% mode: reftex
%%% TeX-master: "../thesis"
%%% End:
