\chapter{Data Models}
\label{cha:data-models}



This chapter details how data is represented in each of the data models mentioned in the previous chapter.
%
In the context of databases, a common definition for the term \emph{data model} is presented
by~\citet{SilberschatzKorthSudarshan:1996aa} as ``a collection of conceptual tools for describing the real-world
entities to be modeled in the database and the relationships among these entities''.
% 
This definition focuses on the (essential) data representation capabilities of a data model; however more fine-grained
definitions, namely by~\citet{Codd:1980aa}, also include in the notion of data model operators and inference rules for
retrieving or deriving data as well as integrity rules for determining accepted database states.
% 
In this chapter, we consider the definition by~\citet{SilberschatzKorthSudarshan:1996aa} and are thus particularly
interested in the data representation aspects of each data model: the relational model, the tree based data models
\ac{XML} and \acs{JSON}, and \ac{RDF}.
%
However, we do touch upon the inferencing capabilities of \ac{RDF} in \cref{sec:rdf-schema} with~\acl{RDFS}; these will
be required later in \cref{cha:anql}.
%
In \cref{cha:query-languages}, we will focus on querying the presented data models.

\subsubsection*{Running example}
\label{sec:example}

In this thesis we will use examples from the music domain, where we are interested in representing \emph{persons},
\emph{bands}, \emph{albums}, and \emph{songs}.
%
In our simplified model, persons can be members of bands and can listen to specific songs.  Bands release albums, which
in turn include songs.
%
For presentation purposes and conciseness of examples, we will use a reduced set of entities, included in
\cref{ex:usecase-data}.

We chose to use the music domain, as opposed to more enterprise oriented examples, due to the availability of
information.
%
We note that data required for this \usecase is available in the \ac{WWW}, for instance, information regarding bands can
be found in Wikipedia (\url{http://www.wikipedia.org/}) or MusicBrainz~\cite{Swartz:2002aa}, while personalised
information about the songs individuals listen to can be found in Last.fm (\url{http://last.fm/}).


\paragraph*{Wikipedia/DBpedia.} The widely known online encyclopaedia Wikipedia relies on user contributions for its
contents.
%
DBpedia~\cite{BizerLehmannKobilarov:2009aa} consists of a partial export of the information from Wikipedia into the RDF
format, accessible using standard query languages such as SPARQL.
%
For our running example, we are interested in extracting information regarding artists, bands, and albums and we often
use DBpedia \acp{URI} as identifiers for entities.


\paragraph*{Last.fm.} The online Web service Last.fm allows users to submit the songs and artists they listen to.
%
These songs are aggregated in order to create a user profile containing the top artists, lists of songs from each user,
and provide personalised recommendations of new artists.
%
The data presented in this thesis was extracted from this author's Last.fm website.\footnote{Last.fm user profile
  available at~\url{http://last.fm/user/jacktrades/}, retrieved on 2012/04/10.}
%
The data retrievable via the Last.fm API contains information such as the five most played bands from a user profile
and, for each band, the most played tracks by the user and the albums they are included in.

\begin{example}[\Usecase data]
  % 
  \label{ex:usecase-data}
  % 
  This example presents the data we are using in the examples of this thesis:\footnote{For the sake of conciseness of
    examples, we restrict the presented data to one band, two members and one album of the band.}
  % 
  \begin{description}[nosep,labelindent=!]
  \item[persons:] Marco Hietala, Tarja Turunen
  \item[bands:] Nightwish
  \item[albums:] Wishmaster
  \item[songs:] FantasMic, Wishmaster%, Deep Silent Complete
  \end{description}
  %
  In this thesis, we are interested in representing this data in four different formats:
  % 
  \begin{inparaenum}[(i)]
  \item a relational database,
  \item XML,
  \item RDF, and
  \item JSON.
  \end{inparaenum}
  % 
  The representation of this data in each specific syntax is accompanied by the description of each data model
  throughout this chapter.
  %
\end{example}



%%% Local Variables:
%%% mode: latex
%%% mode: flyspell
%%% mode: reftex
%%% TeX-master: "../thesis"
%%% End:


\bigskip


In the next sections we describe the relational model, the tree-based \ac{XML} and \ac{JSON} models, and the graph-based
\ac{RDF} data model.  We conclude with an high-level comparison of the presented data models in
\cref{sec:overview-data-models}.
%
In \cref{cha:query-languages}, we will describe the respective query languages associated with each data model.




\section{Relational Model}
\label{sec:relational-model}

Due to the ever-growing need to store information, database systems were one of the most researched software systems and
have evolved from the use of the filesystem to store the data into the currently ubiquitous relational database
management systems~\cite{AbiteboulHullVianu:1995aa}.
%
Initially, the simple use of filesystems to store data did not enforce any structure on the data, where each file could
have its own internal structure. 
% 
One major turning point in the evolution of database systems was the separation of the \emph{logical} definition of the
data from its \emph{physical} representation (known as the \emph{data independence principle}).  Thus, the task of
managing the physical representation is left up to the \emph{database management system} and is usually hidden from the
database user.
%
This separation also led to the development of several logical data models that allowed data to be described
independently of their physical representation.
% 
The logical data models can be composed primarily of a Data Definition Language (DDL) and a Data Manipulation Language
(DML).  The DDL specifies the structure used to represent data while the DML specifies methods to access and update
data.
%
The hierarchical and network data models were the first logical models to be introduced, where the former used a tree
structure for representing its data and the latter a graph structure.
%
However, according to~\citet{AbiteboulHullVianu:1995aa}, major issues with these logical models were:
\begin{inparaenum}[(i)]
\item they were still closely related to the physical representation model; and
\item their DML were limited, focusing mostly on navigating the physical representation.
\end{inparaenum}

The introduction of the relational model by~\citet{Codd:1970aa}, with its strong theoretical foundations, propelled
database management systems forward, allowing for advances in efficient query translation methods (from the relational
logical model into the physical representation model) and query optimisation techniques.
%
In the relational model, data is represented primarily using named \emph{relations}, where each \emph{relational tuple}
(or record) consists of several typed and named \emph{attributes}.  A commonly used alternative representation for
relational data depicts each relation as a table, where the attributes are the columns of this table, and each
relational tuple is represented as a row in the table.
%
Next we present a definition of the relational model, based on~\citet{AbiteboulHullVianu:1995aa}, that relies on the
pairwise disjoint and countably infinite sets \AR for relation names, \AAs for attribute names and \AD for the domain of
values that the attributes can hold.
%
An element~$d \in \AD$ is called a \emph{constant} and for an attribute~$a \in \AAs$ we represent the domain of~$a$
as~$\dom{a}$.
%
Furthermore, a total order is assumed between the elements of \AAs: this is a necessary feature to later allow us to
specify relational instances in a similar fashion to logic programming~\cite{Lloyd:1987aa}.
%
\begin{definition}[Relation and database schema]
  \label{def:database-schema}
  A \emph{relation schema} is represented as~$r[U]$, where~$r \in \AR$ is a relation name and~$U \subset \AAs$ is a set
  of attribute names, called the \emph{sort} of~$r$ and denoted by~$\funcCall{sort}{r}$.  The \emph{arity} of~$r$
  consists of its number of attributes:~$|\funcCall{sort}{r}|$.
  % 
  In turn, a \emph{database schema}~$S$ is a non-empty and finite set of relation schemas.
\end{definition}



\begin{example}[Relational Schema]
  %
  \label{ex:bands-schema}
  %
  A possible schema for a relational database that stores information relevant to our \usecase
  is~$S = \set{\relationName{person}, \relationName{band}, \relationName{album}, \relationName{song}}$, where
  %
  \begin{align*}
    \funcCall{sort}{\relationName{person}} &= \set{\mathit{personId}, \mathit{personName}, \mathit{bandId}}\\
    \funcCall{sort}{\relationName{band}} &= \set{\mathit{bandId}, \mathit{bandName}}\\
    \funcCall{sort}{\relationName{album}} &= \set{\mathit{albumId}, \mathit{albumName}, \mathit{bandId}}\\
    \funcCall{sort}{\relationName{song}} &= \set{\mathit{songId}, \mathit{songName}, \mathit{albumId}}%\\
  \end{align*}%
  %
\end{example}



Other features of the relational model include primary and foreign keys.
%
Intuitively, a primary key consists of a set of attributes that uniquely identify the tuples of a relation.  For
example, in our database schema we assume an artificially generated number that uniquely identifies each person or band
($\mathit{personId}$ and~$\mathit{bandId}$, respectively).
%
Foreign keys are used to specify dependencies between attributes of two different relations: the connected attributes
must have the same value in both relations.  This can be seen in \cref{ex:bands-schema}, where the same attribute
names are used in different relations to specify the foreign keys, \eg~$\mathit{bandId}$ in the
relations~\relationName{person} and~\relationName{band}.

%
Furthermore, the \NULL value is assumed to belong to all domains and, unless otherwise specified by means of
constraints, can be used in place of any valid value for an attribute of a relation. The intended meaning of \NULL
values is to represent missing or unknown information.
%
However, since \NULL values greatly complicate the definition of the algebra operations (presented in
\cref{sec:sql-semantics}), we will, for the most part, ignore \NULL values in the presented definitions.



\subsection*{Database Instances}
\label{sec:database-instances}


\citet{AbiteboulHullVianu:1995aa} present different perspectives for representing relational tuples \ie~\emph{instances}
of relational schemas, the \emph{conventional} and \emph{logic programming} perspectives.
%
The so-called \emph{conventional} perspective on relational databases (used later in \cref{sec:sql-semantics})
represents tuples as functions, where a tuple~$t$ over a finite set of attributes~$U$ consists of a function~$u$ with
domain~$U$.
%
The sort of~$u$ is~$U$ and the value of~$u$ of an attribute~$a \in U$ is denoted~\funcCall{u}{a}.
%
Extending this notion to a set of attributes~$V \subseteq U$, we say that~$u[V] = u|_{V}$ denotes the restriction of the
function~$u$ to~$V$, \ie~$u[V]$ denotes a new tuple~$v$ over~$V$ such that~$\funcCall{v}{a} = \funcCall{u}{a}$ for each
attribute~$a \in V$.


%
An alternate view focuses on the \emph{logic programming} perspective, under which a relational tuple can be viewed as a
\emph{fact}.  For a relation name~$r$ with arity~$n$, a fact is an expression~$\fact{r}{a_{1}, \dots, a_{n}}$, where
each~$a_{i} \in \AD$ is a constant.  Facts can also be represented as~$\fact{r}{u}$, where~$u = \tuple{a_{1}, \dots,
  a_{n}}$.
%
According to this representation, a \emph{relation instance} over a relation schema~$r$ is a finite set of facts
over~$r$ and a \emph{database instance} over a database schema~$S$ is the union of all relation instances over~$r$, for
each relation schema~$r \in S$.
%
Since relations are represented as sets, the standard set operations of intersection, union and difference ($\cap,
\cup$, and~$-$, respectively) can be applied and relations can be compared using the~$\subset, \subseteq, =$, and~$\neq$
operators.
%
\footnote{We note that although the relational model is formally described using a set based semantics, it is common for
  database systems to use multi-sets for representing the data and the results of \acs{SQL} queries.}
%
\cref{ex:relational-data} represents a database instance following the logic programming perspective.
%
\begin{example}[Database Instance]
  %
  \label{ex:relational-data}
  % 
  The database instance containing the \usecase data from \cref{ex:usecase-data}, over the database schema
  presented in \cref{ex:bands-schema}, is as follows:
  %
  \pagebreak[2]
  \begin{align*}
    \{\ 
        & \fact{person}{1, \atom{Marco Hietala}, 1}, \fact{person}{2, \atom{Tarja Turunen}, 1}, \\
        & \fact{band}{1, \atom{Nightwish}}, \\
        & \fact{album}{1, \atom{Wishmaster}, 1}, \\
        & \fact{song}{1,\atom{FantasMic}, 1}, \fact{song}{2, \atom{Wishmaster}, 1} 
        \}
  \end{align*}
\end{example}
%
Both views on relational data are equivalent and are used for different formalisations of the relational model and query
languages (as presented in \cref{sec:querying-rdb}).









%%% Local Variables:
%%% mode: latex
%%% mode: flyspell
%%% mode: reftex
%%% TeX-master: "../thesis"
%%% End:


\section{Extensible Markup Language (XML)}
\label{sec:xml}


As we have highlighted in \cref{cha:introduction}, with the growing success of the \ac{WWW}, where data exposed as
\ac{HTML} is often extracted from relational databases, the need to query Web Data in a structured way and thus consider
the Web as a \emph{global database} increased~\cite{SilberschatzKorthSudarshan:2005aa,AbiteboulBunemanSuciu:1999aa}.
%
Also powered by several data integration projects, research began to focus on the representation and querying of
\sd data following a \emph{graph} or \emph{tree} structure.
%
\Sd data models were devised as the required formats for representing data available on the Web and as a
representation-independent way to transfer data between different database management
systems~\cite{Abiteboul:1997aa,Buneman:1997aa}.

The \acf{XML}~\cite{BrayPaoliSperberg-Mcqueen:2008aa} is a \sd representation format and, with the support of the
\ac{W3C}, it has become the \emph{de facto} standard for data exchange on the
Web~\cite{Suciu:1998aa,AbiteboulBunemanSuciu:1999aa}.  XML is a subset of the Standard Generalized Markup Language
(SGML) ISO standard~\cite{ISO:1986:SGML} and is designed to be compatible with SGML and \ac{HTML}.
%
\ac{XML} represents data in a tree-like format that, when compared to the relational format, is a more flexible data
representation format and is also considered easier to read and write for both humans and machines.


\ac{XML} has also brought forward a new class of databases: \emph{XML databases}.  Although currently most databases
provide easy creation of \ac{XML} data, for example by exporting the data they contain as an \ac{XML} document, \ac{XML}
databases refer to a database management system that manage collections of \ac{XML}
data~\cite{KatzChamberlinKay:2003aa}.  Even though the data may be internally represented in another format, access and
manipulation is based on \ac{XML} formats and languages.


\cref{fig:bands-xml} contains the representation of the \usecase data from \cref{ex:usecase-data} in \ac{XML}.  This
document starts by representing a user and the top bands they listen to, where each band includes information regarding
its members and albums, and for each album, the songs listened to by the user.
%
As per~\citet{BrayPaoliSperberg-Mcqueen:2008aa}, the \acl{XML} describes what are called \emph{\ac{XML} documents},
which are composed primarily of \ac{XML} \emph{elements}.
%
In turn, \ac{XML} elements consist of a \emph{start-tag}, the \emph{element content}, and an \emph{end-tag}. 
%
\begin{data}[t]
  \centering
  \lstinputlisting[language=XML,basicstyle=\small\ttfamily]{0-data+queries/usecaseData.xml}
  \caption{Bands in XML (\fname{bands.xml})}
\label{fig:bands-xml}
\end{data}
%
Consider the following \ac{XML} element: 
%
\begin{lstlisting}[language=XML,basicstyle=\small\ttfamily,frame=none,numbers=none]
<song>Wishmaster</song>
\end{lstlisting}
%
Start- and end-tags are indicated by \stringValue{<song>} and \stringValue{</song>}, respectively, where
\stringValue{song} is called the \emph{element name}, and the element content may consist of text (any string of
characters), other (nested) \ac{XML} elements, CDATA sections, processing instructions or comments.
%
CDATA sections can be used to include text that contains markup characters (such as the start- and end-delimiters) and
processing instructions contain data that is to be sent to the application consuming the document.
%
Comments can be present anywhere in the document and, similar to any programming language, can be ignored by the
\ac{XML} processor.
%
Furthermore, \ac{XML} elements may contain \emph{attributes} enclosed in their start-tags, in this case the
\stringValue{album} element has the \stringValue{name} attribute with value \stringValue{Wishmaster}:
%
\begin{lstlisting}[language=XML,basicstyle=\small\ttfamily,frame=none,numbers=none]
<album name="Wishmaster">
\end{lstlisting}
%
In \ac{XML} text, elements, CDATA, processing instructions, comments, and attributes are collectively referred to as
\emph{\ac{XML} nodes}.

\subsection{XML Namespaces}
\label{sec:xml-namespaces}

\ac{XML} provides a way to disambiguate entities such as element and attribute names by using \ac{XML}
namespaces~\cite{BrayHollanderLayman:2009aa}, where each \ac{XML} namespace is identified by a URI
reference~\cite{Berners-LeeFieldingMasinter:2005aa}.
%
\ac{XML} allows, by means of reserved attributes, to associate partial \acp{URI} with a \emph{prefix name} and/or to
declare a \emph{default namespace}.
%
\emph{Qualified names} (or QNames) provide a convenient form of naming element and attribute names in \ac{XML} and can
be composed of prefixed or unprefixed names.  Prefixed names make use of the previously declared prefixes and are
combined with the \emph{local part} to specify the \ac{URI} reference.
%
For unprefixed names, if a default namespace declaration exists it is taken as the namespace value, otherwise there will
be no namespace value.
%
For example, including the ``\verb+xmlns+'' attribute in an \ac{XML} element declares the default namespace to be used
within that element:
%
\begin{lstlisting}[language=XML,basicstyle=\small\ttfamily,frame=none,numbers=none]
<user xmlns="http://example.org/bands/">
\end{lstlisting}
%
while \acp{URI} can be associated with a prefix in the following manner:
%
\begin{lstlisting}[language=XML,basicstyle=\small\ttfamily,frame=none,numbers=none]
<members xmlns:foaf="http://xmlns.com/foaf/0.1/">
\end{lstlisting} 
%
\ac{XML} namespaces are scoped to the element in which they are declared, including any child elements.


\subsection{XML Validation}
\label{sec:xml-validation}


The \ac{XML} \ac{W3C} specification~\cite{BrayPaoliSperberg-Mcqueen:2008aa} defines two levels of conformance for
\ac{XML} documents: \emph{well-formed} documents and \emph{valid} documents.  \emph{Well-formedness} constraints
primarily ensure that the \ac{XML} document follows syntactic specifications, such as (to name but a few):
\begin{inparaenum}[(i)]
\item they must contain at least one element;
\item a distinct element, called the \emph{root}, is not included in the content of any other element;
\item for all non-root elements, its start- and end-tags must be included within the content of the same element,
  \ie~opening and closing tags must not overlap; and
\item attribute names must be unique within the same element.
\end{inparaenum}

On the other hand, \emph{valid} documents rely on a \emph{schema} that, similar to relational databases, specifies the
structure of a particular class of \ac{XML} documents. Such schemas can be specified using two different formats:
\ac{DTD} or \ac{XML} Schema, both of which are detailed below.\footnote{There are other schema languages for \ac{XML},
  such as the Relax NG language, but for the scope of this thesis we will focus on \ac{W3C} specifications.}
%
In \cref{cha:xsparql} we will define \ac{XML} Schema datatypes for representing \ac{RDF} concepts and thus
incorporating them into XQuery.

\subsubsection{Document Type Definition}
\label{sec:dtd}


\begin{figure}[t]
  \centering
  \lstinputlisting[language=XML,basicstyle=\small\ttfamily]{0-data+queries/usecaseData.dtd}
  \caption{\ac{DTD} definition for the bands XML data}
\label{fig:bands-xml-dtd}
\end{figure}

\ac{DTD} specifications are mostly referenced here for historical reasons, since \ac{XML} Schema is more widely used (as
detailed in the next section).
%
\ac{DTD} specifications consist of \emph{markup declarations}, such as \emph{element type}, \emph{attribute list},
\emph{entity} or \emph{notation} declarations.  
% 
Element type declarations are defined using the \character{ELEMENT} keyword, for instance:
%
\begin{lstlisting}[language=XML,basicstyle=\normalfont\ttfamily,frame=none,numbers=none]
<!ELEMENT album (song*)>
\end{lstlisting}
%
specifies an \stringValue{album} element that is constituted by any number of \stringValue{song} elements. The
\stringValue{album} element is required to have an attribute named \stringValue{name} by the following attribute list
declaration:
%
\begin{lstlisting}[language=XML,basicstyle=\normalfont\ttfamily,frame=none,numbers=none]
<!ATTLIST album name CDATA #REQUIRED>
\end{lstlisting}
%
The complete \ac{DTD} definition for the \usecase \ac{XML} structure is presented in \cref{fig:bands-xml-dtd}.  An
attribute declared as \texttt{CDATA} indicates that its value must be a sequence of characters and/or \ac{XML} markup.
%
On the other hand, \texttt{PCDATA} (meaning ``parsed character data'') indicates that only one text element, and no
other nodes are allowed in the content.
%
Adding this \ac{DTD} definition to the \ac{XML} document from \cref{fig:bands-xml} would ensure that any
validating \emph{\ac{XML} processor} checks the structure of the \ac{XML} data against the provided schema definition.
%


\subsubsection{XML Schema}
\label{sec:xml-schema}

While \acp{DTD} are included in the \ac{W3C} \ac{XML} specification and therefore are widely available, there are some
drawbacks to their use, most noticeably the lack of namespace support.
%
To overcome such drawbacks, the \ac{W3C} has defined the \ac{XML} Schema specification, composed of two parts:
%
\begin{inparaenum}[(i)]
\item an \ac{XML}-based syntax for validating \ac{XML} documents~\cite{ThompsonBeechMaloney:2004aa}; and
\item a specification of \ac{XML} datatypes~\cite{BironMalhotra:2004aa}.
\end{inparaenum}


\begin{figure}[p]
  \centering
  \lstinputlisting[language=XML]{0-data+queries/usecaseData.xsd}
  \caption{XML Schema definition for Bands XML data (partial)}
\label{fig:bands-xml-xsd}
\end{figure}
%
The \ac{XML} Schema definition of the \usecase \ac{XML} data is presented in \cref{fig:bands-xml-xsd}, which has the
same effect as the \ac{DTD} presented in \cref{fig:bands-xml-dtd}: validating the \ac{XML} document from
\cref{fig:bands-xml}.
%
In \ac{XML} Schema, \ac{XML} elements and attributes are declared using an \ac{XML} element named \stringValue{element}
and \stringValue{attribute}, respectively, contained in the \stringValue{\url{http://www.w3.org/2001/XMLSchema}}
namespace.
%
For example, the \stringValue{album}, along with its \stringValue{name} attribute and \stringValue{song} elements, are
defined in lines 41--48 of \cref{fig:bands-xml-xsd}.


The specification of datatypes in \ac{XML}~\cite{ThompsonBeechMaloney:2004aa} introduces a datatype system that is also
used by other \ac{W3C} specifications, such as the \ac{RDF} specification~\cite{ManolaMiller:2004aa}.
%
A datatype is defined by
%
\begin{inparaenum}[(a)]
\item the \emph{value space}: a set of values for a datatype; 
\item the \emph{lexical space}: a set of valid character strings for the datatype; and
\item a \emph{lexical-to-value mapping} linking elements of these two sets.
\end{inparaenum}
%
A datatype is identified by a \ac{URI} and a \emph{datatype map} associates the \ac{URI} with the specific datatype.
%
The defined datatype system allows for the creation of user-defined datatypes, where such datatypes are \emph{derived}
from existing datatypes (called the \emph{base type}) by restricting or extending its value space and lexical space.


The formalisation of \ac{XML} Schema was proposed by~\citet{SimeonWadler:2003aa}, where the authors also describe a more
human readable notation for both \ac{XML} elements and \ac{XML} schema types.  This notation was later adopted by the
XQuery semantics specification~\cite{DraperFankhauserFernandez:2010aa}.
%
Following this notation, the \ac{XML} element
%
\lstinline[basicstyle=\small\ttfamily]{<song>Wishmaster</song>}
%
is represented as
%
\lstinline[basicstyle=\small\ttfamily]+element song { "Wishmaster" }+.
%
The \stringValue{song} and \stringValue{album} elements from the~\ac{XML} Schema in \cref{fig:bands-xml-xsd} can
be represented in the shorthand notation as:
%
\begin{lstlisting}[language=XML,basicstyle=\small\ttfamily,frame=none,numbers=none]
define element song of type xs:string
define element album of type albumType
define type albumType {
  element song*,
  attribute name of type xs:string  }
\end{lstlisting}
%
After performing validation with the presented \ac{XML} Schema, the \ac{XML} element is represented as:
\lstinline[basicstyle=\small\ttfamily]+element song of type xs:string { "Wishmaster" }+.
%
In \cref{cha:xsparql} we specify the types introduced by the XSPARQL language following this notation.


\subsection{XML Abstract Representations}
\label{sec:xml-data-models}

The \ac{W3C} specifications are defined over abstract representations of \ac{XML} documents, with the objective of
omitting the concrete syntax of \ac{XML} documents, namely the \ac{Infoset} and the \ac{XDM}.
%
The \ac{Infoset} provides the basic definitions for describing well-formed \ac{XML} documents, with the purpose of
serving as a reference for other \ac{XML} specifications.
%
On the other hand, the more complex \ac{XDM} is meant to act as a data model for the XPath, XSLT and XQuery languages:
describing their input documents and the values for expressions.  These query languages will be the focus of
\cref{sec:prelim-xquery}.
%
The \ac{Infoset} describes only the basic information contained in an \ac{XML} document, while the XQuery and XPath Data
Model is used mostly for the \ac{XML} Query languages (described in \cref{sec:prelim-xquery}).

\subsubsection{XML Infoset}
\label{sec:xml-infoset}

The \acl{Infoset}, described by~\citet{CowanTobin:2004aa}, provides definitions referring to a well-formed \ac{XML} document
and as such, any well-formed \ac{XML} document, although not necessarily a valid document, will have an \ac{Infoset}.
%
The \ac{Infoset} consists of a set of \emph{information items}, where each information item describes a part of the \ac{XML}
document by means of \emph{properties} that refer to other information items.
%

An \ac{Infoset} contains exactly one document information item that, directly or indirectly, refers to all of the other
information items in the set.  Other information items are used to represent \ac{XML} nodes such as elements,
attributes, processing instructions, or comments.


\subsubsection{XQuery 1.0 and XPath 2.0 Data Model (XDM)}
\label{sec:xpath-data-model}
%
The XPath, XSLT, and XQuery languages use the \emph{\acf{XDM}}~\cite{FernandezMalhotraMarsh:2010aa} for describing their
input \ac{XML} documents.  \ac{XDM} is based on the \ac{XML} \ac{Infoset} and extends it with support for:
%
\begin{inparaenum}[(i)]
\item \ac{XML} Schema types~\cite{ThompsonBeechMaloney:2004aa,BironMalhotra:2004aa};
\item typed atomic values; 
\item collections of documents and complex values; and
\item ordered, heterogeneous sequences.
\end{inparaenum}

The basic element of the data model is called an \emph{item}, which has a type (either an \ac{XML} \emph{node} or an
atomic type) and an associated value.
%
Types in the data model are uniquely represented using (expanded) QNames %, in this case after expansion,
and the pre-defined atomic types include the built-in types presented in~\citet{BironMalhotra:2004aa}, extended with the
following additional types:
%
\begin{inparaenum}[(i)]
\item \qname{xs}{anyAtomicType};
\item \qname{xs}{untyped};
\item \qname{xs}{untypedAtomic};
\item \qname{xs}{dayTimeDuration}; and
\item \qname{xs}{yearMonthDuration}.
\end{inparaenum}


The data model defines seven types of \ac{XML} nodes, namely: document, element, text, attribute, namespace, processing
instruction, and comment.  Each item that represents an \ac{XML} node is associated with the corresponding type and for
every type of node, it is possible to compute a string value.
%
In the data model, each \ac{XML} node has a unique identity, which is equal only to itself.  On the other hand, all
instances of the same atomic values are considered equal, \ie~they do not have a unique identity.
%
XDM defines a total order among all available nodes, called \emph{document order}, which consists of the order that
nodes appear in the original document.

In comparison to the \ac{Infoset}, another addition of the \ac{XDM} is the support for sequences.  A \emph{sequence}
consists of any number of items (\ac{XML} nodes and/or atomic values) and are represented as a
comma-separated~\character{,} list of items, delimited by the~\character{(} and~\character{)} characters.
%
Each item is considered equal to a singleton sequence containing that item and furthermore sequences are not allowed to
include any other sequences and are thus always considered as a flattened sequence.
%
For example, the sequence \inlineExpr{\lstinline{(1,(<a/>), "3")}} is translated to \inlineExpr{\lstinline{(1,<a/>, "3")}}.


%%% Local Variables:
%%% mode: latex
%%% mode: flyspell
%%% mode: reftex
%%% TeX-master: "../thesis"
%%% End:



\section{JavaScript Object Notation (JSON)}
\label{sec:json}


\ac{JSON} is defined as a ``lightweight data-interchange format'' is another tree-based model designed as an alternative
to \ac{XML} for data transmission between different applications.
%
Although it originates from the JavaScript language, its format is language independent and thus can be used by several
programming languages.

The main structure of \ac{JSON} is called an \emph{object}, enclosed between~\character{\textbraceleft}
and~\character{\textbraceright}, and consists of an unordered sequence of \emph{name-value} pairs, separated
by~\character{:}. 
%
In such structures, the \emph{name} is restricted to be a string while \emph{value} may be one of
\begin{inparaenum}[(a)]
\item string;
\item number;
\item object;
\item array;
\item boolean (true or false); or
\item null.
\end{inparaenum}
%
\emph{Arrays} consist of an ordered list of values and are enclosed between~\character{[} and~\character{]}.
%
The \ac{JSON} representation of our \usecase data is presented in \cref{fig:bands-json}.
%
\begin{data}[t]
  \centering
  \lstinputlisting[basicstyle=\small\ttfamily]{0-data+queries/usecaseData.json}
  \caption{Bands in JSON (\fname{bands.json})}
\label{fig:bands-json}
\end{data}
%
This simple and unambiguous structure, coupled with the fact that \ac{JSON} is natively recognised and imported by
JavaScript, made \ac{JSON} extremely popular on the Web.  A comparative study of the uptake of \ac{XML} and JSON was
presented by~\citet{Musser:2011aa}.\footnote{The presentation is available
  at~\url{http://www.slideshare.net/jmusser/j-musser-semtechjun2011/}, retrieved on 2012/04/10.}


Although \ac{JSON} and \ac{XML} serve very similar purposes, commonly presented advantages for using \ac{JSON} over
\ac{XML} are:
%
\begin{inparaenum}[(i)]
\item \ac{JSON} documents are (usually) smaller; and
\item an external schema is not required to unambiguously represent the content.
\end{inparaenum}
%
On the other hand, one of the biggest disadvantages of \ac{JSON} is its lack of support for namespaces: whereas in
\ac{XML} it is possible to distinguish attribute and element names by giving them different namespace prefixes, this is
not possible in \ac{JSON}.


Due to the similarities between the \ac{JSON} and \ac{XML} formats, the question of translating between them has arisen.
Since \ac{XML} is arguably the more expressive language, being more flexible in its format, converting from \ac{XML} to
\ac{JSON} poses some problems:
%
\begin{description}
\item[Namespaces.]  Since \ac{JSON} does not natively support namespaces, a non-trivial issue is how to represent
  \ac{XML} namespaces in such a fashion that can be translated back into \ac{XML};
\item[Attributes.]  Similar to namespaces, \ac{JSON} has no equivalent for \ac{XML} element attributes and similar
  representational questions arise for \ac{XML} attributes;
\item[Mixed Content.]  Since the contents of \ac{XML} elements can consist of text values arbitrarily mixed with other
  elements, an accurate representation of this mixed content in \ac{JSON}, although possible, would yield a very verbose
  representation.
\end{description}
%
On the other hand, converting from \ac{JSON} to \ac{XML} is a straightforward task, relying solely on using predefined
element names to represent \ac{JSON} objects and arrays.  This straightforward conversion will be used in
\cref{cha:xsparql} for the inclusion of \ac{JSON} data into our proposed transformation language.



%%% Local Variables:
%%% mode: latex
%%% mode: flyspell
%%% mode: reftex
%%% TeX-master: "../thesis"
%%% End:




\section{Resource Description Framework (RDF)}
\label{sec:rdf}

In the attempts to transform the Web into a global database, another model, the~\acl{RDF}, was proposed as the data model
for representing machine readable data, also known as \emph{Semantic Web} data.  The \ac{RDF} model allows for the
specification of \emph{statements} about \emph{Web Resources}~\cite{ManolaMiller:2004aa}.
%
However, this general notion of resource may refer not only to virtual entities (that can only be found on the Web) but
also any physical entity that can be \emph{identified} on the Web.
%
Such resources are identified by a \ac{URI}, generally indicating where the resource is located, or a \emph{blank node},
which plays the role of an anonymous resource and allows for the modelling of incomplete or unknown data.  In the
following, we identify blank nodes by using the prefix \character{\_:} followed by a string, called the \emph{blank node
  label}.  Blank nodes are scoped to the document they appear in, \ie~two blank nodes from different documents, even if
they have the same label, must be considered different.
%
Furthermore, \ac{RDF} \emph{literals} can be used to specify string- or datatype-based values for properties.
%
The atomic \emph{statements} of the \ac{RDF} data model are called \emph{RDF triples} consisting of \emph{subject},
\emph{predicate} and \emph{object}, and intuitively state that
%
the subject is connected to the object by the predicate relation.
%
Since triples can also be viewed as part of a labelled directed graph, where subjects and objects correspond to nodes
and predicates to edges of the graph, we refer to a set of such \ac{RDF} triples as an \emph{RDF graph}.
%


For the definitions of the \ac{RDF} syntax, we rely on the the pairwise disjoint alphabets~$\AU$,~$\AB$, and~$\AL$
denoting \emph{URI references}, \emph{blank nodes} and \emph{literals}, respectively.\footnote{We assume~$\AU, \AB$,
  and~$\AL$ fixed, and for presentation purposes we will denote unions of these sets by concatenating their names.}  We
call the elements in~$\AUBL$ \emph{terms}.
%
\begin{definition}[RDF Triple]
  \label{def:rdf-triple}
  An \emph{RDF triple} is~$\tau = \triplee{s}{p}{o} \in \AUBL \times \AU \times \AUBL$, where~$s$ is called the
  \emph{subject},~$p$ the \emph{predicate}, and~$o$ the \emph{object}.
\end{definition}
%
\nd Strictly speaking, according to the \ac{RDF} specification~\cite{Hayes:2004aa} literals are not allowed to be the
subject of \ac{RDF} triples however, as commonly adopted in other
works~\cite{MunozPerezGutierrez:2007aa,PrudhommeauxSeaborne:2008aa,CarrollBizerHayes:2005aa}, this definition considers
a \emph{generalised RDF Triple}, that allows literals for the subject positions.


\begin{definition}[RDF Graph]
  \label{def:rdf-graph}
  Following the definition of an \ac{RDF} triple, an \emph{RDF graph}~$G$ consists of a set of triples.
  % 
  The \universe of~$G$,~$\fc{\universe}{G}$, is the set of elements in~$\AUBL$ that occur in the triples of~$G$ and the
  \emph{vocabulary} of~$G$,~$\fc{\voc}{G}$, is~$\fc{\universe}{G} \cap \AUL$.
  % 
  Furthermore, we say that~$G$ is \emph{ground} \iff~$\fc{\universe}{G} = \fc{\voc}{G}$, \ie~$G$ does not contain blank nodes.
\end{definition}

When combining different \ac{RDF} graphs some care must be taken to ensure the local scope of blank nodes:
%
\begin{definition}[RDF merge]
  \label{def:rdf-merge}
  Let~$S$ be a set of RDF graphs.  The \emph{RDF merge} of~$S$ consists of the set-theoretical union of all the graphs
  in~$S$ after blank nodes have been \emph{standardised apart}: if any two graphs contain the same blank node label, all
  occurrences of these labels within the same graph are replaced by a new blank node label that is not present in any of
  the other graphs.
\end{definition}
%
\noindent This disambiguation of blank node labels is meant to keep any blank nodes between different graphs distinct, thus
maintaining the scope of blank nodes to the graph they occur in.
%

Similar to \ac{XML} namespaces, \acp{URI} can be abbreviated by using a namespace prefix.
%
For example, the URI \stringValue{\qname{foaf}{Person}} from the widely used \ac{FOAF} ontology, consists of the prefix
\stringValue{\term{foaf}}, which is associated with the URI \stringValue{\url{http://xmlns.com/foaf/0.1/}}, and the
local part \stringValue{\term{Person}}.  The complete URI represented by \stringValue{\qname{foaf}{Person}} is thus
\stringValue{\url{http://xmlns.com/foaf/0.1/Person}}.
%
\qname{rdf}{type} predicates can be used to specify that an \ac{RDF} resource is an \emph{instance} of a \emph{class};
for example the triple:
%
\begin{equation}
\label{eq:rdf-triple}
\triplee{\qname{dbpedia}{Marco\mathunderscore{}Hietala}}{\qname{rdf}{type}}{\qname{foaf}{Person}}
\end{equation}
% 
intuitively specifies that the resource \qname{dbpedia}{Marco\mathunderscore{}Hietala} is used to identify a person.

\ac{RDF} literals can be further classified as \emph{plain}, in which case they can optionally contain a \emph{language
  tag}, or \emph{typed} literals.
%
Typed literals include a \ac{URI} that refers to their datatype, usually one of the \ac{XML} Schema built-in datatypes
or the newly defined \ac{RDF} datatype \qname{rdf}{XMLLiteral} (used to indicate the literal contains \ac{XML} data).
%
The specific syntax of literals is presented in the next section.


Another \ac{RDF} feature, although not so commonly used, is \emph{reification}, can be used to represent
meta-information about an \ac{RDF} triple, \eg~provenance information.  Any \ac{RDF} statement can be reified by
representing it as four distinct \ac{RDF} triples with a common subject.  Although it is possible to use a \ac{URI} for
the subject of reified triples, as presented in \cref{ex:reified-triple}, it is common to use a blank node.
%
Reification is later used in \cref{cha:anql} as one possible serialisation for Annotated RDFS graphs.
%
\begin{example}[Reified \ac{RDF} statement]
  %
  \label{ex:reified-triple}
  The \ac{RDF} statement~\eqref{eq:rdf-triple} can be reified as the following triples:
  %
  \vspace{-\abovedisplayskip}
  %
  \begin{align*}
    &\triplee{\bnode{r1}}{\qname{rdf}{type}}{\qname{rdf}{Statement}}\\
    &\triplee{\bnode{r1}}{\qname{rdf}{subject}}{\qname{dbpedia}{Marco\mathunderscore{}Hietala}}\\
    &\triplee{\bnode{r1}}{\qname{rdf}{predicate}}{\qname{rdf}{type}}\\
    &\triplee{\bnode{r1}}{\qname{rdf}{object}}{\qname{foaf}{Person}}
  \end{align*}
  % 
\end{example}


Yet another feature of \ac{RDF} are \emph{collections}, which allow to state that a group of resources are members of
the collection.  This is represented in \ac{RDF} using a list structure following a predefined vocabulary:
\qname{rdf}{List} states the type of the resource and the \qname{rdf}{first} and \qname{rdf}{rest} properties are used
to represent the list.  This list must be terminated by \qname{rdf}{nil}.  Collections are used in
\cref{sec:bf-extended-bgp}.

\medskip

Next, \cref{sec:representation-syntax} presents how \ac{RDF} can be serialised in order to be stored or
exchanged, focusing on the RDF/XML and Turtle syntaxes and \cref{sec:rdf-semantics} presents the semantics of
\ac{RDF}.  Finally, \cref{sec:rdf-schema} focuses on the inferencing capabilities of \ac{RDF} by describing
\acl{RDFS}.

\subsection{Representation Syntaxes}
\label{sec:representation-syntax}

Although the \ac{RDF} specification states that the normative syntax for writing \ac{RDF} graphs is
RDF/XML~\cite{BeckettMcBride:2004aa}, this syntax is not favoured among practitioners and there have been proposals to
support other serialisation formats and move away from \ac{XML} based representations~\cite{Beckett:2010aa}.
% 
Other well known syntaxes for \ac{RDF} are Turtle~\cite{BeckettBerners-Lee:2008aa} and RDFa~\cite{AdidaBirbeck:2008aa},
where Turtle consists of a specialised syntax for RDF and RDFa defines a mechanism to incorporate \ac{RDF} statements into
(X)HTML webpages.
%
In the following, we briefly highlight the constructs of the RDF/XML and Turtle syntaxes.


\subsubsection{RDF/XML}
\label{sec:rdfxml}

Although RDF/XML is the normative syntax for \ac{RDF}, this serialisation is very flexible, and the same \ac{RDF} graph can
be serialised in numerous different ways.
%
As we will point out in \cref{cha:xsparql}, this lack of a canonical RDF/XML serialisation is one of the major
roadblocks to process \ac{RDF} data using \ac{XML} tools.
%


%
The RDF/XML syntax uses \ac{XML} elements to represent \ac{RDF} subjects, predicates and objects:
\stringValue{\qname{rdf}{Description}} elements are used to represent nodes (subjects and objects) of the \ac{RDF}
graph, where the \stringValue{\qname{rdf}{about}} attribute specifies the URI of the node.
%
In turn, predicates are represented as \ac{XML} elements where the name of the element corresponds to the \ac{URI}
(represented as an \ac{XML} QName) of the predicate.
%
A possible RDF/XML serialisation of the \ac{RDF} graph from our running example is presented in
\cref{fig:bands-rdfxml}.
%
\begin{data}[p]
  \centering
  \lstinputlisting[language=XML,basicstyle=\small\ttfamily]{0-data+queries/usecaseData.rdf}
  \caption{Bands in RDF/XML}
  \label{fig:bands-rdfxml}
\end{data}
%
\begin{data}[t]
  \centering
  \lstinputlisting{0-data+queries/usecaseData-abbrev.rdf}
  \caption{Bands in abbreviated RDF/XML}
\label{fig:bands-rdfxml-abbrev}
\end{data}


The RDF/XML serialisation allows the use of several abbreviations and 
%
an abbreviated serialisation of the \ac{RDF} graph in \cref{fig:bands-rdfxml} is presented in
\cref{fig:bands-rdfxml-abbrev}. 
%
One of the possible abbreviations is, if an object node does not contain any other predicates, to omit the
\stringValue{\qname{rdf}{Description}} element and specify the URI reference of the object in the
\stringValue{\qname{rdf}{resource}} attribute of the predicate element node (for example in lines~13 and~14). 
%
Another abbreviation is, in case the subject contains an \stringValue{\qname{rdf}{type}} predicate, to replace the
\stringValue{\qname{rdf}{Description}} element name with the type of the subject (for example lines~9, 12, and~16).
%
Also, several predicates about the same subject can be nested in the same \ac{XML} element (as presented in
lines~$19$--$24$ of \cref{fig:bands-rdfxml-abbrev}).
%


Blank nodes (anonymous resources) can be given a label using the \stringValue{\qname{rdf}{nodeID}} attribute and can
later be referred to (from within the same document).
%
Literals can be specified as the text content of a property \ac{XML} element (\eg~line~$17$ of
\cref{fig:bands-rdfxml-abbrev}) or alternatively as the value of an attribute where the attribute name is the
corresponding property \ac{URI} (as in line~$12$ of \cref{fig:bands-rdfxml-abbrev}).
%
Language tags are specified as the value of the \stringValue{\qname{xml}{lang}} attribute, while typed literals use the
\stringValue{\qname{xml}{datatype}} attribute.




\subsubsection{Turtle}
\label{sec:turtle}
%

Stemming from its \ac{XML} roots and, even with all the proposed abbreviations, the RDF/XML syntax is still very verbose
and neither easy to read nor write for humans.
%
To overcome these problems, the Turtle syntax~\cite{BeckettBerners-Lee:2008aa} aims to be a compact representation for
\ac{RDF} graphs that is easier to read and write for users and includes abbreviations for common \ac{RDF} patterns.
%
Turtle is based on N-Triples, a simple syntax introduced for the \ac{RDF} test cases~\cite{GrantBeckett:2004aa} that
represents one triple per line.  Furthermore, Turtle incorporates features from Notation 3~\cite{Berners-Lee:2005aa},
most notably:
%
\begin{inparaenum}[(i)]
\item namespace declarations,
\item shortcuts for commonly used \ac{RDF} patterns, and
\item a syntax for anonymous blank nodes.
\end{inparaenum}

%
\begin{data}[t]
  \centering
  \lstinputlisting{0-data+queries/usecaseData.ttl}
  \caption{Bands in Turtle (\fname{bands.ttl})}
\label{fig:bands-turtle}
\end{data}
%

The Turtle \ac{RDF} representation of the \usecase data from \cref{ex:usecase-data} is presented in
\cref{fig:bands-turtle}.  In the Turtle syntax, \term{\texttt{@}prefix} declarations can be used to abbreviate common
URIs (similar to \ac{XML} namespaces and QNames) and URIs must be enclosed between the \character{<} and \character{>}
characters.
%
Literals are surrounded by double-quotes, as in \literal{Nightwish}, and may include a suffix to specify the language
tag following the \character{@} separator character, for example \lang{Marco\ Hietala}{en}, or a datatype \ac{URI} after
the \character{\^{}\^{}} separator as in \dt{5}{\uril{http://www.w3.org/2001/XMLSchema\#integer}}.
%
Blank nodes are prefixed with~\character{\_:} \eg~\bnode{song506} where the blank node label is \term{song506}.  A
shortcut for unnamed blank nodes is provided by using the \character{[]} notation.  Another useful shortcut is the
\character{\term{a}} keyword (line~7 of \cref{fig:bands-turtle}), which represents the URI
\uril{http://www.w3.org/1999/02/22-rdf-syntax-ns\#type}, also commonly abbreviated as \qname{rdf}{type}.


Furthermore the \character{;} and \character{,} symbols can be used to create new triples without repeating the subject
or subject and predicate, respectively.  For example, the triples from lines 7--10 of \cref{fig:bands-turtle} can
be written as:
\begin{lstlisting}[frame=none,numbers=none]
dbpedia:Nightwish a mo:MusicGroup ;
                  foaf:name "Nightwish" ; 
                  foaf:member dbpedia:Marco_Hietala, dbpedia:Tarja_Turunen .
\end{lstlisting}
%
Commonly used datatypes can also be abbreviated: for instance \term{5} is equivalent to
\dt{5}{\uril{http://www.w3.org/2001/XMLSchema\#integer}}, while \term{5.0} corresponds to
\dt{5.0}{\uril{http://www.w3.org/2001/XMLSchema\#decimal}}.
%
Turtle also provides abbreviations for \ac{RDF} collections by listing a space-separated sequence of \ac{RDF} terms
enclosed by \lit{(} and \lit{)}.


\subsection{Semantics}
\label{sec:rdf-semantics}

The semantics of \ac{RDF} is specified using a model theory as per~\citet{Hayes:2004aa}, which is a common form of specifying
semantics, for example for first-order logic.
%
Model theoretic semantics of formal languages assign any expression in the language to an element of a possible
``world'' -- called an \emph{interpretation} -- and also specify the necessary conditions for an interpretation to be
considered valid -- called a \emph{model}.  The notion of \emph{entailment} between two expressions,~$A$ entails~$B$,
can then be defined as any interpretation that is a model of~$A$ must also be a model of~$B$.
%
Based on this semantics, it is possible to define what are the \emph{entailed} consequences of the interpretation and
what are valid \emph{inference rules}. 


In the case of \ac{RDF}, language expressions are considered as being the terms in the universe of the graph and also the
individual triples, \ie~each term in the vocabulary is assigned to an interpretation element, where plain literals are
interpreted as themselves and blank nodes are interpreted as existential variables (scoped to the \ac{RDF} graph in which
they occur).


The \ac{RDF} semantics~\cite{Hayes:2004aa} specifies different types of interpretation, and hence of entailment, ranging
from the so-called \emph{simple} interpretation to the more complex \emph{RDFS} and \emph{datatype} interpretations.
%
Simple interpretations consider only the vocabulary of the triples present in the graph while other types of
interpretations, namely \ac{RDF} and \ac{RDFS}-interpretations, consider predefined vocabularies and a set of \ac{RDF}
triples that any interpretation must satisfy by default: the so-called \emph{axiomatic} triples.


\ac{RDF} interpretations consider the terms defined in the \url{http://www.w3.org/1999/02/22-rdf-syntax-ns#} namespace
(commonly abbreviated with the prefix \term{rdf}).
%
For instance, \ac{RDF} interpretations impose conditions that identify a subset of the interpretation resources as being
\emph{properties} (interpretation resources of type \term{rdf{:}Property}) and introduce the new datatype
\term{rdf{:}XMLLiteral} to represent well-formed \ac{XML} literals. % (c.f. \cref{sec:xml}).



\ac{RDFS}-interpretations consider further vocabulary in the \term{rdfs} namespace
(\url{http://www.w3.org/2000/01/rdf-schema#}) that define further conditions on top of \ac{RDF} interpretations and
introduce the notion of a \emph{class}.  A class is itself a resource that denotes a common set of resources, which are
called \emph{instances} of the class and all have the class as the value for their \qname{rdf}{type} property.
%
Informally, the \ac{RDFS} vocabulary states the following:
%
\begin{inparaenum}[(i)]
\item $\triple{p, \term{rdfs{:}subPropertyOf}, q}$ means that any resources related by property~$p$ are also related by
  property~$q$;
\item $\triple{c, \term{rdfs{:}subClassOf}, d}$ means that any instance of class~$c$ is also an instance of class~$d$;
\item $\triple{a, \term{rdf{:}type}, c}$ means that~$a$ is an instance of~$c$;
\item $\triple{p, \term{rdfs{:}domain}, c}$ means that the \emph{domain} of property~$p$ is~$c$, \ie~any resource that
  is the subject of a triple with predicate~$p$ is an instance of~$c$; and
\item $\triple{p, \term{rdfs{:}range}, c}$ means that the \emph{range} of property~$p$ is~$c$, \ie~any resource that is
  an object of a triple with predicate~$p$ is an instance of~$c$.
\end{inparaenum}
%

Further extending \ac{RDFS}-interpretations, a D-interpretation provides an (admittedly minimal) support for \ac{XSD}
datatypes~\cite{BironMalhotra:2004aa} extended with the \term{rdf{:}XMLLiteral} datatype.
%
Another \ac{W3C} specification that provides a more expressive inference system than \ac{RDFS} is the \ac{OWL}, now in
its second version~\cite{HitzlerKrotzschParsia:2009aa}.  The \ac{OWL} language introduces new concepts such as the
distinction between object and datatype properties, class disjointness assertions, and assertions of equality between
individuals, among others.
%
It is noteworthy that D-interpretations, \ac{OWL}, and the \term{rdf{:}XMLLiteral} datatype may introduce
inconsistencies in \ac{RDF}.
%
However, in this thesis, we are mostly interested in \ac{RDFS} inferences and we do not consider D-interpretations,
\ac{OWL} constructs, nor the typing of \term{rdf{:}XMLLiteral} and thus we avoid any inconsistencies in \ac{RDF}.  
%


Although in the \ac{RDF} Semantics specification~\cite{Hayes:2004aa} the semantic conditions for each interpretation are
detailed separately, in this thesis we follow the formalism defined
by~\citet{MunozPerezGutierrez:2007aa,MunozPerezGutierrez:2009aa} and provide a single notion of interpretation that
covers Simple, \ac{RDF}, and \ac{RDFS}-interpretations. 
%
Intuitively, the interpretation of an \ac{RDF} triple~$\triple{s,p,o}$ is true if~$s$,~$p$ and~$o$ belong to the
interpretation vocabulary,~$p$ is a property and the pair~$(s,o)$ belongs to the extension of the property~$p$.  An
interpretation assigns the value true to an \ac{RDF} graph if it assigns the value true to all of its triples.
%

%
Additionally, in order to assign a truth value for a graph containing blank nodes, an interpretation must rely on a
mapping from the set of blank nodes present in the graph to terms in the graph.  This mapping of blank nodes ensures
that all occurrences of the same blank node are mapped to the same interpretation element and, since this mapping is not
an integral part of the interpretation, it also ensures that blank nodes have no visibility outside the graph.
%
Based on~\citet{MunozPerezGutierrez:2007aa}, we define the notion of \emph{map}:
%
\begin{definition}[Map]
  A \emph{map} is a function~$\theta : \AUBL \to \AUBL$ preserving URIs and literals, \ie~$\theta(t) = t$, for all~$t \in
  \AUL$. 
  % 
  Given a graph~$G$, we define~$\theta(G) = \set{ \triple{\theta(s), \theta(p), \theta(o)} \mid \triple{s, p, o} \in G }$. We speak of
  a map~$\theta$ from~$G_{1}$ to~$G_{2}$, and write~$\theta : G_{1} \to G_{2}$, if~$\theta$ is such that~$\theta(G_{1}) \subseteq
  G_{2}$.
  %
  Furthermore, we say that a map~$\theta$ is a \emph{grounding} of a graph~$G$, iff~$\theta(G)$ is a ground graph.
\end{definition}
%

We next present the definition of interpretation according to~\citet{MunozPerezGutierrez:2007aa}:
%
\begin{definition}[Interpretation,~\citet{MunozPerezGutierrez:2007aa}]
  \label{def:interpretation}
  % 
  An \emph{interpretation}~$\I$ over a vocabulary~$V$ is a tuple~$\I =\tuple{\Delta_{R}, \Delta_{P}, \Delta_{C},
    \Delta_{L}, \intP{\cdot}, \intC{\cdot}, \int{\cdot}}$, where~$\Delta_{R}, \Delta_{P}$,~$\Delta_{C}, \Delta_{L}$ are
  the interpretation domains of~$\I$, which are finite non-empty sets, and~$\intP{\cdot}, \intC{\cdot}, \int{\cdot}$ are
  the interpretation functions of~$\I$. They have to satisfy:
  % 
  \begin{enumerate}[noitemsep]
  \item $\Delta_{R}$ are the resources (the domain or universe of~$\I$);
  \item $\Delta_{P}$ are property names (not necessarily disjoint from~$\Delta_{R}$);
  \item $\Delta_{C} \subseteq \Delta_{R}$ are the classes;
  \item $\Delta_{L} \subseteq \Delta_{R}$ are the literal values (containing~$\AL \cap V$);
  \item $\intP{\cdot}$ is a function~$\intP{\cdot}\colon \Delta_{P} \to 2^{\Delta_{R} \times \Delta_{R}}$;
  \item $\intC{\cdot}$ is a function~$\intC{\cdot}\colon \Delta_{C} \to 2^{\Delta_{R}}$;
  \item $\int{\cdot}$ maps each~$t \in \AUL \cap V$ into a value~$\int{t} \in \Delta_{R} \cup \Delta_{P}$ such
    that~$\int{\cdot}$ is the identity for plain literals and assigns an element in~$\Delta_{R}$ to each element
    in~$\AL$.
  \end{enumerate}
\end{definition}



\subsection{RDF Schema}
\label{sec:rdf-schema}

As briefly presented in the previous section, \ac{RDFS} is a vocabulary that allows for the description of relations
between \ac{RDF} resources.
%
For this thesis, we will rely on a fragment of \ac{RDFS}, called~$\rhodf$, presented
by~\citet{MunozPerezGutierrez:2007aa}, that covers essential features of \ac{RDFS}.
%
$\rhodf$ consists of the following subset of the \ac{RDFS} vocabulary:~$\set{ \term{rdfs{:}subPropertyOf},
  \term{rdfs{:}subClassOf}, \term{rdf{:}type}, \term{rdfs{:}domain}, \term{rdfs{:}range}}$.  In the following, for
readability purposes, we are using the following abbreviations:~$\spp$ for \term{rdfs{:}subPropertyOf},~$\subclass$ for
\term{rdfs{:}subClassOf},~$\typeR$ for \term{rdf{:}type},~$\domR$ for \term{rdfs{:}domain}, and~$\range$ for
\term{rdfs{:}range}.


Based on the definition of interpretation (\cref{def:interpretation}), we can define the concept of
\emph{model} of an \ac{RDF} graph:
%
\begin{definition}[Model~\cite{MunozPerezGutierrez:2007aa}]
  % 
  \label{def:rdf-model}
  %
  An interpretation~$\I$ is a \emph{model} of a ground graph~$G$, denoted~$\I \models G$, \iff~$\I$ is an interpretation
  over the vocabulary~$\rhodf \cup \universe(G)$ that satisfies the following conditions:
  % 
  \begin{description}\label{condRDF}
  \item[Simple:] \
    \begin{enumerate}
    \item for each~$\triple{s, p, o} \in G$,~$\intA{p} \in\Delta_{P}$ and~$(\intA{s}, \intA{o}) \in \intP{\intA{p}}$;
    \end{enumerate}
  \item[Subproperty:] \
    \begin{enumerate}
    \item $\intP{\intA{ \spp}}$ is transitive over~$\Delta_{P}$;
    \item if~$(p, q) \in \intP{\intA{ \spp}}$ then~$p, q \in \Delta_{P}$ and~$\intP{p} \subseteq \intP{q}$;
    \end{enumerate}
  \item[Subclass:] \
    \begin{enumerate}
    \item $\intP{\intA{\subclass}}$ is transitive over~$\Delta_{C}$;
    \item if~$(c, d) \in \intP{\intA{\subclass}}$ then~$c, d \in \Delta_{C}$ and~$\intC{c} \subseteq \intC{d}$;
    \end{enumerate}
  \item[Typing I:] \
    \begin{enumerate}
    \item $x \in \intC{c}$ \iff~$(x,c) \in \intP{\intA{\typeR}}$;
    \item if~$(p, c) \in \intP{\intA{\domR}}$ and~$(x, y) \in \intP{p}$ then~$x \in \intC{c}$;
    \item if~$(p, c) \in \intP{\intA{\range}}$ and~$(x, y) \in \intP{p}$ then~$y \in \intC{c}$;
    \end{enumerate}
  \item[Typing II:] \
    \begin{enumerate}
    \item For each~$\eee \in \rhodf$,~$\intA{\eee} \in \Delta_{P}$
    \item if~$(p, c) \in \intP{\intA{\domR}}$ then~$p \in \Delta_{P}$ and~$c \in \Delta_{C}$
    \item if~$(p, c) \in \intP{\intA{\range}}$ then~$p \in \Delta_{P}$ and~$c \in \Delta_{C}$
    \item if~$(x, c) \in \intP{\intA{\typeR}}$ then~$c \in \Delta_{C}$
    \end{enumerate}
  \end{description}
\end{definition}

\nd Entailment among ground graphs~$G$ and~$H$ behaves as per the model-theoretic semantics: any interpretation that is
a model of~$G$ is also a model of~$H$.
%
In the case where~$G$ and~$H$ may contain blank nodes,~$G \models H$ \iff for any grounding~$G'$ of~$G$ there is a
grounding~$H'$ of~$H$ such that~$G' \models H'$.
%

In~\citet{MunozPerezGutierrez:2007aa}, the authors define two variants of the semantics: the default one includes
reflexivity of~$\intP{\int{ \spp}}$ and~$\intC{\int{\subclass}}$ over~$\Delta_{P}$ and~$\Delta_{C}$, respectively, but
herein we are only considering the alternative semantics presented in~\citet[Definition 4]{MunozPerezGutierrez:2007aa},
which omits this requirement.
% 
As a consequence, inferences such as~$G \models \triple{a, \subclass, a}$ are not supported.  However, the drawback of
this is minimal since such inferences do not add expressive power and are thus of marginal interest.
%


\subsubsection*{Deductive System}
% 
In what follows, we present the sound and complete deductive system from~\citet{MunozPerezGutierrez:2007aa}.
%
The system is arranged in groups of rules that capture the semantic conditions of models. In every rule,~$A,B,C,X$,
and~$Y$ are meta-variables representing elements in~$\AUBL$ and~$D,E$ represent elements in~$\AUL$.  
%
The rules are as follows:
% 
\begin{description}
\item[1. Simple:] ~ \\[0.5em]
  \begin{tabular}{llll}
    $(a)$ & $\frac{G}{G'}$ for a map~$\theta:G' \to G$ & $(b)$ & $\frac{G}{G'}$ for~$G' \subseteq G$
  \end{tabular}
\item[2. Subproperty:] ~ \\[0.5em]
  \begin{tabular}{llll}
    $(a)$ & $\frac{\triple{A,  \spp, B},  \triple{B,  \spp, C}}{\triple{A,  \spp, C}}$ & $(b)$ & $\frac{\triple{D,  \spp, E},  \triple{X, D, Y}}{\triple{X, E, Y}}$
  \end{tabular}
\item[3. Subclass:] ~ \\[0.5em]
  \begin{tabular}{llll}
    $(a)$ & $\frac{\triple{A, \subclass, B},  \triple{B, \subclass, C}}{\triple{A, \subclass, C}}$ & $(b)$ & $\frac{\triple{A, \subclass, B},  \triple{X, \typeR, A}}{\triple{X, \typeR, B}}$
  \end{tabular}
\item[4. Typing:] ~ \\[0.5em]
  \begin{tabular}{llll}
    $(a)$ & $\frac{\triple{D, \domR, B},  \triple{X, D, Y}}{\triple{X, \typeR, B}}$ & $(b)$ & $\frac{\triple{D, \range, B},  \triple{X, D, Y}}{\triple{Y, \typeR, B}}$
  \end{tabular}
\item[5. Implicit Typing:] ~ \\[0.5em]
  \begin{tabular}{llll}
    $(a)$ & $\frac{\triple{A, \domR, B},  \triple{D,  \spp, A}, \triple{X, D, Y}}{\triple{X, \typeR, B}}$  &
    $(b)$ & $\frac{\triple{A, \range, B},  \triple{D,  \spp, A}, \triple{X, D, Y}}{\triple{Y, \typeR, B}}$
  \end{tabular}
\end{description}
%
The deductive system presented by~\citet{MunozPerezGutierrez:2007aa} includes 7 rules, where the missing rules (rules
6-7) handle reflexivity.  
%
Furthermore, as noted in~\citet{MunozPerezGutierrez:2007aa}, the ``Implicit Typing'' rules are a necessary addition to
the rules presented in~\citet{Hayes:2004aa} for complete \ac{RDFS} entailment.  These represent the case when variable~$A$
in~$\triple{D, \spp, A}$ and~$\triple{A, \domR, B}$ or~$\triple{A, \range, B}$, is a property implicitly represented by
a blank node.

We denote with~$\set{\tau_{1}, \ldots, \tau_{n}} \vdash_{\mathsf{RDFS}} \tau$ that the consequence~$\tau$ is obtained from
the premise~$\tau_{1}, \ldots, \tau_{n}$ by applying one of the inference rules 2--5 above.
% 


%%% Local Variables:
%%% mode: latex
%%% mode: flyspell
%%% mode: reftex
%%% TeX-master: "../thesis"
%%% End:




\section{Comparison of the Data Models}
\label{sec:overview-data-models}
%
\begin{table} [t]
\centering
\caption{Feature overview of data models}
\label{fig:data-models-overview}
\begin{tabular}{rcccc}
\toprule
                         & RDB        &   \acs{XML}     & \acs{JSON}     & \acs{RDF} \\
\midrule
\multicolumn{1}{l}{\bf Model:}  & & & & \\
logical structure         & relations  &  tree           &  tree          & graph    \\
ordered                   & \no        &  \yes           &  \kindof       & \no      \\
schema validation         & \yes       &  \yes           &  \no           & \no      \\
inference                 & \no        &  \no            &  \no           & \yes     \\
\tabularnewline
\multicolumn{1}{l}{\bf Languages:}  & & & & \\
query                     & \yes       &  \yes           &  \no           & \yes     \\
data manipulation         & \yes       &  \yes            &  \no           & \no      \\
schema manipulation       & \yes       &  \no            &  \no           & \no      \\
\bottomrule
\end{tabular}
\end{table}
%
We present a brief comparison of the data models in \cref{fig:data-models-overview}, focusing on features of the
data model and existing query languages.\footnote{In this table we are using the term RDB as a shorthand for the
  relational model.}
%
The features of the data model we considered were the logical structure it uses to represent data, whether the model is
an intrinsically ordered data model, and if it provides means, possibly external, of performing schema validation.
%
Inference capabilities allow to deduce new data based on existing one by specifying structural and representational
properties.
%
Regarding the languages, we represent the existence of languages for querying data contained in the respective model, as
well as manipulation languages for both data and schema.  Such languages can be used, for example, for inserting and
updating data or changing the representation structure of data.
%


\paragraph*{The Relational Model.}
%
As we have discussed in \cref{cha:introduction}, the relational model is used to store information for many
software applications.  A relational database consists of a set of relations (also commonly known as tables), and data
for each table is called a record.
%
As the data model overview presented in \cref{fig:data-models-overview} shows, the relational model is the most
mature, having stable query and manipulation languages.
%

\paragraph*{XML.}
With the uptake of the \ac{WWW}, more flexible data models were introduced, such as \ac{XML}.
%
Also Web pages are described using the \ac{HTML} language, a syntax that although similar to \ac{XML} is mostly focused
on rendering the contents in a \emph{web browser}.
%
\ac{XML} is more concerned with describing the data it contains while rendering \acs{XSLT} transformation language.
% 
\ac{XML} is a tree-based, ordered data representation format that imposes no restrictions on its element and attribute
names, nor on the nesting structure.
%
The \ac{XML} query and transformation languages were presented in \cref{sec:prelim-xquery}, mostly focusing on the
XQuery language.  A recommendation for an XQuery Update language was also presented
by~\citet{RobieChamberlinDyck:2011aa}.

\paragraph*{JSON.}
%
\cref{sec:json} presented another tree-based representation format, \ac{JSON}, that has recently gained traction and
uptake on the Web due to its easy integration with the JavaScript language, which is supported by all modern web
browsers.
%
\acs{JSON} is mostly regarded as an interchange format, notably lacking the specifications of any type of query language
and schema validation.  The \ac{JSON} data model is also tree-based and it distinguishes different structures
(\emph{objects} and \emph{arrays}), where objects consist of unordered sets, while arrays represent an ordered sequence
of elements.


\paragraph*{RDF.}
%
The advance of the traditional, human-readable Web into a machine-readable \emph{Semantic
  Web}~\cite{LeeHendlerLassila:2001aa} introduces a new data model: \ac{RDF}.
%
\ac{RDF} is a graph-based data model and, as discussed in \cref{sec:model-integr-data}, is suitable for representing
integrated data.
%
One main difference between \ac{RDF} and the other data models relates to its capabilities for deducing new data, based
on a specialised vocabulary called \acl{RDFS}.  \ac{RDFS}, as opposed to XML Schema, does not behave as a form of data
validation but rather as a form of deducing new data.
%
Although the new SPARQL~1.1 query language includes the specification of an update
language~\cite{GearonPassantPolleres:2012aa}, this is still not a finalised \ac{W3C} standard so we chose to omit it
from \cref{fig:data-models-overview}.
%
Possible forms of validating \ac{RDF} data, even though no recommendation exists, may involve
\begin{enumerate*}[label=(\roman*)]
\item using the SPARQL query language (presented in \cref{sec:sparql-preliminaries}) for determining if any triples
  do not match the constraints;
\item SPIN~\cite{KnublauchHendlerIdehen:2011aa} is a vocabulary that also allows to specify constraints for \ac{RDF}
  data; or
\item by using extensions of \ac{OWL} towards integrity constraints, \eg~\citet{TaoSirinBao:2010aa}.
\end{enumerate*}
%
In this thesis we focus primarily on the \ac{RDF} data model; however, several other graph-based database models exist
and a survey is presented by~\citet{AnglesGutierrez:2008ab}.
%



%%% Local Variables:
%%% mode: latex
%%% mode: flyspell
%%% mode: reftex
%%% TeX-master: "../thesis"
%%% End:


\section{Conclusion}
\label{sec:conclusion-data-models}

This chapter introduced the basis for the different data models we are considering in this thesis.  As such we described
the relational, \ac{XML}, and \ac{RDF} data models and included a description of the \ac{JSON} interchange format.
%
As we have discussed in \cref{sec:model-integr-data}, from a data integration perspective, a flexible format for
representing data is desirable, hence the \ac{XML} or \ac{RDF} formats are preferred over the relational model.
%
The major differences between these data models are 
\begin{inparaenum}[(i)]
\item the structure (table vs. tree vs. graph) that is used to represent data; and
\item the ordering of the data model (\ac{XML} is an intrinsically ordered data model, \ac{JSON} included the ordered
  array structure, while relational databases and \ac{RDF} consist of an unordered set of statements).
\end{inparaenum}


The specific query language for each of these data models are presented in \cref{cha:query-languages}.
%
These different data models are bridged in our novel transformation language, described in detail in
\cref{cha:xsparql}.
%
Furthermore, \cref{cha:anql} presents a proposed extension to the \ac{RDF} model to represent context
information, such as temporal or provenance information, a much needed feature when considering integrated data.




%%% Local Variables:
%%% mode: latex
%%% mode: flyspell
%%% mode: reftex
%%% TeX-master: "../thesis"
%%% End:
