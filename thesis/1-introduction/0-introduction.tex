\chapter{Introduction}
\label{cha:introduction}
%
The term \emph{database} is commonly used to denote a \emph{large collection of data} stored within a
computer. 
%
While initial database systems focused mainly on the physical representation of the database, relying on files stored in
the computers' filesystem, new database models were introduced that provided an abstraction layer over the physical
representation of the database~\cite{AbiteboulHullVianu:1995aa,SilberschatzKorthSudarshan:2005aa}.
%
One of these new database models, the relational model, is nowadays an almost-ubiquitous representation model and, since
the introduction of this model by~\citet{Codd:1970aa}, there have been continuous advancements on storage and querying
mechanisms for relational data.
%
Several companies have focused on the commercialisation of relational database products, like Oracle, IBM DB2,
and Microsoft SQL Server or open-source solutions like PostgreSQL and MySQL.
%
The relational model relies on a strict separation between the data and the organisational schema of the data, where the
schema must be provided beforehand to the \emph{database management system}.  An historical evolution of the data models
in databases was presented by~\citet{Navathe:1992aa} and, with the ultimate focus on graph databases,
by~\citet{AnglesGutierrez:2008ab}.

Database research also began to focus on different aspects of their data, for instance maintaining extra information
such as temporal and provenance.
%
Temporal information allows to determine when tuples were inserted into the database or to represent time periods when
the tuple is considered valid.
%
Provenance information becomes especially important when combining data from different sources, as it can be used
determine from which sources information was derived from.


\begin{figure}[t]
  \centering
  \begin{scaletikzpicturetowidth}{\textwidth}
    

\begin{tikzpicture}[scale=\tikzscale,transform shape]
  \tikzstyle{every node}=[anchor=center]
  \tikzset{x=1em,y=1em}
  
  \draw (-1,0) -- (41,0);


  \node[text width=6em,rotate=0,anchor=east] at (-2,2) {Data Models};
  \draw[dashed,very thin] (-4,4) -- (41,4);

  \node[text width=6em,rotate=0,anchor=east] at (-2,6) {Meta-Information};
  \draw[dashed,very thin] (-4,8) -- (41,8);

  \node[text width=6em,rotate=0,anchor=east] at (-2,10) {Query \\Languages};
  
  % draw vertical lines
  \foreach \x in {0,10,20,30,40} \draw (\x,3pt) -- (\x,-3pt);
  
  \draw (5 ,0) node[below=3pt] {1970} ;
  \draw (15,0) node[below=3pt] {1980} ;
  \draw (25,0) node[below=3pt] {1990} ;
  \draw (35,0) node[below=3pt] {2000} ;

  \footnotesize
  \node at (1,2) (rdb) {\textbf{RDB}};
  \node at (4, 10) (sql) {\textbf{SQL}};

  \node[text width=5em,text centered] at (8,6) {Temporal databases};
  \node at (20, 10) {TSQL2};

  \node at (18,6) {Provenance};

  \node at (9,2) {Semantic};

  \node at (16,3) {Object Oriented};
  \node at (16,1) {Graph};

  \node at (25,2) {OEM};

  \node at (26,1) (xml) {\textbf{XML}};
  \node at (29, 9) {XSLT};
  \node at (32, 10) {\textbf{XQuery}};
  \node[text width=5em,text centered] at (30,6) {Temporal XML};

  \node at (28,3) (rdf) {\textbf{RDF}};
  \node at (34, 11) {\textbf{SPARQL}};
  \node[text width=5em, text centered] at (35,6) {\textbf{Temporal RDF}};

  \node at (32,2) (json) {\textbf{JSON}};

\end{tikzpicture}

%%% Local Variables:
%%% mode: latex
%%% mode: flyspell
%%% mode: reftex
%%% TeX-master: "../thesis"
%%% End:


  \end{scaletikzpicturetowidth}
  \caption{Overview of data models and query languages}
  \label{fig:overview-dm-ql}
\end{figure}
%
A timeline of the different data models, approaches for representing temporal and provenance information, and query
languages is presented in \cref{fig:overview-dm-ql}.\footnote{The models and languages we are particularly interested in
  for this thesis are highlighted in \textbf{bold}.}
%
The aim of this figure is to show trends in research rather than exact dates for several topics.  For example, research
in semantic data models (like the Entity-Relationship model), temporality and provenance in databases, as well as graph
databases, has spanned over several years.


\subsubsection*{Web Data Models}


With the increased importance of the \ac{WWW} in our daily lives, we have also witnessed a shift in the focus of
enterprise applications: from the desktop to the Web~\cite{AbiteboulBunemanSuciu:1999aa}.
%
While the Web was initially used to boost the visibility of the enterprise, \eg~the corporate website quickly became an
important form of attracting new business and clients, nowadays more enterprise tasks are accomplished though Web
applications.
%
For example, it is becoming commonplace for enterprises to use online calendars and meeting scheduling systems or even
word processing systems that allow employees to collaboratively and concurrently work on the same document.
%
Many of these Web applications follow a \emph{multitier} approach, where data sources (often residing in relational
databases) are integrated before exposing the result as \ac{HTML} pages, possibly linked to other information sources
across the Web~\cite{SilberschatzKorthSudarshan:2005aa,AbiteboulBunemanSuciu:1999aa}.

For an open environment, such as the \ac{WWW}, the predefined schema requirement of the relational model does not
provide the flexibility necessary to deal with different representations of the same concepts and agreeing on a common
representation of concepts (also referred to as a \emph{global-schema}) is often not an achievable
objective~\cite{AbiteboulBunemanSuciu:1999aa}.
%
Thus, \emph{\sd} data emerged as a possible solution for avoiding the need for a predefined schema and several flexible
data models, well suited for representing integrated data, were
introduced~\cite{PapakonstantinouGarcia-MolinaWidom:1995aa,CluetDelobelSimeon:1998aa,Buneman:1997aa}.
%
Most of the presented data models for \sd data are tree or graph-based.
% 
On the \ac{WWW} the \ac{XML} has become a widely used data representation format and is regarded
by~\citet{AbiteboulBunemanSuciu:1999aa} as the \emph{de-facto} standard for information exchange.  \footnote{More
  recently, \ac{JSON} is becoming the preferred information exchange format, often in determent of \ac{XML}.}
%
\ac{XML} follows a \sd, \emph{tree-like} data model and several information integration projects relied on \ac{XML} for
representing their
data~\cite{DraperHalevyWeld:2001ab,DraperHalevyWeld:2001aa,BaruGuptaLudascher:1999aa,ManolescuFlorescuKossmann:2001aa,YuPopa:2004aa}.
%

Another data model, the \acf{RDF}\footnote{The \ac{RDF} data model is considered essentially a \emph{directed labelled
    graph}, however, in \cref{sec:model-integr-data} we discuss different views on this representation
  model.}~\cite{ManolaMiller:2004aa} has recently been gaining importance on the Web and the Semantic
Web~\cite{LeeHendlerLassila:2001aa}, supported by efforts like the \ac{LOD}
initiative~\cite{BizerHeathBerners-Lee:2009ab}.
%
With the increase of data published in \ac{RDF} as Linked Data, for example the DBpedia
project~\cite{BizerLehmannKobilarov:2009aa}, a valuable and steadily growing source of structured information from
various domains is being made available.
%
The possibility of using \ac{LOD} structured information in integration scenarios, allowing for an easy and low cost
enrichment of enterprise data, provides further incentive for an enterprise to represent its integrated data in
\ac{RDF}~\cite{Stephens:2007aa}.



\subsubsection*{Data Segmentation and Data Integration}

Four decades past the introduction of the relational model, most current software applications still rely on relational
databases (RDB) to store their information.
%
Enterprises commonly use RDB-based software applications to manage each aspect of their business, ranging from human
resources to manufacturing.  However, the use of specialised applications results in data segmentation, where the
enterprise's valuable data is spread across different applications and relational
databases~\cite{Dillnut:2006aa,SilberschatzKorthSudarshan:2005aa,BernsteinHaas:2008aa}.


%
For instance, having an integrated view on customers allows an enterprise salesman to better target its product, or
enables decision support systems to provide the management with a high-level view of all of the enterprise resources,
from manufacturing to human resources and sales.
%
As such, data integration in the context of relational databases has been a research topic in the
past~\cite{HalevyRajaramanOrdille:2006aa} and a good overview of integration techniques is provided
by~\citet{DoanHalevy:2005aa}.
%
Focusing on enterprise data integration, common issues and possible solutions are described
in~\citet{ZieglerDittrich:2004aa,HalevyAshishBitton:2005aa}.


Common approaches for integrating segmented data over relational database systems involve the use of \emph{mediator} or
\emph{data-warehousing} systems~\cite{Wiederhold:1992aa,AbiteboulBunemanSuciu:1999aa,ZieglerDittrich:2004aa}.
%
Mediator systems provide an abstraction layer over the original data sources often using a \emph{global-schema} (or
\emph{mediated-schema}), where queries over this schema are executed over the original sources.  On the other hand in
the data-warehousing approach, data in the original sources is materialised into a common data model.
%
Although both of these approaches have advantages and disadvantages, data-warehousing is particularly unsuitable in
changing environments: 
%
consider for instance that one of the sources is highly dynamic \eg~containing data gathered from sensors: this would
cause the integrated data to become quickly outdated~\cite{AbiteboulBunemanSuciu:1999aa}.
%
Other forms of data integration may also be considered, such as federated databases~\cite{ShethLarson:1990aa}, using Web
services~\cite{AbiteboulBenjellounMilo:2002aa}, or peer-to-peer systems~\cite{ArenasKantereKementsietsidis:2003aa}.

For the scope of this thesis we consider performing the data integration by relying on a newly defined query language
that is capable of accessing and transforming data stored in heterogeneous data sources and models that can be used as
an implementation language for both mediator and data-warehousing scenarios.


\subsubsection*{Meta-information in the Data Integration Process}

Although existing systems provide a way to solve the data segmentation problem, additional questions often arise when
integrating data, such as \emph{which sources} were involved in producing a specific piece of information or how to deal
with \emph{conflicting} information contained in the original sources.  For example, different enterprise systems can
store different addresses for an employee.
%
There are several forms of dealing with conflicting information, for example, maintaing \emph{provenance} information
(also known as \emph{lineage}) allows to determine from which of the original sources the specific information has been
derived~\cite{CuiWidomWiener:2000aa,WoodruffStonebraker:1997aa,BenjellounSarmaHalevy:2008aa} in order to trace the
origin of the contradiction and possibly correct it.
%
Other approaches include maintaining \emph{temporal} or \emph{uncertain} information, which caters for evolving data,
possibly avoiding contractions, and levels of confidence or certainty to be assigned to the conflicting data,
respectively.\footnote{In the following we refer to ``information about data'' commonly as \emph{meta-information}.}
%
These aspects of data have been identified as an important part of the data integration process
by~\citet{HalevyRajaramanOrdille:2006aa}, and for example, the Trio
system~\cite{Widom:2005aa,AgrawalBenjellounSarma:2006aa} extends the relational data model to consider both provenance
and uncertain information.
%


Meta-information can become an important aspect of any data integration process and as such any suitable data model for
representing integrated data needs to cater for this kind of information.
%
Even aside from our core focus on data integration, meta-information is still an important aspect in any software
application.
%
It is common in applications and database schemas to maintain temporal information, for example, keeping logs of
specific changes to the database, records of past employees, or having historical data available for manufacturing
materials in order to predict future needs.
%
In certain scenarios, temporal information even constitutes a critical aspect, where well known examples involve
real-time monitoring such as air-traffic control.
%
In these cases, temporal information is considered an important dimension, warranting its introduction into the
relational model~\cite{AbiteboulHullVianu:1995aa,Snodgrass:1999aa}, which in turn lead to the concept of temporal
databases.\footnote{An historical overview of temporality in databases is presented by~\citet{Snodgrass:1990aa}.}
%
Similar extensions have also been proposed to represent temporal information in
\ac{XML}~\cite{AmagasaYoshikawaUemura:2000aa,RizzoloVaisman:2008aa} and
\ac{RDF}~\cite{GutierrezHurtadoVaisman:2007aa,PuglieseUdreaSubrahmanian:2008aa,TappoletBernstein:2009aa}.
%
However temporal information is not the only kind of meta-information we can consider.  Other extensions to the
relational model also allow to represent ambiguous or approximate data in the form of fuzzy information.  An overview of
fuzzy databases is provided by~\citet{MaYan:2008aa}, where fuzzy extensions were later also proposed for the
\ac{XML}~\cite{MaYan:2007aa} and \ac{RDF} models~\cite{Straccia:2009aa,MazzieriDragoni:2008aa,LvMaYan:2008aa}.




\section{Problem Statement}
\label{sec:problem-statement}


Currently established data models do not easily support the data integration process: a data model suitable for
representing integrated data needs to be flexible and to cater for meta-information.
%
However, even such a flexible data model is not enough for a complete data integration application: while a flexible
data model facilitates the representation of integrated data, it still does not help the task of data gathering and
transformation.
%
For a complete solution, the data integration application must be aware of both the input sources and the target data
model.


Existing solutions for enterprise data integration rely on specialised or custom-built applications, following either
the mediator or data-warehouse approaches, to bridge the distributed sources and different data models.  However, the
costs of such applications quickly becomes too high~\cite{HalevyAshishBitton:2005aa}.
%
Another option is to consider using a query language for the data integration task~\cite{DraperHalevyWeld:2001aa}, but
traditional query languages focus only on one data format and are thus not a possibility when the distributed data
adheres to different data formats.
%
In such cases, the use of a query language requires translating the original data into a common data model, much like
the data-warehousing approach, and then performing the queries over the integrated data.  
%
A scenario that may also complicate such an approach is when the original data is protected by some form of access
control, where these access restrictions would also need to be replicated in the mediator or data-warehouse in order to
avoid information leakage.



With the introduction of the \ac{WWW}, the data integration task can become unfeasible using traditional approaches due
to the large amount of sources and different models.
%
The evolution of the Web into the Semantic Web has also introduced a new data model, \ac{RDF}, which can facilitate the
representation of integrated data.
%
However, the currently standardised \ac{RDF}-based specifications do not cater for any type of meta-information
regarding the individual \ac{RDF} triples.\footnote{One possible way of attaching meta-information to \ac{RDF} triples
  is by using \emph{reification}.  However the use of this feature may be discouraged in future revisions of the
  \ac{RDF} language \cf~\url{http://www.w3.org/2010/09/rdf-wg-charter}.}


\section{A Model for Integrated Data}
\label{sec:model-integr-data}


Software applications are now focused on the Web, having evolved from single-user applications and the personal
computer.
%
When we look at data models we find a similar evolution (\cf~\cref{fig:overview-dm-ql}), starting from data models
that were mostly aimed at storing information in a single computer system to current ones that allow to share and link
information to and from different sources.
%
The widely disseminated relational data model, although perfect for a closed environment such as a specific application
within an enterprise system, is not so well suited for open environments like the Web, or for representing integrated
data since, in both cases, the data schema cannot always be determined \textit{a priori}.
%

Given the variety of data and formats to be integrated, in this thesis we argue for a unifying data model and a query
language capable of integrating data represented in different formats and models.
%
Several \sd data models were presented that cater for dynamic, open, and flexible environments.
%
\paragraph*{OEM.}
One of the most notable \sd data models is the \acf{OEM} model~\cite{PapakonstantinouGarcia-MolinaWidom:1995aa}, defined
in the context of the TSIMMIS data-integration project~\cite{Garcia-MolinaPapakonstantinouQuass:1997aa}.  \ac{OEM} is
considered a \sd data model, consisting of a \emph{graph of objects}.  An object is represented as a
quadruple~$\triple{\mathit{label}, \mathit{oid}, \mathit{type}, \mathit{value}}$, where $\mathit{label}$ aims to be a
human readable description of the object, thus making the data model \emph{self-describing}.
% 
The $\mathit{oid}$ is a unique identifier for each object and $\mathit{type}$ indicates the type of the value.  Finally,
$\mathit{value}$ consists of an atomic value or a set of objects.

\paragraph*{XML.}
%
As presented by~\citet{Suciu:1998aa}, \ac{XML} has important differences to \sd data, one of which is that \ac{XML} more
naturally represents data as \emph{trees} whereas \sd data models, namely the \ac{OEM} model, are usually
graph-based.\footnote{\ac{XML} does cater for representing arbitrary graphs, by assigning unique identifiers to nodes
  and then using \ac{XML} \emph{references} to point to those identifiers.}
%
Another core difference between \ac{XML} and the \ac{OEM} model resides in the \emph{ordering} of the data model:
\ac{XML} is an intrinsically ordered data model, where each element has a specific order among its siblings, while the
\ac{OEM} model consists of an unordered graph similar (in terms of lack of ordering) to the relational model.
%
The tree and ordered data model of \ac{XML} makes integrating different documents a difficult task, even requiring
Turing-complete languages for arbitrary transformations~\cite{Kepser:2004aa}.
%
There are diverging opinions regarding whether \ac{XML} is self-describing.  Undoubtably, its unrestricted modelling
features allow to specify the meaning of the data it contains, however, without the use of an \ac{XML} Schema, it is
impossible to accurately determine the types of the values~\cite{SimeonWadler:2003aa}.
%
Finally, another difference between these models is the use of \ac{XML} attributes and although the \ac{OEM} model could
be trivially extended to cater for similar representations, it does not have a natural equivalent~\cite{Suciu:1998aa}.


\paragraph*{RDF.}
%
The \ac{RDF} data model is closely related to the \ac{OEM} data model for \sd data:
%
\begin{inparaenum}[(i)]
\item its representation model is a graph;
\item it is unordered; and
\item it is schema-less and self-describing, relying on \acp{URI}~\cite{Berners-LeeFieldingMasinter:2005aa} as unique
  identifiers for resources.
\end{inparaenum}
%
The major differences between these data models is that the \ac{RDF} data model is more correctly represented by an
\emph{hypergraph} since, as described by~\citet{HayesGutierrez:2004aa}, properties can themselves be the subject and
object of other \ac{RDF} triples, making the \ac{RDF} model go beyond the theoretical notion of a graph.
%
The existence of blank nodes in \ac{RDF}, akin to existential variables, is another significant difference between the
\ac{RDF} and \ac{OEM} data models.
%
When compared to \ac{XML}, \ac{RDF} is schema-unware: the task of \ac{RDFS} is to deduce implicit information rather
than restricting the structure of the \ac{RDF} data, as is the task of \ac{XML} Schema for \ac{XML} data.
% 
A survey of other graph models, focusing mostly on databases, is presented by~\citet{AnglesGutierrez:2008ab}.


\subsubsection*{Requirements of a Data Model for Integrated Data}
%
In the following we define the characteristics of a data model that is suitable for representing integrated data.
%
Most importantly, any such data model needs to be \emph{composable} \ie~the merging of data should be an easy task and,
inspired by \sd data models, we can present the desired features of a data model for achieving composability as:
%
\begin{description}
\item[Entity-Centric Global Identifiers:] The need for global identifiers is justified by the fact that we can, in any
  closed system, uniquely identify an entity, for example, by giving it a unique sequential identifier.  For a global
  system, we also need to uniquely identify an entity, thus requiring a global identifier.  In the case of \ac{RDF},
  this global identifier is the \ac{URI}.  For global identifiers, we need to make some assumptions, namely that they
  are used consistently \ie~the same identifier is not allowed to be used to identify different entities.  It is however
  possible for the same entity to be identified by distinct identifiers.
  %

\item[Schema-less:] A schema-less data model is one of the premises of \sd data: in an open and possibly global
  environment, obtaining agreement on the schema to represent any domain is a difficult and often impossible task.
  Monetary considerations aside (it would be extremely expensive to develop a global schema) cultural differences or
  simply personal preferences often stand in the way of achieving agreement on a schema~\cite{GohMadnickSiegel:1994aa}.
  % 
  Some common examples are the cultural, and often legislative, differences in the concepts of marriage: in some
  cultures marriage must be monogamous while for others this is not a requirement.  Another example of schema conflicts
  would be modelling monotheistic and polytheistic views of religion.
  % 
  Conversely, under a closed environment, it is possible to obtain a level of agreement over a topic, for example all
  users of a specific system can agree on the use of a single \emph{domain model}.
  % 
  For the \ac{WWW}, one noteworthy attempt at defining a collection of models is \url{schema.org}; supported by the
  Google, Yahoo! and Bing search engines, the major incentive for using this vocabulary is that webpages will be better
  indexed by these search engines.
  %
  However, the concepts defined so far are unambiguous, such as Places, Events, or Organisations.  No vocabulary is yet
  provided that caters for ambiguous concepts such as those presented above.

\item[Self-Describing:] Also related to the previous topic, a self-describing data model is a necessary characteristic
  derived from existing \sd data models that allows arbitrary data to be merged and exchanged without the need for
  domain specialists.
  %
  This requirement allows us to arbitrarily combine information about the same object, where object identity can be
  determined by global identifiers.
  %


\item[Graph-Based:] The need for a graph-based data model is necessary for the data integration step.  If we focus on
  the data models we presented so far, we rapidly come to the conclusion that we need a graph-based data model: \ac{RDF}
  is in itself graph-based and \ac{XML}, although naturally a tree-based model, includes forms of graph representation
  (by means of giving \ac{XML} nodes identifiers and then referring to them).
  % 
  As for the relational model, although it consists of a set of relations, the schema is actually a graph when we take
  into account foreign key constraints between different relations.  When we take a schema-less data-model as the target
  data model, the schema information needs to be encoded in the data and as such, even for the relational model, we
  require a target data model that is a graph.

\end{description}
%



As we have seen, \ac{RDF} has clear advantages over the relational and \ac{XML} formats as a representation model for
integrated data:
%
\begin{inparaenum}[(i)]
\item \ac{RDF} is \textit{per se} schema-unaware; 
\item by relying on \acp{URI}, it uses a standard mechanism for providing global identifiers for entities; and
\item it is self-describing, since according to the \ac{LOD} principles, by accessing each \ac{URI} we obtain further
  information about the resource or by using \ac{RDF} Schema~\cite{BrickleyGuha:2004aa} we can further deduce implicit
  information.
\end{inparaenum}


%%% Local Variables:
%%% mode: latex
%%% mode: flyspell
%%% mode: reftex
%%% TeX-master: "../thesis"
%%% End:


\section{Hypothesis}
\label{sec:hypothesis}


In this thesis we propose the use of \ac{RDF} as a representation model for integrated data and extend \ac{RDF} with
support for meta-information, thus allowing us to deal with temporal and uncertain data.
%
Several data models suitable for representing integrated data have been presented before, mostly tree or
graph-based~\cite{CluetDelobelSimeon:1998aa,PapakonstantinouGarcia-MolinaWidom:1995aa,Abiteboul:1997aa}, and we consider
that \ac{RDF}, being graph-based, is a well suited format for representing integrated data. 
%
We are particularly focusing on the conversion of data stored in legacy models, such as relational databases and
\ac{XML}, into \ac{RDF}.
%
The objective is to facilitate this data integration process by providing a query language that is capable of accessing
the data stored in different source formats and thus, as opposed to traditional query languages, avoid the explicit need
for \textit{a priori} data translation, while allowing the target \ac{RDF} data to be created.


Support for meta-information in \ac{RDF} is necessary not only for representing integrated data, but also
%
if the original source already contains some form of meta-information, such as temporal data, or access restrictions.
To represent temporal, fuzzy, provenance, or access control information, we need to consider an extension of the
\ac{RDF} data model that caters for such kinds of meta-information and further make the query language aware of this
extension.



The main hypothesis of this thesis can be summarised as follows:
%
\begin{quotation}
  \noindent
  % 
  \textbf{%
    %
    Efficient data integration over heterogeneous data sources can be achieved by
    \begin{enumerate*}[label=(\roman*), noitemsep, before=\unskip{: }, after=\unskip{.}, itemjoin={{; }}, itemjoin*={{; and }}]
    \item\label{hip:a} a query language that allows to access data adhering to different formats in the original sources (without the
      need for data transformation)
    \item\label{hip:b} a set of optimisations that allow for efficient query evaluation in such a query language
    \item\label{hip:c} an interchange representation format based on \ac{RDF} with support for meta-information,
      allowing to represent temporal, uncertain, provenance, or even access-control information
    \end{enumerate*}
  }
\end{quotation}
% 
The proposed query language must be expressive enough to represent arbitrary transformations between data models, an
important characteristic since one of the considered data models is the tree-based \ac{XML}, where the merging of
\ac{XML} data needs to based on tree transformations.

When representing the integrated data as \ac{RDF} extended with meta-information, other issues arise: how should
\ac{RDFS} handle such meta-information in the inference process? Or how can we query the meta-information?  Extending
the \ac{RDF} data model with meta-information also requires a similar extension to SPARQL: the \ac{W3C} recommended
query language for \ac{RDF}.
%


\section{Contributions}
\label{sec:contributions}

%
We validate our hypothesis by designing a query language, called XSPARQL, that combines the XQuery, SPARQL, and the
\ac{SQL} query languages, thus providing a cross-model query language suitable for data integration.
%
The initial proposal of combining the XQuery and SPARQL query languages was presented before the start of this thesis
by~\citet{AkhtarKopeckyKrennwallner:2008aa}.  
%
Shortly after this initial paper, I joined the project and started working on the following aspects, which constitute a
substantial part of the presented thesis:
\begin{itemize}[noitemsep]
\item formalising the existing XSPARQL language and extending its semantics to cater for both the \ac{XML} and
  relational models;
  % 
\item a set of optimisations over this novel language, specifically targeted at nested queries, that improve the
  evaluation times for such types of queries, including both formal proof of the correctness and empirical proof of the
  efficiency of such optimisations; and
  % 
\item a general extension of the \ac{RDF} data model, called Annotated RDFS, that allows to represent meta-information
  and forms the target data model for integrated data.
  % 
  We also detail extensions of the \ac{RDFS} inference rules and the SPARQL query language that allow to infer new
  information and query this novel data model.
  %
\end{itemize}
%
The difference in ordering of the data models is bridged at the query language level: even for the unordered data models
(relational and \ac{RDF}) their query languages (\ac{SQL} and SPARQL, respectively) impose an ordering on the query
results.\footnote{This is further detailed in \cref{cha:query-languages}.}
%
The XSPARQL query language is based on XQuery, which is an intrinsically ordered query language for \ac{XML} data, and 
%
we rely on the implicit ordering provided the \ac{SQL} and SPARQL query languages to maintain an ordered query language.
%
For the purposes of data integration, where we are interested in generating \ac{RDF}, the ordering in the query language
is not important (since the target data model is unordered).\footnote{In \cref{cha:optimisations} we exploit features of
  the XQuery language (and thus inherited by XSPARQL) that allow to disregard ordering during query evaluation.}

\subsection{Impact}
\label{sec:scientific-contributions}


For tackling our hypothesis points~\ref{hip:a} and~\ref{hip:b}, the consolidated work on the XSPARQL language, catering
for the integration of the \ac{XML} and \ac{RDF} formats has been published in the Journal on Data
Semantics~\cite{BischofDeckerKrennwallner:2012aa}.  This work formalises XSPARQL and introduces the optimisations for
nested SPARQL queries in our current implementation of the XSPARQL language.
%
Further expanding on our hypothesis point~\ref{hip:a}, the extension of XSPARQL to also support relational
databases~\cite{LopesBischofDecker2011aa} was presented at the Portuguese Conference on Artificial Intelligence
(EPIA2011).


Regarding our novel data model, hypothesis point~\ref{hip:c}, we introduced the initial Annotated RDFS framework, along
with the definition of the Fuzzy and Temporal annotation domains, was accepted to the AAAI Conference on Artificial
Intelligence (AAAI 2010)~\cite{StracciaLopesLukacsy:2010aa}.
%
\citet{LopesPolleresStraccia:2010aa} later introduced the extension of the SPARQL query language that caters for
querying domain annotations.  This work was presented at the International Semantic Web Conference (ISWC-10).
%
The consolidation of the Annotated RDFS model and the AnQL query language~\cite{ZimmermannLopesPolleres:2012aa} was
published in the Journal of Web Semantics.
%

Finally, based on the data model proposed in this thesis,~\citet{LopesKirraneZimmermann:2012aa} specialises Annotated
RDFS to the access control domain.  This is also an important aspect when considering data integration since the
underlying sources often to have their data protected by some form of access control.  
%
Using the combined XSPARQL query language, it is possible to extract the data and access control information from the
underlying sources and replicate it as Annotated RDFS.
%
This work has been accepted to the International Conference on Logic Programming.



\subsection{Other Contributions}
\label{sec:contr-stand}

Since the focus of this thesis is on Web languages and data models, in addition to research publications, another
important aspect of the dissemination of our results is the impact regarding standardisation.
%
As such, parts of the work developed for this thesis have been submitted to the \ac{W3C} in the form of Member
Submissions or at the \ac{W3C} organised workshop on RDF Next Steps:
%
the first contribution was a \ac{W3C} Member Submission describing the XSPARQL language.  The aim of such Member
Submissions is to make the \ac{W3C} aware of technology being developed, which may be considered as input to future
working groups.
%
The XSPARQL \ac{W3C} Member Submission was composed of four documents:
\begin{inparaenum}[(i)]
\item XSPARQL Language Specification~\cite{PolleresKrennwallnerLopes:2009aa};
\item XSPARQL: Implementation and Test-cases~\cite{LopesKrennwallnerPolleres:2009aa};
\item XSPARQL: Semantics~\cite{KrennwallnerLopesPolleres:2009aa}; and
\item XSPARQL: Use cases~\cite{PassantKopeckyCorlosquet:2009aa}. 
\end{inparaenum}
%

Two position papers were accepted to the \ac{W3C} Workshop on RDF Next Steps. The purpose of this workshop was to gather
feedback on possible improvements (if any) for the next iteration of the \ac{RDF} language.  The accepted position
papers argued for the need to integrate \ac{XML} and \ac{RDF} by means of a query
language~\cite{LopesBischofErling:2010aa}, largely inspired by the XSPARQL language, and the need to cater for
meta-information in \ac{RDF}~\cite{LopesZimmermannHogan:2010aa}, calling for a framework similar to Annotated RDFS.

A presentation detailing the XSPARQL language and focusing on the integration of heterogeneous sources on the Web, such
as \ac{XML} and \ac{RDF} in the form of Linked Data, was presented at the 2011 Semantic Technology (SemTech)
Conference~\cite{LopesPolleres:2011aa}.  


Also, my participation in the \ac{W3C} RDB2RDF working group, currently a \ac{W3C} recommendation, resulted in an
implementation of the RDB2RDF Direct Mapping~\cite{ArenasPrudhommeauxSequeda:2011aa} and
R2RML~\cite{DasSundaraCyganiak:2011aa} language specifications, using the XSPARQL language described in this thesis.
The implementation of these specifications in XSPARQL was submitted to the \ac{W3C} RDB2RDF Working Group.




\section{Thesis Outline}
\label{sec:outline}

Next we present an overview of each of the following chapters in this thesis:
%
\begin{description}

\item[\cref{cha:data-models} (\nameref*{cha:data-models})] presents the necessary background information regarding the
  relevant data models considered in the integrated query language: the relational, \ac{XML}, and \ac{RDF} data models.

\item[\cref{cha:query-languages} (\nameref*{cha:query-languages})] gives an overview of the query languages that can be
  used over the different data models: SQL for relational databases, XQuery for \ac{XML} data, and SPARQL for data
  adhering to the \ac{RDF} model.
  % 

\item[\cref{cha:xsparql} (\nameref*{cha:xsparql})] introduces our combined query language, called XSPARQL, that allows
  data adhering to the relational, \ac{XML}, and \ac{RDF} data models to queried using a common language.  The XSPARQL
  language consists of an extension of the XQuery query language with syntactical constructs taken from \ac{SQL} and
  SPARQL.  We present the syntax and semantics of XSPARQL, based on extending the XQuery formal semantics, and show
  correspondences between this novel query language and its composing languages.
  % 

\item[\cref{cha:optimisations} (\nameref*{cha:optimisations})] describes our current implementation of the XSPARQL
  language, presents the experimental evaluation and some possible optimisations for XSPARQL queries.  These
  optimisations focus on the interface between the different data formats, which in the case of nested queries, may
  cause severe evaluation overhead when compared with their single data model counterparts.  We present an evaluation of
  the proposed optimisations based on a newly defined benchmark suite encompassing different data models.

\item[\cref{cha:anql} (\nameref*{cha:anql})] presents a common extension of the \ac{RDF} data model, called Annotated
  RDFS, that caters for different kinds of meta-information, and facilitates the modelling of Temporal, Fuzzy, and
  Provenance meta-information in \ac{RDF}.
  % 
  This chapter also includes the extension of the SPARQL query language to query the \ac{RDF} annotated with
  meta-information.  This extension of the SPARQL language, called AnQL, allows the user to write meta-information aware
  queries and we extend the SPARQL algebra to allow for the propagation of this meta-information in the query.
  %   

\item[\cref{cha:usecase} (\nameref*{cha:usecase})] illustrates an integrated \usecase where the XSPARQL language further
  extended to support Annotated RDFS is used to extract legacy information contained in several enterprise systems and
  convert it to \ac{RDF} while maintaining any existing access control permissions.  We give an overview of a system
  architecture that, based on XSPARQL, extracts the data (along with the access control information) from the original
  sources, converts this data into Annotated RDFS, and enforces the access control permissions.

\item[\cref{cha:conclusions} (\nameref*{cha:conclusions})] contains critical discussion of the presented work,
  highlights future directions of research and finishes with some concluding remarks.

\end{description}



%%% Local Variables:
%%% mode: latex
%%% mode: flyspell
%%% mode: reftex
%%% TeX-master: "../thesis"
%%% End:
