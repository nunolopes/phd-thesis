\section{A Model for Integrated Data}
\label{sec:model-integr-data}


Software applications are now focused on the Web, having evolved from single-user applications and the personal
computer.
%
When we look at data models we find a similar evolution (\cf~\cref{fig:overview-dm-ql}), starting from data models
that were mostly aimed at storing information in a single computer system to current ones that allow to share and link
information to and from different sources.
%
The widely disseminated relational data model, although perfect for a closed environment such as a specific application
within an enterprise system, is not so well suited for open environments like the Web, or for representing integrated
data since, in both cases, the data schema cannot always be determined \textit{a priori}.
%

Given the variety of data and formats to be integrated, in this thesis we argue for a unifying data model and a query
language capable of integrating data represented in different formats and models.
%
Several \sd data models were presented that cater for dynamic, open, and flexible environments.
%
\paragraph*{OEM.}
One of the most notable \sd data models is the \acf{OEM} model~\cite{PapakonstantinouGarcia-MolinaWidom:1995aa}, defined
in the context of the TSIMMIS data-integration project~\cite{Garcia-MolinaPapakonstantinouQuass:1997aa}.  \ac{OEM} is
considered a \sd data model, consisting of a \emph{graph of objects}.  An object is represented as a
quadruple~$\triple{\mathit{label}, \mathit{oid}, \mathit{type}, \mathit{value}}$, where $\mathit{label}$ aims to be a
human readable description of the object, thus making the data model \emph{self-describing}.
% 
The $\mathit{oid}$ is a unique identifier for each object and $\mathit{type}$ indicates the type of the value.  Finally,
$\mathit{value}$ consists of an atomic value or a set of objects.

\paragraph*{XML.}
%
As presented by~\citet{Suciu:1998aa}, \ac{XML} has important differences to \sd data, one of which is that \ac{XML} more
naturally represents data as \emph{trees} whereas \sd data models, namely the \ac{OEM} model, are usually
graph-based.\footnote{\ac{XML} does cater for representing arbitrary graphs, by assigning unique identifiers to nodes
  and then using \ac{XML} \emph{references} to point to those identifiers.}
%
Another core difference between \ac{XML} and the \ac{OEM} model resides in the \emph{ordering} of the data model:
\ac{XML} is an intrinsically ordered data model, where each element has a specific order among its siblings, while the
\ac{OEM} model consists of an unordered graph similar (in terms of lack of ordering) to the relational model.
%
The tree and ordered data model of \ac{XML} makes integrating different documents a difficult task, even requiring
Turing-complete languages for arbitrary transformations~\cite{Kepser:2004aa}.
%
There are diverging opinions regarding whether \ac{XML} is self-describing.  Undoubtably, its unrestricted modelling
features allow to specify the meaning of the data it contains, however, without the use of an \ac{XML} Schema, it is
impossible to accurately determine the types of the values~\cite{SimeonWadler:2003aa}.
%
Finally, another difference between these models is the use of \ac{XML} attributes and although the \ac{OEM} model could
be trivially extended to cater for similar representations, it does not have a natural equivalent~\cite{Suciu:1998aa}.


\paragraph*{RDF.}
%
The \ac{RDF} data model is closely related to the \ac{OEM} data model for \sd data:
%
\begin{inparaenum}[(i)]
\item its representation model is a graph;
\item it is unordered; and
\item it is schema-less and self-describing, relying on \acp{URI}~\cite{Berners-LeeFieldingMasinter:2005aa} as unique
  identifiers for resources.
\end{inparaenum}
%
The major differences between these data models is that the \ac{RDF} data model is more correctly represented by an
\emph{hypergraph} since, as described by~\citet{HayesGutierrez:2004aa}, properties can themselves be the subject and
object of other \ac{RDF} triples, making the \ac{RDF} model go beyond the theoretical notion of a graph.
%
The existence of blank nodes in \ac{RDF}, akin to existential variables, is another significant difference between the
\ac{RDF} and \ac{OEM} data models.
%
When compared to \ac{XML}, \ac{RDF} is schema-unware: the task of \ac{RDFS} is to deduce implicit information rather
than restricting the structure of the \ac{RDF} data, as is the task of \ac{XML} Schema for \ac{XML} data.
% 
A survey of other graph models, focusing mostly on databases, is presented by~\citet{AnglesGutierrez:2008ab}.


\subsubsection*{Requirements of a Data Model for Integrated Data}
%
In the following we define the characteristics of a data model that is suitable for representing integrated data.
%
Most importantly, any such data model needs to be \emph{composable} \ie~the merging of data should be an easy task and,
inspired by \sd data models, we can present the desired features of a data model for achieving composability as:
%
\begin{description}
\item[Entity-Centric Global Identifiers:] The need for global identifiers is justified by the fact that we can, in any
  closed system, uniquely identify an entity, for example, by giving it a unique sequential identifier.  For a global
  system, we also need to uniquely identify an entity, thus requiring a global identifier.  In the case of \ac{RDF},
  this global identifier is the \ac{URI}.  For global identifiers, we need to make some assumptions, namely that they
  are used consistently \ie~the same identifier is not allowed to be used to identify different entities.  It is however
  possible for the same entity to be identified by distinct identifiers.
  %

\item[Schema-less:] A schema-less data model is one of the premises of \sd data: in an open and possibly global
  environment, obtaining agreement on the schema to represent any domain is a difficult and often impossible task.
  Monetary considerations aside (it would be extremely expensive to develop a global schema) cultural differences or
  simply personal preferences often stand in the way of achieving agreement on a schema~\cite{GohMadnickSiegel:1994aa}.
  % 
  Some common examples are the cultural, and often legislative, differences in the concepts of marriage: in some
  cultures marriage must be monogamous while for others this is not a requirement.  Another example of schema conflicts
  would be modelling monotheistic and polytheistic views of religion.
  % 
  Conversely, under a closed environment, it is possible to obtain a level of agreement over a topic, for example all
  users of a specific system can agree on the use of a single \emph{domain model}.
  % 
  For the \ac{WWW}, one noteworthy attempt at defining a collection of models is \url{schema.org}; supported by the
  Google, Yahoo! and Bing search engines, the major incentive for using this vocabulary is that webpages will be better
  indexed by these search engines.
  %
  However, the concepts defined so far are unambiguous, such as Places, Events, or Organisations.  No vocabulary is yet
  provided that caters for ambiguous concepts such as those presented above.

\item[Self-Describing:] Also related to the previous topic, a self-describing data model is a necessary characteristic
  derived from existing \sd data models that allows arbitrary data to be merged and exchanged without the need for
  domain specialists.
  %
  This requirement allows us to arbitrarily combine information about the same object, where object identity can be
  determined by global identifiers.
  %


\item[Graph-Based:] The need for a graph-based data model is necessary for the data integration step.  If we focus on
  the data models we presented so far, we rapidly come to the conclusion that we need a graph-based data model: \ac{RDF}
  is in itself graph-based and \ac{XML}, although naturally a tree-based model, includes forms of graph representation
  (by means of giving \ac{XML} nodes identifiers and then referring to them).
  % 
  As for the relational model, although it consists of a set of relations, the schema is actually a graph when we take
  into account foreign key constraints between different relations.  When we take a schema-less data-model as the target
  data model, the schema information needs to be encoded in the data and as such, even for the relational model, we
  require a target data model that is a graph.

\end{description}
%



As we have seen, \ac{RDF} has clear advantages over the relational and \ac{XML} formats as a representation model for
integrated data:
%
\begin{inparaenum}[(i)]
\item \ac{RDF} is \textit{per se} schema-unaware; 
\item by relying on \acp{URI}, it uses a standard mechanism for providing global identifiers for entities; and
\item it is self-describing, since according to the \ac{LOD} principles, by accessing each \ac{URI} we obtain further
  information about the resource or by using \ac{RDF} Schema~\cite{BrickleyGuha:2004aa} we can further deduce implicit
  information.
\end{inparaenum}


%%% Local Variables:
%%% mode: latex
%%% mode: flyspell
%%% mode: reftex
%%% TeX-master: "../thesis"
%%% End:
