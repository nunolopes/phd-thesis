
\subsection{Nested Queries in XMarkRDF}
\label{sec:optim-xmarkrdf-nest}


From the initial set of 20 queries there are 5 queries ($q_8$--$q_{12}$) that contain nested expressions. They are
described informally in the XMark suite as follows:
\begin{enumerate}[noitemsep]
\item[($q_8$)] ``List the names of persons and the number of items they bought;''
\item[($q_9$)] ``List the names of persons and the names of the items they bought in Europe;''
\item[($q_{10}$)] ``List all persons according to their interest;''
\item[($q_{11}$)] ``List the number of items currently on sale whose price does not exceed 0.02\% of the seller's
  income;'' and
\item[($q_{12}$)] ``For each richer-than-average person, list the number of items currently on sale whose price does not
  exceed 0.02\% of the person's income.''
\end{enumerate}



\begin{figure}[t]
\subfloat[Query~$q_9$ in XQuery (XMark)]{\label{fig:query09-xquery}%
  \begin{minipage}{.53\linewidth}%
    \lstinputlisting[linewidth=.97\linewidth,basicstyle=\ttfamily\scriptsize]{0-data+queries/query09.xq}%
  \end{minipage}%
}
%
\hfill
%
\subfloat[Query~$q_9$ in XSPARQL (XMarkRDB)]{\label{fig:query09-rdb}%
  \begin{minipage}{.53\linewidth}%
    \lstinputlisting[linewidth=.97\linewidth,basicstyle=\ttfamily\scriptsize]{0-data+queries/query09-rdb.xsparql}%
  \end{minipage}%
}%

\subfloat[Query~$q_9$ in XSPARQL (XMarkRDF)]{\label{fig:query09-xsparql}%
  \begin{minipage}{.53\linewidth}%
    \lstinputlisting[linewidth=.97\linewidth,basicstyle=\ttfamily\scriptsize]{0-data+queries/query09.xsparql}%
  \end{minipage}%
}%
%
\hfill
%
\subfloat[Query~$q_9$ in SPARQL2XQuery ($\textrm{XMarkRDF}_{\mathit{S2XQ}}$)]{\label{fig:query09-groppe}%
  \begin{minipage}{.53\linewidth}%
    \lstinputlisting[linewidth=.97\linewidth,basicstyle=\ttfamily\scriptsize]{0-data+queries/query09-groppe.xquery}%
  \end{minipage}%
}%
\caption{Variants of benchmark query~$q_9$}
\label{fig:query09}
\end{figure}
%

\cref{fig:query09-xquery,fig:query09-rdb,fig:query09-xsparql} present XMark query~$q_9$, its translated XSPARQL version
in XMarkRDB and XMarkRDF, respectively.
%
Query~$q_{9}$, as presented in \cref{fig:query09-groppe}, is ready to be evaluated by the SPARQL2XQuery system
over the~$\textrm{XMarkRDF}_{\mathit{S2XQ}}$ dataset.\footnote{Please note that this query follows the syntax presented
by~\citet{GroppeGroppeLinnemann:2008aa} however, we only had access to the implementation of the translation from SPARQL
to XQuery and hence manually replicated the complete query translation.}

The different rewritings presented in \cref{sec:optimisation} can be applied to the four nested
queries~$q_8$--$q_{11}$.
%
Query~$q_{12}$ also consists of a nested expression, however the most accurate translation of this query into XSPARQL
results in the dependent variable not being \emph{strictly bound} since it occurs only in the \FILTER of the inner
query. As such, we cannot apply the different rewritings to this query.
%

XMarkRDF query~$q_9$ is presented in \cref{fig:query09-xsparql}. This query is close to queries~$q_8$,~$q_{10}$, and
$q_{11}$ and consists of a nested expression: the inner \FOR expression of the query
(\crefrange{fig:query09:inner-start}{fig:query09:inner-end}) is executed once for each person matched by the outer
expression (\crefrange{fig:query09:outer-start}{fig:query09:outer-end}), which means that one SPARQL call will be
made for each person separately.  
%
Thus, the number of SPARQL calls performed in the inner expression directly depends on the size of the
dataset~(\cf~\cref{tab:iterations} for details).
%
Queries $q_8$,~$q_{9}$, and~$q_{11}$ evaluates the inner expression for each person, while~$q_{10}$ evaluates the inner
expression for each category.  Each dataset contains approximately 25 times more persons than categories.
%
The rewriting strategies presented in \cref{sec:optimisation} reduce the number of SPARQL calls to two: one to
get all the people (similar to the direct rewriting version), and one additional SPARQL call for retrieving all the
information about all the auctions in the dataset.  
%
Although the query remains exponential, the practical evaluation will show that reducing the number of SPARQL calls
drastically improves query execution times.




\begin{figure}[t]

  \subfloat[Query~$q_{9}$ -- bought items grouped by person]{\label{fig:output-09}
    \begin{minipage}{.5\linewidth}
      \lstinputlisting[language=XML,numbers=none]{0-data+queries/output-09.xml}
    \end{minipage}
  }
  %
  \subfloat[Query~$q_{9}'$ -- flat list of items and buyer]{\label{fig:output-39}
    \begin{minipage}{.5\linewidth}
      \lstinputlisting[language=XML,numbers=none]{0-data+queries/output-39.xml}
    \end{minipage}
  }
  \caption{Example output excerpts of queries~$q_{9}$ and~$q_{9}'$}
  \label{fig:output}
\end{figure}



As mentioned in \cref{sec:query-rewriting}, for the SPARQL based rewritings, we want the query output to be
computable directly in SPARQL without any further processing, \ie~we do not want to use XQuery for further processing
of the SPARQL results and the query should be expressible in SPARQL without features from SPARQL~1.1.  
%
Since the original nested queries~$q_8$--$q_{11}$ group the output results (while optionally applying some aggregation
function), we need to include modified versions of these benchmark queries for the evaluation of the SPARQL based
rewritings.
%
In these modified queries, denoted~$q_{8}'$--$q_{11}'$, we changed the return format of the queries to consist of a
flattened representation of the output of the original query.  An example of the output for queries~$q_{9}$ and~$q_{9}'$
is presented in \cref{fig:output}.
%
All queries~$q_i'$ and~$q_i''$ follow a similar strategy for reformatting the output:
%
the queries resulting from applying~$\funcName{opt_{sr}}$ are named~$q_8'$--$q_{11}'$, while the queries that consist of
an outer \FOR expression -- to which~$\funcName{opt_{ng}}$ was applied -- are~$q_8''$--$q_{11}''$.



\subsection{Evaluation of the Proposed Optimisations}
\label{sec:exper-eval}

In this section we present an experimental evaluation of the different rewritings presented in
\cref{sec:optimisation}.
%
For this evaluation we also rely on the XMarkRDF benchmark suite (presented in \cref{sec:benchm-descr}) and
compare, when possible, the effects of the different rewritings on the SPARQL2XQuery
system~\cite{GroppeGroppeLinnemann:2008aa}.
%



For the evaluation we extend the run configurations presented in \cref{sec:benchm-descr} with the following:
\begin{description}[noitemsep]
\item[$\mathit{XS^{rdf}_{Z}}$:] using the XSPARQL implementation over the XMarkRDF datasets (translated data and
  queries) with nested expresion optimisation \funcName{opt_{Z}} for~$Z \in \set{nl,ng,sr}$;
\item[$\mathit{XS^{rdb}_{nl}}$:] using the XSPARQL implementation over the XMarkRDB datasets (translated data and
  queries) with nested expresion optimisation \funcName{opt_{nl}};
\item[$\mathit{S2XQ}_{Z}$:] using the SPARQL2XQuery implementation over the translation of the XMarkRDF datasets into
  the required \ac{XML} format ($\textrm{XMarkRDF}_{\mathit{S2XQ}}$) with nested expresion optimisation
  \funcName{opt_{Z}} for~$Z \in \set{nl,sr}$.
\end{description}
%
The experimental setup remains the same as presented in \cref{sec:experimental_setup}.
%
We applied the nested-loop join rewriting from \cref{sec:join-impl-xquery} to the XMarkRDB and XMarkRDF translated
queries, which are denoted as~$\mathit{XS^{rdb}_{nl}}$ and~$\mathit{XS^{rdf}_{nl}}$, respectively.  The same
optimisations were applied to the SPARQL2\-XQuery translation to XQuery, denoted~$\mathit{S2XQ_{nl}}$ in the results.
%
The strategies of rewriting to a single SPARQL query, as presented in \cref{sec:join-impl-sparql}, were also applied to
the XSPARQL XMarkRDF and SPARQL2XQuery queries and are denoted as~$\mathit{XS^{rdf}_{sr}}$ and~$\mathit{S2XQ_{sr}}$,
respectively.  The Named Graph rewriting was applied to the XSPARQL XMarkRDF queries and is
denoted~$\mathit{XS^{rdf}_{ng}}$.



\def\scale{.99}
\begin{figure*}[t]\scriptsize
  \centering
  \subfloat[Query~$q_8$]{
    \import{\plotsDir}{optimisations-paper-query08}
    \label{fig:opt08}
  }
  \hspace{8pt}
  \subfloat[Query~$q_9$]{
    \import{\plotsDir}{optimisations-paper-query09}
    \label{fig:opt09}
  }

  \subfloat[Query~$q_8'$]{
    \import{\plotsDir}{optimisations-paper-query38}
    \label{fig:opt38}
  }
  \hspace{8pt}
  \subfloat[Query~$q_9'$]{
    \import{\plotsDir}{optimisations-paper-query39}
    \label{fig:opt39}
  }

  \subfloat[Query~$q_8''$]{
    \import{\plotsDir}{optimisations-paper-query48}
    \label{fig:opt48}
  }
  \hspace{8pt}
  \subfloat[Query~$q_9''$]{
    \import{\plotsDir}{optimisations-paper-query49}
    \label{fig:opt49}
  }
  \caption{Query response times for (variants of)~$q_8$ and~$q_9$ on all XMarkRDF datasets}
  \label{fig:evaluation-times-q8-q9} 
\end{figure*}

\begin{figure*}[t]\scriptsize
  \centering
  \subfloat[Query~$q_{10}$]{
    \import{\plotsDir}{optimisations-paper-query10}
    \label{fig:opt10}
  }
  \hspace{8pt}
  \subfloat[Query~$q_{11}$]{
    \import{\plotsDir}{optimisations-paper-query11}
    \label{fig:opt11}
  }

  \subfloat[Query~$q_{10}'$]{
    \import{\plotsDir}{optimisations-paper-query40}
    \label{fig:opt40}
  }
  \hspace{8pt}
  \subfloat[Query~$q_{11}'$]{
    \import{\plotsDir}{optimisations-paper-query41}
    \label{fig:opt41}
  }

  \subfloat[Query~$q_{10}''$]{
    \import{\plotsDir}{optimisations-paper-query50}
    \label{fig:opt50}
  }
  \hspace{8pt}
  \subfloat[Query~$q_{11}''$]{
    \import{\plotsDir}{optimisations-paper-query51}
    \label{fig:opt51}
  }
  \caption{Query response times for (variants of)~$q_{10}$ and~$q_{11}$ on all XMarkRDF datasets}
  \label{fig:evaluation-times-q10-q11} 
\end{figure*}

The comparison of the response times of the different rewriting functions presented in \cref{sec:optimisation} is
shown graphically in \cref{fig:evaluation-times-q8-q9,fig:evaluation-times-q10-q11}.
%
\begin{table*}[t]
  \caption[Query response times (in seconds) of different optimisations for the 2MB datasets.]{Query response times (in seconds) of different optimisations for the 2MB datasets. Optimisation not applicable (\textit{n/a}).}
  \label{tab:results-opt-2mb}
  \centering
  \import{\plotsDir}{table-optimisations-paper-2}
\end{table*}
%
The response times of these queries for the 2MB are presented in \cref{tab:results-opt-2mb} as a reference,
where~$\textit{n/a}$ indicates that the combination of query and optimisation is not applicable.





As we can see from \cref{tab:results-opt-2mb} and \cref{fig:evaluation-times-q8-q9,fig:evaluation-times-q10-q11}, the
\funcName{opt_{nl}} optimisation provides significant reduction in the query evaluation times when applied to the nested
queries with an inner \SparqlForClause.  For queries~$q_{8}$,~$q_{9}$, and~$q_{11}$ the difference in response times is
one order of magnitude.
%
However, applying a similar rewriting to relational data deteriorates the response times of the query.  This hints that
collecting the data and performing the join in the rewritten XQuery is slower than the nested calls to the relational
database.  
%
There are two possible causes for this discrepancy of behaviours from the different backends.  One possible explanation
for the speed improvement in SPARQL is that the overhead resides on the loading of data by the ARQ engine.  Since this
overhead is not presented in the relational database the queries would not benefit from the optimised rewritings.
%
The other explanation is that the cost of evaluating one unbound query and building the necessary structures for
representing the returned data is greater than the cost of executing nested queries.
%
Further investigation into this would be required to determine the source of the overhead.
%

The improvement in the execution time for query~$q_{10}$ is less drastic. This can be explained by the fact that the
outer expression of~$q_{10}$ iterates over ``categories'', which, as presented in \cref{tab:iterations}, increases at a
much smaller rate than ``persons'' do in the outer expressions of queries~$q_8$,~$q_9$, and~$q_{11}$.


However, for the~$\mathit{S2XQ}$ runs this optimisation provides virtually no improvement in the query response times
for queries~$q_8$ and~$q_9$ and their variants.  In queries~$q_{10}$,~$q_{11}$,~$q_{10}'$, and~$q_{11}'$ we can observe
an improvement in response times. This can be attributed to the fact that the rewriting for queries~$q_{10}$
and~$q_{11}$ and their variants are not as suitable for optimisation by the XQuery engine when compared to queries~$q_8$
and~$q_9$. For these cases our rewriting strategy is capable of performing the optimisation task for the XQuery engine.


For the~$\mathit{XS^{rdf}}$ run, it is possible to see in \cref{fig:opt38,fig:opt39} that~\funcName{opt_{sr}} (presented
in \cref{sec:join-impl-sparql}) is generally more efficient in terms of response times than the XQuery based.
%
This can be justified by the the smaller amount of information that is necessary to transfer from SPARQL to the XQuery
engine.  This effectively reduces the overhead of using an external SPARQL engine for the evaluation of queries.
%
Considering the~$\mathit{S2XQ_{sr}}$ run, \funcName{opt_{sr}} produces no improvement in the
query response times and in some cases ($q_{10}'$ and~$q_{11}'$ from \cref{tab:results-opt-2mb}) even deteriorates
considerably the response times when compared to~$\mathit{S2XQ}$.  This further supports our previous
claims that the XQuery engine is not capable of optimising the rewritten code from complex SPARQL queries.


Furthermore, the~$\mathit{S2XQ_{sr}}$ runs could only evaluate the smaller dataset sizes for query~$q_{8}$: its response
times deteriorated considerably with the larger dataset sizes, as opposed to the~$\mathit{XS^{rdf}_{sr}}$ runs that
behaved consistently similar to~$\mathit{XS^{rdf}_{nl}}$.  This indicates that~$\mathit{S2XQ}$ is not as efficient as
the ARQ-based native SPARQL engine runs~$\mathit{XS^{rdf}_{sr}}$ and~$\mathit{XS^{rdf}_{nl}}$ for larger datasets.


We can draw similar conclusions for the \funcName{opt_{ng}} when comparing the query evaluation times of the
\funcName{opt_{sr}} rewriting.  However, the response times for this approach are deteriorated by the the overhead of
creating, inserting and deleting the \ac{RDF} Named Graph.  This slowdown makes queries~$q_8''$,~$q_{10}''$ and~$q_{11}''$ of
the of the \funcName{opt_{nl}} rewriting outperform this optimisation.



%%% Local Variables:
%%% fill-column: 120
%%% TeX-master: t
%%% TeX-PDF-mode: t
%%% TeX-debug-bad-boxes: t
%%% TeX-parse-self: t
%%% TeX-auto-save: t
%%% reftex-plug-into-AUCTeX: t
%%% mode: latex
%%% mode: flyspell
%%% mode: reftex
%%% TeX-master: "../thesis"
%%% End:
