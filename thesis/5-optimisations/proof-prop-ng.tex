\begin{proof}
  %
  We start by showing the proof for the base case, where \envElem{\ExprSingle}{1} and \envElem{\ExprSingle}{2}
  of~\eqref{eq:expr_ng} do not contain any occurrences of~\eqref{eq:expr_ng}.
  %
  \paragraph*{Base Case.}
  %
  ($\Rightarrow$)
  % 
  Let us start by showing that if~$\dynEnv \grammarRule{\envElem{Q}{}} \Rightarrow \envElem{Val}{}$ then~$\dynEnv
  \ong{\envElem{Q}{}}$~$\Rightarrow \envElem{Val}{}$.
  %
  We now show the proof tree for each of the XQuery core expressions in the~$\funcName{opt_{ng}}$ rewriting.

\begin{description}

\item[\LET expression of line (1).]  Considering~$\envElem{NGP}{} = \set{ \mathtt{[]~{:}value}~\var{VarName} }$, we have

    \begin{prooftreefunction}
      \AxiomC{$\dynEnv{\funcCallR{createNG}{\begin{array}{l}
              \FOR~\var{VarName}~\grammarRule{OptTypeDeclaration}\\
              \grammarRule{OptPositionalVar}~\IN~\envElem{\ExprSingle}{1}\\
              \RETURN~\funcCallR{evalTemplate}{\envElem{NGP}{}}\\
              \end{array}
            }} \Rightarrow \envElem{DS}{} $}
      
      \UnaryInfC{$\dynEnvX{1}{ng}{\grammarRule{Expr}} \Rightarrow \envElem{Res}{}$}
      
      \UnaryInfC{$\dynEnv{\begin{array}{l}
            \LET~\varR{ds}~:=\\
            \quad\funcCallR{createNG}{\begin{array}{l}
                \FOR~\var{VarName}~\grammarRule{OptTypeDeclaration}\\
                \grammarRule{OptPositionalVar}~\IN~\envElem{\ExprSingle}{1}\\
                \RETURN~\funcCallR{evalTemplate}{\envElem{NGP}{}}\\
              \end{array}}\\
            \RETURN~\grammarRule{Expr}
          \end{array} \Rightarrow \envElem{Res}{}
        }$}
    \end{prooftreefunction}%
  where 
  \begin{tree}
    \ \ \dynX{1}{ng} = \dyn \envExtend{varValue}{\varNameR{ds} \Rightarrow \envElem{DS}{}} \enspace .
    \label{tree:dynEvn3-ng}
  \end{tree}%

\item[\LET expression of line (2).] As a shortcut representation, consider the dataset clause~$\gsup{\DatasetClause}{ng}
  = \DatasetClause\ \cup\ \set{ \FROMNAMED\ \varR{ds} }$ and the graph pattern~$\gsup{\SparqlWhereClause}{ng} =
  \SparqlWhereClause\ \cup\ \WHERE\ \set{ \keyword{graph}\ \varR{ds} \set{ \mathtt{[]\ {:}value}\  \var{VarName} } }$.

    \begin{prooftreefunction}
      \AxiomC{$
        \dynEnvX{1}{ng}{\funcCallR{sparqlCall}{
              \begin{array}{l}
                \SELECT~\Vars \cup \set{\var{VarName}} \\
                \gsup{\DatasetClause}{ng}~\gsup{\SparqlWhereClause}{ng}\\
                \SolutionModifier
              \end{array}} \Rightarrow \omg{ng}{}}
        $}
      
      \UnaryInfC{$\dynEnvX{2}{ng}{\grammarRule{Expr}} \Rightarrow \envElem{Res}{}$}
      
      \UnaryInfC{$\dynEnvX{1}{ng}{\begin{array}{l}
            \LET~\varR{results}~:=\\
            \quad\funcCallR{sparqlCall}{
              \begin{array}{l}
                \SELECT~\Vars \cup \set{\var{VarName}} \\
                \gsup{\DatasetClause}{ng}~\gsup{\SparqlWhereClause}{ng}\\
                \SolutionModifier
              \end{array}\!\!} \\ 
            \RETURN~\grammarRule{Expr}
          \end{array} \Rightarrow \envElem{Res}{}}$}
    \end{prooftreefunction}%
  where 
  \begin{small}
    $\dynX{2}{ng} = \dynX{1}{ng}\envExtend{varValue}{\varNameR{results} \Rightarrow \omg{ng}{}}$.
  \end{small}%

  \item[\FOR expression of line (3):]~

    \begin{prooftreefunction}

      \AxiomC{$\dynEnvX{2}{ng}{\varR{results//sr{:}result}} \Rightarrow \sm{i}{}$}
      \UnaryInfC{$\dynEnvX{3}{ng}{\grammarRule{Expr}} \Rightarrow \envElem{Res}{i}$}

      \AxiomC{$\Ddots$}
      
      \BinaryInfC{$\dynEnvX{2}{ng}{
          \begin{array}{l}
            \FOR~\varR{result}~\AT~\varR{result\mathunderscore{}pos}\\
            \qquad\IN~\varR{results//sr{:}result}\\
            \RETURN~\grammarRule{Expr}
          \end{array}  \Rightarrow \envElem{Res}{1},\dotsb,\envElem{Res}{n}}$}
      
    \end{prooftreefunction}%
  where 
  \begin{small}
      $\dynX{3}{ng} = \dynX{2}{ng} \envExtend{varValue}{\begin{array}{l}
        \varNameR{result} \Rightarrow \sm{i}{};\\
        \varNameR{result\mathunderscore{}pos} \Rightarrow i
      \end{array}
    }\enspace .$
\end{small}%

\item[\LET expressions of lines (4)--(5).] Here we consider all the \LET expressions represented by line~(4),
  where~$\var{v} \in \Vars$:

  \begin{prooftreefunction}
      \AxiomC{$\dynEnvX{3}{ng}{\varR{result/sr{:}binding[@name = \var{v}]/*}} \Rightarrow \envElem{V}{}$}

      \UnaryInfC{$\dynEnvX{4}{ng}{\grammarRule{Expr}} \Rightarrow \envElem{Res}{}$}

      \UnaryInfC{$\dynEnvX{3}{ng}{\begin{array}{l}
            \LET~\var{v} := \varR{result/sr{:}binding[@name = \var{v}]/*} \\
            \RETURN~\envElem{\ExprSingle}{2}
          \end{array} \Rightarrow \envElem{Res}{}}$}

    \end{prooftreefunction}%
  where 
  \begin{small}
    $\dynX{4}{ng} = \dynX{3}{ng} \envExtend{varValue}{\envElem{v}{} \Rightarrow \envElem{V}{}}$.
  \end{small}%

  \end{description}

  %
  Let \omg{xs}{in} be the solution sequence returned by the evaluation of the inner \SparqlForClause of \envElem{Q}{}.
  %
  Furthermore let \dynX{i}{expr} be the dynamic environment such that \dynEnvX{i}{expr}{\ExprSingle \Rightarrow
    \envElem{Val}{}}.  \dynX{i}{expr} results from extending \dyn with bindings for the outer variable \var{VarName} and
  with variable bindings from a solution mapping~$\sm{xs}{in} \in \omg{xs}{in}$
  where~$\funcCall{\sm{xs}{in}}{\grammarRule{VarName}} = \var{VarName}$, \ie~the value for the join variable in the
  solution mapping \sm{xs}{in} is the same as assigned to \var{VarName}.


  The new merged dataset, \gsup{\DatasetClause}{ng}, is created based on \DatasetClause and the newly created named
  graph~$\envElem{NG}{}$. Since the URI that identifies the newly created named graph~$\envElem{NG}{}$ is distinct from
  any URI of named graphs present in \DatasetClause, the triples included in~$\envElem{NG}{}$ will never be a solution
  for \SparqlWhereClause, and will be matched only by the graph pattern~``$\WHERE \set{ \keyword{graph}\ \varR{ds} \set{
      \mathtt{[]\ {:}value}\ \var{VarName} } }$''.

  Let~$C$ be the expression context where~$\dyn$ is included,~$\mu_C$ the XSPARQL instance mapping of~$C$
  and~$\gsup{P}{out}$ and~$\gsup{P}{in}$ the graph patterns obtained from replacing the variables
  in~$\SparqlWhereClause$ and~``$\WHERE\set{ \keyword{graph}~\varR{ds} \set{ \mathtt{ []~{:}value}~\var{VarName} } }$''
  according to~$\mu_C$, respectively.

  Furthermore, let~$\omg{ng}{out} = \evalS{\gsup{\DatasetClause}{ng}}{\gsup{P}{out}}$ and~$\omg{ng}{in} =
  \evalS{\gsup{\DatasetClause}{ng}}{\gsup{P}{in}}$.
  %
  According to SPAR\-QL semantics, the pattern solution that results from evaluating~$\SparqlWhereClause$,~$\Omega_{ng}
  = \omg{ng}{out} \bowtie \omg{ng}{in}$ consists of all the solution mappings~$\sm{out}{} \in \omg{ng}{out}$
  and~$\sm{in}{} \in \omg{ng}{in}$ such that~$\sm{out}{}$ and~$\sm{in}{}$ are \emph{compatible}.
  % 
  Similar to the proof of \cref{prop:sr-correct}, we are only considering \ORDERBY solution modifiers, these only change
  the order of the solution sequences and thus can be safely ignored for this proof.

  The evaluation of the outer XQuery \FOR clause (lines~(1)--(2) of \envElem{Q}{}) performed over \dyn is translated, by
  the \ong{\envElem{Q}{}} function, into the \funcNameR{sparqlCall} from line~(2), which is evaluated over \dynX{1}{ng}.
  %
  However, as we can see from~\eqref{tree:dynEvn3-ng}, \dynX{1}{ng} is based on \dyn by adding the value for
  the~$\varNameR{ds}$ variable and, since this variable belongs to the~$\varNameR{}$ reserved namespace, it is not
  allowed to appear in the \SparqlWhereClause and we have that the results of evaluating the \funcNameR{sparqlCall}
  function over \dyn or \dynX{1}{ng} will be the same.
 

  The inner \SparqlForClause (lines~(3)--(4) of \envElem{Q}{}) is evaluated over some dynamic environment \dynX{}{expr},
  is incorporated by the \ong{\envElem{Q}{}} into the \funcNameR{sparqlCall} from line~(2), which is evaluated over
  \dynX{1}{ng}.
  %
  Considering that \dynX{1}{ng} is less restrictive than \dynX{}{expr}, \ie~\dynX{1}{ng} contains less bindings for
  variables than \dynX{}{expr}\!, the evaluation of the inner \SparqlForClause over \dynX{1}{ng} will contain all the
  solution mappings from \omg{xs}{in} and specifically~$\sm{in}{}$.
  % 
  As~$\sm{out}{}$ and~$\sm{in}{}$ are \emph{compatible} we have that~$\dynEnv \funcCall{ng}{expr} \Rightarrow
  \envElem{Val}{}$.

  
  \medskip 
  \noindent ($\Leftarrow$)
  %
  Next we will show that if~$\dynEnv \ong{\envElem{Q}{}}$~$\Rightarrow \envElem{Val}{}$ then~$\dynEnv
  \envElem{Q}{} \Rightarrow \envElem{Val}{}$.
  %
  Let us turn to the evaluation of~$\dynEnv \envElem{Q}{} \Rightarrow \envElem{Val}{}$. 
  %
  \begin{description}
    % 
  \item[XQuery \FOR clause from lines (1)--(2).] Here \grammarRule{Expr} corresponds to the \SparqlForClause from
    lines~(3)--(4) of \envElem{Q}{}.
    %

      \begin{prooftreefunction}
      \AxiomC{$\dyn.\ecomp{globalPosition} = \seq{ \envElem{Pos}{1}, \cdots, \envElem{Pos}{m} } $}
      \UnaryInfC{$\dynEnv{\envElem{\grammarRule{ExprSingle}}{1} \Rightarrow \envElem{V}{i}}$}
        \UnaryInfC{$\dynX{i}{xs} \proofs \grammarRule{Expr} \Rightarrow \envElem{Value}{i}$}

        \AxiomC{$\Ddots$}

        \BinaryInfC{$\dynEnv{\begin{array}{l}
              \FOR~\var{VarName}~\grammarRule{OptTypeDeclaration}\\
              \quad\grammarRule{OptPositionalVar}~\IN~\envElem{\grammarRule{ExprSingle}}{1} \\
              \RETURN~\grammarRule{Expr}
            \end{array} \Rightarrow \envElem{Value}{i},\dots,\envElem{Value}{n}}$}

      \end{prooftreefunction}%
    % 
    we have for each~$\envElem{V}{i}$:
    \begin{tree}
      \ \dynX{i}{xs} = 
      \dyn \envExtend{globalPosition}{\seq{ \envElem{Pos}{1}, \cdots, \envElem{Pos}{m}, i }} \envExtend{varValue}{
        \e{VarName}{}{} \Rightarrow \envElem{V}{i}
      } \enspace .
      \label{proof:dynenv_i_ng}
    \end{tree}%

  \item[\SparqlForClause of lines (2)--(4):]~ 
    %
      \begin{prooftreefunction}
      \AxiomC{$\dyn.\ecomp{globalPosition} = \seq{ \envElem{Pos}{1}, \cdots, \envElem{Pos}{m} } $}
      \UnaryInfC{$ \dynEnvX{i}{xs}{
              \funcCall{fs{:}dataset}{\grammarRule{\DatasetClause}} \Rightarrow \grammarRule{DS}
          }$}
        \UnaryInfC{$\dynEnvX{i}{xs}{
              \funcCall{fs{:}sparql}{\!\!\begin{array}{l}
                  \grammarRule{DS},  \SparqlWhereClause, \\
                  \SolutionModifier
                \end{array}\!\!}  \Rightarrow \sm{j}{}}$}
        \UnaryInfC{$\dynX{j}{xs} \proofs \envElem{\ExprSingle}{2} \Rightarrow \envElem{Value}{j}$}

        \AxiomC{$\!\!\!\Ddots$}

        \BinaryInfC{$\dynEnvX{i}{xs}{\begin{array}{l}
            \FOR~\Vars~\DatasetClause\\
            \SparqlWhereClause~\SolutionModifier\\
            \RETURN~\envElem{\ExprSingle}{2}
          \end{array}}~\Rightarrow \envElem{Value}{1} \dotsb \envElem{Value}{m}$}
      \end{prooftreefunction}%
    % 
    where, considering~$\Vars = \var{Var_1}\dots\var{Var_n}$, we have for each~$\sm{j}{}$:
    \begin{small}
      \begin{equation*}
        \ \ \dynX{j}{xs} = \begin{array}{l}
          \dynX{i}{xs} \envExtend{activeDataset}{\grammarRule{DS}}\envExtend{globalPosition}{\seq{ \envElem{Pos}{1}, \cdots, \envElem{Pos}{m}, j }}\\
          \envExtend{varValue}{\begin{array}{l}
              \e{Var}{1}{} \Rightarrow \funcCall{fs{:}value}{\sm{j}{}, \e{Var}{1}{}}; \\
              \dotsb;\\
              \e{Var}{n}{} \Rightarrow \funcCall{fs{:}value}{\sm{j}{},\e{Var}{n}{}}
            \end{array}}
        \end{array}\enspace .
      \end{equation*}
    \end{small}%
  \end{description}

  As we have seen in the ($\Rightarrow$) direction, we have that~$\omg{ng}{out} = \omg{xs}{out}$ and so we have
  that~$\sm{ng}{out} \in \omg{xs}{out}$.
  % 
  Let~$\omg{ng}{out}$ and~$\omg{ng}{in}$ be the solution sequences returned by the evaluation of the new
  \gsup{\SparqlWhereClause}{ng} and \SparqlWhereClause, respectively.
  % 
  As we have seen~$\Omega_{ng}$ contains all the solution mappings~$\sm{}{} = \sm{ng}{out} \bowtie \sm{ng}{in}$,
  where~$\sm{ng}{out} \in \omg{ng}{out}$ and~$\sm{ng}{in} \in \omg{ng}{in}$, such that~$\sm{ng}{out}$ and~$\sm{ng}{in}$
  are \emph{compatible}.
  % 
  Again, consider~$\sm{ng}{out}$ and~$\sm{ng}{in}$ the pattern solutions where~$\envElem{Val}{}$ is deduced from.


  Let~$C$ be the expression context where~$\dyn$ is included and~$\mu_C$ the XSPARQL instance mapping of~$C$.
  Furthermore let~$\gsup{P}{in}$ be the graph pattern obtained from replacing the variables
  in~$\gsup{\SparqlWhereClause}{in}$ according to~$\mu_C$.
  % 
  Since we know that~$\vars{\gsup{\SparqlWhereClause}{in}} \subseteq \vars{\gsup{P}{in}}$, all solutions mappings
  returned by evaluating \SparqlWhereClause\gsup{\!\!\!}{in} under XSPARQL semantics are included in the pattern
  solution of evaluating~$\e{P}{}{in}$ under SPARQL semantics \ie~$\omg{xs}{in} \preceq \omg{ng}{in}$.
  % 
  We obtain two cases:
  \begin{inparaenum}[(i)]
  \item\label{en:inxsin_ng}~$\sm{ng}{in} \in \omg{xs}{in}$; or
  \item\label{en:notinxsin_ng}~$\sm{ng}{in} \not\in \omg{xs}{in}$. 
  \end{inparaenum}
  %
  In \eqref{en:inxsin_ng} we immediately get that~$\dynEnv \envElem{Q}{} \Rightarrow \envElem{Val}{}$.
  %
  For~\eqref{en:notinxsin_ng}, consider \sm{C_1}{xs} the XSPARQL instance of the inner \SparqlForClause (created based
  on \dynX{1}{xs}). As we can see from~\eqref{proof:dynenv_i_ng}, \dynX{1}{xs} (and thus also \sm{C_1}{xs}) includes the
  bindings for variables from each solution mapping~$\sm{i}{} \in \omg{xs}{out}$.  Thus, according to the XSPARQL \ac{BGP}
  matching (\cf~\cref{def:xsparql-bgp-matching}), \omg{xs}{in} will contain all the solution mappings that are
  compatible with any solution mapping~$\sm{i}{} \in \omg{xs}{out}$ and specifically those compatible with \sm{ng}{out}.
  Since we know that \sm{ng}{in} is compatible with \sm{ng}{out}, we have that \sm{ng}{in} must belong to \omg{xs}{in},
  thus we can deduce that~$\dynEnv \envElem{Q}{} \Rightarrow \envElem{Val}{}$. 
  % 
  \paragraph*{Inductive Step.}
  % 
  Let us assume that, for some arbitrary \dynX{i}{}, $\dynEnvX{i}{} \envElem{\ExprSingle}{1} \Rightarrow
  \envElem{Val}{i}$ if and only if $\dynEnvX{i}{} \funcCall{opt_{ng}}{\envElem{\ExprSingle}{1}} \Rightarrow
  \envElem{Val}{i}$.
  % 
  According to the \funcName{opt_{ng}} rewriting, there must exist a \dynX{j}{} that is the extension of \dynX{i}{}
  with \envElem{Val}{i} and thus
  $\dynEnvX{j}{} \envElem{\ExprSingle}{2} \Rightarrow \envElem{Val}{}$ if and only if $\dynEnvX{j}{}
  \funcCall{opt_{ng}}{\envElem{\ExprSingle}{2}} \Rightarrow \envElem{Val}{}$.  Consequently, we have that $\dynEnv Q
  \Rightarrow \envElem{Val}{}$ if and only if $\dynEnv \funcCall{opt_{ng}}{Q} \Rightarrow \envElem{Val}{}$.
  % 
\end{proof}



%%% Local Variables:
%%% fill-column: 120
%%% TeX-master: "../thesis" 
%%% TeX-PDF-mode: t
%%% TeX-debug-bad-boxes: t
%%% TeX-parse-self: t
%%% TeX-auto-save: t
%%% reftex-plug-into-AUCTeX: t
%%% End:
