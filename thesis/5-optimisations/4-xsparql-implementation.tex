\section{Implementation}
\label{sec:implementation}

\begin{figure}[t]
  \centering
  
  \begin{tikzpicture}[
    terminal/.style={ rectangle,minimum size=6mm,rounded corners=3mm,text centered,draw=black!50,fill=gray!20},%
    rdb/.style={cylinder, shape border rotate=90, draw,minimum width=1.5cm,minimum height=1cm, shape aspect=.35}%
    ]
    \tikzstyle{every node}=[font=\footnotesize]


    % XSPARQL engine
    \begin{scope}[node distance = 20pt]
      \node[terminal,text width=1.7cm] (rewriter) {Query Rewriter}; %
      \node[right=of rewriter,doc=1cm] (rewritenXQ) {XQuery query};%
      \node[right=of rewritenXQ, terminal,text width=1.7cm] (EXQuery) {Enhanced XQuery engine}; %
    \end{scope}


    % XSPARQL query
    \node[above=2.5em of rewriter, doc=1.4cm] (XS-Q) {XSPARQL query};%

    \coordinate[below=1.5em of EXQuery.south] (input-separator);
    % input sources
    \begin{scope}[node distance = 7pt]
      \documentSet{below=2.5em of input-separator}{xml}{XML / JSON};
      \documentSet{right=of xml}{rdf}{RDF};
      \node[left=of xml, rdb]  (rdb)  {RDB};%
    \end{scope}
  
    \draw[-latex] (input-separator) -- (EXQuery.south);%
    \draw[-latex] (input-separator) -- (xml.north) node [midway,inner sep=0,fill=white] {XQuery};%
    \draw[-latex] (input-separator) -| (rdb.north) node [near end,inner sep=0,fill=white] {SQL};%
    \draw[-latex] (input-separator) -| (rdf.north) node [near end, inner sep=0,fill=white] {SPARQL};%

    % outputs
    \begin{scope}[node distance = 2pt]
      \node[above=2em of EXQuery, doc=1cm] (output) {XML~/ RDF};%
    \end{scope}
  
    % inner diagram arrows
    \draw[-latex] (rewriter.east) -- (rewritenXQ.west);%
    \draw[-latex] (rewritenXQ.east) -- (EXQuery.west);%
    
    % outer box
    \node[draw=black,dotted,thick,inner sep=10pt,rectangle,fit=(rewriter) (EXQuery)] (XSPARQL) {} ;%
    \node[inner sep=0,fill=white] at (XSPARQL.north) {XSPARQL};

    % outer diagram arrows
    \draw[-latex] (XS-Q.south) -- (rewriter.north);%

    % output lines
    \draw[-latex] (EXQuery.north) -- (output.south) ;%

  \end{tikzpicture}

%%% Local Variables:
%%% mode: latex
%%% mode: flyspell
%%% mode: reftex
%%% TeX-master: "../thesis"
%%% End:


  \caption{XSPARQL implementation architecture}
  \label{fig:arch}
\end{figure}
%
In this section we present our prototype implementation of the XSPARQL language, which translates an XSPARQL query into
an XQuery query with interleaved calls to a relational database and/or a SPARQL engine.  The architecture of our
implementation is shown in \cref{fig:arch} and consists of the following main components:
%
\begin{inparaenum}[(1)]
\item\label{item:arch-1} a query rewriter, which turns an XSPARQL query into an XQuery; and
\item\label{item:arch-2} an (enhanced) XQuery engine for evaluating the rewritten XQuery.
\end{inparaenum}
%
This enhanced XQuery engine relies on a \ac{SQL} relational database and on a SPARQL engine, for accessing the
heterogeneous data sources from within the rewritten XQuery.

 
We implement the XSPARQL language syntax and query rewriter by using the ANTLR parser
generator,\footnote{\url{http://www.antlr.org/}, retrieved on 2012/07/17.}
%
which produces an XQuery with calls to the \ac{SQL} and SPARQL engines.
%
For the XQuery engine we use Saxon\footnote{\url{http://saxon.sourceforge.net/}, retrieved on 2012/07/17.} and use the
ARQ SPARQL engine\footnote{\url{http://jena.sourceforge.net/ARQ/}, retrieved on 2012/07/17.} for querying the \ac{RDF}
data.
%
As for accessing relational databases, we rely on a JDBC interface to the relational database and we have tested the
connection to MySQL,\footnote{\url{http://www.mysql.com/}, retrieved on 2012/07/17.}
PostgreSQL,\footnote{\url{http://www.postgresql.org/}, retrieved on 2012/07/17.} and Microsoft SQL
Server.\footnote{\url{http://www.microsoft.com/sqlserver/}, retrieved on 2012/07/17.}
%
This interface between the different engines is implemented using the Saxon Extension API, which allows to create custom
XQuery functions associated with Java methods.
%
The functions that the rewritten queries use to access the relational and \ac{RDF} data are called \funcNameR{sqlCall}
and \funcNameR{sparqlCall}, which translate a \SQLForClause or a \SparqlForClause into a \ac{SQL} or SPARQL \SELECT
query respectively, and evaluate it, returning the results according to the types presented in
\cref{sec:xsparql-types}.\footnote{In the produced XQuery expressions we assume the reserved namespace prefix~$\prefixR$
  associated with~\url{http://xsparql.deri.org/demo/xquery/xsparql.xquery}. 
  % 
  This prefix is not allowed in an XSPARQL query and is used not only as the namespace for the XQuery functions
  \funcNameR{sqlCall} and \funcNameR{sparqlCall} but also as the namespace for any auxiliary variables introduced by the
  rewriting, effectively avoiding clashes with variables from the XSPARQL query.  }
%
However, instead of implementing all the newly introduced types as custom types in XQuery, we reuse the XML Schema of
the SPARQL Query Results XML Format,\footnote{See {\url{http://www.w3.org/2007/SPARQL/result.xsd}}, we assume this
  schema is associated with the namespace prefix~$\mathtt{sr}$.}  where the~\qname{sr}{binding} type corresponds
directly to XSPARQL's \type{RDFTerm} type.
%
An \type{RDFGraph}, \eg~the result of a \ConstructClause, is serialised using Turtle syntax by building the output as
\type{\qname{xs}{string}}.  The remaining types \type{RDFDataset} and \type{RDFNamedGraph} are adapted accordingly.


A more general form of using a SPARQL engine would be to rely on a SPARQL endpoint, as presented in the initial XSPARQL
prototype described by~\citet{AkhtarKopeckyKrennwallner:2008aa}.
%
However, by using Saxon's extension mechanism the query engines are more tightly integrated and allow for a more
efficient communication of results (opposed to using a SPARQL endpoint via HTTP).
%
As we will describe later in \cref{sec:sparqlforclause-imple}, we can still rely on this feature if necessary, and we
use it for the implementation of the remote endpoint feature that allows us to mimic SPARQL~1.1 \keyword{SERVICE}
feature.



\paragraph*{Blank Node Matching in Nested Queries}
\label{sec:blank-node-matching-nested}
%
The \funcNameR{sparqlCall} function also implements the \emph{matching blank nodes in nested queries} feature (as
described in \cref{sec:bgpmatch}), for which we rely on custom Java code that uses the ARQ API to preserve blank node
labels in consecutive SPARQL calls over the same dataset.
%
The custom Java code maintains a \emph{stack} of the previously used datasets during the query execution: upon the
execution of a \SparqlForClause with a \DatasetClause, the code stores the blank node identifiers in the dataset and
when executing a \SparqlForClause without an explicit \DatasetClause, we use the first element of the stack as the
implicit dataset along with its existing blank node identifiers.


\paragraph*{Creating Distinct Blank Nodes in \ConstructClause{s}}
\label{sec:blank-node-construct}
%
In the XSPARQL semantics (\cref{sec:constructsem}), we use the new \ecomp{globalPosition} dynamic environment
component to cater for creating fresh blank node identifiers for each instantiation of the \ConstructClause.
%
In our implementation we rely on \emph{position variables} in XQuery \FOR expressions~(\cf~\cref{sec:xquery}) for
generating the distinct identifiers.
%
In the query rewriting step we normalise all the \FOR expressions to include a position variable and also keep a list of
all previous position variables in the query.  Hence, when a blank node is found in a \ConstructClause, we can generate
the blank node label based on the label provided in the query and the values of the existing position variables.




\bigskip

Next we present how \SQLForClause{s}, \SparqlForClause{s} and \ConstructClause{s} are processed by using what we call
\emph{rewriting functions}, which operate on syntactic objects of XSPARQL and return an XQuery expression.


\subsection{\SQLForClause and \SparqlForClause}
\label{sec:sparqlforclause-imple}
%
Our implementation defers the \ac{SQL} and SPARQL query fragments to the respective external engines and extracts the
bindings for the existing variables from the returned XML results document. 
%
For the definitions of the rewriting functions, let $\mathbb{XS}$ and $\mathbb{XQ}$ denote the set of all XSPARQL and
XQuery core expressions, respectively.  The rewriting function~$\funcName{tr} \colon \mathbb{XS} \to \mathbb{XQ}$
details our translation from XSPARQL to XQuery core.
%
We now describe the rewriting function for the translation of \SparqlForClause{s}, given an XSPARQL expression
$\envElem{Q}{}$ of form
%
\begin{queryF}
  \[
  \begin{array}{l}
    \FOR~\Vars~\DatasetClause\\
    \SparqlWhereClause~\SolutionModifier\\
    \ReturnExpr
  \end{array}
  \]
\label{eq:expr_tr}
\end{queryF}%
%
\noindent
then $\funcCall{tr}{\envElem{Q}{}}$ is defined as the XQuery Core expression
%
\begin{small}
\begin{equation*}
  \begin{array}{l}
    \funcCall{tr}{\envElem{Q}{}} =  \\
    \mathrm{(1)~~}\LET~\varR{results}~\text{\texttt{:=}}~\funcCallR{sparqlCall}{\begin{array}{l}
        \SELECT~\Vars~\DatasetClause\\
        \SparqlWhereClause~\SolutionModifier
      \end{array}
    }~\RETURN\\
    \mathrm{(2)~~}\FOR~\varR{result}~\AT~\varR{posvar}~\IN~\varR{results//sr{:}result}~\RETURN \\
    \mathrm{(3)~~}\LET~\var{v} := \varR{result/sr{:}binding[@name = \varName{v}]/*}~\RETURN \hfill \qquad\qquad\qquad\qquad\qquad \textrm{\smaller for each~$\var{v} \in \Vars$}\\ 
    \mathrm{(4)~~}\grammarRule{ExprSingle} 
  \end{array}
\end{equation*}
\end{small}%
%
That is, we implement the $\funcName{fs{:}sparql}$ formal semantics function by translating $\envElem{Q}{}$ into a
SPARQL \SELECT query, which is then executed by the custom runtime function \funcNameR{sparql\-Call}.  that returns the
result in SPARQL's XML result format.  This is represented in line~(1) of the rewritten query.
%
The \FOR expression in line~(2) selects all solutions from the \ac{XML} representation of the query results, while the
\LET expressions in line~(3) assign the result value of each variable to XSPARQL variables.  Finally, the \RETURN
expression~\grammarRule{ExprSingle} of line~(4) is evaluated with the new variables available.

For XSPARQL \SQLForClause{s} of the form
%
\begin{queryF}
  \[
  \begin{array}{l}%
    \FOR~\envElem{AttrSpec}{1}~\keyword{as}~\envElem{\var{Var}}{1}, \dots, \envElem{AttrSpec}{n}~\keyword{as}~\envElem{\var{Var}}{n}\\
    \grammarRule{RelationList}~\grammarRule{SQLWhereClause}\\
    \ReturnExpr
  \end{array} 
  \]
  \label{eq:expr_tr_sql}
\end{queryF}%
%
\noindent
then $\funcCall{tr}{\envElem{Q}{}}$ is defined as the XQuery Core expression
%
\begin{small}
\begin{equation*}
  \begin{array}{l}
    \funcCall{tr}{\envElem{Q}{}} =  \\
    \mathrm{(1)~~}\LET~\varR{results}~\text{\texttt{:=}}~\funcCallR{sqlCall}{\begin{array}{l}
        \SELECT~\envElem{AttrSpec}{1}, \dots, \envElem{AttrSpec}{n}\\
        \grammarRule{RelationList}~\grammarRule{SQLWhereClause}
      \end{array}
    }~\RETURN\\
    \mathrm{(2)~~}\FOR~\varR{result}~\AT~\varR{posvar}~\IN~\varR{results//sr{:}result}~\RETURN \\
    \mathrm{(3)~~}\LET~\envElem{\var{Var}}{i} := \hfill \qquad\qquad\qquad \textrm{ for each } \envElem{AttrSpec}{i}, \var{Var}{i} \in \envElem{AttrSpec}{1}~\keyword{as}~\envElem{\var{Var}}{1}, \dots, \envElem{AttrSpec}{n}~\keyword{as}~\envElem{\var{Var}}{n}\\ 
    \qquad\qquad \varR{result/sr{:}binding[@name = \envElem{AttrSpec}{i}]/*}~\RETURN \\
    \mathrm{(4)~~}\grammarRule{ExprSingle} 
  \end{array}
\end{equation*}
\end{small}%
%
Furthermore, in case the XSPARQL specifies the attribute selection as a \lit{FOR *} the translation function requires
access to the input relational database during the rewriting in order to determine the relation attributes and the names
of the XSPARQL variables to be generated.


\subsubsection*{Implementation of the XSPARQL Semantics}
\label{sec:impl-xsparql-semant}

The presented~\funcNameR{sparqlCall} function also implements XSPARQL's \ac{BGP} matching, as described in
\cref{sec:semantics}, by replacing any previously assigned variables in the SPARQL query with their current value
according to the rules presented in \cref{def:xml2rdfterm}.
%
This behaviour implements XSPARQL's \ac{BGP} matching while relying on an off-the-shelf SPARQL engine.  In
\cref{sec:soundn-compl} we present the formal correspondence between the variable replacement approach and
XSPARQL's semantics.
%
This replacement of variables can be statically determined during the query rewriting step and generate the respective
\ac{SQL} or SPARQL query string (the parameters to the~\funcNameR{sqlCall} and~\funcNameR{sparqlCall} functions,
respectively) by using XQuery's \concat function.  The \concat allows for an arbitrary number of arguments and when
executed concatenates the string value resulting from the evaluation of each argument.
%
When parsing a \SparqlForClause we have access to the set of previously declared variables and, whenever we find a
variable it is possible to determine whether to replace it by its previously assigned value or keep it as a variable.
%
If the variable has been previously declared, the variable name is inserted as an argument of the \concat function,
which upon evaluation accesses the value of the variable and use it in the creation of the \SELECT query.
%
If the variable is fresh, \ie~has not been declared before, we leave the variable name as a (quoted) string within the
\concat, which effectively postpones the evaluation of the variable to the SPARQL engine.
%

\begin{example}[\SELECT query generation]
  %
  Consider the following simple XSPARQL query:
  % 
  \lstinputlisting[frame=none,numbers=none]{0-data+queries/example-select.xsparql}
  %
  The rewritten code that generates the SPARQL \SELECT query is as follows:
  % 
  \lstinputlisting[frame=none,numbers=none]{0-data+queries/example-select-rewritten.xquery}
  %
  Note that the \lit{for *} has also been replaced to select only the \emph{unbound} variables in the
  \synt{WhereClause}.
\end{example}
%
For \SQLForClause{s} we follow a similar approach but since \ac{SQL} does not allow for \var{}-prefixed variable names,
we always leave the variable name unquoted, which means that for \SQLForClause{s} all variables used in
\synt{SQLWhereClauses} must be previously bound.


\paragraph*{Querying External SPARQL Endpoints}
\label{sec:external-endpoint}
%
Since our implementation of XSPARQL rewrites \SparqlForClause{s} into SPARQL queries, we can execute the rewritten
SPARQL query in different ways: the typical way is to use a local instance of the ARQ engine to execute the query.
%
One of the new features of SPARQL~1.1 (presented in \cref{sec:sparql-11}) is the \keyword{SERVICE} keyword, which
specifies that the following subquery will be executed in a remote SPARQL endpoint.  
%
In XSPARQL we can also enable this behaviour by specifying the \lit{endpoint} \ac{URI} in a \SparqlForClause, after the
\DatasetClause.  This instructs the XSPARQL engine to use the remote endpoint specified by \ac{URI} for executing the
query and incorporate the bindings into the query as usual.
%
As opposed to SPARQL~1.1, which does not allow to inject bindings of results into the \keyword{SERVICE}
subquery,\footnote{Please note that the \keyword{BINDINGS} can only take fixed values for variables, preventing to use
  results from the execution of other parts of the query} our variable replacement operation allows to inject values
from the outer query into the inner query.
%
This feature allows to write queries that are otherwise unavailable or impractical in SPARQL, either by design
restrictions of the language or practical restrictions of SPARQL endpoints (as illustrated in the following example).
%
\begin{query}[t]
  \lstinputlisting[firstline=2]{0-data+queries/birthday-marco.xsparql}
  \caption{Querying a remote endpoint with XSPARQL}
  \label{qr:endpoint-xsparql}
\end{query}
%
\begin{query}[t]
  \lstinputlisting{0-data+queries/birthday-marco2.sparql}
  \caption{Querying a remote endpoint with SPARQL}
  \label{qr:service-sparql}
\end{query}
%
\begin{example}[Querying Remote SPARQL Endpoints]
  Consider as an example we want to retrieve from DBPedia persons that have the same birthday as Marco Hietala.  For
  this we first need to retrieve Marco's birthday (we are taking this information from DBPedia but we could rather use
  the artists personal \ac{FOAF} file) and then retrieve from DBPedia the persons with the same birthday.
  % 
  \cref{qr:endpoint-xsparql,qr:service-sparql} present possible XSPARQL and SPARQL versions of this query, respectively.
  These queries both involve querying the remote DBPedia SPARQL endpoint.
  %
  The SPARQL version (\cref{qr:service-sparql}) quickly runs into limits imposed by the DBPedia SPARQL endpoint, since
  the \keyword{SERVICE} nested query attempts to retrieve the all persons that have any birthday specified, due to not
  being aware of the bindings of variable \var{MB}.
  %
  On the other hand, in the XSPARQL version in \cref{qr:endpoint-xsparql} the inner \keyword{endpoint} query only
  retrieves the required birthdays.
  %
\end{example}


\subsection{\ConstructClause}
\label{sec:constructclause}
%
\begin{figure}[t]
  \subfloat[\funcNameR{rdfTerm} function]{\label{ex:rdfTerm}
    \begin{minipage}{.55\linewidth}
      \lstinputlisting[numbers=none]{0-data+queries/rdfTerm.xquery}
    \end{minipage}
  }
  % 
  \subfloat[\funcNameR{validTriple} function]{\label{ex:validtriple}
    \begin{minipage}{.45\linewidth}
      \lstinputlisting[numbers=none]{0-data+queries/validTriple.xquery}
    \end{minipage}
  }
  \caption{Implementation functions example}
\end{figure}
%
As for the construction of \ac{RDF} graphs (\ie~whenever the~\grammarRule{ReturnClause} is a \ConstructClause), our
implementation XQuery rewriting produces a string in Turtle syntax, where we ensure that each generated RDF triple is
syntactically valid.
%
For this we rely on a number of additional auxiliary XQuery functions: firstly, the
function~\funcCallR{rdfTerm}{\var{VarName}} (presented in \cref{ex:rdfTerm}), when given a variable of type
\type{RDFTerm}, returns the correctly formatted \ac{RDF} term (according to the Turtle syntax) of \var{VarName}.
%
Next, the~\funcNameR{validTriple} presented in \cref{ex:validtriple} implements the semantics
function~\funcName{fs{:}validTriple} by calling the~\funcNameR{rdfTerm} function to correctly format triples to Turtle
syntax. 
%
This function further uses the auxiliary functions \funcNameR{validSubject}, \funcNameR{validPredi\-cate} and
\funcNameR{validObject} that determine, according to the \ac{RDF} semantics, if their argument is a valid subject,
predicate, or object, respectively.


Our implementation of the~\funcName{fs{:}skolemConstant} function, that ensures blank nodes in \CONSTRUCT expressions
are distinct between different solutions, consists of appending the position variables from all the surrounding \FOR
expressions to the respective blank node identifier using ``\_'' as a separator.  This is represented by the following
rewriting function
%
\begin{small}
\begin{equation*}
\begin{array}{l}
  \funcCall{tr_{sk}}{\var{BNodeName},\set{\var{PosVar_1}, \cdots,\var{PosVar_n}}} = \\
  \qquad\funcCall{\concat}{
    \texttt{"}\mathunderscore{:}\texttt{"},\var{BNodeName},\texttt{"}\mathunderscore\texttt{"},\var{PosVar_1}, \cdots, \texttt{"}\mathunderscore\texttt{"}, \var{PosVar_n} 
  } \enspace .
\end{array} 
\end{equation*}
\end{small}%
%
Finally, the function~\funcNameR{evalCT} implements~\funcName{fs{:}evalCT} by simply concatenating all the triples
generated by the~\funcNameR{validTriple} function to a string representation of the RDF graph to be constructed.


\paragraph*{Implementation of Constructed Datasets}
\label{sec:constructed-dataset}
%
As described in \cref{sec:syntax}, it is possible to assign the result of a \CONSTRUCT query to an XSPARQL
variable, which can later be used in the \DatasetClause of a \SparqlForClause.
%
In order to make this constructed graph available to the ARQ SPARQL engine, we need to materialise it as a temporary
file and specify this temporary file's location in the SPARQL query.
%
To enable this feature, during the query rewriting step, whenever we find a \ConstructClause assigned to an XSPARQL
variable, we create a temporary \ac{RDF} file with the result of the \CONSTRUCT expression represented as Turtle and
assign the local path of this generated file to the XSPARQL variable.


\subsection{Soundness \& Completeness of the Implementation}
\label{sec:soundn-compl}
%
We next present the equivalence between our implementation of the XSPARQL language and the XSPARQL semantics presented
in \cref{sec:semantics}.  
%
We start by presenting a lemma stating that the results of the evaluation of a \ac{BGP}~$P$ under XSPARQL \ac{BGP}
matching semantics can be determined based on the results of evaluating~$\funcCall{\mu}{P}$
(\cf~\cref{def:variable-substitution}) under SPARQL semantics.  Similar correspondence for \SQLForClause{s} is
presented was \cref{lemma:xsparql-relation-instance}.
%
\begin{lemma}
  \label{lem:replacement-equivalence}
  Let~$P$ be a BGP,~$D$ a dataset and~$\mu$ the XSPARQL instance mapping of $P$.  Considering~$P' = \funcCall{\mu}{P}$,
  we have that~$\evalXS{D}{P}{\mu} = \evalS{D}{P'} \bowtie \set{\mu}$.
\end{lemma}
%
\begin{proof}
  % 
  Since, according to the variable substitution operation we have that $\vars{P'} = \vars{P} \setminus \dom{\mu}$, we
  also have that $\vars{P'} \cap \dom{\mu} = \emptyset$ and it follows directly from
  \cref{lemma:xsparql-instance-mapping} that $\evalXS{D}{P}{\mu} = \evalS{D}{P'} \bowtie \set{\mu}$.
  %
\end{proof}

The following result presents the equivalence of our implementation function $\funcName{tr}$ and the XSPARQL
semantics.\footnote{Please note that, for presentation purposes, we are omitting the initial \emph{empty line} in case
  the proof trees require no premises and the variable expansion premises.}
%
\begin{proposition}
  \label{prop:sparqlCall}
  Let \envElem{Q}{} be a \SparqlForClause of form~\eqref{eq:expr_tr} or a \SQLForClause of form~\eqref{eq:expr_tr_sql}
  and \dyn the dynamic environment of \envElem{Q}{}, then $\dynEnv \envElem{Q}{} \Rightarrow \envElem{Val}{}$ if and
  only if $\dynEnv \funcCall{tr}{\envElem{Q}{}} \Rightarrow \envElem{Val}{}.$
\end{proposition}


\begin{proof} 
  % 
  We present here only the proof for \SparqlForClause{s} of form~\eqref{eq:expr_tr}, the proof for \SQLForClause{s} is
  analogous.
  
  \noindent ($\Leftarrow$) 
  % 
  Let us show that if~$\dynEnv \funcCall{tr}{\envElem{Q}{}} \Rightarrow \envElem{Val}{}$ then~$\dynEnv \envElem{Q}{}
  \Rightarrow \envElem{Val}{}$.
  % 
  The evaluation of~$\envElem{Q}{}$ consists of the application of Rule~\eqref{eq:sparqlForClause} as 
  % 
    \begin{prooftreefunction}
      \AxiomC{$\dyn.\ecomp{globalPosition} = \seq{ \envElem{Pos}{1}, \cdots, \envElem{Pos}{j} } $}
      % 
      \UnaryInfC{$ \dynEnv{
          \funcCall{fs{:}dataset}{\DatasetClause}
          \Rightarrow \envElem{Dataset}{}
        }$}
      \UnaryInfC{$\dynEnv{
            \funcCall{fs{:}sparql}{
              \begin{array}{l}
                \envElem{Dataset}{}, \SparqlWhereClause, \\
                \SolutionModifier 
              \end{array}}
            \Rightarrow \sm{i}{xs}
        } $}
      \UnaryInfC{$\dynEnvX{1}{xs}{\gsup{ExprSingle}{} \Rightarrow \envElem{Value}{i}}$}

      \AxiomC{$\Ddots$}

      \BinaryInfC{$\dynEnv{\begin{array}{l}
            \FOR~\envElem{\var{Var}}{1}\dotsb\envElem{\var{Var}}{n}~\DatasetClause\\ 
            \SparqlWhereClause~\SolutionModifier\\
            \ReturnExpr
          \end{array} \Rightarrow \envElem{Value}{1} \dotsb \envElem{Value}{m}
        }$}
    \end{prooftreefunction}%
  % 
  where, for each~$\sm{i}{xs}$,
  %
  \begin{tree}
    \begin{tabular}{r@{\hspace{-0.1em}}l@{\hspace{-0.1em}}}
      \dynX{1}{xs} &~= \dyn\envExtend{activeDataset}{\envElem{Dataset}{}} \envExtend{globalPosition}{\seq{ \envElem{Pos}{1}, \cdots, \envElem{Pos}{j}, i }}\\
      &\envExtend{varValue}{\begin{array}{l}
          \envElem{Var}{1} \Rightarrow \funcCall{fs{:}value}{\sm{i}{xs}, \envElem{Var}{1}}; \\
          \dotsb;\\
          \envElem{Var}{n} \Rightarrow \funcCall{fs{:}value}{\sm{i}{xs},\envElem{Var}{n}}
        \end{array}
      }  \enspace .
    \end{tabular}
    \label{proof:sparqlForClause_dynenv_i}
  \end{tree}%
  %
  Let~$\sm{C}{}$ be the XSPARQL instance mapping of the expression context that includes \dyn and~$\omg{tr}{}$ the pattern
  solution resulting from the evaluation of the \funcNameR{sparqlCall} function, \ie~$\omg{tr}{} =
  \evalS{\DatasetClause}{P}$, where~$P$ is the rewriting of~$\SparqlWhereClause$ according to~$\sm{C}{}$.
  % 
  Furthermore, let~$\sm{i}{} \in \omg{tr}{}$ be the solution mapping from which~$\envElem{Val}{}$ is generated,
  \ie~there exists some dynamic environment~$\dynX{}{tr}$ based on \dyn and extended with the variable bindings
  from~$\sm{i}{}$ such that~$\dynEnvX{}{tr}{\ExprSingle} \Rightarrow \envElem{Val}{}$.

  Consider~$\omg{xs}{} = \evalXS{\DatasetClause}{\SparqlWhereClause}{\sm{C}{}}$ as the solution sequence resulting from
  the evaluation of the~$\funcName{fs{:}sparql}$ function.  As we know from
  \cref{lem:replacement-equivalence},~$\omg{xs}{} = \omg{tr}{} \bowtie \set{\sm{C}{}}$ and thus there must exist a
  solution mapping~$\sm{xs}{} \in \omg{xs}{}$ such that~$\sm{xs}{} = \sm{i}{} \bowtie \sm{C}{}$.
  % 
  From~\eqref{proof:sparqlForClause_dynenv_i} we know that there exists a dynamic environment~$\dynX{}{xs}$ that results
  from extending~$\dyn$ with the variable bindings from~$\sm{xs}{}$ and thus this environment will also contain all the
  variable mappings from~$\sm{i}{}$ (and from~\dynX{}{tr},  respectively).  Since we know that~$\dynEnvX{}{tr}{\ExprSingle
    \Rightarrow \envElem{Val}{}}$, we also have that~$\dynEnvX{}{xs}{\ExprSingle \Rightarrow \envElem{Val}{}}$ and
  thus~$\dynEnv{\envElem{Q}{} \Rightarrow \envElem{Val}{}}$. 


  \medskip
  % 
  \noindent ($\Rightarrow$) 
  % 
  Next we will show that if~$\dynEnv \envElem{Q}{} \Rightarrow \envElem{Val}{}$ then~$\dynEnv
  \funcCall{tr}{\envElem{Q}{}} \Rightarrow \envElem{Val}{}$.
  % 
  We present the proof tree for each of the XQuery core expressions in the~$\funcCall{tr}{\envElem{Q}{}}$
  rewriting.  The proof trees are presented for each line of the~$\funcCall{tr}{\envElem{Q}{}}$ rewriting and, in each
  proof tree, \grammarRule{Expr} corresponds to the XQuery expressions of the following lines.
  %
  \begin{description}
  \item[\LET expression of line (1):]~
      \begin{prooftreefunction}
        \AxiomC{$\dynEnv{\funcCallR{sparqlCall}{\begin{array}{l}
                \SELECT~\Vars~\DatasetClause\\
                \SparqlWhereClause~\SolutionModifier
              \end{array}
            }} \Rightarrow \omg{tr}{}
          $}
        
        \UnaryInfC{$\dynEnvX{1}{tr}{\grammarRule{Expr}} \Rightarrow \envElem{Res}{}$}
        
        \UnaryInfC{$\dynEnv{\begin{array}{l}
              \LET~\varR{results}~\keyword{:=}\\ 
              \qquad\funcCallR{sparqlCall}{\begin{array}{l}
                  \SELECT~\Vars~\DatasetClause\\
                  \SparqlWhereClause~\SolutionModifier
                \end{array}
              } \\ 
              \RETURN~\grammarRule{Expr}
            \end{array} \Rightarrow \envElem{Res}{}
          }$}
      \end{prooftreefunction}%
    \noindent where 
    \begin{small}
      \begin{tree}
        \ \ \dynX{1}{tr} = \dyn \envExtend{varValue}{\varNameR{results} \Rightarrow \omg{tr}{}} \enspace .
        \label{tree:dynEvn1}
      \end{tree}%
    \end{small}%

    \pagebreak[3]
  \item[\FOR expression of line (2):]~
      \begin{prooftreefunction}
        \AxiomC{$\dynEnvX{1}{tr}{\varR{results//sr{:}result}} \Rightarrow \sm{i}{}$}
        \UnaryInfC{$\dynEnvX{2}{tr}{\grammarRule{Expr}} \Rightarrow \envElem{Res}{i}$}

        \AxiomC{$\Ddots$}
        
        \BinaryInfC{$\dynEnvX{1}{tr}{
            \begin{array}{l}
              \FOR~\varR{result}~\AT~\varR{posvar} \\ \quad\IN~\varR{results//sr{:}result}\\
              \RETURN~\grammarRule{Expr}
            \end{array} \Rightarrow \envElem{Res}{1},\dotsb,\envElem{Res}{n}
          }$}
      \end{prooftreefunction}%
    where 
    % 
    \begin{small}
      \dynX{2}{tr} = \dynX{1}{tr} \envExtend{varValue}{
        \begin{array}{l}
          \varNameR{result} \Rightarrow \sm{i}{};\\
          \varNameR{posvar} \Rightarrow i      
        \end{array}
      }
    \end{small}%
    
  \item[\LET expressions of lines (3)--(4).] Here we consider all the \LET expressions represented by line~(3),
    where~$\var{v} \in \Vars$, \ie~this rule is repeated for each~$\var{v} \in \Vars$:
      \begin{prooftreefunction}
        \AxiomC{$\dynEnvX{2}{tr}{\varR{result/sr{:}binding[@name = \varName{v}]/*}} \Rightarrow V$}

        \UnaryInfC{$\dynEnvX{3}{tr}{\grammarRule{ExprSingle}} \Rightarrow \envElem{Res}{}$}

        \UnaryInfC{$\dynEnvX{2}{tr}{\begin{array}{l}%
              \LET~\var{v} := \varR{result/sr{:}binding[@name = \varName{v}]/*}\\
              \RETURN~\grammarRule{ExprSingle}
            \end{array} \Rightarrow \envElem{Res}{}
          }$}

      \end{prooftreefunction}%
    where~$\dynX{3}{tr} = \dynX{2}{tr} \envExtend{varValue}{\varName{v} \Rightarrow V}$
  \end{description}


  

  Consider the dynamic environment \dynX{i}{xs} such that \dynEnvX{i}{xs}{\grammarRule{ExprSingle} \Rightarrow
    \envElem{Val}{}} where, as we know from~\eqref{proof:sparqlForClause_dynenv_i}, \dynX{i}{xs} extends \dyn with the
  bindings from a solution mapping~\sm{i}{xs}.
  % 
  Furthermore, consider~$\sm{C}{}$,~$\omg{xs}{}$, and~$\omg{tr}{}$ as per the ($\Leftarrow$) part of the proof.
  
  From the proof trees of~$\funcCall{tr}{\envElem{Q}{}}$ we can see that the \LET expression from line~(1) extends \dyn
  with the value for the reserved variable \varR{results}, which cannot be included in~$Q$. 
  %
  The \FOR expression from line~(2) iterates over all the solution mappings~$\sm{tr}{} \in \omg{tr}{}$
  %
  and, as we know from \cref{lem:replacement-equivalence},~$\omg{xs}{} = \omg{tr}{} \bowtie \set{\sm{C}}$.
  Since~$\sm{C}{}$ is created based on~$\dyn$.\ecomp{var\-Value}, all the variable bindings from~$\sm{C}{}$ are already
  included in \dyn and all solution mappings~$\sm{tr}{} \in \omg{tr}{}$ are guaranteed to be compatible with~\sm{C}{}
  and thus we have that~$\sm{i}{xs} \in \omg{tr}{}$.

  Finally, the \LET expressions from lines~(3) and~(4) ensure that there exists a \dynX{2}{tr} such
  that~$\dynX{2}{tr}$.\ecomp{varValue} contains all the variable bindings from~$\sm{i}{xs}$, and we have
  that~$\dynX{2}{tr} \proofs \mathit{Expr\-Single} \Rightarrow \envElem{Val}{}$ and~$\dyn \proofs
  \funcCall{tr}{\envElem{Q}{}} \Rightarrow \envElem{Val}{}$.
  %
\end{proof}





%%% Local Variables:
%%% fill-column: 120
%%% TeX-master: "../thesis" 
%%% TeX-PDF-mode: t
%%% TeX-debug-bad-boxes: t
%%% TeX-parse-self: t
%%% TeX-auto-save: t
%%% reftex-plug-into-AUCTeX: t
%%% End:










%%% Local Variables:
%%% fill-column: 120
%%% TeX-master: t
%%% TeX-PDF-mode: t
%%% TeX-debug-bad-boxes: t
%%% TeX-parse-self: t
%%% TeX-auto-save: t
%%% reftex-plug-into-AUCTeX: t
%%% mode: latex
%%% mode: flyspell
%%% mode: reftex
%%% TeX-master: "../thesis"
%%% End:
