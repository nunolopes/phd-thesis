\section{The XMarkRDF benchmark}
\label{sec:benchm-descr}

For the evaluation of our implementation we created a benchmark suite based on the XMark benchmark
suite~\cite{SchmidtWaasKersten:2002aa}, which according to~\citet{AfanasievMarx:2008aa} is the most widely used
benchmark suite for XQuery.  It provides a data generator that produces \ac{XML} data simulating an auction website
(including information about persons and items they bid for) and includes 20 XQuery queries, henceforth referred to
as~$q_{1}$ to $q_{20}$, over this generated data.


In order to benchmark the XSPARQL language we also require data in the relational and \ac{RDF} formats, hence we provide
transformations (in fact, using XSPARQL queries) from the XMark \ac{XML} datasets into \ac{RDF} triples and a relational
instance, following a manually created schema for representing the XMark data.
%
These transformations replicate all the data in the original XMark datasets as \ac{RDF} triples and relational tuples.
% 
Next, we converted the XMark queries into corresponding XSPARQL queries using \SparqlForClause{s} and \SQLForClause{s}
to access the \ac{RDF} data and the relational database, respectively.
%
We call this new benchmark suite the XMarkRDF benchmark. 

We have made two changes to the original XMark queries:
\begin{inparaenum}[(1)]
\item SPARQL queries do not guarantee any default ordering, hence all original XMark queries were declared
  unordered -- as a consequence the XQuery engine is not required to follow document order when executing the query; and
\item we added the external variables \var{xml} and \var{rdf} in the XQuery and XSPARQL queries as parameters used to
  specify the URI identifying the input benchmark instance.
\end{inparaenum}


We also included in our own comparisons the SPAR\-QL2\-XQuery system~\cite{GroppeGroppeLinnemann:2008aa}, which is
similar in spirit to XSPARQL.  While the SPARQL2XQuery language allows to perform similar queries to the \ac{RDF} and
\ac{XML} fragment of the XSPARQL language, the implementation follows a different approach to integrate the \ac{XML} and
\ac{RDF} data: rather than performing interleaved calls to a SPARQL engine, the SPARQL2XQuery system relies on
translating the \ac{RDF} data into a pre-defined \ac{XML} format and transforming SPARQL queries into equivalent XQuery
over this pre-defined \ac{XML} format.  The translated queries can be directly executed using a native XQuery engine.
%
We focussed our experimental evaluation on query response time rather than on data transformation time, and as
SPARQL2XQuery requires an additional translation step from \ac{RDF} to a custom RDF/XML format, we converted the
XMarkRDF \ac{RDF} data into the format required by the SPAR\-QL2\-XQuery system.  We denote these new datasets,
containing the RDF/XML format required for the SPARQL2XQuery, by~$\textrm{XMarkRDF}_{\mathit{S2XQ}}$.


\begin{table}[t]
\caption{XMark (and variants) benchmark dataset description}
\label{tab:iterations}
\centering
\begin{tabular}{p{1.5cm}rr|r|rr|rr}
 \toprule
  Scaling                  &         &            & \multicolumn{1}{c|}{XMark}      & \multicolumn{2}{c|}{XMarkRDF}  & \multicolumn{2}{c}{XMarkRDB} \tabularnewline
   factor & Persons & Categories & Size (MB)  & Size (MB) & \# Triples & Size (MB) & \# Tuples \tabularnewline
 \midrule
  0.01 &   255 &   10 &   1.1 &   1.2 & 14745 & 1 & 4112 \tabularnewline
  0.02 &   510 &   20 &   2.3 &   2.3 & 27519 & 2 & 7799 \tabularnewline
  0.05 &  1275 &   50 &   5.8 &   5.8 & 70859 & 5 & 20190 \tabularnewline
  0.10 &  2550 &  100 &  11.7 &  12.4 & 142721 & 10 & 40183 \tabularnewline
  0.20 &  5100 &  200 &  23.5 &  24.9 & 283639 & 20 & 80622 \tabularnewline
  0.50 & 12750 &  500 &  58.0 &  61.7 & 706723 & 50 & 200496 \tabularnewline
  1.00 & 25500 & 1000 & 116.5 & 124.8 & 1414469 & 101 & 400620\tabularnewline
 \bottomrule
\end{tabular}
\end{table}


Using the data generators and translators, provided by the XMark benchmark and the XSPARQL translation to \ac{RDF} (as
presented in \cref{sec:optimisation}), we created datasets with scaling factors of 0.01, 0.02, 0.05, 0.1, 0.2,
0.5, and~1.0 and translated them into XMarkRDF and XMarkSQL.  An overview of the generated data is presented in
\cref{tab:iterations}, including the number of persons and item categories modelled, dataset sizes,\footnote{For
  the dataset sizes we determined the dataset size based on a Turtle representation of the \ac{RDF} graph and the
  \ac{SQL} \textsf{INSERT} statements that populate the database.}, the number of relational tuples and \ac{RDF}
triples.



\begin{table}[t]
\caption{$\textrm{XMarkRDF}_{\mathit{S2XQ}}$ dataset and translation times}
\label{tab:XMarkRDF-groppe}
\centering
\begin{tabular}{rrr}
 \toprule
  Scaling factor & Dataset size (MB) & Translation times (sec)\tabularnewline
 \midrule
  0.01 &   3.3 & 18.94\tabularnewline
  0.02 &   6.4 & 18.30\tabularnewline
  0.05 &  16.1 & 26.08\tabularnewline
  0.10 &  32.7 & 39.01\tabularnewline
  0.20 &  65.3 & 62.35\tabularnewline
  0.50 & 162.3 & 143.35\tabularnewline
  1.00 & 326.2 & 329.93\tabularnewline
 \bottomrule
\end{tabular}
\end{table}


Furthermore, we converted the XMarkRDF datasets into the RDF/XML format required by the SPAR\-QL2\-XQuery system. The
resulting dataset sizes and translation times for the different scaling factors of the XMarkRDF dataset are presented in
\cref{tab:XMarkRDF-groppe}.



\subsection{Experimental Setup}
\label{sec:experimental_setup}



The benchmark system consists of a dual core Intel XEON E5606 2.13GHz, 64GB memory running a 64 bit installation of
Debian 6.0.3 (stable distribution).
%
For the XQuery engine, we rely on Saxon version 9.4 Home Edition and Java version 1.6.0 64 bit. For evaluating SPARQL
queries we used ARQ 2.8.7.  We ran each query with a timeout of 10 minutes per query and with the Java Heap size set to
1GB.  Each query was run 10 times and the response time was measured using GNU time~1.7.  For each query we discard the
fastest and slowest response time and calculate the average of the remaining times.  From this result we deduce the
process startup time, determined by following the same procedure and executing an empty query.


For the evaluation we defined the following run configurations:
\begin{description}[noitemsep]
\item[$\mathit{XQ}$:] original XQuery queries, evaluated using the Saxon engine;
\item[$\mathit{XS^{rdf}}$:] using the XSPARQL implementation over the XMarkRDF datasets (translated data and queries);
\item[$\mathit{XS^{rdb}}$:] using the XSPARQL implementation over the XMarkRDB datasets stored on a PostgreSQL 8.4.11
  relational database management system (translated data and queries); and
\item[$\mathit{S2XQ}$:] using the SPARQL2XQuery implementation over the translation of the XMarkRDF datasets into the
  required \ac{XML} format ($\textrm{XMarkRDF}_{\mathit{S2XQ}}$).
\end{description}


\subsection{Base System Results}
\label{sec:results}

In this section we present an experimental evaluation of our prototype presented in \cref{sec:implementation}
using the novel XMarkRDF and XMarkRDB benchmark suites.
%
We also compare our XSPARQL prototype with the SPARQL2XQuery engine, an implementation of the direct translation of
SPARQL to XQuery presented by~\citet{GroppeGroppeLinnemann:2008aa}.


\begin{table*}[t]
  \caption[Query response times (in seconds) of the 2MB dataset.]{Query response times (in seconds) of the 2MB dataset.  Query rewriting error (\textit{err}).}
  \label{tab:results-2mb}
  \centering
  \import{\plotsDir}{table-2a.tex}

  \vspace{5pt}  

  \import{\plotsDir}{table-2b.tex}
\end{table*}



The response times of the~$\mathit{XQ}$,~$\mathit{XS^{rdb}}$,~$\mathit{XS^{rdf}}$, and~$\mathit{S2XQ}$ runs for the
benchmark queries over the 2MB dataset size are shown in \cref{tab:results-2mb}.\footnote{Queries~$q_{15}$ and~$q_{16}$
  involve applying an XPath expression to data that is stored in \ac{RDF} or in the relational database as a string.
  %
  Since parsing this string representation back into an \ac{XML} element is not available to the Saxon HE engine we are
  using for benchmarking, these queries were considered as errors (\textit{err}) for the~$\mathit{XS^{rdf}}$
  and~$\mathit{XS^{rdb}}$ configurations.
  %
    The errors in~$\mathit{S2XQ}$ were due to translation errors from the application that was provided to us.} We present
the 2MB dataset as it is the largest dataset our $\mathit{XS^{rdf}}$ implementation can process within the time limit of
10 minutes.
%
Both the data and query translation times for the~$\mathit{S2XQ}$ configuration are not included in the presented
results since this process can be done a priori.
%
The~$\mathit{XQ}$ response times are presented as a baseline measure, however it is noteworthy that these queries do not
cater for our heterogeneous data sources scenario.
%
The~$\mathit{XS^{rdb}}$ configuration often presents the best response times undoubtably due to the underlying
relational database system.  Since the inner queries in nested queries access the primary keys of relations, their
response time is fast.  \cref{tab:results-100mb} presents the results for our largest dataset, where we can see that
the~$\mathit{XS^{rdb}}$ approach is able to evaluate most of the queries (except~$q_{11}$) within the time limit.
%
\begin{table*}[t]
  \caption[Query response times (in seconds) of the 100MB dataset.]{Query response times (in seconds) of the 100MB dataset.  Query rewriting error (\textit{err}).}
  \label{tab:results-100mb}
  \centering
  \import{\plotsDir}{table-100a.tex}

  \vspace{5pt}  

  \import{\plotsDir}{table-100b.tex}
\end{table*}


\cref{tab:results-2mb} shows that for most of the queries in the~$\mathit{S2XQ}$ runs are faster than the
interleaved calls to a SPARQL engine in the~$\mathit{XS^{rdf}}$ runs.
%
Even considering that the response times do not include the data translation times (presented in
\cref{tab:XMarkRDF-groppe}), this suggests that an implementation of XSPARQL where the SPARQL queries are translated
into native XQuery is a viable alternative to interleaving calls to a SPARQL engine.  However, for such translations to
be possible we need access to the full \ac{RDF} dataset to perform the query translation, which is not possible for
example in the case where we are querying data behind a SPARQL endpoint.
%
Another issue related to the implementation of the SPARQL2XQuery system is that response times deteriorate
considerably for larger datasets.  This was observed for all the queries in the benchmark and can be seen in the graphs
of \cref{fig:evaluation-times-q8-q9,fig:evaluation-times-q10-q11}.

Queries~$q_{8}$--$q_{12}$ have the highest execution times of all the benchmark queries (especially noticeable in
the~$\mathit{XS^{rdf}}$ configuration) since they contain nested expressions.  For these nested queries, our interleaved
\SparqlForClause{s} XSPARQL implementation can only handle small datasets: the 2MB dataset is the largest for which all
queries finish within the time limit and for the 20MB dataset all queries result in a \emph{timeout}.

Based on these results, we propose a set of different rewritings that aim at reducing the response times of nested
queries.




%%% Local Variables:
%%% fill-column: 120
%%% TeX-master: t
%%% TeX-PDF-mode: t
%%% TeX-debug-bad-boxes: t
%%% TeX-parse-self: t
%%% TeX-auto-save: t
%%% reftex-plug-into-AUCTeX: t
%%% mode: latex
%%% mode: flyspell
%%% mode: reftex
%%% TeX-master: "../thesis"
%%% End:
