\section{Optimisations of Nested \texttt{for} Expressions}
\label{sec:optimisation}

Following our current implementation of the XSPARQL language, this section presents different rewriting strategies for
XSPARQL queries containing nested expressions.  
%
Based on the experimental evaluation results from the previous section, we are especially interested in nested
expressions with an inner \SparqlForClause, as the number of interleaved calls to the SPARQL engine can be reduced
drastically by using these rewritings.
%
Intuitively, these rewritings rely on executing the inner SPARQL query only once in an unbounded manner, and then either
performing the a nested loop over the results of the queries directly in XQuery, or, if possible, transforming the
nested queries into a single SPARQL query.

We start by presenting the definitions and conditions under we can perform these rewritings.
%
\begin{definition}[Dependent Join]
  We call two nested XSPARQL \FOR expressions (\ForClause, \SparqlForClause, or \SQLForClause), where the inner
  expression is a \SparqlForClause and at least one variable in the inner expression is bound by the outer expression, a
  \emph{dependent join}.  The shared variables between the \FOR expressions are called \emph{dependent variables}.
\end{definition}
%
\noindent Note that the strategies presented here are only applicable for dependent joins satisfying the following restrictions:
%
\begin{enumerate}[noitemsep]
\item An explicit~$\DatasetClause$ of the inner query needs to be statically determined \ie~it cannot be determined
  based on variables bound from the outer expression;
\item The return clause of the inner expression can not be a \ConstructClause; and
\item The dependent variable in the inner query's graph pattern must be \emph{strictly bounded} as defined next.
\end{enumerate}
%
\begin{definition}[Strict Boundedness]
  The set of \emph{strictly bound variables} in a graph pattern~$P$,
  denoted~$\bVars(P)$, is recursively defined as follows: if~$P$ is
  \begin{itemize}[nosep]
  \item a \ac{BGP}, then~$\bVars(P) = \vars{P}$;
  \item $(P_1~\AND~P_2)$, then~$\bVars(P) = \bVars(P_1) \cup \bVars(P_2)$;
  \item $(P_1~\OPTIONAL~P_2)$, then~$\bVars(P) = \bVars(P_1)$;
  \item $(P_1~\UNION~P_2)$, then~$\bVars(P) = \bVars(P_1) \cap \bVars(P_2)$;
  \item $(\GRAPH~i~P_1)$, then~$\bVars(P) = \bVars(P_1) \cup (\set{i} \cap\AV)$; and
  \item $(P_1~\FILTER~R)$, then~$\bVars(P) = \bVars(P_1)$.
  \end{itemize}
\end{definition}
%
Informally, the dependent variables must occur
\begin{inparaenum}[(i)]
\item in a \ac{BGP},
\item in every alternative of \UNION{s} pattern, and 
\item it must also occur outside of the optional graph pattern in case of optionals.
\end{inparaenum}
% 
Strict boundedness essentially ensures that the join variable does not occur only in a \FILTER expression, which would
lead to problems in case the inner expression is called unconstrained, see below.

Next, we define the notion of inclusion of solution sequences.  
%
\begin{definition}[Solution sequence inclusion]
  Let~$\Omega_1$ and~$\Omega_2$ be solution sequences. We say~\emph{$\Omega_1$ is included in~$\Omega_2$},
  denoted~$\Omega_1 \preceq \Omega_2$, if for all solution mappings~$\mu_1 \in \funcCall{ToMultiset}{\Omega_1}$ there
  exists a solution mapping~$\mu_2 \in \funcCall{ToMultiset}{\Omega_2}$ such that~$\mu_1 \subseteq \mu_2$. 
\end{definition}
%
Please note that this definition extends the notion of subset between multisets by considering also the subset relation
between their elements, \ie~solution mappings. 


The following rewritings for the implementation of dependent joins can be grouped into two categories, depending whether
the join is performed in XQuery or SPARQL.  For performing the join in XQuery, we use already known join algorithms from
relational databases, namely nested-loop joins.
%
For performing the join in SPARQL, if the outer expression is a \SparqlForClause we can implement the join by rewriting
both the inner and the outer expressions into a single SPARQL call.  In case the outer query consists of an XQuery
\ForClause, we can still consider this approach, but we need to convert the result of the outer XQuery \ForClause to an
\ac{RDF} graph, for instance relying on a SPARQL engine that supports SPARQL Update~\cite{GearonPassantPolleres:2012aa}
to add this temporary graph to a triple store.


\subsection{Dependent Join implementation in XQuery}
\label{sec:join-impl-xquery}
The intuitive idea with these rewritings is, instead of using the na\"{\i}ve rewriting that performs one SPARQL query
for each iteration of the outer expression, to execute only one unconstrained SPARQL query, before the outer query.  The
resulting sequence of SPARQL solution mappings is then joined in XQuery with the results of the outer expression, using
one of the following strategies. 


The straightforward way to implement the join over dependent variables directly in XQuery is by nesting two XQuery \FOR
expressions, much like a regular nested-loop join~\cite{AbiteboulHullVianu:1995aa} in standard relational databases.
The join consists of restricting the values of variables from the inner expression to the values taken from the current
iteration of the outer expression. 


Similar to \cref{sec:implementation}, we will describe the implementation of this nested-loop join by means of
the rewriting function~$\funcName{opt_{nl}}$. We use~$A \triangle B = \left(A \cup B\right)
\setminus \left(A \cap B\right)$ to denote the symmetric difference of two sets~$A$ and~$B$.


Let~$\envElem{Q}{}$ be an XSPARQL expression of form

\begin{queryF}
  \[
  \begin{array}{l}
  \mathrm{(1)~~~~}\FOR~\gsup{\var{Var}}{out}~\AT~\gsup{\var{PosVar}}{out}~\IN~\envElem{\ExprSingle}{1}~\RETURN\\
  \mathrm{(2)~~~~}\quad\FOR~\gsup{\Vars}{in}~\DatasetClause~\SparqlWhereClause~\SolutionModifier\\
  \mathrm{(3)~~~~}\quad\RETURN~\envElem{\ExprSingle}{2}
\end{array}
\]
\label{eq:expr_sdep_xquery}
\end{queryF}%

\noindent
the application of the rewriting function~\onl{\envElem{Q}{}} can be split into two cases:
%
\begin{itemize}
\item if \envElem{\ExprSingle}{1} and \envElem{\ExprSingle}{2} do not contain any occurrences
  of~\eqref{eq:expr_sdep_xquery} then, assuming $\gsup{\Vars}{sp} = \funcCall{\Vars}{\SparqlWhereClause}$, we have that:
%
\begin{small}
\begin{equation*}
\begin{array}{l}
  \onl{\envElem{Q}{}} =\\
  \mathrm{(1)~~~~} \LET~\varR{results}~:= \funcCallR{sparqlCall}{\begin{array}{l}
      \SELECT~\{ \gsup{\var{Var}}{out} \}  \cup \gsup{\Vars}{in}~\DatasetClause \\
      \SparqlWhereClause~\SolutionModifier
    \end{array}} \RETURN\\
  \mathrm{(2)~~~~} \FOR~\gsup{\var{Var}}{out}~\AT~\gsup{\var{PosVar}}{out}~\IN~\envElem{\ExprSingle}{1}~\RETURN\\
  \mathrm{(3)~~~~} \quad \FOR~\varR{result}~\AT~\varR{posvar\mathunderscore{}in}~\IN~\varR{results//sr{:}result}~\RETURN \\
  \mathrm{(4)~~~~} \quad \IF~\left(\funcCall{join_{nl}}{\begin{array}{l}
        \{ \gsup{\var{Var}}{out} \} \cap \gsup{\Vars}{sp},\\
        \varR{result}
      \end{array}
    }\right)~\THEN\\
  \mathrm{(5)~~~~} \qquad \LET~\var{v} := \varR{result/sr{:}binding[@name = \varName{v}]/*}~\qquad\qquad \hfill {\textrm{\smaller for each~$\var{v} \in \set{ \gsup{Var}{out} } \triangle \gsup{\Vars}{sp}$}}\\
  \mathrm{(6)~~~~} \qquad \envElem{\ExprSingle}{2}\\
  \mathrm{(7)~~~~} \quad \ELSE~\left(\right)
\end{array}
\end{equation*}
\end{small}%
%
\item otherwise:
%
\begin{small}
\begin{equation*}
\begin{array}{r@{~~~}l}
  \multicolumn{2}{l}{\onl{\envElem{Q}{}} =}\\[.3em]
   & \onl{\begin{array}{l}
       \FOR~\gsup{\var{Var}}{out}~\AT~\gsup{\var{PosVar}}{out}~\IN~\onl{\envElem{\ExprSingle}{1}}~\RETURN\\
       \quad \FOR~\gsup{\Vars}{in}~\DatasetClause~\SparqlWhereClause~\SolutionModifier\\
       \quad \RETURN~\onl{\envElem{\ExprSingle}{2}}
\end{array}
}
\end{array}
\end{equation*}
\end{small}%

\end{itemize}
%
The auxiliary function~$\funcName{join_{nl}}$ consists of an XPath expression that determines if an XQuery tuple stream
is compatible with a SPARQL solution mapping. 
%
More specifically, this function considers two variables as compatible if their values are equal, the outer value is a
blank node, or the inner value~($\var{VarRes}_{i}$) is unbound.
%
These cases represent the semantics of XQuery nested queries, behaving similar to a left outer join~($\leftouterjoin$).

\begin{small}
\begin{equation*}
\begin{array}{l}
  \funcCall{join_{nl}}{\set{ \var{Var_1}, \dotsb, \var{Var_n} }, \var{res}} = \\[.5em]
  \begin{array}{c}
    \left(
      \begin{array}{l}
        \funcCallR{isBlank}{\var{Var_1}}~\OR \\
        \funcCall{\mathtt{fn{:}empty}}{\var{res}/\mathtt{sr{:}binding}\mathtt{[@name~=~\envElem{Var}{1}]/*}}~\OR\\
        \left( \var{Var_1}~\EQ~\var{res}/\mathtt{sr{:}binding[@name~=~\envElem{Var}{1}]/*} \right)
      \end{array}
    \right)~\AND \\ 
     \dotsb \\
    \AND~\left(\begin{array}{l}
        \funcCallR{isBlank}{\var{Var_n}}~\OR\\
        \funcCall{\mathtt{fn{:}empty}}{\var{res}/\mathtt{sr{:}binding[@name~=~\envElem{Var}{n}]/*}}~\OR\\
        \left( \var{Var_n}~\EQ~\var{res}/\mathtt{sr{:}binding[@name~=~\envElem{Var}{n}]/*} \right)
    \end{array}
  \right) \\
  \end{array}
\end{array}
\end{equation*}
\end{small}%
%
When~$\envElem{Q}{}$ is an XSPARQL expression of form
\begin{queryF}
\[\begin{array}{l}
  \mathrm{(1)~~~~}\FOR~\gsup{\Vars}{out}~\gsup{\DatasetClause}{out}~\gsup{\SparqlWhereClause}{out}~\gsup{SolutionModifier}{out}\\
  \mathrm{(2)~~~~}\RETURN\\
  \mathrm{(3)~~~~}\quad\FOR~\gsup{\Vars}{in}~\gsup{\DatasetClause}{in}~\gsup{\SparqlWhereClause}{in}~\gsup{SolutionModifier}{in}\\
  \mathrm{(4)~~~~}\quad\ReturnExpr
\end{array}\]
\label{eq:expr_sdep}
\end{queryF}%
%
\noindent the application of the rewriting function~\onl{\envElem{Q}{}} can be split into two cases:
%
\begin{itemize}
\item in case \envElem{\ExprSingle}{} does not contain any occurrences of~\eqref{eq:expr_sdep} then, considering
  $\gsup{\Vars}{sp} = \vars{\gsup{\SparqlWhereClause}{in}}$ the set of variables from the inner \SparqlWhereClause, we
  have that:
%
\begin{small}
  \begin{equation*}
  \begin{array}{l}
  \onl{\envElem{Q}{}} =\\[.3em]
    \mathrm{(1)~~~~} \LET~\varR{res\mathunderscore{}in}~:=~\funcCallR{sparqlCall}{
                    \begin{array}{l}
                      \SELECT~\gsup{\Vars}{in} \cup \gsup{\Vars}{out} \cap \gsup{\Vars}{sp}\\
                      \gsup{\DatasetClause}{in}~\gsup{\SparqlWhereClause}{in} \\
                      \gsup{\SolutionModifier}{in}
                    \end{array}
                  }~\RETURN\\
    \mathrm{(2)~~~~}\LET~\varR{res\mathunderscore{}out}~:=~\funcCallR{sparqlCall}{
                    \begin{array}{l}
                      \SELECT~\gsup{\Vars}{out}~\gsup{\DatasetClause}{out} \\
                      \gsup{\SparqlWhereClause}{out}~\gsup{\SolutionModifier}{out}
                    \end{array}
                  }~\RETURN\\
    \mathrm{(3)~~~~}\FOR~\varR{rout}~\AT~\varR{posvar\mathunderscore{}out}~\IN~\varR{res\mathunderscore{}out//sr{:}result}~~\RETURN\\
    \mathrm{(4)~~~~} \quad \LET~\var{v}~:=~\varR{rout/sr{:}binding[@name=\varName{v}]/*}~\RETURN \qquad \hfill \textrm{\smaller for each~$\var{v}~\in~\gsup{\Vars}{out}$}\\
    \mathrm{(5)~~~~} \quad \FOR~\varR{rin}~\AT~\varR{posvar\mathunderscore{}out}~\IN~\varR{res\mathunderscore{}in//sr{:}result}~\RETURN\\
    \mathrm{(6)~~~~} \quad\quad \IF~\left(\funcCall{join_{sr}}{\begin{array}{l}
          \gsup{\Vars}{out} \cap \gsup{\Vars}{sp},
          \varR{res\mathunderscore{}out},
          \varR{res\mathunderscore{}in}
        \end{array}
      }\right)~\THEN\\
    \mathrm{(7)~~~~} \quad\qquad \LET~\var{v} :=~\varR{res\mathunderscore{}in/sr{:}binding[@name=\varName{v}]/*}~\RETURN \qquad \hfill \textrm{\smaller for each~$\var{v} \in \gsup{\Vars}{out} \triangle \gsup{\Vars}{sp}$}\\
    \mathrm{(8)~~~~} \quad\quad~~~\grammarRule{ExprSingle} \\
    \mathrm{(9)~~~~} \quad\quad \ELSE~\left(\right)
  \end{array}
\end{equation*}
\end{small}%
%
\item otherwise:
%
\begin{small}
\begin{equation*}
\begin{array}{r@{~~~}l}
  \multicolumn{2}{l}{\onl{\envElem{Q}{}} =}\\[.3em]
   & \onl{\begin{array}{l}
       \FOR~\gsup{\Vars}{out}~\gsup{\DatasetClause}{out}~\gsup{\SparqlWhereClause}{out}~\gsup{SolutionModifier}{out}\\
       \RETURN\\
       \quad\FOR~\gsup{\Vars}{in}~\gsup{\DatasetClause}{in}~\gsup{\SparqlWhereClause}{in}~\gsup{SolutionModifier}{in}\\
       \quad\RETURN~\onl{\envElem{ExprSingle}{}}
\end{array}
}
\end{array}
\end{equation*}
\end{small}%
\end{itemize}
% 
The~$\funcName{join_{sr}}$ function is defined as:
\begin{small}
\begin{equation*}
\begin{array}{c}
  \funcCall{join_{sr}}{\set{ \var{Var_1}, \dotsb, \var{Var_n} }, \var{resOut}, \var{resIn}} = \\[.5em]
  \quad \funcCall{join_{nl}}{\set{ \var{resOut}/\mathtt{sr{:}binding[@name~=~\envElem{Var}{1}]/*} }, \var{resIn}}\\
  \quad \AND~\dotsb~\AND\\
  \quad \funcCall{join_{nl}}{\set{ \var{resOut}/\mathtt{sr{:}binding[@name~=~\envElem{Var}{n}]/*} }, \var{resIn}} \enspace .
\end{array}
\end{equation*}
\end{small}%
%
\noindent The~$\funcName{join_{sr}}$ function behaves in a similar fashion to the~$\funcName{join_{nl}}$ function with
the difference that it compares two SPARQL solution sequences.
%
For nested expressions with an outer \SQLForClause, \ie~when~$\envElem{Q}{}$ is an XSPARQL expression of form
\begin{queryF}
\[\begin{array}{l}
  \mathrm{(1)~~~~}\FOR~\envElem{AttrSpec}{1}~\keyword{as}~\envElem{\var{Var}}{1}, \dots, \envElem{AttrSpec}{n}~\keyword{as}~\envElem{\var{Var}}{n}~\grammarRule{RelationList}~\grammarRule{SQLWhereClause}\\
  \mathrm{(2)~~~~}\RETURN\\
  \mathrm{(3)~~~~}\quad\FOR~\gsup{\Vars}{in}~\gsup{\DatasetClause}{in}~\gsup{\SparqlWhereClause}{in}~\gsup{SolutionModifier}{in}\\
  \mathrm{(4)~~~~}\quad\ReturnExpr
\end{array}\]
\label{eq:expr_sdep_sql}
\end{queryF}%
%
\noindent the application of the rewriting function~\onl{\envElem{Q}{}} can also be split into two cases:
%
\begin{itemize}
\item in case \envElem{\ExprSingle}{} does not contain any occurrences of~\eqref{eq:expr_sdep_sql} then, considering
  $\gsup{\Vars}{sp} = \vars{\gsup{\SparqlWhereClause}{in}}$ is the set of variables from the inner \SparqlWhereClause
  and~$\gsup{\Vars}{out} = \set{\envElem{\var{Var}}{1}, \dots, \envElem{\var{Var}}{n}}$ is a shorthand notation for the
  variables in the outer \SQLForClause, we have that:
  % 
\begin{small}
  \begin{equation*}
  \begin{array}{l}
  \onl{\envElem{Q}{}} =\\[.3em]
    \mathrm{(1)~~~~} \LET~\varR{res\mathunderscore{}in}~:=~\funcCallR{sparqlCall}{
                    \begin{array}{l}
                      \SELECT~\gsup{\Vars}{in} \cup \gsup{\Vars}{out} \cap \gsup{\Vars}{sp}\\
                      \gsup{\DatasetClause}{in}~\gsup{\SparqlWhereClause}{in} \\
                      \gsup{\SolutionModifier}{in}
                    \end{array}
                  }~\RETURN\\
    \mathrm{(2)~~}\LET~\varR{res\mathunderscore{}out}~\text{\texttt{:=}}~\funcCallR{sqlCall}{\begin{array}{l}
        \SELECT~\envElem{AttrSpec}{1}, \dots, \envElem{AttrSpec}{n}\\
        \grammarRule{RelationList}~\grammarRule{SQLWhereClause}
      \end{array}
    }~\RETURN\\
    \mathrm{(3)~~~~}\FOR~\varR{rout}~\AT~\varR{posvar\mathunderscore{}out}~\IN~\varR{res\mathunderscore{}out//sr{:}result}~~\RETURN\\
    \mathrm{(4)~~~~} \quad \LET~\var{v}~:=~\varR{rout/sr{:}binding[@name=\varName{v}]/*}~\RETURN \qquad \hfill \textrm{\smaller for each~$\var{v}~\in~\gsup{\Vars}{out}$}\\
    \mathrm{(5)~~~~} \quad \FOR~\varR{rin}~\AT~\varR{posvar\mathunderscore{}out}~\IN~\varR{res\mathunderscore{}in//sr{:}result}~\RETURN\\
    \mathrm{(6)~~~~} \quad\quad \IF~\left(\funcCall{join_{sr}}{\begin{array}{l}
          \gsup{\Vars}{out} \cap \gsup{\Vars}{sp},
          \varR{res\mathunderscore{}out},
          \varR{res\mathunderscore{}in}
        \end{array}
      }\right)~\THEN\\
    \mathrm{(7)~~~~} \quad\qquad \LET~\var{v} :=~\varR{res\mathunderscore{}in/sr{:}binding[@name=\varName{v}]/*}~\RETURN \qquad \hfill \textrm{\smaller for each~$\var{v} \in \gsup{\Vars}{out} \triangle \gsup{\Vars}{sp}$}\\
    \mathrm{(8)~~~~} \quad\quad~~~\grammarRule{ExprSingle} \\
    \mathrm{(9)~~~~} \quad\quad \ELSE~\left(\right)
  \end{array}
\end{equation*}
\end{small}%
%
\item otherwise:
%
\begin{small}
\begin{equation*}
\begin{array}{r@{~~~}l}
  \multicolumn{2}{l}{\onl{\envElem{Q}{}} =}\\[.3em]
   & \onl{\begin{array}{l}
       \FOR~\envElem{AttrSpec}{1}~\keyword{as}~\envElem{\var{Var}}{1}, \dots, \envElem{AttrSpec}{n}~\keyword{as}~\envElem{\var{Var}}{n}~\grammarRule{RelationList}~\grammarRule{SQLWhereClause}\\
       \RETURN\\
       \quad\FOR~\gsup{\Vars}{in}~\gsup{\DatasetClause}{in}~\gsup{\SparqlWhereClause}{in}~\gsup{SolutionModifier}{in}\\
       \quad\RETURN~\onl{\envElem{ExprSingle}{}}
\end{array}
}
\end{array}
\end{equation*}
\end{small}%
\end{itemize}


The following proposition states that the~$\funcName{opt_{nl}}$ rewriting function is sound and complete.
%
\begin{restatable}{proposition}{depJoinCorrectRestatable}
\label{prop:depjoincorrect}
Let \envElem{Q}{} be an XSPARQL expression of form~\eqref{eq:expr_sdep_xquery},~\eqref{eq:expr_sdep},
or~\eqref{eq:expr_sdep_sql} and \dyn the dynamic environment of \envElem{Q}{}, then~$\dynEnv \envElem{Q}{} \Rightarrow
\envElem{Val}{}$ if and only if~$\dynEnv \onl{\envElem{Q}{}} \Rightarrow \envElem{Val}{}.$
\end{restatable}
%
\begin{proof}
  We now present the proof of the~$\funcName{opt_{nl}}$ rewriting function for expressions of the
  form~\eqref{eq:expr_sdep}.
  %
  We start by showing the proof for the base case, where \ExprSingle of~\eqref{eq:expr_sdep} does not
  contain any occurrences of~\eqref{eq:expr_sdep}.
  %
  \paragraph*{Base Case.}
  \noindent ($\Rightarrow$)
  % 
  We start by showing that if~$\dynEnv \envElem{Q}{} \Rightarrow \envElem{Val}{}$ then~$\dynEnv \onl{\envElem{Q}{}}
  \Rightarrow \envElem{Val}{}$.
  % 
  We present the proof tree for each of the XQuery core expressions in the~$\onl{\envElem{Q}{}}$ rewriting where, in
  each proof tree, \grammarRule{Expr} corresponds to the XQuery expressions of the following lines.
  % 
  \begin{description}
    % 
  \item[\LET expression of line (1).] For this rule let~$\Vars = \gsup{\Vars}{in} \cup \left( \gsup{\Vars}{out} \cap
      \vars{\gsup{\SparqlWhereClause}{in}} \right)$ be the set of variables from the inner \SparqlForClause and any
    variables from the outer \SparqlForClause used in the inner \SparqlWhereClause. Thus, we have that:

    \begin{prooftreefunction}
      \AxiomC{$\begin{array}{r@{\hspace{-1pt}}l}
          \dyn &~\proofs \funcCallR{sparqlCall}{
            \begin{array}{l}
              \SELECT~\Vars~\gsup{\DatasetClause}{in}\\
              \gsup{\SparqlWhereClause}{in}~\gsup{\SolutionModifier}{in}
            \end{array}
          }~\Rightarrow \omg{nl}{in}
        \end{array}$}
      
      \UnaryInfC{$\dynEnvX{1}{nl}{\grammarRule{Expr}} \Rightarrow \envElem{Res}{}$}
      
      \UnaryInfC{$\dynEnv{\begin{array}{l}
            \LET~\varR{res\mathunderscore{}in}~:=~\funcCallR{sparqlCall}{
              \begin{array}{l}
                \SELECT~\Vars~\gsup{\DatasetClause}{in}\\
                \gsup{\SparqlWhereClause}{in}~\gsup{\SolutionModifier}{in}
              \end{array}\!} \\ 
            \quad \RETURN~\grammarRule{Expr}
          \end{array} \Rightarrow \envElem{Res}{}}$}
      
    \end{prooftreefunction}%
    where 
    \begin{tree}
      \dynX{1}{nl} = \dyn \envExtend{varValue}{\varNameR{res\mathunderscore{}in} \Rightarrow \omg{nl}{in}} \enspace .
      \label{tree:dynEvn2}
    \end{tree}%
    % 

  \item[\LET expression of line (2):]~

    \begin{prooftreefunction}
      \AxiomC{$\dynEnvX{1}{nl}{\funcCallR{sparqlCall}{
            \begin{array}{l}
              \SELECT~\gsup{\Vars}{out}~\gsup{\DatasetClause}{out} \\
              \gsup{\SparqlWhereClause}{out}~\gsup{\SolutionModifier}{out}
            \end{array}}}~\Rightarrow \omg{nl}{out}~$}

      \UnaryInfC{$\dynEnvX{2}{nl}{\grammarRule{Expr}} \Rightarrow \envElem{Res}{}$}

      \UnaryInfC{$\dynEnvX{1}{nl}{\begin{array}{l}
            \keyword{let}~\varR{res\mathunderscore{}out}~:=\\
            \qquad\funcCallR{sparqlCall}{
              \begin{array}{l}
                \SELECT~\gsup{\Vars}{out}~\gsup{\DatasetClause}{out} \\
                \gsup{\SparqlWhereClause}{out}~\gsup{\SolutionModifier}{out}
              \end{array}} \\
            \quad \RETURN~\grammarRule{Expr}
          \end{array}}~ \Rightarrow \envElem{Res}{}$}
    \end{prooftreefunction}%
    %
    where 
    % 
    \begin{small}
      $\dynX{2}{nl} = \dynX{1}{nl} \envExtend{varValue}{\varNameR{res\mathunderscore{}out} \Rightarrow \omg{nl}{out}}
      \enspace .$
    \end{small}%    
    

  \item[\FOR expression of line (3):]~

    \begin{prooftreefunction}

      \AxiomC{$\dynEnvX{2}{nl}{\varR{res\mathunderscore{}out//sr{:}result}} \Rightarrow \sm{i}{out}$}
      \UnaryInfC{$\dynEnvX{3}{nl}{\grammarRule{Expr}} \Rightarrow \envElem{Res}{i}$}

      \AxiomC{$\Ddots$}
      
      \BinaryInfC{$\dynEnvX{2}{nl}{
          \begin{array}{l}
            \FOR~\varR{rout}~\AT~\varR{posvar\mathunderscore{}out}\\
            \quad\IN~\varR{res\mathunderscore{}out//sr{:}result}\\
            \RETURN~\grammarRule{Expr}
          \end{array} \Rightarrow \envElem{Res}{1},\dotsb,\envElem{Res}{n}}$}
      
    \end{prooftreefunction}%
    %
    where 
    \begin{small}
      $\dynX{3}{nl} = \dynX{2}{nl} \envExtend{varValue}{\begin{array}{l}
          \varNameR{rout} \Rightarrow \sm{i}{out};\\
          \varNameR{posvar\mathunderscore{}out} \Rightarrow i
        \end{array}
      } \enspace .$
    \end{small}%


  \item[\LET expressions of line (4).] Here we consider all the \LET expressions represented by line~(4), where~$\var{v}
    \in \gsup{\Vars}{out}$:

    \begin{prooftreefunction}
      \AxiomC{$\dynEnvX{3}{nl}{\varR{rout/sr{:}binding[@name = \varName{v}]/*}} \Rightarrow \envElem{V}{}$}

      \UnaryInfC{$\dynEnvX{4}{nl}{\grammarRule{Expr}} \Rightarrow \envElem{Res}{}$}

      \UnaryInfC{$\dynEnvX{3}{nl}{\begin{array}{l}
            \LET~\var{v} :=~\varR{rout/sr{:}binding[@name = \varName{v}]/*} \\
            \RETURN~\grammarRule{Expr}
          \end{array} \Rightarrow \envElem{Res}{}}$}

    \end{prooftreefunction}%
    %
    where 
    \begin{small}
      $\dynX{4}{nl} = \dynX{3}{nl}\envExtend{varValue}{\mathit{v} \Rightarrow V} \enspace .$
    \end{small}%
    
  \item[\FOR expression of line (5):]~

    \begin{prooftreefunction}

      \AxiomC{$\dynEnvX{4}{nl}{\varR{res\mathunderscore{}in//sr{:}result}} \Rightarrow \sm{j}{out}$}
      \UnaryInfC{$\dynEnvX{5}{nl}{\grammarRule{Expr}} \Rightarrow \envElem{Res}{j}$}

      \AxiomC{$\Ddots$}

      \BinaryInfC{$\dynEnvX{4}{nl}{
          \begin{array}{l}
            \FOR~\varR{rin}~\AT~\varR{posvar\mathunderscore{}in}\\
            \IN~\varR{res\mathunderscore{}in//sr{:}result}\\
            \RETURN~\grammarRule{Expr}
          \end{array} \Rightarrow \envElem{Res}{1},\dotsb,\envElem{Res}{n}}$}
      
    \end{prooftreefunction}%
    %
    where 
    \begin{small}
      %
      $\dynX{5}{nl} = \dynX{4}{nl} \envExtend{varValue}{\begin{array}{l}
          \varNameR{rin} \Rightarrow \sm{j}{out};\\
          \varNameR{posvar\mathunderscore{}in} \Rightarrow j
        \end{array}
      }\enspace .$
      % 
    \end{small}%
    
  \item[\IF expression of lines (6)--(9).] In this rule,~\grammarRule{Expr} represents the \LET expressions from lines
    (7)--(8):

    \begin{prooftreefunction}
      \AxiomC{$\dynEnvX{5}{nl}{\funcCall{join_{sr}}{\begin{array}{l}
              \gsup{\Vars}{out} \cap \vars{\SparqlWhereClause},\\
              \varR{res\mathunderscore{}out}, \varR{res\mathunderscore{}in}
            \end{array}
          }} \Rightarrow \mathtt{true}$}
      \UnaryInfC{$\dynEnvX{5}{nl}{\ExprSingle} \Rightarrow \envElem{Res}{1}$}

      \UnaryInfC{$\dynEnvX{5}{nl}{
          \begin{array}{l}
            \IF~\left(\funcCall{join_{sr}}{
                \begin{array}{l}
                  \gsup{\Vars}{out} \cap \vars{\SparqlWhereClause},\\
                  \varR{res\mathunderscore{}out}, \varR{res\mathunderscore{}in}
                \end{array}
              }\right)\\
            \THEN~\grammarRule{Expr}~\ELSE~()
          \end{array}  \Rightarrow \envElem{Res}{1}}$}
      
    \end{prooftreefunction}%

    \pagebreak[3]
  \item[\LET expressions of lines (7)--(8).] Again, we consider all the \LET expressions represented by line~(7),
    where~$\var{v} \in \gsup{\Vars}{out} \triangle \vars{\gsup{\SparqlWhereClause}{in}}$:

    \begin{prooftreefunction}
      \AxiomC{$\dynEnvX{5}{nl}{\varR{res\mathunderscore{}in/sr{:}binding[@name = \varName{v}]/*}} \Rightarrow \envElem{V}{}$}

      \UnaryInfC{$\dynEnvX{6}{nl}{\grammarRule{ExprSingle}} \Rightarrow \envElem{Res}{}$}

      \UnaryInfC{$\dynEnvX{5}{nl}{\begin{array}{l}
            \keyword{let}~\var{v} :=~\varR{res\mathunderscore{}in/sr{:}binding[@name = \varName{v}]/*}\\
            \ReturnExpr
          \end{array} \Rightarrow \envElem{Res}{}}$}
    \end{prooftreefunction}%
    %
    where 
    \begin{small}
      %
      $\dynX{6}{nl} = \dynX{5}{nl}\envExtend{varValue}{\mathit{v} \Rightarrow \envElem{V}{}} \enspace .$
      %
    \end{small}%
  \end{description}

  Consider~$\omg{xs}{out}$ and~$\omg{xs}{in}$ the solution sequences returned by the evaluation of the outer and inner
  \SparqlForClause{s} of~$\envElem{Q}{}$, respectively, and the set of join variables~$J = \gsup{\Vars}{out} \cap
  \vars{\gsup{\SparqlWhereClause}{in}}$.
  %
  Furthermore consider~$\sm{xs}{out} \in \omg{xs}{out}$ and~$\sm{xs}{in} \in \omg{xs}{in}$ the solution mappings that
  agree on the value of each join variable~$j \in J$ from where \envElem{Val}{} is generated, \ie~there exists some
  dynamic environment \dynX{i}{xs} based on \dyn and extended with the variable mappings from \sm{xs}{out} and
  \sm{xs}{in} such that \dynEnvX{i}{xs}{\ExprSingle \Rightarrow \envElem{Val}{}}.


  \paragraph*{Outer \SparqlForClause:} Regarding the \SparqlForClause of lines~(1)--(2) of~$\envElem{Q}{}$ (evaluated
  considering~$\dyn$), the~$\onl{\envElem{Q}{}}$ translates it into the~$\funcNameR{sparqlCall}$ from line~(2), which is
  evaluated over~$\dynX{1}{nl}$.
  %
  Consider~$C_1$ the expression context where~$\dynX{1}{nl}$ is included,~$\mu_{C_1}$ the XSPARQL instance mapping
  of~$C_1$ and~$\e{P}{}{out} = \fc{\mu_{C_1}}{\gsup{\SparqlWhereClause}{out}}$ the graph pattern obtained from replacing
  the variables in~$\gsup{\SparqlWhereClause}{out}$ according to~$\mu_{C_1}$.
  % 
  From~\eqref{tree:dynEvn2} we can see that~$\funcCall{dom}{\mu_{C_1}} = \dom{\mu_{C}} \cup
  \set{\varR{res\mathunderscore{}in}}$ but~$\varR{res\mathunderscore{}in}$ belongs to the~$\varR{}$ reserved namespace
  so it cannot be included in the variables of $\gsup{\SparqlWhereClause}{out}$ and we can observe that we obtain the
  same graph pattern~$\e{P}{}{out}$ by replacing~$\gsup{\SparqlWhereClause}{out}$ according to~$\mu_C$,
  \ie~$\e{P}{}{out} = \fc{\mu_{C_1}}{\gsup{\SparqlWhereClause}{out}} = \fc{\mu_{C}}{\gsup{\SparqlWhereClause}{out}}$.
  %
  Furthermore, let~$\omg{xs}{out} = \evalXS{\gsup{\DatasetClause}{out}}{\gsup{\SparqlWhereClause}{out}}{\mu_C}$ be the solution
  sequence resulting from evaluating the outer \SparqlForClause according to XSPARQL semantics and~$\omg{nl}{out} =
  \evalS{\gsup{\DatasetClause}{out}}{\gsup{P}{out}}$ be the pattern solution resulting from evaluating the rewritten
  outer \SparqlForClause according to SPARQL semantics.
  % 
  Following \cref{lem:replacement-equivalence}, we have that~$\omg{xs}{out} = \omg{nl}{out} \bowtie \set{\mu_C}$
  and, as we have seen from the proof of \cref{prop:sparqlCall}, since~$\mu_C$ is already included in~$\dyn$,
  we have that~$\omg{xs}{out} = \omg{nl}{out}$.


  \paragraph*{Inner \SparqlForClause:} The inner \SparqlForClause from lines~(3)--(4) of \envElem{Q}{} is evaluated
  considering some dynamic environment \dynX{i}{xs} (with expression context~$C_i$).
  %
  On the other hand, the~\onl{\envElem{Q}{}} translates this inner expression into the \funcNameR{sparqlCall} of
  line~(1), which is evaluated over the dynamic environment~$\dyn$ (with expression context~$C$).
  % 
  Consider~$\mu_C$ the XSPARQL instance mapping of~$C$ and~$\mu_{C_i}$ the XSPARQL instance mapping of~$C_i$.
  %
  Since \dynX{i}{xs} is an extension of \dyn we have that~$\dom{\mu_C} \subseteq \dom{\mu_{C_i}}$.
  % 
  Let~$\omg{xs}{in} = \evalXS{\gsup{\DatasetClause}{in}}{\gsup{\SparqlWhereClause}{in}}{\mu_{C_i}}$ be the solution
  sequence resulting from the evaluation of the inner \SparqlForClause of \envElem{Q}{} and the solution sequence
  resulting from the evaluation of the~$\funcNameR{sparqlCall}$ function be~$\omg{nl}{in} =
  \evalS{\gsup{\DatasetClause}{in}}{\gsup{P}{in}}$, where~$\gsup{P}{in} = \fc{\mu_C}{P}$ is the graph pattern obtained
  from replacing the variables in~$\gsup{\SparqlWhereClause}{in}$ according to~$\mu_C$.
  % 
  As~$\dom{\mu_C} \subseteq \dom{\mu_{C_i}}$, \ie~$\mu_C$ contains less bindings for variables than~$\mu_{C_i}$, the
  rewritten graph pattern~$\envElem{P}{in}$ contains more variables and we get that~$\omg{xs}{in} \preceq \omg{nl}{in}$.

  \medskip

  Since we know that~$\omg{nl}{out} = \omg{xs}{out}$ and~$\omg{xs}{in} \preceq \omg{nl}{in}$, we obtain that~$\sm{xs}{out}
  \in \omg{nl}{out}$ and~$\sm{xs}{in} \in \omg{nl}{in}$.  Since~$\onl{\envElem{Q}{}}$ performs a nested-loop
  iteration over~$\omg{nl}{out}$ and~$\omg{nl}{in}$, the~$\funcName{join_{sr}}$ function will  join the two solution
  mappings successfully since~$\sm{xs}{out}$ and~$\sm{xs}{in}$ share the same values for the join variables, and
  thus we have that~$\dynEnv \onl{\envElem{Q}{}} \Rightarrow \envElem{Val}{}$.


  \medskip \noindent ($\Leftarrow$)
  %
  We now proceed by showing that if~$\dynEnv \onl{\envElem{Q}{}} \Rightarrow \envElem{Val}{}$ then~$\dynEnv \envElem{Q}{}
  \Rightarrow \envElem{Val}{}$.
  %
  \noindent Let us turn to the evaluation of~$\dynEnv \envElem{Q}{} \Rightarrow \envElem{Val}{}$. 
  \begin{description}
  \item[\SparqlForClause from lines (1)--(2).] Considering that \grammarRule{Expr} corresponds to the \SparqlForClause
    from lines~(3)--(4) of~\envElem{Q}{}, the evaluation of this \SparqlForClause consists of the application of
    Rule~\eqref{eq:sparqlForClause}:

    \begin{prooftreefunction}
      \AxiomC{$\dyn.\ecomp{globalPosition} = \seq{ \envElem{Pos}{1}, \cdots, \envElem{Pos}{m} } $}
      \UnaryInfC{$\dynEnv{
          \funcCall{fs{:}dataset}{\gsup{\DatasetClause}{out}}
          \Rightarrow \gsup{DS}{out}
        }$}
      \UnaryInfC{$\dynEnv{
          \funcCall{fs{:}sparql}{
            \begin{array}{l}
              \gsup{DS}{out},~\SparqlWhereClause, \\ 
              \SolutionModifier 
            \end{array}}
          \Rightarrow \sm{i}{}}$}
      \UnaryInfC{$\dynX{1}{xs} \proofs \grammarRule{Expr} \Rightarrow \envElem{Value}{i}$}

      \AxiomC{$\Ddots$}

      \BinaryInfC{$\dynEnv{\begin{array}{l}
            \FOR~\gsup{\Vars}{out}~\gsup{\DatasetClause}{out}\\
            \gsup{\SparqlWhereClause}{out}~\gsup{\SolutionModifier}{out}\\
            \RETURN~\grammarRule{Expr}
          \end{array} \Rightarrow \envElem{Value}{1}, \dotsb, \envElem{Value}{m}}$}

    \end{prooftreefunction}%
    % 
    with~$\gsup{\Vars}{out} = \e{\var{\!Var}}{\!1}{out}\dotsb\e{\var{\!Var}}{n}{out}$, we have for each~$\sm{i}{}$
    \begin{tree}
      $\dynX{1}{xs} = \begin{array}{l}
        \dyn \envExtend{activeDataset}{\gsup{DS}{out}}\envExtend{globalPosition}{\seq{ \envElem{Pos}{1}, \cdots, \envElem{Pos}{m}, i }}\\
        \envExtend{varValue}{\begin{array}{l}
            \e{Var}{1}{out} \Rightarrow \funcCall{fs{:}value}{\sm{i}{}, \e{Var}{1}{out}}; \\
            \dotsc;\\
            \e{Var}{n}{out} \Rightarrow \funcCall{fs{:}value}{\sm{i}{}, \e{Var}{n}{out}}
          \end{array}
          }
      \end{array}  \enspace~$
      \label{proof:dynenv_i}
    \end{tree}
    
  \item[\SparqlForClause of lines (3)--(4).] The evaluation of~$\dynX{1}{xs}$~$\proofs \grammarRule{Expr} \Rightarrow
    \envElem{Value}{i}$ is given by:

    \begin{prooftreefunction}
      % 
      \AxiomC{$\dyn.\ecomp{globalPosition} = \seq{ \envElem{Pos}{1}, \cdots, \envElem{Pos}{m} } $}
      \UnaryInfC{$ \dynEnvX{1}{xs}{
          \funcCall{fs{:}dataset}{\gsup{\DatasetClause}{in}} 
          \Rightarrow \gsup{DS}{in}
        }$}
      \UnaryInfC{$\dynEnvX{1}{xs}{
          \funcCall{fs{:}sparql}{\!\!\begin{array}{l}
              \gsup{DS}{in}, \gsup{\SparqlWhereClause}{in}, \\
              \gsup{SolutionModifier}{in}
            \end{array}\!\!}
          \Rightarrow \sm{j}{}}$}
      \UnaryInfC{$\dynEnvX{2}{xs}{ \ExprSingle \Rightarrow \envElem{Value}{j}}$}

      \AxiomC{$\Ddots$}

      \BinaryInfC{$\dynEnvX{1}{xs}{\begin{array}{l}
            \FOR~\gsup{\Vars}{in}~\gsup{\DatasetClause}{in}\\
            \gsup{\SparqlWhereClause}{in}~\gsup{\mathit{SolutionModifier}}{in}\\
            \RETURN~\ExprSingle
          \end{array} \Rightarrow \envElem{Value}{1} \dotsb \envElem{Value}{m}}$}
    \end{prooftreefunction}%
    % 
    where, considering~$\gsup{\Vars}{in} = \e{\var{Var}}{1}{in}\dots\e{\var{Var}}{n}{in}$, we have for each~$\sm{j}{}$
    \begin{small}
      \begin{equation*}
        \ \ \dynX{2}{xs} = \begin{array}{l}
          \dynX{1}{xs} \envExtend{activeDataset}{\gsup{DS}{in}}\envExtend{globalPosition}{\seq{ \envElem{Pos}{1}, \cdots, \envElem{Pos}{m}, j }}\\
          \envExtend{varValue}{\!\begin{array}{l}
              \e{Var}{1}{in} \Rightarrow \funcCall{fs{:}value}{\sm{j}{}, \e{Var}{1}{in}}; \\
              \dotsb;\\
              \e{Var}{n}{in} \Rightarrow \funcCall{fs{:}value}{\sm{j}{},\e{Var}{n}{in}}
            \end{array}\!\!} \enspace .
        \end{array}
      \end{equation*}
    \end{small}%
  \end{description}
  % 
  Let~$\omg{nl}{out}$ and~$\omg{nl}{in}$ be the pattern solutions returned by the outer and inner \SparqlForClause{s},
  respectively, and let~$\sm{nl}{out} \in \omg{nl}{out}$ and~$\sm{nl}{in} \in \omg{nl}{in}$ be the solution mappings.
  Without loss of generality we can assume these are the solution mappings from where~$\envElem{Val}{}$ is deduced,
  \ie~\sm{nl}{out} and~\sm{nl}{in} are compatible.
  %
  We also know that there exists a dynamic environment \dynX{}{nl}, based on \dyn and extended with the variable
  mappings~$\sm{nl}{out}$ and~$\sm{nl}{in}$ such that~$\dynEnvX{}{nl}{\ExprSingle \Rightarrow \envElem{Val}{}}$.
  
  As we know from the ($\Rightarrow$) direction of the proof,~$\omg{nl}{out} = \omg{xs}{out}$ and so we have
  that~$\sm{nl}{out} \in \omg{xs}{out}$.  Regarding the evaluation of the inner \SparqlForClause we also know
  that~$\omg{xs}{in} \preceq \omg{nl}{in}$ and as such, we must consider two cases:
  \begin{inparaenum}[(i)]
  \item\label{en:inxsin}~$\sm{nl}{in} \in \omg{xs}{in}$ or
  \item\label{en:notinxsin}~$\sm{nl}{in} \not\in \omg{xs}{in}$.
  \end{inparaenum}
  % 
  From~\eqref{en:inxsin}, we immediately get the desired result that~$\dynEnv \envElem{Q}{} \Rightarrow
  \envElem{Val}{}$.
  % 
  For~\eqref{en:notinxsin}, we know from~\eqref{proof:dynenv_i} that the inner \SparqlForClause is executed over
  \dynX{1}{xs} (and the respective XSPARQL instance mapping \sm{C_1}{xs}), which include the bindings for variables from
  each solution mapping~$\sm{i}{} \in \omg{xs}{out}$.
  %
  Thus, according to the XSPARQL \ac{BGP} matching (\cf~\cref{def:xsparql-bgp-matching}), \omg{xs}{in} will
  contain all the solution mappings that are compatible with any solution mapping~$\sm{i}{} \in \omg{xs}{out}$ and,
  since~$\sm{nl}{out} \in \omg{xs}{out}$, specifically those compatible with \sm{nl}{out}.  
  %
  However, we know that \sm{nl}{in} is compatible with \sm{nl}{out} and thus we have that \sm{nl}{in} must also belong
  to \omg{xs}{in} and we can deduce that~$\dynEnv \envElem{Q}{} \Rightarrow \envElem{Val}{}$.

  \paragraph*{Inductive Step.}
  % 
  The proof follows from the recursive application of the base case, over a new dynamic environment determined by the
  \funcName{opt_{nl}} rewriting to $\dynEnvX{i}{}{\onl{\ExprSingle}}$.
  % 

  \medskip

  The proof for nested queries with an XQuery \FOR outer expression~\eqref{eq:expr_sdep_xquery} is analogous where, in
  the preceding, the evaluation of the \SparqlForClause from lines~(1)--(2) of~\eqref{eq:expr_sdep} is replaced by the
  evaluation of an XQuery \ForClause, as presented by~\cite[Section~4.8.2]{DraperFankhauserFernandez:2010aa}.
  %
\end{proof}


%%% Local Variables:
%%% fill-column: 120
%%% TeX-master: "../thesis" 
%%% TeX-PDF-mode: t
%%% TeX-debug-bad-boxes: t
%%% TeX-parse-self: t
%%% TeX-auto-save: t
%%% reftex-plug-into-AUCTeX: t
%%% End:


\subsection{Dependent Join implementation in SPARQL}
\label{sec:join-impl-sparql}

This form of rewriting of nested expressions aims at improving the runtime of the query by delegating the execution of
the join to the SPARQL engine, as opposed to performing the join within XQuery (as in the previous optimisation).
%
We start by presenting the rewriting function for the case when both nested expressions are \SparqlForClause{s}:
%
for such nested expressions we can implement the join by rewriting the \SparqlForClause{}s into a single SPARQL query.

\subsubsection{\SparqlForClause within a \SparqlForClause}
\label{sec:query-rewriting}
%
The idea with these rewritings is that nested \SparqlForClause{s} in XSPARQL can be implemented by a SPARQL query that
merges the \WHERE clauses of the outer and inner \SparqlForClause.
%
However, there are some restrictions to the applicability of this rewriting: 
\begin{inparaenum}[(i)]
\item both queries must be done over the same dataset;
\item apart from \ORDERBY, no other solution modifiers can be used in the queries; and
\item the original queries must not require any nesting of the \ac{XML} output or use of aggregators.
\end{inparaenum}
%
The use of aggregators is restricted since in SPARQL queries they are only possible in the not yet standardised
SPARQL~1.1.  Thus it is not possible to generate the nested \ac{XML} structure required by some queries, for example the
query presented in \cref{fig:query09}, by using a single SPARQL query or alternatively further processing of the
SPARQL results in XQuery.
%
As indicated before, for the next rewriting we are only allowing the \ORDERBY solution modifier and the concatenation of
``\ORDERBY \$o1'' and ``\ORDERBY \$o2'' is ``\ORDERBY \$o1 \$o2''.


For an XSPARQL query~$\envElem{Q}{}$ of form:\footnote{For presentation purposes,~$\mathit{GGP}$ and $\mathit{OC}$ are a
  short representation for~$\mathit{GroupGraphPattern}$ and~$\mathit{OrderCondition}$, respectively.}
%
\begin{queryF}
\[\begin{array}{l}
  \mathrm{(1)~~~~}\FOR~\gsup{\Vars}{out}~\DatasetClause~\WHERE~\gsup{\GroupGraphPattern}{out}~\ORDERBY~\gsup{\OrderCondition}{out}\\
  \mathrm{(2)~~~~}\RETURN\\
  \mathrm{(3)~~~~}\quad \FOR~\gsup{\Vars}{in}~\DatasetClause~\WHERE~\gsup{\GroupGraphPattern}{in}~\ORDERBY~\gsup{\OrderCondition}{in}\\
  \mathrm{(4)~~~~}\quad \ReturnExpr
\end{array}\]
\label{eq:expr_sr}
\end{queryF}%
%
\noindent then 
%
\begin{itemize}
\item in case \envElem{\ExprSingle}{} does not contain any occurrences of~\eqref{eq:expr_sr}, we have that:
\begin{small}
\begin{equation*}
\begin{array}{l}
  \funcCall{opt_{sr}}{\envElem{Q}{}} = \\[.3em]
    \mathrm{(1)~~~~}\LET~\varR{results}~\mathtt{:=}~\funcCallR{sparqlCall}{
                    \begin{array}{l}
                      \SELECT~\gsup{\Vars}{out}\cup\gsup{\Vars}{in}~\DatasetClause\\
                      \WHERE~\set{\gsup{\GroupGraphPattern}{out}~.~\gsup{\GroupGraphPattern}{in}}\\
                      \ORDERBY~\gsup{\OrderCondition}{out}~\gsup{\OrderCondition}{in}\\
                    \end{array}}~\RETURN\\
    \mathrm{(2)~~~~} \FOR~\varR{result}~\AT~\varR{posvar}~\IN~\varR{results//sr{:}result}~\RETURN \\
    \mathrm{(3)~~~~} \quad \LET~\var{v} := \varR{result/sr{:}binding[@name = \var{v}]/*}~\RETURN \hfill\qquad\quad \textrm{\smaller for each \var{v}~$\in \gsup{\Vars}{out}\cup\gsup{\Vars}{in}$}\\
    \mathrm{(4)~~~~} \quad\grammarRule{ExprSingle} \\
 \end{array}
\end{equation*}
\end{small}%
%
Please note that the group graph patterns~$\e{GGP}{1}{}$ and~$\e{GGP}{2}{}$ include the surrounding curly braces: \{ and
\}.
\item otherwise:
%
\begin{small}
\begin{equation*}
\begin{array}{r@{~~~}l}
  \multicolumn{2}{l}{\sr{\envElem{Q}{}} =}\\[.3em]
   & \sr{\begin{array}{l}
       \FOR~\gsup{\Vars}{out}~\DatasetClause~\WHERE~\gsup{\GroupGraphPattern}{out}~\ORDERBY~\gsup{\OrderCondition}{out}\\
       \RETURN\\
       \quad \FOR~\gsup{\Vars}{in}~\DatasetClause~\WHERE~\gsup{\GroupGraphPattern}{in}~\ORDERBY~\gsup{\OrderCondition}{in}\\
       \quad \RETURN~\sr{\envElem{ExprSingle}{}}
     \end{array}
   }
 \end{array}
\end{equation*}
\end{small}%
\end{itemize}
%
\begin{restatable}{proposition}{srCorrectRestatable}
  \label{prop:sr-correct}
  Let \envElem{Q}{} an XSPARQL expression of form~\eqref{eq:expr_sr} and \dyn the dynamic environment of \envElem{Q}{},
  then~$\dynEnv{\envElem{Q}{} \Rightarrow \envElem{Val}{}}$ if and only if~$\dynEnv{\funcCall{opt_{sr}}{\envElem{Q}{}}
    \Rightarrow \envElem{Val}{}}$.
\end{restatable}
%
\begin{proof}
  % 
  We start by showing the proof for the base case, where \ExprSingle of~\eqref{eq:expr_sr} does not contain any
  occurrences of~\eqref{eq:expr_sr}.
  % 
  \paragraph*{Base Case.}
  ($\Rightarrow$)
  % 
  We start by showing that if~$\dynEnv \grammarRule{\envElem{Q}{}} \Rightarrow \envElem{Val}{}$ then~$\dynEnv
  \sr{\envElem{Q}{}}$~$\Rightarrow \envElem{Val}{}$.
  % 
  Next, we show the proof tree for each of the XQuery core expressions in each line of the~$\funcName{opt_{sr}}$
  rewriting where, for each line, \grammarRule{Expr} represents the expressions of the following lines.
  % 
  \begin{description}
  \item[\LET expression of line (1):]~
    % 
    \begin{prooftreefunction}
      % 
      \AxiomC{$\dynEnv{\begin{array}{c}
            \funcCallR{sparqlCall}{
              \begin{array}{l}
                \SELECT~\gsup{\Vars}{out}\cup\gsup{\Vars}{in}\\
                \DatasetClause\\
                \WHERE~\set{\gsup{\GroupGraphPattern}{out}~.~\gsup{\GroupGraphPattern}{in}}\\
                \ORDERBY~\gsup{\OrderCondition}{out}~\gsup{\OrderCondition}{in}\\
              \end{array}} 
          \end{array}
        } \Rightarrow \omg{sr}{}$}
      % 
      \UnaryInfC{$\dynEnvX{1}{sr}{\grammarRule{Expr}} \Rightarrow \envElem{Res}{}$}
      
      \UnaryInfC{$\dynEnv{\begin{array}{l}
            \LET~\varR{results}~:= \\
            \quad\funcCallR{sparqlCall}{
              \begin{array}{l}
                \SELECT~\gsup{\Vars}{out}\cup\gsup{\Vars}{in}~\DatasetClause\\
                \WHERE~\set{\gsup{\GroupGraphPattern}{out}~.~\gsup{\GroupGraphPattern}{in}}\\
                \ORDERBY~\gsup{\OrderCondition}{out}~\gsup{\OrderCondition}{in}\\
              \end{array}} \\ 
            \RETURN~\grammarRule{Expr}
          \end{array} \Rightarrow \envElem{Res}{}}$}
      
    \end{prooftreefunction}%
    where 
    \begin{small}
      $\dynX{1}{sr} = \dyn\!\envExtend{varValue}{\varNameR{results} \Rightarrow \omg{sr}{}} \enspace .$
    \end{small}%

  \item[\FOR expression of line (2):]~
    \begin{prooftreefunction}

      \AxiomC{$\dynEnvX{1}{sr}{\varR{results//sr{:}result}} \Rightarrow \sm{i}{}$}
      \UnaryInfC{$\dynEnvX{2}{sr}{\grammarRule{ExprSingle}} \Rightarrow \envElem{Res}{i}$}

      \AxiomC{$\Ddots$}
      
      \BinaryInfC{$\dynEnvX{1}{sr}{
          \begin{array}{l}
            \FOR~\varR{result}~\AT~\varR{posvar}\\
            \quad\IN~\varR{results//sr{:}result}~\\
            \ReturnExpr 
          \end{array}}\Rightarrow \envElem{Res}{1},\dotsb,\envElem{Res}{n}$}
      
    \end{prooftreefunction}%
    % 
    where 
    \begin{small}
      $\dynX{2}{sr} = \dynX{1}{sr} \envExtend{varValue}{\begin{array}{l}
          \varNameR{result} \Rightarrow \sm{i}{};\\
          \varNameR{posvar} \Rightarrow i
        \end{array}
        } \enspace .$
    \end{small}  

  \item[\LET expressions of lines (3)--(4).] Here we consider all the \LET expressions represented by line~(3),
    where~$\var{v} \in \gsup{\Vars}{out}\cup\gsup{\Vars}{in}$:
    % 
    \begin{prooftreefunction}
      % 
      \AxiomC{$\dynEnvX{2}{sr}{\varR{result/sr{:}binding[@name = \var{v}]/*}} \Rightarrow \envElem{V}{}$}

      \UnaryInfC{$\dynEnvX{3}{sr}{\grammarRule{ExprSingle}} \Rightarrow \envElem{Res}{}$}

      \UnaryInfC{$\dynEnvX{2}{sr}{\begin{array}{l}
            \LET~\var{v} := \varR{result/sr{:}binding[@name = \varName{v}]/*} \\
            \ReturnExpr
          \end{array} \Rightarrow \envElem{Res}{}}$}

    \end{prooftreefunction}%
    %
    where
    \begin{small}
        $\dynX{3}{sr} = \dynX{2}{sr} \envExtend{varValue}{\envElem{v}{} \Rightarrow \envElem{V}{}}\enspace .$
    \end{small}%
  \end{description}


  Let~$\omg{xs}{out}$ and~$\omg{xs}{in}$ be the solution sequences returned by the evaluation of the outer and inner
  \SparqlForClause{s} of~$\envElem{Q}{}$, respectively. 
  % 
  Furthermore, let~$\sm{xs}{out} \in \e{\Omega}{xs}{out}$ and~$\sm{xs}{in} \in \e{\Omega}{xs}{in}$ be compatible
  solution mappings and \dynX{i}{expr} the dynamic environment that results from extending \dyn with the variable
  mappings from~$\sm{xs}{out}$ and~$\sm{xs}{in}$, such that \dynEnvX{i}{expr}{\ExprSingle \Rightarrow \envElem{Val}{}}.


  According to the SPARQL semantics, the solution sequence that results from evaluating the graph
  pattern~``$\set{\gsup{\GroupGraphPattern}{out}~.~\gsup{\GroupGraphPattern}{in}}$'',~$\omg{sr}{} = \omg{sr}{out} \bowtie
  \omg{sr}{in}$ consists of all the solution mappings~$\sm{sr}{out} \in \omg{sr}{out}$ and~$\sm{sr}{in} \in
  \omg{sr}{in}$ such that~$\sm{sr}{out}$ and~$\sm{sr}{in}$ are \emph{compatible}.
  % 
  The evaluation of the outer \SparqlForClause (lines~(1)--(2) of \envElem{Q}{}), evaluated over \dyn, is translated by
  \sr{\envElem{Q}{}} into the \funcNameR{sparqlCall} from line~(1), which is also evaluated over \dyn.
  %
  In this case, according to \cref{lem:replacement-equivalence}, we have that~$\e{\Omega}{sr}{out} =
  \e{\Omega}{xs}{out}$ and then~$\sm{xs}{out} \in \e{\Omega}{sr}{out}$.
  

  The inner \SparqlForClause (lines~(3)--(4) of \envElem{Q}{}), which is evaluated over some dynamic environment
  \dynX{i}{xs}, is incorporated by the \sr{\envElem{Q}{}} rewriting into the \funcNameR{sparql\-Call} from line~(1),
  which is also evaluated over \dyn.
  %
  Considering that \dyn is less restrictive than \dynX{i}{xs}, \ie~\dyn contains less bindings for variables than
  \dynX{i}{xs}, and thus the evaluation of the inner \SparqlForClause over \dyn will contain all the solution mappings
  from \omg{xs}{in} and specifically~$\sm{xs}{in}$.
  % 
  As~$\sm{xs}{out}$ and~$\sm{xs}{in}$ are \emph{compatible} we have that~$\dynEnv \sr{\envElem{Q}{}} \Rightarrow
  \envElem{Val}{}$.


  Please note that we are only considering \ORDERBY solution modifiers, thus the number of results of each query is not
  modified.  The ordering of the results may be changed but this does not interfere with this proof and solution
  modifiers can be safely ignored.


  \bigskip 
  \noindent ($\Leftarrow$)
  %
  Next we show that if~$\dynEnv \sr{\envElem{Q}{}}$~$\Rightarrow \envElem{Val}{}$ then~$\dynEnv \envElem{Q}{}
  \Rightarrow \envElem{Val}{}$.
  %
  Let us turn to the evaluation of~$\dynEnv \envElem{Q}{} \Rightarrow \envElem{Val}{}$.
  \begin{description}[noitemsep]
    % 
  \item[\SparqlForClause from lines (1)--(2).] Where \grammarRule{Expr} corresponds to the \SparqlForClause from
    lines~(3)--(4) of \envElem{Q}{}.  The evaluation of this \SparqlForClause consists of the application of
    Rule~\eqref{eq:sparqlForClause}:
    % 
    \begin{prooftreefunction}
      \AxiomC{$\dyn.\ecomp{globalPosition} = \seq{ \envElem{Pos}{1}, \cdots, \envElem{Pos}{m} } $}
      \UnaryInfC{$\dynEnv{\funcCall{fs{:}dataset}{\DatasetClause} \Rightarrow \grammarRule{DS}}$}
      \UnaryInfC{$\dynEnv{\funcCall{fs{:}sparql}{
            \begin{array}{l}
              \grammarRule{DS},\ \WHERE~\gsup{\GroupGraphPattern}{out}\\
              \ORDERBY~\gsup{\OrderCondition}{out}\\
            \end{array}} \Rightarrow \sm{i}{}}$}
      \UnaryInfC{$\dynX{1}{xs} \proofs \grammarRule{Expr} \Rightarrow \envElem{Value}{i}$}

      \AxiomC{$\Ddots$}

      \BinaryInfC{$\dynEnv{\begin{array}{l}
            \FOR~\gsup{\Vars}{out}~\DatasetClause\\
            \WHERE~\gsup{\GroupGraphPattern}{out}~\ORDERBY~\gsup{\OrderCondition}{out}\\
            \RETURN~\grammarRule{Expr}
          \end{array}} \Rightarrow \envElem{Value}{1}, \dotsb, \envElem{Value}{m}$}

    \end{prooftreefunction}%
    % 
    where~$\gsup{\Vars}{out} = \e{\var{Var}}{1}{out}\dotsb\e{\var{Var}}{n}{out}$, we have for each~$\sm{i}{}$
    \begin{tree}
      $\dynX{1}{xs} = \begin{array}{l}
        \dyn \envExtend{activeDataset}{\grammarRule{DS}\,}\envExtend{globalPosition}{\seq{ \envElem{Pos}{1}, \cdots, \envElem{Pos}{m}, i }}\\
        \envExtend{varValue}{\begin{array}{l}
            \e{Var}{1}{out} \Rightarrow \funcCall{fs{:}value}{\sm{i}{}, \e{Var}{1}{out}}; \\
            \dotsb;\\
            \e{Var}{n}{out} \Rightarrow \funcCall{fs{:}value}{\sm{i}{}, \e{Var}{n}{out}}
          \end{array}
        }
      \end{array} \ .$
      \label{proof:dynenv_i_sr}
    \end{tree}
    % 

  \item[\SparqlForClause of lines (3)--(4).] The evaluation of~$\dynX{i}{xs}$~$\proofs \gsup{ExprSingle}{out}
    \Rightarrow \envElem{Value}{i}$ is shown next:

    \begin{prooftreefunction}

      \AxiomC{$\dyn.\ecomp{globalPosition} = \seq{ \envElem{Pos}{1}, \cdots, \envElem{Pos}{m} } $}
      \UnaryInfC{$ \dynEnvX{1}{xs}{
          \funcCall{fs{:}dataset}{\grammarRule{\DatasetClause}} \Rightarrow \grammarRule{DS}
        }$}
      \UnaryInfC{$\dynEnvX{1}{xs}{
          \funcCall{fs{:}sparql}{\!\!\begin{array}{l}
              \grammarRule{DS},~\WHERE~\gsup{\GroupGraphPattern}{in}\\
              \ORDERBY~\gsup{\OrderCondition}{in}\\
            \end{array}\!\!}  \Rightarrow \sm{j}{}}$}
      \UnaryInfC{$\dynX{2}{xs} \proofs \ExprSingle \Rightarrow \envElem{Value}{j}$}

      \AxiomC{$\Ddots$}

      \BinaryInfC{$\dynEnvX{1}{xs}{\begin{array}{l}
            \FOR~\gsup{\Vars}{in}~\DatasetClause\\
            \WHERE~\gsup{\GroupGraphPattern}{in}~\ORDERBY~\gsup{\OrderCondition}{in}\\
            \RETURN~\ExprSingle
          \end{array}\!\!\!}~\Rightarrow \envElem{Value}{1}, \dotsb, \envElem{Value}{m}$}
    \end{prooftreefunction}%
    % 
    where~$\gsup{\Vars}{in} = \e{\var{Var}}{1}{in}\dotsb\e{\var{Var}}{n}{in}$, for each~$\sm{j}{}$ we have that:
    \begin{small}
      \begin{equation*}
        \ \  \dynX{2}{xs} = \begin{array}{l}
          \dynX{1}{xs} \envExtend{activeDataset}{\grammarRule{DS}}\envExtend{globalPosition}{\seq{ \envElem{Pos}{1}, \cdots, \envElem{Pos}{m}, j }}\\
          \envExtend{varValue}{\!\begin{array}{l}
              \e{Var}{1}{in} \Rightarrow \funcCall{fs{:}value}{\sm{j}{}, \e{Var}{1}{in}}; \\
              \dotsb;\\
              \e{Var}{n}{in} \Rightarrow \funcCall{fs{:}value}{\sm{j}{},\e{Var}{n}{in}}
            \end{array}\!\!}
        \end{array}\ .
      \end{equation*}
    \end{small}%
  \end{description}

  As we have seen in the ($\Rightarrow$) direction, we have that~$\omg{sr}{out} = \omg{xs}{out}$ and so we have
  that~$\sm{sr}{out} \in \omg{xs}{out}$.
  % 
  Furthermore let~$\omg{sr}{out}$ and~$\omg{sr}{in}$ be as per the~($\Rightarrow$) direction of the proof.
  % 
  As we have seen,~$\Omega_{sr}$ contains all the solution mappings~$\sm{}{} = \sm{sr}{out} \bowtie \sm{sr}{in}$ such
  that~$\sm{sr}{out} \in \omg{sr}{out}$ and~$\sm{sr}{in} \in \omg{sr}{in}$ and~$\sm{sr}{out}$ and~$\sm{sr}{in}$ are
  \emph{compatible}.
  % 
  Without loss of generality let us consider~$\sm{sr}{out}$ and~$\sm{sr}{in}$ the solution mappings
  where~$\envElem{Val}{}$ is deduced from.


  Let~$C$ be the expression context where~$\dyn$ is included and~$\mu_C$ the XSPARQL instance mapping of~$C$.
  Furthermore, let~$\gsup{P}{in} = \fc{\mu_C}{\gsup{\GroupGraphPattern}{in}}$ be the graph pattern obtained from
  replacing the variables in~$\gsup{\GroupGraphPattern}{in}$ according to~$\mu_C$.
  Since~$\vars{\gsup{\GroupGraphPattern}{in}} \subseteq \vars{\gsup{P}{in}}$ all solutions mappings returned by
  evaluating~$\gsup{\GroupGraphPattern}{in}$ under XSPARQL semantics are included in the solution sequence of
  evaluating~$\e{P}{}{in}$ under SPARQL semantics \ie~$\omg{xs}{in} \preceq \omg{sr}{in}$.
  % 
  We obtain two cases:
  \begin{inparaenum}[(i)]
  \item\label{en:inxsin_sr}~$\sm{sr}{in} \in \omg{xs}{in}$ or
  \item\label{en:notinxsin_sr}~$\sm{sr}{in} \not\in \omg{xs}{in}$.
  \end{inparaenum}
  % 
  From~\eqref{en:inxsin_sr} we immediately get that~$\dynEnv \envElem{Q}{} \Rightarrow \envElem{Val}{}$.
  % 
  For~\eqref{en:notinxsin_sr}, consider \sm{C_1}{xs} the XSPARQL instance of the inner \SparqlForClause (created based
  on \dynX{1}{xs}). As we can see from~\eqref{proof:dynenv_i_sr}, \dynX{1}{xs} (and thus also \sm{C_1}{xs}) includes the
  bindings for variables from each solution mapping~$\sm{i}{} \in \omg{xs}{out}$.  Thus, according to the XSPARQL \ac{BGP}
  matching (\cf~\cref{def:xsparql-bgp-matching}), \omg{xs}{in} will contain all the solution mappings that are
  compatible with any solution mapping~$\sm{i}{} \in \omg{xs}{out}$ and specifically those compatible with \sm{sr}{out}.
  Since we know that \sm{sr}{in} is compatible with \sm{sr}{out}, we have that \sm{sr}{in} must belong to \omg{xs}{in},
  thus we can deduce that~$\dynEnv \envElem{Q}{} \Rightarrow \envElem{Val}{}$.

  \paragraph*{Inductive Step.}
  % 
  The proof follows from the recursive application of the base case, over a new dynamic environment determined by the
  \funcName{opt_{sr}} rewriting to $\dynEnvX{i}{}{\sr{\ExprSingle}}$.
  % 
\end{proof}



%%% Local Variables:
%%% fill-column: 120
%%% TeX-master: "../thesis" 
%%% TeX-PDF-mode: t
%%% TeX-debug-bad-boxes: t
%%% TeX-parse-self: t
%%% TeX-auto-save: t
%%% reftex-plug-into-AUCTeX: t
%%% End:




\subsubsection{\SparqlForClause within an XQuery \FOR}
\label{sec:named-graphs}
%
In case the outer expression is an XQuery \FOR or an XSPARQL \SQLForClause a similar strategy of deferring the join to a
single SPARQL query is still possible.  
%
This optimisation relies on first transforming the outer expressions' \ac{XML} results into \ac{RDF} and then joining this newly created
\ac{RDF} graph with the inner \SparqlForClause's \WHERE pattern in a single SPARQL query.  
%
For the implementation of this optimisation we can rely on a triple store with support for named graphs and temporarily
store the bindings for dependent variables from the outer XQuery \FOR expression's as \ac{RDF} triples.
%
We can then execute a combined query with an adapted graph pattern, that joins the pattern in the \WHERE clause of the
inner \SparqlForClause with the bindings stored in the newly created named graph.
%
The~$\funcName{opt_{ng}}$ rewriting function (presented below) starts by creating \ac{RDF} triples representing the
\ac{XML} input, which are then collected into the variable~$\varR{ds}$ corresponding to the \ac{RDF} graph to be
inserted into the triple store.  This operation is achieved by the XSPARQL functions~$\funcNameR{createNG}$ that returns
a URI for the newly inserted \ac{RDF} named graph, which is distinct from any other URIs for named graphs used in the
query or present in the triple store, while finally the function~$\funcNameR{deleteNG}$ takes care of deleting the
temporary graph.
%
We will show this optimisation only for the case where the outer expression is an XQuery \FOR, the case of an outer
XSPARQL \SQLForClause expression is analogous. 
%
Let~$\envElem{Q}{}$ be an XSPARQL expression of form

\begin{queryF}
\[\begin{array}{l}
  \mathrm{(1)~~~~}\FOR~\var{VarName}~\grammarRule{OptTypeDeclaration}~\grammarRule{OptPositionalVar}~\IN~\envElem{\ExprSingle}{1}\\
  \mathrm{(2)~~~~}\RETURN \\
  \mathrm{(3)~~~~}\quad\FOR~\Vars~\DatasetClause~\SparqlWhereClause~\SolutionModifier\\
  \mathrm{(4)~~~~}\quad\RETURN~\envElem{\ExprSingle}{2}
  \end{array}\]
\label{eq:expr_ng}
\end{queryF}%
%
then
%
\begin{itemize}
\item in case \envElem{\ExprSingle}{1} and \envElem{\ExprSingle}{2} do not contain any occurrences of~\eqref{eq:expr_ng},
    we have that:
%
\begin{small}
\begin{equation*}
\begin{array}{l}
  \ong{\envElem{Q}{}} =\\[.3em]
    \mathrm{(1)~~~~}\LET~\varR{ds}~:=~\funcCallR{createNG}{\begin{array}{l}
                      \FOR~\var{VarName}~\grammarRule{OptTypeDeclaration}\\
                      \grammarRule{OptPositionalVar}~\IN~\envElem{\ExprSingle}{1} \\
                      \RETURN~\funcCallR{evalCT}{ \envElem{NGP}{}}\\
                    \end{array}}~\RETURN\\
    \mathrm{(2)~~~~}\LET~\varR{results}~:= \\
    \qquad\qquad\funcCallR{sparqlCall}{
                    \begin{array}{l}
                      \SELECT~\Vars \cup \set{\var{VarName}} \\
                      \DatasetClause~\cup \set{ \FROMNAMED~\varR{ds} }\\
                      \SparqlWhereClause~\cup \WHERE\set{ \keyword{graph}~\varR{ds}~\envElem{NGP}{} } \\
                      \SolutionModifier
                    \end{array}}~\RETURN\\
    \mathrm{(3)~~~~} \FOR~\varR{result}~\AT~\varR{result\mathunderscore{}pos}~\IN~\varR{results//sr{:}result}~\RETURN \\
    \mathrm{(4)~~~~} \quad \LET~\var{v} :=~ \varR{result/sr{:}binding[@name = \var{v}]/*} \hfill \qquad \textrm{\smaller for each \var{v}~$\in \Vars \cup \set{\var{VarName}}$}\\
    \mathrm{(5)~~~~}\quad\RETURN \left(\envElem{\ExprSingle}{2}, \funcCallR{deleteNG}{\varR{ds}}\right)
\end{array}
\end{equation*}
\end{small}%
%
where~$\envElem{NGP}{}$ is the graph pattern~$\set{ \mathtt{[]~{:}value}~\var{VarName} }$.
%
\item otherwise:
%
\begin{small}
\begin{equation*}
\begin{array}{r@{~~~}l}
  \multicolumn{2}{l}{\funcCall{opt_{ng}}{\envElem{Q}{}} =}\\[.3em]
  & \ong{\begin{array}{l}
      \FOR~\var{VarName}~\grammarRule{OptTypeDeclaration}~\grammarRule{OptPositionalVar}~\IN~\ong{\envElem{\ExprSingle}{1}}\\
      \RETURN \\
      \quad\FOR~\Vars~\DatasetClause~\SparqlWhereClause~\SolutionModifier\\
      \quad\RETURN~\ong{\envElem{\ExprSingle}{2}}
  \end{array}
}
 \end{array}
\end{equation*}
\end{small}%
\end{itemize}



Let~$\envElem{Q}{}$ be an XSPARQL expression of form
%
\begin{queryF}
\[\begin{array}{l}
  \mathrm{(1)~~~~}\FOR~\envElem{AttrSpec}{1}~\keyword{as}~\envElem{\var{Var}}{1}, \dots, \envElem{AttrSpec}{n}~\keyword{as}~\envElem{\var{Var}}{n}~\grammarRule{RelationList}~\grammarRule{SQLWhereClause}\\
  \mathrm{(2)~~~~}\RETURN \\
  \mathrm{(3)~~~~}\quad\FOR~\Vars~\DatasetClause~\SparqlWhereClause~\SolutionModifier\\
  \mathrm{(4)~~~~}\quad\RETURN~\ExprSingle
  \end{array}\]
\label{eq:expr_ng_sql}
\end{queryF}%
%
then
%
\begin{itemize}
\item in case \ExprSingle  does not contain any occurrences of~\eqref{eq:expr_ng_sql},
    we have that:
%
\begin{small}
\begin{equation*}
\begin{array}{l}
  \ong{\envElem{Q}{}} =\\[.3em]
  \mathrm{(1)~~~~}\LET~\varR{ds}~:=\\
  \qquad\qquad\funcCallR{createNG}{
      \funcCall{tr}{\begin{array}{l}
          \FOR~\envElem{AttrSpec}{1}~\keyword{as}~\envElem{\var{Var}}{1},\dots,\envElem{AttrSpec}{n}~\keyword{as}~\envElem{\var{Var}}{n}\\
          \grammarRule{RelationList}~\grammarRule{SQLWhereClause}\\
          \RETURN~\funcCallR{evalCT}{ \envElem{NGP}{}}\\
      \end{array}
    }}~\RETURN\\
\end{array}
\end{equation*}
\end{small}%
\begin{small}
\begin{equation*}
\begin{array}{l}
    \mathrm{(2)~~~~}\LET~\varR{results}~:=\\
    \qquad\qquad\funcCallR{sparqlCall}{
                    \begin{array}{l}
                      \SELECT~\Vars \cup \set{\envElem{\var{Var}}{1}, \dots, \envElem{\var{Var}}{n}} \\
                      \DatasetClause~\cup \set{ \FROMNAMED~\varR{ds} }\\
                      \SparqlWhereClause~\cup \WHERE\set{ \keyword{graph}~\varR{ds}~\envElem{NGP}{} } \\
                      \SolutionModifier
                    \end{array}}~\RETURN\\
    \mathrm{(3)~~~~} \FOR~\varR{result}~\AT~\varR{result\mathunderscore{}pos}~\IN~\varR{results//sr{:}result}~\RETURN \\
    \mathrm{(4)~~~~} \quad \LET~\var{v} :=~ \varR{result/sr{:}binding[@name = \var{v}]/*} \hfill \quad \textrm{\smaller for each \var{v}~$\in \Vars \cup \set{\envElem{\var{Var}}{1}, \dots, \envElem{\var{Var}}{n}}$}\\
    \mathrm{(5)~~~~}\quad\RETURN \left(\ExprSingle, \funcCallR{deleteNG}{\varR{ds}}\right)
\end{array}
\end{equation*}
\end{small}%
%
where~$\envElem{NGP}{}$ is the graph pattern~$\set{ \mathtt{[]}~\qname{}{\envElem{Var\!}{1}}~\envElem{\var{Var}}{1} ;~ \dots ;~\qname{}{\envElem{Var\!}{n}}~\envElem{\var{Var}}{n} }$.
%
\item otherwise:
%
\begin{small}
\begin{equation*}
\begin{array}{r@{~~~}l}
  \multicolumn{2}{l}{\funcCall{opt_{ng}}{\envElem{Q}{}} =}\\[.3em]
  & \ong{\begin{array}{l}
      \FOR~\envElem{AttrSpec}{1}~\keyword{as}~\envElem{\var{Var}}{1}, \dots, \envElem{AttrSpec}{n}~\keyword{as}~\envElem{\var{Var}}{n}~\grammarRule{RelationList}~\grammarRule{SQLWhereClause}\\
      \RETURN \\
      \quad\FOR~\Vars~\DatasetClause~\SparqlWhereClause~\SolutionModifier\\
      \quad\RETURN~\ong{\ExprSingle}
  \end{array}
}
 \end{array}
\end{equation*}
\end{small}%
\end{itemize}



\begin{restatable}{proposition}{ngCorrectRestatable}
  \label{prop:ng-correct}
  Let \envElem{Q}{} be an XSPARQL expression of form~\eqref{eq:expr_ng} or~\eqref{eq:expr_ng_sql} and \dyn the dynamic
  environment of \envElem{Q}{}, then~$\dynEnv \envElem{Q}{} \Rightarrow \envElem{Val}{}$ if and only if~$\dynEnv
  \ong{\envElem{Q}{}} \Rightarrow \envElem{Val}{}.$
\end{restatable}
%
\begin{proof}
  %
  We start by showing the proof for the base case, where \envElem{\ExprSingle}{1} and \envElem{\ExprSingle}{2}
  of~\eqref{eq:expr_ng} do not contain any occurrences of~\eqref{eq:expr_ng}.
  %
  \paragraph*{Base Case.}
  %
  ($\Rightarrow$)
  % 
  Let us start by showing that if~$\dynEnv \grammarRule{\envElem{Q}{}} \Rightarrow \envElem{Val}{}$ then~$\dynEnv
  \ong{\envElem{Q}{}}$~$\Rightarrow \envElem{Val}{}$.
  %
  We now show the proof tree for each of the XQuery core expressions in the~$\funcName{opt_{ng}}$ rewriting.

\begin{description}

\item[\LET expression of line (1).]  Considering~$\envElem{NGP}{} = \set{ \mathtt{[]~{:}value}~\var{VarName} }$, we have

    \begin{prooftreefunction}
      \AxiomC{$\dynEnv{\funcCallR{createNG}{\begin{array}{l}
              \FOR~\var{VarName}~\grammarRule{OptTypeDeclaration}\\
              \grammarRule{OptPositionalVar}~\IN~\envElem{\ExprSingle}{1}\\
              \RETURN~\funcCallR{evalTemplate}{\envElem{NGP}{}}\\
              \end{array}
            }} \Rightarrow \envElem{DS}{} $}
      
      \UnaryInfC{$\dynEnvX{1}{ng}{\grammarRule{Expr}} \Rightarrow \envElem{Res}{}$}
      
      \UnaryInfC{$\dynEnv{\begin{array}{l}
            \LET~\varR{ds}~:=\\
            \quad\funcCallR{createNG}{\begin{array}{l}
                \FOR~\var{VarName}~\grammarRule{OptTypeDeclaration}\\
                \grammarRule{OptPositionalVar}~\IN~\envElem{\ExprSingle}{1}\\
                \RETURN~\funcCallR{evalTemplate}{\envElem{NGP}{}}\\
              \end{array}}\\
            \RETURN~\grammarRule{Expr}
          \end{array} \Rightarrow \envElem{Res}{}
        }$}
    \end{prooftreefunction}%
  where 
  \begin{tree}
    \ \ \dynX{1}{ng} = \dyn \envExtend{varValue}{\varNameR{ds} \Rightarrow \envElem{DS}{}} \enspace .
    \label{tree:dynEvn3-ng}
  \end{tree}%

\item[\LET expression of line (2).] As a shortcut representation, consider the dataset clause~$\gsup{\DatasetClause}{ng}
  = \DatasetClause\ \cup\ \set{ \FROMNAMED\ \varR{ds} }$ and the graph pattern~$\gsup{\SparqlWhereClause}{ng} =
  \SparqlWhereClause\ \cup\ \WHERE\ \set{ \keyword{graph}\ \varR{ds} \set{ \mathtt{[]\ {:}value}\  \var{VarName} } }$.

    \begin{prooftreefunction}
      \AxiomC{$
        \dynEnvX{1}{ng}{\funcCallR{sparqlCall}{
              \begin{array}{l}
                \SELECT~\Vars \cup \set{\var{VarName}} \\
                \gsup{\DatasetClause}{ng}~\gsup{\SparqlWhereClause}{ng}\\
                \SolutionModifier
              \end{array}} \Rightarrow \omg{ng}{}}
        $}
      
      \UnaryInfC{$\dynEnvX{2}{ng}{\grammarRule{Expr}} \Rightarrow \envElem{Res}{}$}
      
      \UnaryInfC{$\dynEnvX{1}{ng}{\begin{array}{l}
            \LET~\varR{results}~:=\\
            \quad\funcCallR{sparqlCall}{
              \begin{array}{l}
                \SELECT~\Vars \cup \set{\var{VarName}} \\
                \gsup{\DatasetClause}{ng}~\gsup{\SparqlWhereClause}{ng}\\
                \SolutionModifier
              \end{array}\!\!} \\ 
            \RETURN~\grammarRule{Expr}
          \end{array} \Rightarrow \envElem{Res}{}}$}
    \end{prooftreefunction}%
  where 
  \begin{small}
    $\dynX{2}{ng} = \dynX{1}{ng}\envExtend{varValue}{\varNameR{results} \Rightarrow \omg{ng}{}}$.
  \end{small}%

  \item[\FOR expression of line (3):]~

    \begin{prooftreefunction}

      \AxiomC{$\dynEnvX{2}{ng}{\varR{results//sr{:}result}} \Rightarrow \sm{i}{}$}
      \UnaryInfC{$\dynEnvX{3}{ng}{\grammarRule{Expr}} \Rightarrow \envElem{Res}{i}$}

      \AxiomC{$\Ddots$}
      
      \BinaryInfC{$\dynEnvX{2}{ng}{
          \begin{array}{l}
            \FOR~\varR{result}~\AT~\varR{result\mathunderscore{}pos}\\
            \qquad\IN~\varR{results//sr{:}result}\\
            \RETURN~\grammarRule{Expr}
          \end{array}  \Rightarrow \envElem{Res}{1},\dotsb,\envElem{Res}{n}}$}
      
    \end{prooftreefunction}%
  where 
  \begin{small}
      $\dynX{3}{ng} = \dynX{2}{ng} \envExtend{varValue}{\begin{array}{l}
        \varNameR{result} \Rightarrow \sm{i}{};\\
        \varNameR{result\mathunderscore{}pos} \Rightarrow i
      \end{array}
    }\enspace .$
\end{small}%

\item[\LET expressions of lines (4)--(5).] Here we consider all the \LET expressions represented by line~(4),
  where~$\var{v} \in \Vars$:

  \begin{prooftreefunction}
      \AxiomC{$\dynEnvX{3}{ng}{\varR{result/sr{:}binding[@name = \var{v}]/*}} \Rightarrow \envElem{V}{}$}

      \UnaryInfC{$\dynEnvX{4}{ng}{\grammarRule{Expr}} \Rightarrow \envElem{Res}{}$}

      \UnaryInfC{$\dynEnvX{3}{ng}{\begin{array}{l}
            \LET~\var{v} := \varR{result/sr{:}binding[@name = \var{v}]/*} \\
            \RETURN~\envElem{\ExprSingle}{2}
          \end{array} \Rightarrow \envElem{Res}{}}$}

    \end{prooftreefunction}%
  where 
  \begin{small}
    $\dynX{4}{ng} = \dynX{3}{ng} \envExtend{varValue}{\envElem{v}{} \Rightarrow \envElem{V}{}}$.
  \end{small}%

  \end{description}

  %
  Let \omg{xs}{in} be the solution sequence returned by the evaluation of the inner \SparqlForClause of \envElem{Q}{}.
  %
  Furthermore let \dynX{i}{expr} be the dynamic environment such that \dynEnvX{i}{expr}{\ExprSingle \Rightarrow
    \envElem{Val}{}}.  \dynX{i}{expr} results from extending \dyn with bindings for the outer variable \var{VarName} and
  with variable bindings from a solution mapping~$\sm{xs}{in} \in \omg{xs}{in}$
  where~$\funcCall{\sm{xs}{in}}{\grammarRule{VarName}} = \var{VarName}$, \ie~the value for the join variable in the
  solution mapping \sm{xs}{in} is the same as assigned to \var{VarName}.


  The new merged dataset, \gsup{\DatasetClause}{ng}, is created based on \DatasetClause and the newly created named
  graph~$\envElem{NG}{}$. Since the URI that identifies the newly created named graph~$\envElem{NG}{}$ is distinct from
  any URI of named graphs present in \DatasetClause, the triples included in~$\envElem{NG}{}$ will never be a solution
  for \SparqlWhereClause, and will be matched only by the graph pattern~``$\WHERE \set{ \keyword{graph}\ \varR{ds} \set{
      \mathtt{[]\ {:}value}\ \var{VarName} } }$''.

  Let~$C$ be the expression context where~$\dyn$ is included,~$\mu_C$ the XSPARQL instance mapping of~$C$
  and~$\gsup{P}{out}$ and~$\gsup{P}{in}$ the graph patterns obtained from replacing the variables
  in~$\SparqlWhereClause$ and~``$\WHERE\set{ \keyword{graph}~\varR{ds} \set{ \mathtt{ []~{:}value}~\var{VarName} } }$''
  according to~$\mu_C$, respectively.

  Furthermore, let~$\omg{ng}{out} = \evalS{\gsup{\DatasetClause}{ng}}{\gsup{P}{out}}$ and~$\omg{ng}{in} =
  \evalS{\gsup{\DatasetClause}{ng}}{\gsup{P}{in}}$.
  %
  According to SPAR\-QL semantics, the pattern solution that results from evaluating~$\SparqlWhereClause$,~$\Omega_{ng}
  = \omg{ng}{out} \bowtie \omg{ng}{in}$ consists of all the solution mappings~$\sm{out}{} \in \omg{ng}{out}$
  and~$\sm{in}{} \in \omg{ng}{in}$ such that~$\sm{out}{}$ and~$\sm{in}{}$ are \emph{compatible}.
  % 
  Similar to the proof of \cref{prop:sr-correct}, we are only considering \ORDERBY solution modifiers, these only change
  the order of the solution sequences and thus can be safely ignored for this proof.

  The evaluation of the outer XQuery \FOR clause (lines~(1)--(2) of \envElem{Q}{}) performed over \dyn is translated, by
  the \ong{\envElem{Q}{}} function, into the \funcNameR{sparqlCall} from line~(2), which is evaluated over \dynX{1}{ng}.
  %
  However, as we can see from~\eqref{tree:dynEvn3-ng}, \dynX{1}{ng} is based on \dyn by adding the value for
  the~$\varNameR{ds}$ variable and, since this variable belongs to the~$\varNameR{}$ reserved namespace, it is not
  allowed to appear in the \SparqlWhereClause and we have that the results of evaluating the \funcNameR{sparqlCall}
  function over \dyn or \dynX{1}{ng} will be the same.
 

  The inner \SparqlForClause (lines~(3)--(4) of \envElem{Q}{}) is evaluated over some dynamic environment \dynX{}{expr},
  is incorporated by the \ong{\envElem{Q}{}} into the \funcNameR{sparqlCall} from line~(2), which is evaluated over
  \dynX{1}{ng}.
  %
  Considering that \dynX{1}{ng} is less restrictive than \dynX{}{expr}, \ie~\dynX{1}{ng} contains less bindings for
  variables than \dynX{}{expr}\!, the evaluation of the inner \SparqlForClause over \dynX{1}{ng} will contain all the
  solution mappings from \omg{xs}{in} and specifically~$\sm{in}{}$.
  % 
  As~$\sm{out}{}$ and~$\sm{in}{}$ are \emph{compatible} we have that~$\dynEnv \funcCall{ng}{expr} \Rightarrow
  \envElem{Val}{}$.

  
  \medskip 
  \noindent ($\Leftarrow$)
  %
  Next we will show that if~$\dynEnv \ong{\envElem{Q}{}}$~$\Rightarrow \envElem{Val}{}$ then~$\dynEnv
  \envElem{Q}{} \Rightarrow \envElem{Val}{}$.
  %
  Let us turn to the evaluation of~$\dynEnv \envElem{Q}{} \Rightarrow \envElem{Val}{}$. 
  %
  \begin{description}
    % 
  \item[XQuery \FOR clause from lines (1)--(2).] Here \grammarRule{Expr} corresponds to the \SparqlForClause from
    lines~(3)--(4) of \envElem{Q}{}.
    %

      \begin{prooftreefunction}
      \AxiomC{$\dyn.\ecomp{globalPosition} = \seq{ \envElem{Pos}{1}, \cdots, \envElem{Pos}{m} } $}
      \UnaryInfC{$\dynEnv{\envElem{\grammarRule{ExprSingle}}{1} \Rightarrow \envElem{V}{i}}$}
        \UnaryInfC{$\dynX{i}{xs} \proofs \grammarRule{Expr} \Rightarrow \envElem{Value}{i}$}

        \AxiomC{$\Ddots$}

        \BinaryInfC{$\dynEnv{\begin{array}{l}
              \FOR~\var{VarName}~\grammarRule{OptTypeDeclaration}\\
              \quad\grammarRule{OptPositionalVar}~\IN~\envElem{\grammarRule{ExprSingle}}{1} \\
              \RETURN~\grammarRule{Expr}
            \end{array} \Rightarrow \envElem{Value}{i},\dots,\envElem{Value}{n}}$}

      \end{prooftreefunction}%
    % 
    we have for each~$\envElem{V}{i}$:
    \begin{tree}
      \ \dynX{i}{xs} = 
      \dyn \envExtend{globalPosition}{\seq{ \envElem{Pos}{1}, \cdots, \envElem{Pos}{m}, i }} \envExtend{varValue}{
        \e{VarName}{}{} \Rightarrow \envElem{V}{i}
      } \enspace .
      \label{proof:dynenv_i_ng}
    \end{tree}%

  \item[\SparqlForClause of lines (2)--(4):]~ 
    %
      \begin{prooftreefunction}
      \AxiomC{$\dyn.\ecomp{globalPosition} = \seq{ \envElem{Pos}{1}, \cdots, \envElem{Pos}{m} } $}
      \UnaryInfC{$ \dynEnvX{i}{xs}{
              \funcCall{fs{:}dataset}{\grammarRule{\DatasetClause}} \Rightarrow \grammarRule{DS}
          }$}
        \UnaryInfC{$\dynEnvX{i}{xs}{
              \funcCall{fs{:}sparql}{\!\!\begin{array}{l}
                  \grammarRule{DS},  \SparqlWhereClause, \\
                  \SolutionModifier
                \end{array}\!\!}  \Rightarrow \sm{j}{}}$}
        \UnaryInfC{$\dynX{j}{xs} \proofs \envElem{\ExprSingle}{2} \Rightarrow \envElem{Value}{j}$}

        \AxiomC{$\!\!\!\Ddots$}

        \BinaryInfC{$\dynEnvX{i}{xs}{\begin{array}{l}
            \FOR~\Vars~\DatasetClause\\
            \SparqlWhereClause~\SolutionModifier\\
            \RETURN~\envElem{\ExprSingle}{2}
          \end{array}}~\Rightarrow \envElem{Value}{1} \dotsb \envElem{Value}{m}$}
      \end{prooftreefunction}%
    % 
    where, considering~$\Vars = \var{Var_1}\dots\var{Var_n}$, we have for each~$\sm{j}{}$:
    \begin{small}
      \begin{equation*}
        \ \ \dynX{j}{xs} = \begin{array}{l}
          \dynX{i}{xs} \envExtend{activeDataset}{\grammarRule{DS}}\envExtend{globalPosition}{\seq{ \envElem{Pos}{1}, \cdots, \envElem{Pos}{m}, j }}\\
          \envExtend{varValue}{\begin{array}{l}
              \e{Var}{1}{} \Rightarrow \funcCall{fs{:}value}{\sm{j}{}, \e{Var}{1}{}}; \\
              \dotsb;\\
              \e{Var}{n}{} \Rightarrow \funcCall{fs{:}value}{\sm{j}{},\e{Var}{n}{}}
            \end{array}}
        \end{array}\enspace .
      \end{equation*}
    \end{small}%
  \end{description}

  As we have seen in the ($\Rightarrow$) direction, we have that~$\omg{ng}{out} = \omg{xs}{out}$ and so we have
  that~$\sm{ng}{out} \in \omg{xs}{out}$.
  % 
  Let~$\omg{ng}{out}$ and~$\omg{ng}{in}$ be the solution sequences returned by the evaluation of the new
  \gsup{\SparqlWhereClause}{ng} and \SparqlWhereClause, respectively.
  % 
  As we have seen~$\Omega_{ng}$ contains all the solution mappings~$\sm{}{} = \sm{ng}{out} \bowtie \sm{ng}{in}$,
  where~$\sm{ng}{out} \in \omg{ng}{out}$ and~$\sm{ng}{in} \in \omg{ng}{in}$, such that~$\sm{ng}{out}$ and~$\sm{ng}{in}$
  are \emph{compatible}.
  % 
  Again, consider~$\sm{ng}{out}$ and~$\sm{ng}{in}$ the pattern solutions where~$\envElem{Val}{}$ is deduced from.


  Let~$C$ be the expression context where~$\dyn$ is included and~$\mu_C$ the XSPARQL instance mapping of~$C$.
  Furthermore let~$\gsup{P}{in}$ be the graph pattern obtained from replacing the variables
  in~$\gsup{\SparqlWhereClause}{in}$ according to~$\mu_C$.
  % 
  Since we know that~$\vars{\gsup{\SparqlWhereClause}{in}} \subseteq \vars{\gsup{P}{in}}$, all solutions mappings
  returned by evaluating \SparqlWhereClause\gsup{\!\!\!}{in} under XSPARQL semantics are included in the pattern
  solution of evaluating~$\e{P}{}{in}$ under SPARQL semantics \ie~$\omg{xs}{in} \preceq \omg{ng}{in}$.
  % 
  We obtain two cases:
  \begin{inparaenum}[(i)]
  \item\label{en:inxsin_ng}~$\sm{ng}{in} \in \omg{xs}{in}$; or
  \item\label{en:notinxsin_ng}~$\sm{ng}{in} \not\in \omg{xs}{in}$. 
  \end{inparaenum}
  %
  In \eqref{en:inxsin_ng} we immediately get that~$\dynEnv \envElem{Q}{} \Rightarrow \envElem{Val}{}$.
  %
  For~\eqref{en:notinxsin_ng}, consider \sm{C_1}{xs} the XSPARQL instance of the inner \SparqlForClause (created based
  on \dynX{1}{xs}). As we can see from~\eqref{proof:dynenv_i_ng}, \dynX{1}{xs} (and thus also \sm{C_1}{xs}) includes the
  bindings for variables from each solution mapping~$\sm{i}{} \in \omg{xs}{out}$.  Thus, according to the XSPARQL \ac{BGP}
  matching (\cf~\cref{def:xsparql-bgp-matching}), \omg{xs}{in} will contain all the solution mappings that are
  compatible with any solution mapping~$\sm{i}{} \in \omg{xs}{out}$ and specifically those compatible with \sm{ng}{out}.
  Since we know that \sm{ng}{in} is compatible with \sm{ng}{out}, we have that \sm{ng}{in} must belong to \omg{xs}{in},
  thus we can deduce that~$\dynEnv \envElem{Q}{} \Rightarrow \envElem{Val}{}$. 
  % 
  \paragraph*{Inductive Step.}
  % 
  Let us assume that, for some arbitrary \dynX{i}{}, $\dynEnvX{i}{} \envElem{\ExprSingle}{1} \Rightarrow
  \envElem{Val}{i}$ if and only if $\dynEnvX{i}{} \funcCall{opt_{ng}}{\envElem{\ExprSingle}{1}} \Rightarrow
  \envElem{Val}{i}$.
  % 
  According to the \funcName{opt_{ng}} rewriting, there must exist a \dynX{j}{} that is the extension of \dynX{i}{}
  with \envElem{Val}{i} and thus
  $\dynEnvX{j}{} \envElem{\ExprSingle}{2} \Rightarrow \envElem{Val}{}$ if and only if $\dynEnvX{j}{}
  \funcCall{opt_{ng}}{\envElem{\ExprSingle}{2}} \Rightarrow \envElem{Val}{}$.  Consequently, we have that $\dynEnv Q
  \Rightarrow \envElem{Val}{}$ if and only if $\dynEnv \funcCall{opt_{ng}}{Q} \Rightarrow \envElem{Val}{}$.
  % 
\end{proof}



%%% Local Variables:
%%% fill-column: 120
%%% TeX-master: "../thesis" 
%%% TeX-PDF-mode: t
%%% TeX-debug-bad-boxes: t
%%% TeX-parse-self: t
%%% TeX-auto-save: t
%%% reftex-plug-into-AUCTeX: t
%%% End:




\subsection{Nested Queries in XMarkRDF}
\label{sec:optim-xmarkrdf-nest}


From the initial set of 20 queries there are 5 queries ($q_8$--$q_{12}$) that contain nested expressions. They are
described informally in the XMark suite as follows:
\begin{enumerate}[noitemsep]
\item[($q_8$)] ``List the names of persons and the number of items they bought;''
\item[($q_9$)] ``List the names of persons and the names of the items they bought in Europe;''
\item[($q_{10}$)] ``List all persons according to their interest;''
\item[($q_{11}$)] ``List the number of items currently on sale whose price does not exceed 0.02\% of the seller's
  income;'' and
\item[($q_{12}$)] ``For each richer-than-average person, list the number of items currently on sale whose price does not
  exceed 0.02\% of the person's income.''
\end{enumerate}



\begin{figure}[t]
\subfloat[Query~$q_9$ in XQuery (XMark)]{\label{fig:query09-xquery}%
  \begin{minipage}{.53\linewidth}%
    \lstinputlisting[linewidth=.97\linewidth,basicstyle=\ttfamily\scriptsize]{0-data+queries/query09.xq}%
  \end{minipage}%
}
%
\hfill
%
\subfloat[Query~$q_9$ in XSPARQL (XMarkRDB)]{\label{fig:query09-rdb}%
  \begin{minipage}{.53\linewidth}%
    \lstinputlisting[linewidth=.97\linewidth,basicstyle=\ttfamily\scriptsize]{0-data+queries/query09-rdb.xsparql}%
  \end{minipage}%
}%

\subfloat[Query~$q_9$ in XSPARQL (XMarkRDF)]{\label{fig:query09-xsparql}%
  \begin{minipage}{.53\linewidth}%
    \lstinputlisting[linewidth=.97\linewidth,basicstyle=\ttfamily\scriptsize]{0-data+queries/query09.xsparql}%
  \end{minipage}%
}%
%
\hfill
%
\subfloat[Query~$q_9$ in SPARQL2XQuery ($\textrm{XMarkRDF}_{\mathit{S2XQ}}$)]{\label{fig:query09-groppe}%
  \begin{minipage}{.53\linewidth}%
    \lstinputlisting[linewidth=.97\linewidth,basicstyle=\ttfamily\scriptsize]{0-data+queries/query09-groppe.xquery}%
  \end{minipage}%
}%
\caption{Variants of benchmark query~$q_9$}
\label{fig:query09}
\end{figure}
%

\cref{fig:query09-xquery,fig:query09-rdb,fig:query09-xsparql} present XMark query~$q_9$, its translated XSPARQL version
in XMarkRDB and XMarkRDF, respectively.
%
Query~$q_{9}$, as presented in \cref{fig:query09-groppe}, is ready to be evaluated by the SPARQL2XQuery system
over the~$\textrm{XMarkRDF}_{\mathit{S2XQ}}$ dataset.\footnote{Please note that this query follows the syntax presented
by~\citet{GroppeGroppeLinnemann:2008aa} however, we only had access to the implementation of the translation from SPARQL
to XQuery and hence manually replicated the complete query translation.}

The different rewritings presented in \cref{sec:optimisation} can be applied to the four nested
queries~$q_8$--$q_{11}$.
%
Query~$q_{12}$ also consists of a nested expression, however the most accurate translation of this query into XSPARQL
results in the dependent variable not being \emph{strictly bound} since it occurs only in the \FILTER of the inner
query. As such, we cannot apply the different rewritings to this query.
%

XMarkRDF query~$q_9$ is presented in \cref{fig:query09-xsparql}. This query is close to queries~$q_8$,~$q_{10}$, and
$q_{11}$ and consists of a nested expression: the inner \FOR expression of the query
(\crefrange{fig:query09:inner-start}{fig:query09:inner-end}) is executed once for each person matched by the outer
expression (\crefrange{fig:query09:outer-start}{fig:query09:outer-end}), which means that one SPARQL call will be
made for each person separately.  
%
Thus, the number of SPARQL calls performed in the inner expression directly depends on the size of the
dataset~(\cf~\cref{tab:iterations} for details).
%
Queries $q_8$,~$q_{9}$, and~$q_{11}$ evaluates the inner expression for each person, while~$q_{10}$ evaluates the inner
expression for each category.  Each dataset contains approximately 25 times more persons than categories.
%
The rewriting strategies presented in \cref{sec:optimisation} reduce the number of SPARQL calls to two: one to
get all the people (similar to the direct rewriting version), and one additional SPARQL call for retrieving all the
information about all the auctions in the dataset.  
%
Although the query remains exponential, the practical evaluation will show that reducing the number of SPARQL calls
drastically improves query execution times.




\begin{figure}[t]

  \subfloat[Query~$q_{9}$ -- bought items grouped by person]{\label{fig:output-09}
    \begin{minipage}{.5\linewidth}
      \lstinputlisting[language=XML,numbers=none]{0-data+queries/output-09.xml}
    \end{minipage}
  }
  %
  \subfloat[Query~$q_{9}'$ -- flat list of items and buyer]{\label{fig:output-39}
    \begin{minipage}{.5\linewidth}
      \lstinputlisting[language=XML,numbers=none]{0-data+queries/output-39.xml}
    \end{minipage}
  }
  \caption{Example output excerpts of queries~$q_{9}$ and~$q_{9}'$}
  \label{fig:output}
\end{figure}



As mentioned in \cref{sec:query-rewriting}, for the SPARQL based rewritings, we want the query output to be
computable directly in SPARQL without any further processing, \ie~we do not want to use XQuery for further processing
of the SPARQL results and the query should be expressible in SPARQL without features from SPARQL~1.1.  
%
Since the original nested queries~$q_8$--$q_{11}$ group the output results (while optionally applying some aggregation
function), we need to include modified versions of these benchmark queries for the evaluation of the SPARQL based
rewritings.
%
In these modified queries, denoted~$q_{8}'$--$q_{11}'$, we changed the return format of the queries to consist of a
flattened representation of the output of the original query.  An example of the output for queries~$q_{9}$ and~$q_{9}'$
is presented in \cref{fig:output}.
%
All queries~$q_i'$ and~$q_i''$ follow a similar strategy for reformatting the output:
%
the queries resulting from applying~$\funcName{opt_{sr}}$ are named~$q_8'$--$q_{11}'$, while the queries that consist of
an outer \FOR expression -- to which~$\funcName{opt_{ng}}$ was applied -- are~$q_8''$--$q_{11}''$.



\subsection{Evaluation of the Proposed Optimisations}
\label{sec:exper-eval}

In this section we present an experimental evaluation of the different rewritings presented in
\cref{sec:optimisation}.
%
For this evaluation we also rely on the XMarkRDF benchmark suite (presented in \cref{sec:benchm-descr}) and
compare, when possible, the effects of the different rewritings on the SPARQL2XQuery
system~\cite{GroppeGroppeLinnemann:2008aa}.
%



For the evaluation we extend the run configurations presented in \cref{sec:benchm-descr} with the following:
\begin{description}[noitemsep]
\item[$\mathit{XS^{rdf}_{Z}}$:] using the XSPARQL implementation over the XMarkRDF datasets (translated data and
  queries) with nested expresion optimisation \funcName{opt_{Z}} for~$Z \in \set{nl,ng,sr}$;
\item[$\mathit{XS^{rdb}_{nl}}$:] using the XSPARQL implementation over the XMarkRDB datasets (translated data and
  queries) with nested expresion optimisation \funcName{opt_{nl}};
\item[$\mathit{S2XQ}_{Z}$:] using the SPARQL2XQuery implementation over the translation of the XMarkRDF datasets into
  the required \ac{XML} format ($\textrm{XMarkRDF}_{\mathit{S2XQ}}$) with nested expresion optimisation
  \funcName{opt_{Z}} for~$Z \in \set{nl,sr}$.
\end{description}
%
The experimental setup remains the same as presented in \cref{sec:experimental_setup}.
%
We applied the nested-loop join rewriting from \cref{sec:join-impl-xquery} to the XMarkRDB and XMarkRDF translated
queries, which are denoted as~$\mathit{XS^{rdb}_{nl}}$ and~$\mathit{XS^{rdf}_{nl}}$, respectively.  The same
optimisations were applied to the SPARQL2\-XQuery translation to XQuery, denoted~$\mathit{S2XQ_{nl}}$ in the results.
%
The strategies of rewriting to a single SPARQL query, as presented in \cref{sec:join-impl-sparql}, were also applied to
the XSPARQL XMarkRDF and SPARQL2XQuery queries and are denoted as~$\mathit{XS^{rdf}_{sr}}$ and~$\mathit{S2XQ_{sr}}$,
respectively.  The Named Graph rewriting was applied to the XSPARQL XMarkRDF queries and is
denoted~$\mathit{XS^{rdf}_{ng}}$.



\def\scale{.99}
\begin{figure*}[t]\scriptsize
  \centering
  \subfloat[Query~$q_8$]{
    \import{\plotsDir}{optimisations-paper-query08}
    \label{fig:opt08}
  }
  \hspace{8pt}
  \subfloat[Query~$q_9$]{
    \import{\plotsDir}{optimisations-paper-query09}
    \label{fig:opt09}
  }

  \subfloat[Query~$q_8'$]{
    \import{\plotsDir}{optimisations-paper-query38}
    \label{fig:opt38}
  }
  \hspace{8pt}
  \subfloat[Query~$q_9'$]{
    \import{\plotsDir}{optimisations-paper-query39}
    \label{fig:opt39}
  }

  \subfloat[Query~$q_8''$]{
    \import{\plotsDir}{optimisations-paper-query48}
    \label{fig:opt48}
  }
  \hspace{8pt}
  \subfloat[Query~$q_9''$]{
    \import{\plotsDir}{optimisations-paper-query49}
    \label{fig:opt49}
  }
  \caption{Query response times for (variants of)~$q_8$ and~$q_9$ on all XMarkRDF datasets}
  \label{fig:evaluation-times-q8-q9} 
\end{figure*}

\begin{figure*}[t]\scriptsize
  \centering
  \subfloat[Query~$q_{10}$]{
    \import{\plotsDir}{optimisations-paper-query10}
    \label{fig:opt10}
  }
  \hspace{8pt}
  \subfloat[Query~$q_{11}$]{
    \import{\plotsDir}{optimisations-paper-query11}
    \label{fig:opt11}
  }

  \subfloat[Query~$q_{10}'$]{
    \import{\plotsDir}{optimisations-paper-query40}
    \label{fig:opt40}
  }
  \hspace{8pt}
  \subfloat[Query~$q_{11}'$]{
    \import{\plotsDir}{optimisations-paper-query41}
    \label{fig:opt41}
  }

  \subfloat[Query~$q_{10}''$]{
    \import{\plotsDir}{optimisations-paper-query50}
    \label{fig:opt50}
  }
  \hspace{8pt}
  \subfloat[Query~$q_{11}''$]{
    \import{\plotsDir}{optimisations-paper-query51}
    \label{fig:opt51}
  }
  \caption{Query response times for (variants of)~$q_{10}$ and~$q_{11}$ on all XMarkRDF datasets}
  \label{fig:evaluation-times-q10-q11} 
\end{figure*}

The comparison of the response times of the different rewriting functions presented in \cref{sec:optimisation} is
shown graphically in \cref{fig:evaluation-times-q8-q9,fig:evaluation-times-q10-q11}.
%
\begin{table*}[t]
  \caption[Query response times (in seconds) of different optimisations for the 2MB datasets.]{Query response times (in seconds) of different optimisations for the 2MB datasets. Optimisation not applicable (\textit{n/a}).}
  \label{tab:results-opt-2mb}
  \centering
  \import{\plotsDir}{table-optimisations-paper-2}
\end{table*}
%
The response times of these queries for the 2MB are presented in \cref{tab:results-opt-2mb} as a reference,
where~$\textit{n/a}$ indicates that the combination of query and optimisation is not applicable.





As we can see from \cref{tab:results-opt-2mb} and \cref{fig:evaluation-times-q8-q9,fig:evaluation-times-q10-q11}, the
\funcName{opt_{nl}} optimisation provides significant reduction in the query evaluation times when applied to the nested
queries with an inner \SparqlForClause.  For queries~$q_{8}$,~$q_{9}$, and~$q_{11}$ the difference in response times is
one order of magnitude.
%
However, applying a similar rewriting to relational data deteriorates the response times of the query.  This hints that
collecting the data and performing the join in the rewritten XQuery is slower than the nested calls to the relational
database.  
%
There are two possible causes for this discrepancy of behaviours from the different backends.  One possible explanation
for the speed improvement in SPARQL is that the overhead resides on the loading of data by the ARQ engine.  Since this
overhead is not presented in the relational database the queries would not benefit from the optimised rewritings.
%
The other explanation is that the cost of evaluating one unbound query and building the necessary structures for
representing the returned data is greater than the cost of executing nested queries.
%
Further investigation into this would be required to determine the source of the overhead.
%

The improvement in the execution time for query~$q_{10}$ is less drastic. This can be explained by the fact that the
outer expression of~$q_{10}$ iterates over ``categories'', which, as presented in \cref{tab:iterations}, increases at a
much smaller rate than ``persons'' do in the outer expressions of queries~$q_8$,~$q_9$, and~$q_{11}$.


However, for the~$\mathit{S2XQ}$ runs this optimisation provides virtually no improvement in the query response times
for queries~$q_8$ and~$q_9$ and their variants.  In queries~$q_{10}$,~$q_{11}$,~$q_{10}'$, and~$q_{11}'$ we can observe
an improvement in response times. This can be attributed to the fact that the rewriting for queries~$q_{10}$
and~$q_{11}$ and their variants are not as suitable for optimisation by the XQuery engine when compared to queries~$q_8$
and~$q_9$. For these cases our rewriting strategy is capable of performing the optimisation task for the XQuery engine.


For the~$\mathit{XS^{rdf}}$ run, it is possible to see in \cref{fig:opt38,fig:opt39} that~\funcName{opt_{sr}} (presented
in \cref{sec:join-impl-sparql}) is generally more efficient in terms of response times than the XQuery based.
%
This can be justified by the the smaller amount of information that is necessary to transfer from SPARQL to the XQuery
engine.  This effectively reduces the overhead of using an external SPARQL engine for the evaluation of queries.
%
Considering the~$\mathit{S2XQ_{sr}}$ run, \funcName{opt_{sr}} produces no improvement in the
query response times and in some cases ($q_{10}'$ and~$q_{11}'$ from \cref{tab:results-opt-2mb}) even deteriorates
considerably the response times when compared to~$\mathit{S2XQ}$.  This further supports our previous
claims that the XQuery engine is not capable of optimising the rewritten code from complex SPARQL queries.


Furthermore, the~$\mathit{S2XQ_{sr}}$ runs could only evaluate the smaller dataset sizes for query~$q_{8}$: its response
times deteriorated considerably with the larger dataset sizes, as opposed to the~$\mathit{XS^{rdf}_{sr}}$ runs that
behaved consistently similar to~$\mathit{XS^{rdf}_{nl}}$.  This indicates that~$\mathit{S2XQ}$ is not as efficient as
the ARQ-based native SPARQL engine runs~$\mathit{XS^{rdf}_{sr}}$ and~$\mathit{XS^{rdf}_{nl}}$ for larger datasets.


We can draw similar conclusions for the \funcName{opt_{ng}} when comparing the query evaluation times of the
\funcName{opt_{sr}} rewriting.  However, the response times for this approach are deteriorated by the the overhead of
creating, inserting and deleting the \ac{RDF} Named Graph.  This slowdown makes queries~$q_8''$,~$q_{10}''$ and~$q_{11}''$ of
the of the \funcName{opt_{nl}} rewriting outperform this optimisation.



%%% Local Variables:
%%% fill-column: 120
%%% TeX-master: t
%%% TeX-PDF-mode: t
%%% TeX-debug-bad-boxes: t
%%% TeX-parse-self: t
%%% TeX-auto-save: t
%%% reftex-plug-into-AUCTeX: t
%%% mode: latex
%%% mode: flyspell
%%% mode: reftex
%%% TeX-master: "../thesis"
%%% End:



%%% Local Variables:
%%% fill-column: 120
%%% TeX-master: t
%%% TeX-PDF-mode: t
%%% TeX-debug-bad-boxes: t
%%% TeX-parse-self: t
%%% TeX-auto-save: t
%%% reftex-plug-into-AUCTeX: t
%%% mode: latex
%%% mode: flyspell
%%% mode: reftex
%%% TeX-master: "../thesis"
%%% End:
