\section{Implementation Notes}
\label{sec:implementation-notes}

\begin{figure}[t]
  \centering
  
\begin{tikzpicture}[node distance = 35pt,%
  terminal/.style={ rectangle,minimum size=6mm,rounded corners=3mm,text centered,draw=black!50,fill=gray!20},%
  component/.style={ rectangle,minimum size=6mm,rounded corners=3mm,text centered,draw=black!50},%
  rdb/.style={cylinder, shape border rotate=90, draw,minimum width=1.5cm,minimum height=1cm, shape aspect=.35}%
  ]

  \tikzstyle{every node}=[font=\footnotesize]


  %  reasoner
  \node[terminal, text width=1.5cm, minimum height=1cm] (reasoner) {Reasoner}; %

  \coordinate[left= 5em of reasoner.west] (left-reasoner);

  % domains
  \begin{scope}[node distance = 5pt,text width=1.2cm,anchor=east]
    \node[component] at (left-reasoner) (ac) {Access Control}; %
    \node[component, above=of ac]  (frdf) {Fuzzy}; %
    \node[component,left =of frdf] (trdf) {Temporal}; %
    \node[component,below =of trdf] (prov) {Prove\-nance}; %
  \end{scope}

  % domains outer box
  \node[draw=black,dotted,thick,inner sep=10pt,rectangle,fit=(trdf) (ac) (frdf) (prov)] (domains) {} ;%
  \node[inner sep=0,fill=white] at (domains.north) (domains-label) { \bf Domains};

  \draw[-latex] (reasoner.west) -- (domains.east);%


  \coordinate[right= 5em of reasoner.north east] (right-reasoner);
  % domains
  \begin{scope}[node distance = 5pt,text width=1.1cm,anchor=north west]
    \node[component] at (right-reasoner) (crules) {Custom Rules}; %
    \node[component, above=of crules]  (rhodf) {$\rhodf$}; %
  \end{scope}

  % rules outer box
  \node[draw=black,dotted,thick,inner sep=10pt,rectangle,fit=(rhodf) (crules)] (rules) {} ;%
  \node[inner sep=0,fill=white] at (rules.north) {\bf Rules};

  \draw[-latex] (reasoner.east) -- (rules.west);%


  % anql 
  \node[terminal,text width=1.5cm, above =1em of reasoner] (anql) {AnQL}; %
  \node[draw=black,dotted,thick,inner ysep=15pt,inner xsep=20pt,rectangle,fit=(anql) (reasoner)] (ardf) {} ;%

  \node[doc, above = of anql] (anqlQuery) {AnQL Query}; %
  \draw[-latex] (anqlQuery.south) -- (anql.north);%

  \draw[dashed,-latex] (anqlQuery.west) -| (domains-label.north);%

  \node[inner sep=0,fill=white] at (ardf.north) { \bf Annotated RDFS};


  \draw[latex-] (reasoner.north) -- (anql.south);%


  % input sources
  \documentSetL{below=2.5em of reasoner}{ardf}{Annotated RDF};%

  \draw[dashed,-latex] (ardf.west) -| (domains.south);%
  
  \draw[latex-latex] (reasoner.south) -- (ardf.north);%

 
  


\end{tikzpicture}



%%% Local Variables:
%%% mode: latex
%%% mode: flyspell
%%% mode: reftex
%%% TeX-master: "../thesis"
%%% End:


  \caption{Annotated RDFS implementation architecture}
  \label{fig:aRDF-schema}
\end{figure}

Our prototype implementation is split into two distinct modules: one that implements the Annotated RDFS inferencing
and the second module is an implementation of the AnQL query language that relies on the first module to retrieve
the data.  Our prototype implementation is based on SWI-Prolog's Semantic Web
library~\cite{WielemakerHuangvan-der-Meij:2008aa} and we present the architecture of the implementation in
\cref{fig:aRDF-schema}.


For the syntax of the annotated \ac{RDF} dataset we do not rely on any special serialisation but instead reuse other
existing proposals \eg~using reification~\cite{GutierrezHurtadoVaisman:2007aa} or
N-Quads~\cite{CyganiakHarthHogan:2009aa}.  Our engine parses the input \ac{RDF} datasets into an internal
representation, where each triple is represented using the \verb+rdf/4+ predicate, the arguments represent the
\emph{subject}, \emph{predicate}, \emph{object}, and \emph{annotation value} of the triple.


The ``Reasoner'' module consists of a bottom-up engine that calculates the closure of a given ``Annotated RDF'' graph
(or dataset).
%
The variable components correspond to the specification of the given annotation domain (``Domains''); and the ruleset
describing the inference rules and the way the annotation values should be propagated (``Rules'').  
%
The annotation domains are specified by the appropriate semi-ring operations and describe the default annotations for
non-annotated triples.

Rules are specified using a high-level language to specify domain independent rules that abstracts away peculiarities of
the underlying representation syntax:
%
\begin{example}[RDFS subclass implementation rule]
  In our implementation, the following rule provides \textit{subclass inference} in the \ac{RDFS} ruleset:
%
\begin{Verbatim}[fontsize=\small]
rdf(O, rdf:type, C2, V) <==
         rdf(O, rdf:type, C1, V1),
         rdf(C1, rdfs:subClassOf, C2, V2),
         infimum(V1, V2, V).
\end{Verbatim}
%
\end{example}
%
The Rules and Domains are independent of each other: it is possible to combine arbitrary rulesets and domains (see
above).


The AnQL module also implemented in Prolog relies on the SPARQL implementation provided by the ClioPatria Semantic Web
Server.\footnote{\url{http://www.swi-prolog.org/web/ClioPatria/}, retrieved on 2012/04/10.} For the AnQL implementation,
the domain specification needs to be extended with the grammar rules to parse an annotation value and any built-in
functions specific to the domain.


\subsection*{Implementation of Specific Domains}
\label{sec:impl-spec-doma}

For example, for the fuzzy domain the default value is considered to be~$1$ and the~$\otimes$ and~$\oplus$ operations
are the \textit{min} and \textit{max} operations, respectively.  The AnQL grammar rules consist simply of calling the
parser predicate that parses a decimal value.

As for the temporal domain, we represent triple annotations as \emph{ordered} lists of disjoint time intervals.  This
implies some additional care in the construction of the~$\otimes$ and~$\oplus$ operations.  For the representation
of~$-\infty$ and~$+\infty$ we use the \stt{inf} and \stt{sup} Prolog atoms, respectively. Concrete time points are
represented as integers and we use a standard constraint solver over finite domains (CLPFD) in the~$\otimes$ and
$\oplus$ operations. The default value for non-annotated triples is \stt{[inf,sup]}.
%
The~$\otimes$ operation is implemented as the recursive intersection of all elements of the annotation values,
\ie~temporal intervals.  The~$\oplus$ operation is handled by constructing CLPFD expressions that evaluate the union of
all temporal intervals.
%
Again, the AnQL grammar rules take care of adapting the parser to the specific domain and we have defined the domain
built-in operations described in \cref{sec:doma-spec-issu}.



%%% Local Variables:
%%% mode: latex
%%% TeX-master: "../thesis"
%%% End:
