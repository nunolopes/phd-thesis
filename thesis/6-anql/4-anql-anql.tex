\section{AnQL: Annotated SPARQL}
\label{sec:annotated-sparql}

The query language introduced so far allows for conjunctive queries.  Languages like \ac{SQL} and SPARQL allow to pose
more complex queries including built-in predicates to filter solutions and advanced features such as negation or
aggregates.
%
In this section we will present an extension of the SPARQL~\cite{PrudhommeauxSeaborne:2008aa} query language, called
AnQL, that enables querying annotated graphs.

For the rest of this section we fix a specific annotation domain,~$D = \tuple{L, \oplus, \otimes, \bot, \top}$, as
defined in \cref{sec:rdfs-annot-doma}.

\subsection{Syntax}
\label{sec:aSPARQL-syntax}

A \emph{simple AnQL query} is defined -- analogously to a SPARQL query in \cref{sec:sparql-syntax} -- as a
quadruple~$Q=(P,G,V, A)$ with the differences that:
%
\begin{enumerate}[(1),noitemsep]
\item $G$ is an annotated \ac{RDF} graph;
\item we allow annotated graph patterns as presented in \cref{def:pattern}; and
\item $A$ is the set of annotation variables taken from an infinite set~$\AAs$ (distinct from~$\AV$).
\end{enumerate}
%
We now introduce the definition of \emph{Annotated Graph Patterns}:
%
\begin{definition}[Annotated Graph Patterns]
  \label{def:pattern}
  % 
  Let~$\AU$,~$\AB$,~$\AL$, and~$\AV$ be defined as before. Furthermore, let~$\lambda$ be an annotation value from~$L$ or
  an annotation variable from~$\AAs$, we call~$\lambda$ an \emph{annotation label}.
  %
  An \emph{annotated graph pattern} in AnQL is defined (similar to SPARQL) inductively as follows:
  %
  \begin{itemize}[noitemsep]
  \item for a triple pattern~$\tau$,~$\fuzzyg{\tau}{\lambda}$ (called an \emph{annotated triple pattern}) is an
    annotated graph pattern;
  \item a set of annotated triple patterns, called a \acf{BAP}, is an annotated graph pattern;
  \item if~$P$ and~$P'$ are annotated graph patterns, then~$(P\ \AND\ P')$,~$(P\ \OPTIONAL\ P')$,~$(P\ \UNION\ P')$ are
    annotated graph patterns;
  \item if~$P$ is an annotated graph pattern and~$R$ is a filter expression, then~$(P\ \FILTER\ R)$ is an annotated
    graph pattern.
  \end{itemize}
  %
  We further denote by~$\funcCall{avars}{P}$ the set of annotation variables present in a graph pattern~$P$ and~\vars{P}
  is extended to include also the annotation variables.
\end{definition}

\nd The \OPTIONAL operator in the annotated case may cause the values of annotation variables outside the \OPTIONAL to
change depending if the optional pattern is matched.  This is presented in \cref{exx1}.
%
Please note that in query examples we will use a simple extension of the SPARQL syntax that caters for a fourth element
in triple patterns and, for convenience, we will use the notation~$\mu = \set{x_{1}/t_{1}, \ldots, x_{n}/t_{n}}$ to
indicate that~$\funcCall{\mu}{x_{i}} = t_{i}$, \ie~variable~$x_{i}$ is assigned to term~$t_{i}$.

\begin{example}[AnQL \OPTIONAL]
  \label{exx1}
  % 
  Suppose we are looking for Nightwish members during some time period and optionally the instrument they played.  This
  query can be posed as follows:
  %
  \lstinputlisting[numbers=none,frame=none]{0-data+queries/select-optional.sparql}
  % 
  Take our example dataset from \cref{fig:dataset-example} extended with the following triples that indicate the
  instrument:
    \[
    \fuzzyg{\triple{\qname{dbpedia}{Marco\mathunderscore{}Hietala}, \qname{mo}{instrument}, \qname{}{bass}}}{[2005,2009]} 
    \]
  we will get the following answers: 
  %
    \[
    \begin{array}{lcl}
      \mu_{1} & = & \set{\var{p}/\qname{dbpedia}{Tarja\mathunderscore{}Turunen}, \var{l}/[1996, 2005]} \\
      \mu_{2} & = & \set{\var{p}/\qname{dbpedia}{Marco\mathunderscore{}Hietala}, \var{l}/[2001, 2012]} \\
      \mu_{3} & = & \set{\var{p}/\qname{dbpedia}{Marco\mathunderscore{}Hietala}, \var{l}/[2005, 2009], \var{i}/\qname{}{bass} } \enspace .
    \end{array}
    \]
    % 
    The first two answers ($\mu_{1}$ and~$\mu_{2}$) correspond to the answers in which the~\OPTIONAL pattern is not
    satisfied, so we get the annotation values of~$[1996, 2005]$ and~$[2001, 2012]$, respectively, corresponding to the
    time that Tarja and Marco were members of Nightwish.  In the third answer, the~\OPTIONAL pattern is also matched
    and, in this case, the annotation value is restricted to the time when Marco is a member of Nightwish and we have
    information regarding the instrument he played.
    % 
\end{example}
%
Note that -- as we will see -- this first query will return as a binding for the annotation variable \var{l} the periods
where an instrument was played.
%
A different query can be written that returns the periods of time an artist was a member of a band.

\begin{example}[AnQL \OPTIONAL with \FILTER] 
  \label{exx2} 
  % 
  The following query returns the Nightwish members during some time period that optionally played an instrument at some
  point during this time:
  %
  \lstinputlisting[numbers=none,frame=none,,escapechar=\%]{0-data+queries/select-optional-filter.sparql}
  %
  Using the input data from  \cref{exx1}, we obtain the following answers: 
  %
  \[
  \begin{array}{lcl}
    \mu_{1} & = & \set{ \var{p}/\qname{dbpedia}{Tarja\mathunderscore{}Turunen}, \var{l}/[1996, 2005] } \\
    \mu_{2} & = & \set{ \var{p}/\qname{dbpedia}{Marco\mathunderscore{}Hietala}, \var{l}/[2001, 2012] } \\
    \mu_{3} & = & \set{ \var{p}/\qname{dbpedia}{Marco\mathunderscore{}Hietala}, \var{l}/[2001, 2012], \var{i}/\qname{}{bass} } \enspace .
  \end{array}
  \]
  %
  In this example the~\FILTER behaves as in SPARQL by removing from the answer set the mappings that do not make
  the~\FILTER expression true.
  % 
\end{example}


\nd This query also exposes the issue of unsafe filters, noted in~\citet{AnglesGutierrez:2008aa} and we presented the
semantics to deal with this issue in \cref{def:optsemantics}.



\subsection{Semantics}
\label{sec:aSPARQL-semantics}

We can now define the semantics of AnQL queries by extending the notion of SPARQL BGP matching.  
%
Just as matching \acp{BGP} against \ac{RDF} graphs is at the core of SPARQL semantics, matching \acp{BAP} against
annotated \ac{RDF} graphs is the heart of the evaluation semantics of AnQL.

We extend the notion of \emph{substitution} to include a substitution of annotation variables in which we do not allow
any assignment of an annotation variable to~$\bot$ (of the domain~$D$).  
%
An annotation value of~$\bot$, although it is a valid answer for any triple, does not provide any additional information
and thus is of minor interest. Furthermore this would contribute to increasing the number of answers unnecessarily.

\begin{definition}[BAP evaluation]
  \label{def:btpmatch}
  Let~$P$ be a \ac{BAP} and~$G$ an annotated \ac{RDF} graph. We define \emph{evaluation}~$\eval{P}_G$ as the list
  of substitutions that are \emph{solutions} of~$P$, \ie~$\eval{P}_G = \set{ \mu \mid G \models \mu(P) }$, where~$G
  \models \mu(P)$ means that any annotated triple in~$\mu(P)$ is entailed by~$G$.
\end{definition}
%
\nd We can define the notion of solutions for \ac{BAP} as the equivalent notion of answer sets for annotated conjunctive
queries. As for SPARQL, we have:
%
\begin{proposition}
  \label{ppp}
  %
  Given an annotated graph~$G$ and a \ac{BAP}~$P$, the solutions of~$P$ are the same as the answers of the annotated
  query~$\fc{q}{\vars{P}} \leftarrow P$ (where~$\vars{P}$ is the vector of variables in~$P$), \ie~$ans(G,q) =
  \eval{P}_{G}$.
\end{proposition}

\nd For the extension of the SPARQL relational algebra to the annotated case we introduce -- inspired by the definitions
in~\cite{PerezArenasGutierrez:2009aa} -- definitions of compatibility and union of substitutions:

\begin{definition}[$\otimes$-compatibility]
  Two substitutions~$\mu_1$ and~$\mu_2$ are \emph{$\otimes$-compatible} \iff:
  \begin{enumerate}[label=(\roman*),noitemsep]
  \item $\mu_1$ and~$\mu_2$ are compatible for all the non-annotation variables, \ie~$\mu_1(x) = \mu_2(x)$ for any
    non-annotation variable~$x\in \dom{\mu_1} \cap \dom{\mu_2}$; and
  \item $\funcCall{\mu_1}{\lambda} \otimes \funcCall{\mu_2}{\lambda} \neq \bot$ for any annotation variable~$\lambda \in
    \dom{\mu_1} \cap \dom{\mu_2}$.
  \end{enumerate}
\end{definition}
%
\nd This definition of \emph{compatible solutions} is the same for non-annotated variables.
%
For the case of shared annotation variables, consider as an example the temporal domain and two solutions~$\mu_1$
and~$\mu_2$ that share an annotation variable~$x$.  
%
For~$\mu_1$ and~$\mu_2$ to be considered compatible the value their values for~$x$ must overlap:
%
\begin{itemize}[noitemsep]
\item if~$\mu_1(x) = [2001, 2005]$ and~$\mu_2(x) = [2003, 2009]$, then~$[2001, 2005] \otimes [2003, 2009] = [2003,
  2005]$ and~$\mu_1$ and~$\mu_2$ will be considered compatible;
\item on the other hand, if~$\mu_1(x) = [2001, 2003]$ and~$\mu_2(x) = [2005, 2009]$ then~$\mu_1$ and~$\mu_2$
  will are not compatible.
\end{itemize}



\begin{definition}[$\otimes$-union of substitutions]
  Given two~$\otimes$-compatible substitutions~$\mu_1$ and~$\mu_2$, the \emph{$\otimes$-union} of~$\mu_1$ and~$\mu_2$,
  denoted~$\mu_1 \otimes \mu_2$, is as~$\mu_{1} \cup \mu_{2}$, with the exception that any annotation variable~$\lambda
  \in \dom{\mu_1} \cap \dom{\mu_2}$ is mapped to~$\funcCall{\mu_1}{\lambda} \otimes \funcCall{\mu_2}{\lambda}$.
\end{definition}

We now present the notion of evaluation for generic AnQL graph patterns.  This consists of an extension of
\cref{def:semantics-sparql}:

\begin{definition}[Evaluation, extends {\cite[Definition 2]{PerezArenasGutierrez:2009aa}}]
  \label{def:semantics-anql}
  Let~$P$ be a \ac{BAP},~$P_{1}, P_{2}$ annotated graph patterns,~$G$ an annotated graph and~$R$ a filter expression, then
  the evaluation~$\eval{\cdot}_{G}$, is recursively defined as:
  \begin{center}
    \begin{itemize}[noitemsep]
    \item~$\eval{P}_{G} = \set{\mu \mid \dom{\mu} = \vars{P}\textrm{ and }G\models \funcCall{\mu}{P} }$
    \item~$\eval{P_1\ \AND\ P_2}_{G} = \set{ \mu_1\otimes\mu_2 \mid \mu_1 \in \eval{P_1}_{G}, \mu_2 \in
      \eval{P_2}_{G},\mu_1\textrm{ and }\mu_2\textrm{~$\otimes$-compatible} }$
    \item~$\eval{P_1\ \UNION\ P_2}_{G} = \eval{P_1}_{G} \cup \eval{P_2}_{G}$
    \item~$\eval{P_1\ \FILTER\ R}_{G} = \left\{\mu \mid \mu \in \eval{P_1}_{G} \mbox{ and } R\mu \textrm{ is true} \right\}$
    \item~$\eval{P_1\ \OPTIONAL\ P_2[R]}_{G} = \{ \mu \mid \mu$ meets one of the following conditions: 
      %
      \begin{enumerate}[nosep]
      \item $\mu = \mu_1 \otimes \mu_2$ if~$\mu_1 \in \eval{P_1}_{G}, \mu_2 \in \eval{P_2}_{G}, \mu_1$ and~$\mu_2~\otimes$-compatible, and~$R\mu$ is true
      \item $\mu = \mu_1 \in \eval{P_1}_{G}$ and~$\forall\mu_2 \in \eval{P_2}_{G}$ such that~$\mu_1$
        and~$\mu_2$~$\otimes$-compatible,~$\funcCall{R}{\mu_1\otimes\mu_2}$ is true, and for all annotation
        variables~$\lambda\in \dom{\mu_1} \cap \dom{\mu_2}$,~$\funcCall{\mu_2}{\lambda} \prec \funcCall{\mu_1}{\lambda}$
      \item $\mu = \mu_1 \in \eval{P_1}_{G}$ and~$\forall\mu_2 \in \eval{P_2}_{G}$ such that~$\mu_1$
        and~$\mu_2$~$\otimes$-compatible,~$\funcCall{R}{\mu_1\otimes\mu_2}$ is false~$\}$
      \end{enumerate}
      %
    \end{itemize}
    %
  \end{center}
  Let~$R$ be a~$\FILTER$ expression and~$x,y \in \AAs \cup L$, in addition to the~$\FILTER$ expressions presented in
  \cref{def:semantics-sparql} we further allow the expressions presented next.  The valuation of~$R$ on a
  substitution~$\mu$, denoted~$R\mu$, is true if:\footnote{We consider a simple evaluation of filter expressions where the
    ``error'' result is ignored, see~\citet[Section 11.3]{PrudhommeauxSeaborne:2008aa} for details.}
  
  \nd(9)~$R = (x \preceq y)$ with~$x,y \in \dom{\mu} \cup L \wedge\ \funcCall{\mu}{x} \preceq \funcCall{\mu}{y}$;\\
  (10)~$R = \funcCall{p}{\vec{z}}$ with~$\funcCall{p}{\vec{z}}\mu=\mbox{ true \iff }\ \funcCall{p}{\funcCall{\mu}{\vec{z}}} =\textrm{ true}$, where~$p$ is a built-in
  predicate.
  
  \noindent Otherwise~$R\mu$ is false.
\end{definition}

Please note that the cases for the evaluation of~$\OPTIONAL$ are compliant with the SPARQL
specification~\cite{PrudhommeauxSeaborne:2008aa}, covering the notion of unsafe~$\FILTER{s}$ as presented
by~\citet{AnglesGutierrez:2008aa}.  However, there are some peculiarities inherent to the annotated case. More
specifically case 2.) introduces the side effect that annotation variables that are compatible between the mappings may
have different values in the answer depending if the~$\OPTIONAL$ is matched or not.  This is the behaviour demonstrated
in \cref{exx1}.

 In the~$\FILTER$ expressions above, a built-in predicate~$p$ is any~$n$-ary predicate~$p$, where~$p$'s arguments may
be variables (annotation and non-annotation ones), domain values of~$D$, or values from~$\AUL$;~$p$ has a fixed
interpretation and we assume that the evaluation of the predicate can be decided in finite time.

Annotation domains may define their own built-in predicates that range over annotation values as in the following query:
%
\begin{example}[AnQL query]
  Consider our example dataset from \cref{fig:dataset-example} and that we want to know which band
  \qname{dbpedia}{Marco\mathunderscore{}Hietala} was a member of before 2005. This query can be expressed in the
  following way:
  %
  \lstinputlisting[frame=none,numbers=none]{0-data+queries/select-before.sparql}%
  % 
\end{example}

\nd For practical convenience, we retain in~$\eval{\cdot}_{G}$ only ``domain maximal answers''. That is, let us
define~$\mu' \prec \mu$ \iff:
% 
\begin{enumerate}[label=(\roman*),noitemsep]
\item $\mu' \neq \mu$;
\item $\dom{\mu'} = \dom{\mu}$;
\item $\fc{\mu'}{x} = \fc{\mu}{x}$ for any non-annotation variable~$x$; and
\item $\fc{\mu'}{\lambda} \preceq \fc{\mu}{\lambda}$ for all annotation variable~$\lambda$.
\end{enumerate}
%
Then, for any~$\mu \in \eval{P}_{G}$ we remove any~$\mu' \in \eval{P}_{G}$ such that~$\mu' \prec \mu$.



\subsection{Further Extensions of AnQL}
\label{sec:asparql-aggregates}

In this section we will present extensions of \cref{def:semantics-anql} to include features from the SPARQL~1.1
specification, such as variable assignments, aggregates, and solution modifiers.



\begin{definition}[Assignment in AnQL]
  \label{def:anql-assign}
  Let~$P$ be an annotated graph pattern and~$G$ an annotated graph, the evaluation of an~$\keyword{ASSIGN}$ statement is
  defined as:
  %
    \[
    \eval{P\ \keyword{ASSIGN}\ f(\vec{z})\ AS\ z}_{G} = \set{ \mu \mid \mu_1 \in \eval{P}_{G},\mu =\mu_1[z/f(\mu_{1}(\vec{z}))]}
    \]
    %
    \nd where
    \[
    \mu[z/t] = \left\{
    \begin{array}{ll}
      \mu \cup \{z/t\} &\textrm{ if } z \not\in \mathit{dom}(\mu)\\
      \left(\mu \setminus \{z/t'\}\right) \cup \{z/t\} &\textrm{ otherwise} \ .
    \end{array}
    \right.
    \]
    
\end{definition}

\nd Essentially, we assign to the variable~$z$ the value~$f(\mu_{1}(\vec{z}))$, which is the evaluation of the function~$f(\vec{z})$ with respect to a substitution~$\mu_{1} \in \eval{P}_{G}$.

\begin{example}[Assignment in AnQL]
  \label{exx3}
  %
  Using a built-in function we can retrieve for each artist the length of membership for any band:
  %
  \lstinputlisting[frame=none,numbers=none]{0-data+queries/select-assign.sparql}
  % 
  \nd Here, the~$\funcName{length}$ built-in predicate returns, given a set of temporal intervals, the overall total
  length of the intervals.
  % 
\end{example}


\nd We also introduce the~$\keyword{ORDERBY}$ clause where the evaluation of
a~$\eval{P\ \keyword{ORDERBY}\ \var{x}}_{G}$ statement is defined as the ordering of the solutions -- for any~$\mu \in
\eval{P}_G$ -- according to the values of~$\mu(\var{x})$. Ordering for non-annotation variables follows the rules
in~\citet[Section~9.1]{PrudhommeauxSeaborne:2008aa}.
%
Similar to ordering in the query answering setting, we require that the set of values over which~$x$ ranges can be
ordered.
%
We can further extend the evaluation of AnQL queries with aggregate functions
\[
\aggr \in \{\keyword{SUM},\keyword{AVG}, \keyword{MAX}, \keyword{MIN}, \keyword{COUNT}, \oplus, \otimes\}
\]%
\vspace{-\belowdisplayskip}%
\nd as follows:
%
\begin{definition}[Grouping in AnQL]
  \label{def:groupby}
  % 
  The evaluation of a~$\keyword{GROUPBY}$ statement is defined as:\footnote{In the
    expression,~$\vec{\aggr}\vec{f}(\vec{z})\ AS\ \vec{\alpha}$ is a concise representation of~$n$ aggregations of the
    form~$\aggr_{i} f_{i}(\vec{z}_{i})\ AS\ \alpha_{i}$ and~$\{\ldots\}_{\mathsf{DISTINCT}}$ represents a duplicate
    removal operation.}
  %
  \[
  \eval{P\ \keyword{GROUPBY}(\vec{w})\ \vec{\aggr}\vec{f}(\vec{z})\ AS\ \vec{\alpha}}_{G} = \set{ \mu \mid \mu'\mbox{ in
    }\eval{P}_{G}, \mu = \mu'|_{\vec{w}}[\alpha_i/\aggr_{i}{}f_i(\mu'(\vec{z}_{i}))] }_{\mathsf{DISTINCT}}
  \]
  %
  \nd where the variables~$\alpha_i \not\in var(P)$,~$\vec{z}_i \in var(P)$ and none of the~$\keyword{GROUPBY}$
  variables~$\vec{w}$ are included in the aggregation function variables~$\vec{z}_i$.  Here, we denote
  by~$\mu|_{\vec{w}}$ the restriction of variables in~$\mu$ to variables in~$\vec{w}$. Using this notation, we can also
  straightforwardly introduce projection, \ie~sub-SELECTs as an algebraic operator in the language covering another new
  feature of SPARQL~1.1: 
  %
  \[
  \eval{\keyword{SELECT}\ \vec{V}\ \{ P \} }_{G} = \set{ \mu \mid \mu'\mbox{ in }\eval{P}_{G}, \mu = \mu'|_{\vec{v}} } \enspace .
  \]
  %
\end{definition}

Please note that the aggregator functions have a domain of definition and thus can only be applied to values of their
respective domain.  For example,~$\keyword{SUM}$ and~$\keyword{AVG}$ can only be used on numeric values,
while~$\keyword{MAX}, \keyword{MIN}$ are applicable to any total order.
% 
The~$\keyword{COUNT}$ aggregator can be used for any finite set of values. The last two aggregation functions,
namely~$\oplus$ and~$\otimes$, are defined by the annotation domain and thus can be used on any annotation variable.


\begin{example}[Grouping in AnQL]
  \label{ex:AA}
  Suppose we want to know, for each artist, the average length of their membership with different bands. Then such a
  query will be expressed as:
  %
  \lstinputlisting[frame=none,numbers=none]{0-data+queries/select-average.sparql}
  % 
  \nd Essentially, we group by the artists, compute for each artist the time he was a member of a band (by means of the
  built-in function~$\funcName{length}$), and compute the average value for each group.
  % 
  That is, if~$g = \{\tuple{t, t_{1}},\ldots, \tuple{t, t_{n}}\}$ is a group of tuples with the same value~$t$ for
  artist~$x$, and value~$t_{i}$ for~$y$, where each length of membership for~$t_{i}$ is~$l_{i}$ (computed
  as~$length(\cdot)$), then the value of~$avgL$ for the group~$g$ is~$(\sum_{i} l_{i})/n$. 
  %
\end{example}


%%% Local Variables:
%%% mode: latex
%%% TeX-master: "../thesis"
%%% End:
