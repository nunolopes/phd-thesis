\subsection{Deductive system}
\label{sec:anql-deductive-system}
An important feature of our framework is that we are able to provide a deductive system in the style of the one for
classical \ac{RDFS}.  Moreover, \emph{the schemata of the rules are the same for any annotation domain} (only support for the
domain dependent~$\otimes$ and~$\oplus$ operations has to be provided).
%
The rules of our deductive system, as in \cref{sec:rdf-schema}, are arranged in groups that capture the semantic
conditions of models, where~$A,B,C,X$, and~$Y$ are meta-variables representing elements in~$\AUBL$ and~$D,E$ represent
elements in~$\AUL$.
%
The rule set contains two rules,~$(1a)$ and $(1b)$, that are the same as for the crisp case, while rules~$(2a)$
to~$(5b)$ are the annotated rules homologous to the crisp ones. 
%
\ifnormalisedardf%
\else%
Finally, rule~$(6)$ is specific to the annotated case.
\fi%
%
  \begin{description}
  \item[1. Simple:]~ \\[0.5em]
    \begin{tabular}{ll}
      $(a) \quad \frac{G}{G'}$ for an annotated map $\theta:G' \to G$ & 
      \ifnormalisedardf%
      $(b) \quad \frac{G}{G'}$ for $G' \preceq G$
      \else%
      $(b) \quad \frac{G}{G'}$ if $\forall \fuzzyg{\tau}{\lambda_{1}}
      \in G', \exists\fuzzyg{\tau}{\lambda_{2}} \in G$ such that~$\lambda_{1} \preceq \lambda_{2}$
      \fi
    \end{tabular}
  \item[2. Subproperty:]~ \\[0.5em]
    \begin{tabular}{ll}
      $(a) \quad \frac{\fuzzyg{\triple{A,  \spp, B}}{\lambda_{1}},  \fuzzyg{\triple{B,  \spp, C}}{\lambda_{2}}}{\fuzzyg{\triple{A,  \spp, C}}{\lambda_{1}\otimes \lambda_{2}}}$ &
      $(b) \quad \frac{\fuzzyg{\triple{D, \spp, E}}{\lambda_{1}},  \fuzzyg{\triple{X, D, Y}}{\lambda_{2}}}{\fuzzyg{\triple{X, E, Y}}{\lambda_{1} \otimes \lambda_{2}}}$
    \end{tabular}
  \item[3. Subclass:]~ \\[0.5em]
    \begin{tabular}{ll}
      $(a) \quad \frac{\fuzzyg{\triple{A, \subclass, B}}{\lambda_{1}},  \fuzzyg{\triple{B, \subclass, C}}{\lambda_{2}}}{\fuzzyg{\triple{A, \subclass, C}}{\lambda_{1} \otimes \lambda_{2}}}$ &
      $(b) \quad \frac{\fuzzyg{\triple{A, \subclass, B}}{\lambda_{1}},  \fuzzyg{\triple{X, \typeR, A}}{\lambda_{2}}}{\fuzzyg{\triple{X, \typeR, B}}{\lambda_{1} \otimes \lambda_{2}}}$
    \end{tabular}
  \item[4. Typing:]~ \\[0.5em]
    \begin{tabular}{ll}
      $(a) \quad \frac{\fuzzyg{\triple{D, \domR, B}}{\lambda_{1}},  \fuzzyg{\triple{X, D, Y}}{\lambda_{2}}}{\fuzzyg{\triple{X, \typeR, B}}{\lambda_{1} \otimes \lambda_{2}}}$ &
  $(b) \quad \frac{\fuzzyg{\triple{D, \range, B}}{\lambda_{1}},  \fuzzyg{\triple{X, D, Y}}{\lambda_{2}}}{\fuzzyg{\triple{Y, \typeR, B}}{\lambda_{1} \otimes \lambda_{2}}}$
    \end{tabular}
  \item[5. Implicit Typing:]~ \\[0.5em]
    \begin{tabular}{l}
      $(a) \quad \frac{\fuzzyg{\triple{A, \domR, B}}{\lambda_{1}},  \fuzzyg{\triple{D,  \spp, A}}{\lambda_{2}},
        \fuzzyg{\triple{X, D, Y}}{\lambda_{3}}}{\fuzzyg{\triple{X, \typeR, B}}{\lambda_{1} \otimes \lambda_{2} \otimes
          \lambda_{3}}}$ \\[1.5em]
      $(b) \quad \frac{\fuzzyg{\triple{A, \range, B}}{\lambda_{1}},  \fuzzyg{\triple{D,  \spp, A}}{\lambda_{2}}, \fuzzyg{\triple{X, D, Y}}{\lambda_{3}}}{\fuzzyg{\triple{Y, \typeR, B}}{\lambda_{1} \otimes \lambda_{2} \otimes \lambda_{3}}}$
    \end{tabular}
\ifnormalisedardf%
\else%
  \item[6. Generalisation:]~ \\[0.5em]
    \begin{tabular}{l}
      $\frac{\fuzzyg{\triple{X, A, Y}}{\lambda_{1}},  \fuzzyg{\triple{X, A, Y}}{\lambda_{2}}}{\fuzzyg{\triple{X, A, Y}}{\lambda_{1} \oplus \lambda_{2}}}$
    \end{tabular}
\fi%
\end{description}
%
\ifnormalisedardf%
Please note we assume that a rule is not applied if the consequence is of the form~$\fuzzyg{\tau}{\bot}$ (see
\cref{sec:anql-semantics}).
%  
\else%
Please note that rule~$(6)$ is destructive \ie~this rule removes the premises as the conclusion is inferred.
%
We also assume that a rule is not applied if the consequence is of the form~$\fuzzyg{\tau}{\bot}$ 
%
(see \cref{sec:anql-semantics}).
\fi%


\begin{comment}
%
Alternatively, rules~2--5 can be represented concisely using the following inference rule:
%
\begin{center}
  \begin{tabular}{cl}
    $(AG)$ & $\frac{
      \fuzzyg{\tau_{1}}{\lambda_{1}},\ \ldots,\ \fuzzyg{\tau_{n}}{\lambda_{n}, \{\tau_{1}, \ldots \tau_{n}\} \vdash_{\mathsf{RDFS}} \tau}
    }
    {
      \fuzzyg{\tau}{\bigotimes_{i} \lambda_{i}}
    }$
  \end{tabular}
\end{center}
%
\nd Essentially, this rule says that if a classical \ac{RDFS} triple~$\tau$ can be inferred by applying a classical
\ac{RDFS} inference rule to triples~$\tau_{1}, \ldots, \tau_{n}$ (denoted~$\{\tau_{1}, \ldots, \tau_{n}\}
\vdash_{\mathsf{RDFS}} \tau$), then the annotation term of~$\tau$ will be~$\bigotimes_{i} \lambda_{i}$,
where~$\lambda_{i}$ is the annotation of triple~$\tau_{i}$.  It follows immediately that, using rule~$(AG)$, in
addition to~\ifnormalisedardf rule~$1$~\else rules~$1$ and~$6$~\fi from the deductive system above, it is easy to
extend these rules to cover complete \ac{RDFS}.
%
\end{comment}

\ifnormalisedardf%
Like for the classical case, the \emph{closure} is defined as~$\fc{cl}{G} = \set{ \fuzzyg{\tau}{\lambda} \mid G
  \vdash^{*} \fuzzyg{\tau}{\lambda} }$, where~$\vdash^{*}$ is as~$\vdash$ without rule~$(1a)$.  Note again that the size
of the closure of~$G$ is polynomial in~$|G|$ and can be computed in polynomial time, provided that the computational
complexity of operations~$\otimes$ and~$\oplus$ are polynomially bounded (from a computational complexity point of view,
it is as for the classical case, plus the cost of the operations~$\otimes$ and~$\oplus$ in~$L$).
%
\fi


\ifnormalisedardf%
Furthermore note that~$\fc{cl}{G}$ is not guaranteed to be a normalised annotated \ac{RDF} graph.  In order to ensure a
normalised graph we can apply the following rule, denoted \emph{generalisation rule}:
%
$$\frac{\fuzzyg{\triple{X, A, Y}}{\lambda_{1}},  \fuzzyg{\triple{X, A, Y}}{\lambda_{2}}}{\fuzzyg{\triple{X, A,
      Y}}{\lambda_{1} \oplus \lambda_{2}}} \enspace ,$$
%
where each application of this rule removes the premises from the graph. 
%
\fi%
%
Let us show an example of the application of the generalisation rule\ifnormalisedardf\else~$(6)$\fi.
%
\begin{example}[Generalisation operation]
  %
  Consider the following triples along with our running example from \cref{fig:dataset-example}:
  %
    \[
    \begin{array}{l}
      %
      \fuzzyg{\triple{\qname{foaf}{name}, \domR, \qname{foaf}{Person}}}{[-\infty,+\infty]}\footnote{We assume this
        triple is valid to reduce the number of triples shown, the domain of \qname{foaf}{name} is in fact \qname{owl}{Thing}.} \\
      \fuzzyg{\triple{\qname{foaf}{Person}, \subclass, \qname{foaf}{Agent}}}{[-\infty,+\infty]} \\
      \fuzzyg{\triple{\qname{mo}{MusicArtist}, \subclass, \qname{foaf}{Agent}}}{[-\infty,+\infty]} \enspace ,
      % 
    \end{array}
    \]
  % 
  we infer the following triples:
  % 
    \[
    \begin{array}{l}
      \fuzzyg{\triple{\qname{dbpedia}{Marco\mathunderscore{}Hietala}, \typeR, \qname{foaf}{Agent}}}{[1984,2012]} \\
      \fuzzyg{\triple{\qname{dbpedia}{Marco\mathunderscore{}Hietala}, \typeR, \qname{foaf}{Agent}}}{[1966,2012]} \enspace .
    \end{array}
    \]
  % 
  The application of the ``Generalisation'' rule will collapse the triples:
  % 
    \[
    \begin{array}{l}
      \fuzzyg{\triple{\qname{dbpedia}{Marco\mathunderscore{}Hietala}, \typeR, \qname{foaf}{Agent}}}{[1966,2012]}  \enspace ,
    \end{array}
    \]
  % 
  since~$[1984,2012] \oplus [1966,2012] = [1966,2012]$.
  %
\end{example}

\ifnormalisedardf%
\else%
%
Finally, like for the classical case, the \emph{closure} is defined as~$\fc{cl}{G} = \{\fuzzyg{\tau}{\lambda} \mid G
\vdash^{*} \fuzzyg{\tau}{\lambda} \}$, where~$\vdash^{*}$ is as~$\vdash$ without rule~$(1a)$.  Note again that the size
of the closure of~$G$ is polynomial in~$|G|$ and can be computed in polynomial time, provided that the computational
complexity of operations~$\otimes$ and~$\oplus$ are polynomially bounded (from a computational complexity point of view,
it is as for the classical case, plus the cost of the operations~$\otimes$ and~$\oplus$ in~$L$).
%
\fi


\begin{proposition}[Soundness and completeness]
  For two annotated graphs~$G$ and~$G'$, the proof system~$\vdash$ is sound and complete for~$\models$, that is~$G
  \vdash G'$ \iff~$G \models G'$.
\end{proposition}


\begin{proof}[Extends~\citep{MunozPerezGutierrez:2009aa}]
  ($\Rightarrow$) 
  %
  Let~$\I=\tuple{\Delta_{R}, \Delta_{P}, \Delta_{C}, \Delta_{L}, \intP{\cdot}, \intC{\cdot},\int{\cdot}}$ be an
  interpretation such that~$\I \models G$ \ie~$\I$ satisfies all the conditions of \cref{def:annotated-model}.
  %
  Furthermore let~$\fuzzyg{\tau}{\lambda}$ be the result from an instantiation of a \ifnormalisedardf rule~$2-5$\else
  rule~$2-6$\fi, such that~$\frac{R}{\fuzzyg{\tau}{\lambda}}$, where~$R \subseteq G$ and
  let~$G'=G\cup\set{\fuzzyg{\tau}{\lambda}}$,
  then~$\I \models G'$.
  % 
  \begin{description}[nosep]
  \item[1. Simple:]~
    \begin{enumerate}[label=(\alph*),nosep]
    \item We show that if~$G' \to G$ then~$G \models G'$.  Let~$\theta$ be an annotated map such that~$\fc{\theta}{G'}
      \preceq G$. Consider the function~$A' \colon \AB \to \Delta_R$ such that~$$\fc{A'}{x} =
      \left\{\begin{array}{l}\fc{A}{\fc{\theta}{x}}~\mathrm{if}~\fc{\theta}{x} \in \AB\\
          \int{\fc{\theta}{x}}~\mathrm{if}~\fc{\theta}{x} \not\in \AB
        \end{array}\right. \enspace .$$
      %
      Note that:
      %
      \begin{enumerate*}
      \item if~$x\in\AB$ and~$\fc{\theta}{x} \in \AB$ we get that~$\intA{\fc{\theta}{x}} = \fc{A}{\fc{\theta}{x}} =
        \fc{A'}{x} = \intAP{x}$;
      \item if~$x\in\AB$ and~$\fc{\theta}{x} \not\in \AB$ we get that~$\intA{\fc{\theta}{x}} = \int{\fc{\theta}{x}} =
        \fc{A'}{x} = \intAP{x}$;
      \item if~$x\not\in\AB$ we get that~$\fc{\theta}{x} = x$ and~$\intA{\fc{\theta}{x}} = \int{x} = \fc{A'}{x} =
        \intAP{x}$.
      \end{enumerate*}
      %
      Thus we have that~$\intA{\fc{\theta}{x}} = \intAP{x}$ for all~$x\in\AUB$.
      %
      Let~$\fuzzyg{\triple{s,p,o}}{\lambda} \in G'$,
      then~$\fuzzyg{\triple{\fc{\theta}{s},\fc{\theta}{p},\fc{\theta}{o}}}{\lambda} =
      \fuzzyg{\triple{\fc{\theta}{s},p,\fc{\theta}{o}}}{\lambda} \in G$.
      %
      Since~$\I \models G$ we have that~$\int{p} \in \Delta_P$ and~$\intP{\intA{\fc{\theta}{s}},\intA{\fc{\theta}{s}}}
      \succeq \lambda$ and~$\intP{\intAP{\fc{\theta}{s}},\intAP{\fc{\theta}{s}}} \succeq \lambda$.  
      %
      Thus~$\I$ satisfies condition~$(1)$ from \cref{def:annotated-model} for~$G'$ (with function~$A'$) and satisfies
      all other conditions of \cref{def:annotated-model}, so~$\I \models G'$.

      \ifnormalisedardf%
    \item if~$G' \preceq G$ then~$G' \to G$ and thus~$G \models G'$.%
      \else%
    \item if $\forall \fuzzyg{\tau}{\lambda_{1}} \in G', \exists\fuzzyg{\tau}{\lambda_{2}} \in G$ such
      that~$\lambda_{1} \preceq \lambda_{2}$ then~$G' \to G$ and thus~$G \models G'$.%
      \fi
    \end{enumerate}
  \item[2. Subproperty:]~ 
    \begin{enumerate}[label=(\alph*),nosep]
    \item Let~$\I \models \fuzzyg{\triple{A, \spp, B}}{\lambda_{1}}$ and~$\I \models \fuzzyg{\triple{B, \spp,
          C}}{\lambda_{2}}$.  
      %
      It follows that~$\fc{\intP{\int{\spp}}}{\int{A}, \int{B}} \succeq \lambda_{1}$
      and~$\fc{\intP{\int{\spp}}}{\int{B}, \int{C}} \succeq \lambda_{2}$.
      % 
      According to condition ``Subproperty 1.'' from \cref{def:annotated-model}, we have
      that~$\fc{\intP{\int{\spp}}}{\int{A}, \int{B}} \otimes \fc{\intP{\int{ \spp}}}{\int{B}, \int{C}} \preceq
      \fc{\intP{\int{ \spp}}}{\int{A}, \int{C}}$ and thus~$\lambda_{1} \otimes \lambda_{2} \preceq
      \fc{\intP{\int{\spp}}}{\int{A}, \int{C}}$.
      %
      Therefore~$\I \models \fuzzyg{\triple{A, \spp, C}}{\lambda_{1}\otimes\lambda_{2}}$.

    \item Let~$\I \models \fuzzyg{\triple{D, \spp, E}}{\lambda_{1}}$ and~$\I \models \fuzzyg{\triple{X, D,
          Y}}{\lambda_{2}}$.
      %
      It follows that~$\fc{\intP{\int{\spp}}}{\int{D}, \int{E}} \succeq \lambda_{1}$
      and~$\fc{\intP{\int{D}}}{\int{X}, \int{Y}} \succeq \lambda_{2}$.
      % 
      According to condition ``Subproperty 2.'' from \cref{def:annotated-model}, we have
      that~$\fc{\intP{\int{\spp}}}{\int{D}, \int{E}} \otimes \fc{\intP{\int{D}}}{\int{X}, \int{Y}} \preceq
      \fc{\intP{\int{E}}}{\int{X}, \int{Y}}$ and thus~$\lambda_{1} \otimes \lambda_{2} \preceq
      \fc{\intP{\int{E}}}{\int{X}, \int{Y}}$.
      %
      Therefore~$\I \models \fuzzyg{\triple{X, E, Y}}{\lambda_{1}\otimes\lambda_{2}}$.
    \end{enumerate}
  \item[3. Subclass:]~ 
    \begin{enumerate}[label=(\alph*),nosep]
    % \item $\frac{\fuzzyg{\triple{A, \subclass, B}}{\lambda_{1}}, \fuzzyg{\triple{B, \subclass,
    %         C}}{\lambda_{2}}}{\fuzzyg{\triple{A, \subclass, C}}{\lambda_{1} \otimes \lambda_{2}}}$
    \item Let~$\I \models \fuzzyg{\triple{A, \subclass, B}}{\lambda_{1}}$ and~$\I \models \fuzzyg{\triple{B, \subclass,
          C}}{\lambda_{2}}$.  
      %
      It follows that~$\fc{\intP{\int{\subclass}}}{\int{A}, \int{B}} \succeq \lambda_{1}$
      and~$\fc{\intP{\int{\subclass}}}{\int{B}, \int{C}} \succeq \lambda_{2}$.
      % 
      According to condition ``Subclass 1.'' from \cref{def:annotated-model}, we have
      that~$\fc{\intP{\int{\subclass}}}{\int{A}, \int{B}} \otimes \fc{\intP{\int{ \subclass}}}{\int{B}, \int{C}} \preceq
      \fc{\intP{\int{\subclass}}}{\int{A}, \int{C}}$ and thus~$\lambda_{1} \otimes \lambda_{2} \preceq
      \fc{\intP{\int{\subclass}}}{\int{A}, \int{C}}$.
      %
      Therefore~$\I \models \fuzzyg{\triple{A, \subclass, C}}{\lambda_{1}\otimes\lambda_{2}}$.

    % \item $\frac{\fuzzyg{\triple{A, \subclass, B}}{\lambda_{1}}, \fuzzyg{\triple{X, \typeR,
    %         A}}{\lambda_{2}}}{\fuzzyg{\triple{X, \typeR, B}}{\lambda_{1} \otimes \lambda_{2}}}$
    \item Let~$\I \models \fuzzyg{\triple{A, \subclass, B}}{\lambda_{1}}$ and~$\I \models \fuzzyg{\triple{X, \typeR,
          A}}{\lambda_{2}}$.
      %
      It follows that~$\fc{\intP{\int{\subclass}}}{\int{A}, \int{B}} \succeq \lambda_{1}$
      and~$\fc{\intP{\int{\typeR}}}{\int{X}, \int{A}} \succeq \lambda_{2}$.
      % 
      From condition ``Typing I, 1.'' (\cref{def:annotated-model}), we have that~$\fc{\intC{\int{A}}}{\int{X}} \succeq
      \lambda_{2}$.
      % 
      According to condition ``Subclass 2.'' from \cref{def:annotated-model}, we have
      that~$\fc{\intP{\int{\subclass}}}{\int{A}, \int{B}} \otimes \fc{\intC{\int{A}}}{\int{X}} \preceq
      \fc{\intC{\int{B}}}{\int{X}}$ and thus~$\lambda_{1} \otimes \lambda_{2} \preceq \fc{\intC{\int{B}}}{\int{X}}$ and,
      again from condition ``Typing I, 1.'', $\lambda_{1} \otimes \lambda_{2} \preceq \fc{\intP{\int{\typeR}}}{\int{X}, \int{B}}$.
      %
      Therefore~$\I \models \fuzzyg{\triple{X, \typeR, B}}{\lambda_{1}\otimes\lambda_{2}}$.
      


    \end{enumerate}
  \item[4. Typing:]~ 
    \begin{enumerate}[label=(\alph*),nosep]
    \item Let~$\I \models \fuzzyg{\triple{D, \domR, B}}{\lambda_{1}}$ and~$\I \models \fuzzyg{\triple{X, D,
          Y}}{\lambda_{2}}$.
      %
      It follows that~$\fc{\intP{\int{\domR}}}{\int{D}, \int{B}} \succeq \lambda_{1}$
      and~$\fc{\intP{\int{D}}}{\int{X}, \int{Y}} \succeq \lambda_{2}$.
      % 
      From condition ``Typing I, 2.'' (\cref{def:annotated-model}), we have that~$\fc{\intP{\int{\domR}}}{\int{D},
        \int{B}} \otimes \fc{\intP{\int{D}}}{\int{X}, \int{Y}} \preceq \fc{\intC{\int{B}}}{\int{X}}$ and
      thus~$\lambda_{1} \otimes \lambda_{2} \preceq \fc{\intC{\int{B}}}{\int{X}}$.
      % 
      From condition ``Typing I, 1.'' (\cref{def:annotated-model}), we have that~$\lambda_{1} \otimes \lambda_{2}
      \preceq \fc{\intP{\int{\typeR}}}{\int{X},\int{B}}$.
      % 
      Therefore~$\I \models \fuzzyg{\triple{X, \typeR, B}}{\lambda_{1} \otimes \lambda_{2}}$.

    \item Let~$\I \models \fuzzyg{\triple{D, \range, B}}{\lambda_{1}}$ and~$\I \models \fuzzyg{\triple{X, D,
          Y}}{\lambda_{2}}$.
      %
      It follows that~$\fc{\intP{\int{\range}}}{\int{D}, \int{B}} \succeq \lambda_{1}$ and~$\fc{\intP{\int{D}}}{\int{X},
        \int{Y}} \succeq \lambda_{2}$.
      % 
      From condition ``Typing I, 3.'' (\cref{def:annotated-model}), we have that~$\fc{\intP{\int{\range}}}{\int{D},
        \int{B}} \otimes \fc{\intP{\int{D}}}{\int{X}, \int{Y}} \preceq \fc{\intC{\int{B}}}{\int{Y}}$ and
      thus~$\lambda_{1} \otimes \lambda_{2} \preceq \fc{\intC{\int{B}}}{\int{Y}}$.
      % 
      From condition ``Typing I, 1.'' (\cref{def:annotated-model}), we have that~$\lambda_{1} \otimes \lambda_{2}
      \preceq \fc{\intP{\int{\typeR}}}{\int{Y},\int{B}}$.
      % 
      Therefore~$\I \models \fuzzyg{\triple{Y, \typeR, B}}{\lambda_{1} \otimes \lambda_{2}}$.

    \end{enumerate}
  \item[5. Implicit Typing:]~ 
    \begin{enumerate}[label=(\alph*),nosep]
      %
    \item Let~$\I \models \fuzzyg{\triple{A, \domR, B}}{\lambda_{1}}$, $\I \models \fuzzyg{\triple{D, \spp,
          A}}{\lambda_{2}}$, and~$\I \models \fuzzyg{\triple{X, D, Y}}{\lambda_{3}}$.
      %
      It follows that~$\fc{\intP{\int{\domR}}}{\int{A}, \int{B}} \succeq \lambda_{1}$, $\fc{\intP{\int{\spp}}}{\int{D},
        \int{A}} \succeq \lambda_{2}$, and~$\fc{\intP{\int{D}}}{\int{X}, \int{Y}} \succeq \lambda_{3}$.
      % 
      According to condition ``Subproperty 2.'' from \cref{def:annotated-model}, we have
      that~$\fc{\intP{\int{\spp}}}{\int{D}, \int{A}} \otimes \fc{\intP{\int{D}}}{\int{X}, \int{Y}} \preceq
      \fc{\intP{\int{A}}}{\int{X}, \int{Y}}$ and thus~$\lambda_{1} \otimes \lambda_{2} \preceq
      \fc{\intP{\int{A}}}{\int{X}, \int{Y}}$.
      % 
      From condition ``Typing I, 2.'' (\cref{def:annotated-model}), we have that~$\fc{\intP{\int{\domR}}}{\int{A},
        \int{B}} \otimes \fc{\intP{\int{A}}}{\int{X}, \int{Y}} \preceq \fc{\intC{\int{B}}}{\int{X}}$ and
      thus~$\lambda_{1} \otimes \lambda_{2} \otimes \lambda_{3} \preceq \fc{\intC{\int{B}}}{\int{X}}$.
      % 
      From condition ``Typing I, 1.'' (\cref{def:annotated-model}), we have that~$\lambda_{1} \otimes \lambda_{2}
      \otimes \lambda_{3} \preceq \fc{\intP{\int{\typeR}}}{\int{X},\int{B}}$.
      % 
      Therefore~$\I \models \fuzzyg{\triple{X, \typeR, B}}{\lambda_{1} \otimes \lambda_{2} \otimes \lambda_{3}}$.
      %
    \item Let~$\I \models \fuzzyg{\triple{A, \range, B}}{\lambda_{1}}$, $\I \models \fuzzyg{\triple{D, \spp,
          A}}{\lambda_{2}}$, and~$\I \models \fuzzyg{\triple{X, D, Y}}{\lambda_{3}}$.
      %
      It follows that~$\fc{\intP{\int{\range}}}{\int{A}, \int{B}} \succeq \lambda_{1}$, $\fc{\intP{\int{\spp}}}{\int{D},
        \int{A}} \succeq \lambda_{2}$, and~$\fc{\intP{\int{D}}}{\int{X}, \int{Y}} \succeq \lambda_{3}$.
      % 
      According to condition ``Subproperty 2.'' from \cref{def:annotated-model}, we have
      that~$\fc{\intP{\int{\spp}}}{\int{D}, \int{A}} \otimes \fc{\intP{\int{D}}}{\int{X}, \int{Y}} \preceq
      \fc{\intP{\int{A}}}{\int{X}, \int{Y}}$ and thus~$\lambda_{1} \otimes \lambda_{2} \preceq
      \fc{\intP{\int{A}}}{\int{X}, \int{Y}}$.
      % 
      From condition ``Typing I, 3.'' (\cref{def:annotated-model}), we have that~$\fc{\intP{\int{\range}}}{\int{A},
        \int{B}} \otimes \fc{\intP{\int{A}}}{\int{X}, \int{Y}} \preceq \fc{\intC{\int{B}}}{\int{X}}$ and
      thus~$\lambda_{1} \otimes \lambda_{2} \otimes \lambda_{3} \preceq \fc{\intC{\int{B}}}{\int{X}}$.
      % 
      From condition ``Typing I, 1.'' (\cref{def:annotated-model}), we have that~$\lambda_{1} \otimes \lambda_{2}
      \otimes \lambda_{3} \preceq \fc{\intP{\int{\typeR}}}{\int{X},\int{B}}$.
      % 
      Therefore~$\I \models \fuzzyg{\triple{X, \typeR, B}}{\lambda_{1} \otimes \lambda_{2} \otimes \lambda_{3}}$.
    \end{enumerate}
    \ifnormalisedardf%
    \else%
  \item[6. Generalisation:]~ 
    Let~$\I \models \fuzzyg{\triple{X, A, Y}}{\lambda_{1}}$ and $\I \models \fuzzyg{\triple{X, A, Y}}{\lambda_{2}}$.
    % 
    It follows that~$\fc{\intP{\int{A}}}{\int{X}, \int{Y}} \succeq \lambda_{1}$ and $\fc{\intP{\int{A}}}{\int{X},
      \int{Y}} \succeq \lambda_{2}$.
    % 
    The~$\oplus$ operation, which corresponds to the least upper bound in the partial order induced over~$L$, guarantees
    that~$\lambda_{1} \oplus \lambda_{2} \succeq \lambda_{1}$ and~$\lambda_{1} \oplus \lambda_{2} \succeq \lambda_{2}$
    and thus
    % 
    we have that~$\fc{\intP{\int{A}}}{\int{X}, \int{Y}} \succeq \lambda_{1} \oplus \lambda_{2}$.
    % 
    Therefore~$\I \models \fuzzyg{\triple{X, A, Y}}{\lambda_{1} \oplus \lambda_{2}}$.
    % 
    Since~$\lambda_{1} \oplus \lambda_{2} \succeq \lambda_{1}$ and~$\lambda_{1} \oplus \lambda_{2} \succeq \lambda_{2}$,
    the triples~$\fuzzyg{\triple{X, A, Y}}{\lambda_{1}}$ and $\fuzzyg{\triple{X, A, Y}}{\lambda_{2}}$ are redundant and
    can be removed from~$G$.
    % 
    \fi
  \end{description}
  
  ($\Leftarrow$)
  %
  Given an annotated graph~$G$, let~$\I=\tuple{\Delta_{R}, \Delta_{P}, \Delta_{C}, \Delta_{L}, \intP{\cdot},
    \intC{\cdot},\int{\cdot}}$ be an interpretation defined as follows:
  %
  \begin{itemize}[nosep]
  \item $\Delta_{R} = \fc{\universe}{G} \cup \rhodf$;
  \item $\Delta_{P} = \set{p \in \fc{\voc}{G} \mid \fuzzyg{\triple{s,p,o}}{\lambda} \in \fc{cl}{G}} \cup $ \\ $ \rhodf \cup
    \set{p \in \fc{\universe}{G} \mid \fuzzyg{\triple{p,\spp,x}}{\lambda}, \fuzzyg{\triple{y,\spp,p}}{\lambda},
      \fuzzyg{\triple{p,\domR,z}}{\lambda}~\mathrm{or}~\fuzzyg{\triple{p,\range,v}}{\lambda} \in G}$;
  \item $\Delta_{C} = \set{c \in \fc{\universe}{G} \mid \fuzzyg{\triple{x,\typeR,c}}{\lambda} \in G} \cup $ \\ $\set{c
      \in \fc{\universe}{G} \mid \fuzzyg{\triple{c,\subclass,x}}{\lambda}, \fuzzyg{\triple{y,\subclass,c}}{\lambda},
        \fuzzyg{\triple{z,\domR,c}}{\lambda}~\mathrm{or}~\fuzzyg{\triple{v,\range,c}}{\lambda} \in G}$;
  \item $\Delta_{L} = \AL \cap \fc{\universe}{G}$;

  \item $\intP{\cdot}: \Delta_{P} \to 2^{\Delta_{R}\times\Delta_{R}}$ is an interpretation function such that:
    \begin{itemize}[nosep]
    \item if~$p \in \AU \cap \Delta_{P}~\mathrm{and}~\fuzzyg{\triple{x,p,y}}{\lambda} \in \fc{cl}{G}$
      then~$\fc{\intP{p}}{x,y} \succeq \lambda$;
    \item if~$p \in \AB \cap
      \Delta_{P}~\mathrm{and}~\fuzzyg{\triple{p,\spp,p'}}{\lambda_{1}},\fuzzyg{\triple{x,p',y}}{\lambda_{2}} \in
      \fc{cl}{G}$ then~$\fc{\intP{p}}{x,y} \succeq \lambda_{1} \otimes \lambda_{2}$ such that~$\lambda_{1} \otimes
      \lambda_{2} \neq \bot$.
    \end{itemize}

  \item $\intC{\cdot}: \Delta_{C} \to L$ is an interpretation function such that~$\fuzzyg{\triple{x,\typeR,c}}{\lambda}
    \in \fc{cl}{G}, \fc{\intC{c}}{x} \succeq \lambda$;

  \item $\int{\cdot}$ is the identity function over~$\fc{\universe}{G} \cup \rhodf$.
  \end{itemize}
  %
  We have that~$\I \models G$ if $\I$ satisfies all the conditions from \cref{def:annotated-model}:
  %
  \begin{description}[nosep]
  \item[Simple:]~
    \begin{enumerate}[label=(\alph*),nosep]
    %   
    \item First note that from the construction of~$\fc{cl}{G}$, $\fc{\universe}{\fc{cl}{G}} = \fc{\universe}{G} \cup
      \rhodf$.  
      % 
      Let~$\fuzzyg{\triple{s,p,o}}{\lambda} \in G$ then, from the construction of~$\I$, we have that~$\int{p} = p \in
      \Delta_P$ and~$\fc{\intP{p}}{\int{s},\int{o}} = \lambda$ and thus~$\I$ satisfies condition (i) for~$G$.

    \item We now show that if~$G \models G'$ then there is an annotated map~$G' \to \fc{cl}{G}$.  From the construction
      of~$\I$ we have that~$\I \models G$ and since~$G \models G'$, $\I \models G'$.  Since~$\I$ satisfies
      condition~$(i)$ there exists a function~$A \colon \AB \to \fc{\universe}{G}\cup\rhodf$ such that for
      each~$\fuzzyg{\triple{s,p,o}}{\lambda} \in G'$, $p\in\Delta_P$ and~$\fc{\intP{\int{p}}}{\intA{s}, \intA{o}} =
      \lambda$.  Since~$p\in\AU$, we have that~$\int{p} = \intA{p} = p$ and thus~$\fc{\intP{\int{p}}}{\intA{s},
        \intA{o}} = \fc{\intP{p}}{\intA{s}, \intA{o}} = \lambda$ for each~$\fuzzyg{\triple{s,p,o}}{\lambda} \in
      \fc{cl}{G}$.
      %
      Since~$\fc{\intP{\int{p}}}{\intA{s}, \intA{o}} = \lambda$, we have
      that~$\fuzzyg{\triple{\intA{s},\intA{p},\intA{o}}}{\lambda} \in\fc{cl}{G}$ for
      each~$\fuzzyg{\triple{s,p,o}}{\lambda} \in G'$.
      % 
      Thus~$\IA \colon G' \to \fc{cl}{G}$ is an annotated map~$G' \to \fc{cl}{G}$.
    \end{enumerate}
  \item[Subproperty:]~ 
    \begin{enumerate}[label=(\alph*),nosep]
    \item Let~$\fc{\intP{\int{\spp}}}{\int{A}, \int{B}} \succeq \lambda_{1}$ and~$\fc{\intP{\int{\spp}}}{\int{B},
        \int{C}} \succeq \lambda_{2}$.
      % 
      From the construction of~$\I$ we have that~$\fuzzyg{\triple{A,\spp,B}}{\lambda_{1}},
      \fuzzyg{\triple{B,\spp,C}}{\lambda_{2}} \in \fc{cl}{G}$ and~$A,B,C \in \Delta_P$.  Since~$\fc{cl}{G}$ is closed
      under application of rule~$(2a)$ we have that~$\fuzzyg{\triple{A,\spp,C}}{\lambda_{1} \otimes \lambda_{2}} \in
      \fc{cl}{G}$ and thus~$\fc{\intP{\int{\spp}}}{\int{A}, \int{B}} \succeq \lambda_{1} \otimes \lambda_{2}$.

    \item Let~$\fc{\intP{\int{D}}}{\int{X}, \int{Y}} \succeq \lambda_1$ and~$\fc{\intP{\int{\spp}}}{\int{D}, \int{E}}
      \succeq \lambda_2$, thus~$\fuzzyg{\triple{D,\spp,E}}{\lambda_2} \in \fc{cl}{G}$ and~$D,E\in\Delta_P$.
      %
      We must consider the following cases:
      \begin{itemize}[nosep]
      \item if~$D \in \AU$ then, from the construction of~$\I$ we have that~$\fuzzyg{\triple{X,D,Y}}{\lambda_1} \in
        \fc{cl}{G}$.
        %
        If~$E \in \AU$ and since~$\fc{cl}{G}$ is closed under the application of rule~$(2b)$, we also have
        that~$\fuzzyg{\triple{X,E,Y}}{\lambda_1 \otimes \lambda_2} \in \fc{cl}{G}$.
        % 
        Therefore~$\fc{\intP{\int{E}}}{\int{X}, \int{Y}} \succeq \lambda_1 \otimes \lambda_2$.
        %
        If~$E \in \AB$, then~$\fuzzyg{\triple{X,D,Y}}{\lambda_1}, \fuzzyg{\triple{D,\spp,E}}{\lambda_2} \in \fc{cl}{G}$,
        and from the construction of~$\I$ we have that~$\fc{\intP{\int{E}}}{\int{X}, \int{Y}} \succeq \lambda_1 \otimes
        \lambda_2$.
        %
      \item if~$D \in \AB$ by the construction of~$\I$ there exists~$D'$ such
        that~$\fuzzyg{\triple{D',\spp,D}}{\lambda_3}, \fuzzyg{\triple{X,D',Y}}{\lambda_4} \in \fc{cl}{G}$ and~$D' \in
        \Delta_{P}$.
        % 
        Since~$\fc{cl}{G}$ is closed under the application of rule~$(2a)$, we also have
        that~$\fuzzyg{\triple{D',\spp,E}}{\lambda_2 \otimes \lambda_3} \in \fc{cl}{G}$.
        %
        If~$E \in \AU$ as~$\fuzzyg{\triple{D',\spp,E}}{\lambda_2 \otimes \lambda_3}, \fuzzyg{\triple{X,D',Y}}{\lambda_4}
        \in \fc{cl}{G}$ and since~$\fc{cl}{G}$ is closed under the application of rule~$(2b)$, we also have
        that~$\fuzzyg{\triple{X,E,Y}}{\lambda_2 \otimes \lambda_3 \otimes \lambda_4} \in \fc{cl}{G}$.
        % 
        Therefore, from the construction of~$\I$, we have that~$\fc{\intP{\int{E}}}{\int{X}, \int{Y}} \succeq \lambda_2
        \otimes \lambda_3 \otimes \lambda_4$.
        %
        If~$E \in \AB$, then~$\fuzzyg{\triple{X,D,Y}}{\lambda_1}, \fuzzyg{\triple{D,\spp,E}}{\lambda_2} \in \fc{cl}{G}$,
        and from the construction of~$\I$ we have that~$\fc{\intP{\int{E}}}{\int{X}, \int{Y}} \succeq \lambda_1 \otimes
        \lambda_2$.
        %
      \end{itemize}
    \end{enumerate}
  \item[Subclass:]~ 
    \begin{enumerate}[label=(\alph*),nosep]
    \item Let~$\fc{\intP{\int{\subclass}}}{\int{A}, \int{B}} \succeq \lambda_{1}$
      and~$\fc{\intP{\int{\subclass}}}{\int{B}, \int{C}} \succeq \lambda_{2}$.
      % 
      From the construction of~$\I$ we have that~$\fuzzyg{\triple{A,\subclass,B}}{\lambda_{1}},
      \fuzzyg{\triple{B,\subclass,C}}{\lambda_{2}} \in \fc{cl}{G}$ and~$A,B,C \in \Delta_C$.  Since~$\fc{cl}{G}$ is
      closed under application of rule~$(3a)$ we have that~$\fuzzyg{\triple{A,\subclass,C}}{\lambda_{1} \otimes
        \lambda_{2}} \in \fc{cl}{G}$ and thus~$\fc{\intP{\int{\subclass}}}{\int{A}, \int{B}} \succeq \lambda_{1} \otimes
      \lambda_{2}$.

    \item Let~$\fc{\intC{\int{A}}}{\int{X}} \succeq \lambda_{1}$ and~$\fc{\intP{\int{\subclass}}}{\int{A}, \int{B}}
      \succeq \lambda_{2}$.
      %
      From condition ``Typing I, 1.'' (\cref{def:annotated-model}), we have that~$\fc{\intP{\int{\typeR}}}{\int{X},
        \int{A}} = \lambda_{2}$ and thus~$\I \models \fuzzyg{\triple{X, \typeR, A}}{\lambda_{2}}$.
      % 
      Since~$\fc{cl}{G}$ is closed under application of rule~$(3b)$ we have that~$\fuzzyg{\triple{X, \typeR, B}}{\lambda_1
        \otimes \lambda_2}$.
      %
      Then~$\fc{\intP{\int{\typeR}}}{\int{X}, \int{B}} \succeq \lambda_{1} \otimes \lambda_{2}$ and
      thus~$\fc{\intC{\int{B}}}{\int{X}} \succeq \lambda_{1} \otimes \lambda_{2}$.
      %
    \end{enumerate}
  \item[Typing I:]~
    \begin{enumerate}[label=(\alph*),nosep]
    \item Let~$\fc{\intP{\int{\typeR}}}{X,C} \succeq \lambda_1$, by construction of~$\I$ we have that~$C\in\Delta_C$
      and~$\fuzzyg{\triple{X,\typeR,C}}{\lambda_1} \in \fc{cl}{G}$.  Also by the construction of~$\intC{\cdot}$ we have
      that~$\fc{\intC{\int{C}}}{X} \succeq \lambda_1$.
      %
      On the other hand, if~$C\in\Delta_C$ and~$\fc{\intC{\int{C}}}{X} \succeq \lambda_1$, by construction
      of~$\intC{\cdot}$ we have that~$\fuzzyg{\triple{X, \typeR, C}}{\lambda_1} \in \fc{cl}{G}$ and
      so~$\fc{\intP{\int{\typeR}}}{X,C} \succeq \lambda_1$.
      %
    \item Let~$\fc{\intP{\int{\domR}}}{D, B} \succeq \lambda_1$ and~$\fc{\intP{D}}{X,Y} \succeq \lambda_2$.
      By construction of~$\I$ we have that~$D\in\Delta_P$ and~$B\in\Delta_C$. 
      % 
      Since~$\fc{cl}{G}$ is closed under application of rule~$(4a)$ we have that~$\fuzzyg{\triple{X, \typeR, B}}{\lambda_1
        \otimes \lambda_2} \in \fc{cl}{G}$.
      %
      Then~$\fc{\intP{\typeR}}{\int{X},\int{B}} \succeq \lambda_1 \otimes \lambda_2$ and~$\fc{\intC{\int{B}}}{\int{X}}
      \succeq \lambda_1 \otimes \lambda_2$.
      %
    \item Let~$\fc{\intP{\int{\range}}}{D, B} \succeq \lambda_1$ and~$\fc{\intP{D}}{X,Y} \succeq \lambda_2$.
      By construction of~$\I$ we have that~$D\in\Delta_P$ and~$B\in\Delta_C$. 
      % 
      Since~$\fc{cl}{G}$ is closed under application of rule~$(4b)$ we have that~$\fuzzyg{\triple{Y, \typeR, B}}{\lambda_1
        \otimes \lambda_2} \in \fc{cl}{G}$.
      %
      Then~$\fc{\intP{\typeR}}{\int{Y},\int{B}} \succeq \lambda_1 \otimes \lambda_2$ and~$\fc{\intC{\int{B}}}{\int{Y}}
      \succeq \lambda_1 \otimes \lambda_2$.
      %
    \end{enumerate}
  \item[Typing II:] The definition of~$\Delta_R$ and~$\Delta_P$ satisfy all of these conditions.
  \end{description}%
\end{proof}


%%% Local Variables:
%%% mode: latex
%%% TeX-master: "../thesis"
%%% End:



%%% Local Variables:
%%% mode: latex
%%% TeX-master: "../thesis"
%%% End:
