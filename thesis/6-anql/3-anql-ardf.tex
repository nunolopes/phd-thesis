\section{RDF(S) with Annotations}
\label{sec:annotated-rdfs}


For extending our running example we use the temporal domain, which allows us to annotate the RDF data with temporal
information. For instance, we can annotate the band members' triples to reflect their active years with the band. 
%
A possible temporal query that can be easily performed over data represented in this format is ``What were the members
of a band at a certain time?''
%
The use of other domains would allow to represent different views on the data, for example in the fuzzy domain, we can
represent information regarding part-time members of bands.



\begin{data}[t]
  \centering
  \lstinputlisting[basicstyle=\ttfamily\footnotesize]{0-data+queries/usecasedata-temporalRDF.ttl}
  %
  \caption{Temporal Annotated RDFS}
  \label{fig:dataset-example}
\end{data}



\cref{fig:dataset-example} represents an extension of our \usecase data from Data~\ref{fig:bands-turtle} annotated
with information from the temporal domain, which intuitively means that the annotated triple is valid in dates contained
in the annotation interval (the exact meaning of the annotations will be explained later).
%
In \cref{fig:dataset-example}, we are representing the annotated triples using N-Quads~\cite{CyganiakHarthHogan:2009aa},
a format that allows to attach a forth element to an \ac{RDF} triple.\footnote{A similar approach is followed for
  extending SPARQL's syntax.}
%
However, in examples and definitions of the rest of this chapter we will use a representation of the
form~$\fuzzyg{\triple{s,p,o}}{\lambda}$, which is considered equivalent to its N-Quad counterpart. 





\newif\ifnormalisedardf\normalisedardftrue
% \normalisedardffalse

\subsection{Syntax}

Our approach is to extend triples with annotations, where an annotation is taken from a specific domain.
%
This extension follows a similar approach to the annotated logic programming framework~\cite{KiferSubrahmanian:1992aa}.

\ifnormalisedardf%
\begin{definition}[Annotated RDF triple and graph]
  An \emph{annotated triple} is an expression~$\fuzzyg{\tau}{\lambda}$, where~$\tau$ is an \ac{RDF} triple and~$\lambda$
  is an \emph{annotation value} (defined below). An \emph{annotated graph} is a finite set of annotated triples.
  % 
  Furthermore we call an annotated graph~$G$ a \emph{normalised annotated graph}
  iff~$\not\exists\fuzzyg{\tau}{\lambda_{1}},\fuzzyg{\tau}{\lambda_{2}} \in G$ s.t.~$\lambda_{1}\neq\lambda_{2}$.
\end{definition}
% 
\else%
An \emph{annotated triple} is an expression~$\fuzzyg{\tau}{\lambda}$, where~$\tau$ is an \ac{RDF} triple and~$\lambda$
is an \emph{annotation value} (defined below). An \emph{annotated graph} is a finite set of annotated triples.
% 
\fi
%
The intended semantics of annotated triples depends of course on the meaning we associate to the annotation values. For
instance, in a temporal setting~\cite{GutierrezHurtadoVaisman:2007aa},
\[
\fuzzyg{\triple{\qname{dbpedia}{Nightwish}, \qname{foaf}{member}, \qname{dbpedia}{Marco\mathunderscore{}Hietala}}}{{[2001,2012]}}
\]
\nd has the intended meaning ``Marco Hietala was a member of Nightwish during the period from~$2001$ to~$2012$'',
% 
while in the fuzzy setting~\cite{Straccia:2009aa}, we can represent part-time members of a band:
\[
\fuzzyg{\triple{\qname{dbpedia}{Nightwish}, \qname{foaf}{member}, \qname{dbpedia}{Troy\mathunderscore{}Donockley}}}{0.7}
\]
\nd with the intended meaning ``Troy Donockley is a member of Nightwish to a degree not less than 0.7''.\footnote{The
  membership degree was chosen as an example, Troy has collaborated with Nightwish on different albums and live
  concerts.}


\subsection{Annotation Domain Specification}
\label{sec:rdfs-annot-doma}



To start with, let us consider a non-empty set~$L$, where the elements in~$L$ are our annotation values. For example, in
a fuzzy setting,~$L = [0,1]$, while in a typical temporal setting,~$L$ may be time points or time intervals.  In our
annotation framework, we extend the notion of interpretation (presented in \cref{def:interpretation}) to map
statements to elements of the annotation domain.  
%
But first let us define an annotation domain:




\begin{definition}[Annotation Domain]
  We say that an \emph{annotation domain} for \ac{RDFS} is an idempotent, commutative semi-ring
  \[
  D = \tuple{L,  \oplus, \otimes, \bot, \top} \ ,
  \]
  \nd where,~$\top,\bot$ are specific annotation values and~$\oplus$
  is~$\top$-annihilating~\citep{BunemanKostylev:2010aa}. That is, for~$\lambda, \lambda_{i} \in L$:

  \begin{enumerate}[noitemsep]
  \item $\oplus$ is idempotent, commutative, associative;
  \item $\otimes$ is commutative and associative;
  \item $\bot \oplus \lambda = \lambda$,~$\top \otimes \lambda = \lambda$,~$\bot \otimes \lambda = \bot$, and~$\top \oplus \lambda = \top$;
  \item $\otimes$ is distributive over~$\oplus$, \ie~$\lambda_{1} \otimes (\lambda_{2} \oplus \lambda_{3}) =
    (\lambda_{1} \otimes \lambda_{2}) \oplus (\lambda_{1} \otimes \lambda_{3})$.
  \end{enumerate}
\end{definition}
%
\nd Please note that there is a natural partial order on any idempotent semi-ring: an annotation domain~$D = \tuple{L,
  \oplus, \otimes, \bot, \top}$ induces a partial order~$\preceq$ over~$L$ defined as:
%
\[
\lambda_{1} \preceq \lambda_{2} \mbox{ \ \iff  \ }  \lambda_{1} \oplus \lambda_{2} = \lambda_{2} \ .
\]
%
The~$\top$ and~$\bot$ respective represent the highest and lowest element in the partial order.  This partial
order~$\preceq$ is used to express redundant information: for instance, for temporal intervals, an annotated
triple~$\fuzzyg{\triple{s,p,o}}{[2000,2006]}$ includes~$\fuzzyg{\triple{s,p,o}}{[2003,2004]}$, since~$[2003,2004]
\subseteq [2000,2006]$ (here,~$\subseteq$ plays the role of~$\preceq$).


In previous work~\cite{StracciaLopesLukacsy:2010aa,LopesPolleresStraccia:2010aa}, an annotation domain was assumed to be
a more specific structure, namely a residuated bounded lattice.  In~\citet{BunemanKostylev:2010aa} it was shown that we
may use a slightly weaker structure than residuated lattices for annotation domains.




We use~$\oplus$ to combine information about the same statement.  For instance, in temporal logic,
from~$\fuzzyg{\tau}{[2000,2006]}$ and~$\fuzzyg{\tau}{[2003,2008]}$, we infer~$\fuzzyg{\tau}{[2000,2008]}$,
as~$[2000,2008] = [2000,2006]\cup [2003,2008]$ (where~$\cup$ plays the role of~$\oplus$).
%
In the fuzzy context, from~$\fuzzyg{\tau}{0.7}$ and~$\fuzzyg{\tau}{0.6}$, we infer~$\fuzzyg{\tau}{0.7}$, since~$0.7 =
\max(0.7, 0.6)$ (here,~$\max$ plays the role of~$\oplus$).

We use~$\otimes$ to model the ``conjunction'' of information, where~$\otimes$ is a generalisation of boolean conjunction
to the many-valued case. In fact,~$\otimes$ satisfies also that:
%
\begin{enumerate}[noitemsep]
\item $\otimes$ is bounded: \ie~$\lambda_{1} \otimes \lambda_{2} \preceq \lambda_{1}$.
\item $\otimes$ is~$\preceq$-monotone, \ie~for~$\lambda_{1} \preceq \lambda_{2}$,~$\lambda \otimes \lambda_{1} \preceq
  \lambda \otimes \lambda_{2}$
\end{enumerate}
%
For instance, on interval-valued temporal logic, from~$\fuzzyg{\triple{a, \subclass, b}}{[2000,2006]}$ and
$\fuzzyg{\triple{b, \subclass, c}}{[2003,2008]}$, we will infer~$\fuzzyg{\triple{a, \subclass, c}}{[2003,2006]}$, as
$[2003,2006] = [2000,2006] \cap [2003,2008]$ (where~$\cap$ plays the role of~$\otimes$).
%
In the fuzzy context, one may chose any triangular norm (t-norm)~\cite{KlementMesiarPap:2000aa}, \eg~product, and, thus,
from~$\fuzzyg{\triple{a, \subclass, b}}{0.7}$ and~$\fuzzyg{\triple{b, \subclass, c}}{0.6}$, we will
infer~$\fuzzyg{\triple{a, \subclass, c}}{0.42}$, as $0.42 = 0.7 \cdot 0.6$ (here,~$\cdot$ plays the role of~$\otimes$).

The distributivity condition guarantees that we obtain the same annotation~$\lambda_{1} \otimes (\lambda_{2} \oplus
\lambda_{3}) = (\lambda_{1} \otimes \lambda_{2}) \oplus (\lambda_{1} \otimes \lambda_{3})$ of the triple~$\triple{a,
  \subclass, c}$ that can be inferred from triples~$\fuzzyg{\triple{a, \subclass, b}}{\lambda_{1}}$, $\fuzzyg{\triple{b,
    \subclass, c}}{\lambda_{2}}$ and~$\fuzzyg{\triple{b, \subclass, c}}{\lambda_{3}}$.
%
Finally, note that, conceptually, in order to build an annotation domain, one has to:
% 
\begin{enumerate}[noitemsep]
\item determine the set of annotation values~$L$ (typically a countable set\footnote{Note that one may use \ac{XML}
    decimals in~$[0,1]$ in place of real numbers for the fuzzy domain.}),
  identifying the top and bottom elements;

\item define suitable operations~$\otimes$ and~$\oplus$ that acts as ``conjunction'' and ``disjunction'' functions, to
  support the intended inference over schema axioms, such as
  \begin{quote}
    ``from~$\fuzzyg{\triple{a, \subclass, b}}{\lambda}$ and~$\fuzzyg{\triple{b, \subclass,
        c}}{\lambda'}$ infer~$\fuzzyg{\triple{a, \subclass, c}}{\lambda \otimes \lambda'}$''
  \end{quote}

  \nd and
  \begin{quote}
    ``from~$\fuzzyg{\tau}{\lambda}$ and~$\fuzzyg{\tau}{\lambda'}$ infer~$\fuzzyg{\tau}{\lambda \oplus \lambda'}$''
  \end{quote}
\end{enumerate}

Another desirable feature is to use annotated and non-annotated triples in parallel, possibly even in the same dataset.
In~\citet{ZimmermannLopesPolleres:2012aa}, we presented several approaches for combining annotated and non-annotated
triples, such as assuming a default annotation for any non-annotated triple or creating a new compound domain.
% 
For simplicity, and since we are considering the issue of compound domains as out of scope for this thesis, we follow
the approach of assuming a default annotation for non-annotated triples. 
%
This default annotation can be specified on a per-domain basis however, if unspecified, we assume the~$\top$ element
from the domain as the default annotation.





\subsection{Semantics}
\label{sec:anql-semantics}
%
For this section, we fix an annotation domain~$D = \tuple{L, \oplus, \otimes, \bot, \top}$.
%
Similar to \cref{sec:rdf-semantics} we rely on the \rhodf fragment of \ac{RDFS} and do not consider datatype
interpretations.
%
\begin{definition}[Annotated Map]
  An \emph{annotated map} is a function~$\theta : \AUBL \to \AUBL$ preserving \acp{URI} and literals,
  \ie~$\fc{\theta}{t} = t$, for all~$t \in \AUL$.
  % 
  Given an annotated graph~$G$, we define~$\fc{\theta}{G} = \set{ \fuzzyg{\triple{\fc{\theta}{s}, \fc{\theta}{p},
        \fc{\theta}{o}}}{\lambda'} \mid \fuzzyg{\triple{s, p, o}}{\lambda} \in G }$, where~$\lambda' \in \AL$
  and~$\lambda' \preceq \lambda$.
  %
  Similarly to the classical case, we speak of an annotated map~$\theta$ from~$G_{1}$ to~$G_{2}$, and write~$\theta :
  G_{1} \to G_{2}$, if~$\theta$ is such that~$\forall \fuzzyg{\tau}{\lambda_{2}} \in G_{2},
  \exists\fuzzyg{\tau}{\lambda_{1}} \in G_{1}$ such that~$\lambda_{2} \preceq \lambda_{1}$.\footnote{As a shorthand
    notation, from herein we will use~$G_{2} \preceq G_{1}$ to denote~$\forall \fuzzyg{\tau}{\lambda_{2}} \in G_{2},
    \exists\fuzzyg{\tau}{\lambda_{1}} \in G_{1}$ such that~$\lambda_{2} \preceq \lambda_{1}$.}
  %
\end{definition}
%
Informally, an interpretation~$\I$ will assign to a triple~$\tau$ an element of the annotation domain~$\lambda \in L$:
%
\begin{definition}[Annotated Interpretation, extends \cref{def:interpretation}]
  An \emph{annotated interpretation}~$\I$ over a vocabulary~$V$ is a tuple
  % 
  \[\I=\tuple{\Delta_{R}, \Delta_{P}, \Delta_{C}, \Delta_{L}, \intP{\cdot}, \intC{\cdot}, \int{\cdot}}\]
  % 
  where~$\Delta_{R}, \Delta_{P}, \Delta_{C}, \Delta_{L}$ are interpretation domains of~$\I$ and~$\intP{\cdot},
  \intC{\cdot}, \int{\cdot}$ are interpretation functions of~$\I$.
  %
  They have to satisfy:
  \begin{enumerate}
  \item $\Delta_{R}$ is a nonempty finite set of resources, called the domain or universe of~$\I$;
  \item $\Delta_{P}$ is a finite set of property names (not necessarily disjoint from~$\Delta_{R}$);
  \item $\Delta_{C} \subseteq \Delta_{R}$ is a distinguished subset of~$\Delta_{R}$ identifying if a resource denotes a
    class of resources;
  \item $\Delta_{L} \subseteq \Delta_{R}$, the set of literal values,~$\Delta_{L}$ contains all plain literals in~$\AL
    \cap V$;
  \item\label{item5} $\intP{\cdot}$ maps each property name~$p \in \Delta_{P}$ into a function~$\intP{p}: \Delta_{R}
    \times \Delta_{R}\to L$, \ie~assigns an annotation value to each pair of resources;
  \item\label{item6} $\intC{\cdot}$ maps each class~$c \in \Delta_{C}$ into a function~$\intC{c} : \Delta_{R} \to L$,
    \ie~assigns an annotation value representing class membership in~$c$ to every resource;
  \item $\int{\cdot}$ maps each~$t \in \AUL \cap V$ into a value~$\int{t} \in \Delta_{R} \cup \Delta_{P}$ and such
    that~$\int{\cdot}$ is the identity for plain literals and assigns an element in~$\Delta_{R}$ to each element
    in~$\AL$.
  \end{enumerate}
\end{definition}


Similar to the classical case we provide a single notion of interpretation that covers Simple, \ac{RDF}, and \ac{RDFS}
(\rhodf) entailment.  Furthermore, we extend the definition of model:
%
\begin{definition}[Model, extends \cref{def:rdf-model}]
  \label{def:annotated-model}%
  \nd An interpretation~$\I$ is a \emph{model} of an annotated ground graph~$G$, denoted~$\I \models G$, \iff~$\I$ is an
  interpretation over the vocabulary~$\rhodf \cup \fc{\universe}{G}$ that satisfies the following conditions, where~$A
  \colon \AB \maps \Delta_{R}$ and~$\I_{A}$ is the extension of~$\I$ with~$A$:
  %
  \begin{description}
  \item[Simple:] \
    \begin{enumerate}
    \item $\fuzzyg{\triple{s, p, o}}{\lambda} \in G$ implies~$\intA{p} \in\Delta_{P}$ and~$\intP{\intA{p}}(\intA{s}, \intA{o})
      \succeq \lambda$;
    \end{enumerate}
  \item[Subproperty:] \
    \begin{enumerate}
    \item $\intP{\int{ \spp}}(p, q) \otimes \intP{\int{ \spp}}(q, r) \preceq \intP{\int{ \spp}}(p, r)$;
    \item $\intP{\int{ p}}(x, y) \otimes \intP{\int{ \spp}}(p, q) \preceq \intP{\int{ q}}(x, y)$;
    \end{enumerate}
  \item[Subclass:] \
    \begin{enumerate}
    \item $\intP{\int{ \subclass}}(c, d) \otimes \intP{\int{ \subclass}}(d, e) \preceq \intP{\int{ \subclass}}(c, e)$;
    \item $\intC{\int{c}}(x) \otimes \intP{\int{ \subclass}}(c, d) \preceq \intC{\int{ d}}(x)$;
    \end{enumerate}
  \item[Typing I:] \
    \begin{enumerate}
    \item $\intC{c}(x) =  \intP{\int{\typeR}}(x,c)$;
    \item $\intP{\int{\domR}}(p, c) \otimes  \intP{p}(x,y) \preceq  \intC{c}(x)$;
    \item $\intP{\int{\range}}(p, c) \otimes  \intP{p}(x,y) \preceq  \intC{c}(y)$;
    \end{enumerate}
  \item[Typing II:] \
    \begin{enumerate}
    \item For each~$\eee \in \rhodf$,~$\int{\eee} \in \Delta_{P}$; 
    \item $\intP{\int{\spp}}(p, q)$ is defined only for~$p,q \in \Delta_{P}$;
    \item $\intC{\int{\subclass}}(c, d)$ is defined only for~$c,d \in \Delta_{C}$;
    \item $\intP{\int{\domR}}(p, c)$ is defined only for~$p \in \Delta_{P}$ and~$c \in \Delta_{C}$;
    \item  $\intP{\int{\range}}(p, c)$ is defined only for~$p \in \Delta_{P}$ and~$c \in \Delta_{C}$;
    \item $\intP{\int{\typeR}}(s, c)$ is defined only for~$c \in \Delta_{C}$.
    \end{enumerate}
  \end{description}
\end{definition}
%
\nd Intuitively, a triple~$\fuzzyg{\triple{s, p, o}}{\lambda}$ is satisfied by $\I$ if~$(s, o)$ belongs to the extension
of~$p$ to a ``wider'' extent than~$\lambda$.
%
Note that the major differences from the classical setting reside on \cref{item5,item6}.


Finally, entailment among annotated ground graphs~$G$ and~$H$ is as usual.
Now,~$G \models H$, where~$G$ and~$H$ may contain blank nodes, \iff for any grounding~$G'$ of~$G$ there is a grounding~$H'$ of~$H$ such that~$G' \models H'$.

Please note that we always have that~$G \models \fuzzyg{\tau}{\bot}$, however, triples of the form~$\fuzzyg{\tau}{\bot}$
are uninteresting and, thus, in the following we do not consider them as part of the language.


\subsection{Examples of Annotation Domains}
\label{sec:domain-examples}


Next we specify some domains in Annotated RDFS, namely the classical domain, fuzzy~\cite{Straccia:2009aa},
temporal~\cite{GutierrezHurtadoVaisman:2007aa}, and provenance.

\subsubsection{The Classical Domain}
\label{sec:crisp}

The classical \ac{RDF} setting corresponds to the case in which the annotation values are~$L = \set{0,1}$.
%
Thus, the classical domain can be specified as~$D_{01} = \tuple{\set{0, 1}, \max, \min, 0, 1}$.  In this case, Annotated
RDFS turns out to be the same as standard \ac{RDFS}.


\subsubsection{The Temporal Domain}
\label{sec:temporal}


For our representation of the temporal domain we aim at using non-discrete time as it is necessary to model temporal
intervals with any precision.
% 
However, for presentation purposes we will show the dates as years only.
%
To start with, \emph{time points} are elements of the value space~$\mathbb{Q} \cup\ \set{-\infty, +\infty}$ and a
\emph{temporal interval} is a non-empty interval~$[\alpha_1, \alpha_2]$, where $\alpha_{i}$ are time points.  An empty
interval is denoted as~$\emptyset$.
%
We define the partial order on intervals as~$I_{1} \leq I_{2} \mbox{ \iff\ } I_{1} \subseteq I_{2}$.  The intuition here
is that if a triple is true at time points in~$I_{2}$ and~$I_{1} \leq I_{2}$ then, in particular, it is true at any time
point in~$I_{1} \neq \emptyset$.

Now, apparently the set of intervals would be a candidate for~$L$, however, in order to represent the upper bound
interval of~$\fuzzyg{\tau}{[2001,2005]}$ and~$\fuzzyg{\tau}{[2008,2009]}$ we rather need the union of intervals,
denoted~$\set{[2001,2005], [2008,2009]}$, meaning that a triple is true both in the former as well as in the latter
interval.
%
Now, we define~$L$ as:
%
\[
L = \set{t \mid t \mbox{ is a finite set of disjoint temporal intervals} } \cup \set{\bot,\top} \enspace ,
\]
%
where~$\bot = \{\emptyset\}, \top = \{[-\infty,+\infty]\}$.
%
Therefore, a \emph{temporal term} is an element~$t \in L$, \ie~a set of pairwise disjoint time intervals. 
% 
We allow the following variations:
\begin{enumerate}[noitemsep,label=(\roman*)]
\item $[\alpha]$ as a shorthand for~$[\alpha, \alpha]$;
\item $\fuzzyg{\tau}{\alpha}$ as a shorthand of $\fuzzyg{\tau}{\{[\alpha]\}}$; and
\item $\fuzzyg{\tau}{[\alpha, \alpha']}$ as a shorthand of~$\fuzzyg{\tau}{\{[\alpha, \alpha']\}}$.
\end{enumerate}
%
Furthermore, on~$L$ we define the following partial order:
\[
t_1 \preceq t_2 \mbox{ \iff }~ \forall I_{1} \in t_1 \exists I_2 \in t_2, \mbox{ such that } I_1 \leq I_2 \enspace .
\]
%
\noindent 
%
\begin{comment}
  \todo{check pre-order vs partial order} 
  % 
  Please note that~$\preceq$ is the Hoare order on power sets~\cite{AbramskyJung:1994aa}, which is a pre-order.
  % 
  For the anti-symmetry property, assume that~$t_1 \preceq t_2$ and~$t_2 \preceq t_1$: so for an~$I_{1} \in t_{1}$,
  there is $I_{2} \in t_{2}$ for which there is~$I_{3} \in t_{1}$ such that~$I_{1} \subseteq I_{2} \subseteq I_{3}$.
  % 
  But,~$t_{1}$ is maximal and, thus,~$I_{1} = I_{3} = I_{2}$. So,~$t_{1} = t_{2}$ and, thus,~$\preceq$ is a partial
  order.
\end{comment}

Similarly as for time intervals, the intuition for~$\preceq$ is that if a triple is true during the time points in all
the intervals in~$t_{2}$ and~$t_{1} \preceq t_{2}$, then, in particular, the triple is true at any time point in
intervals of~$t_{1}$. 
%
Essentially, if~$t_{1} \preceq t_{2}$ then a temporal triple~$\fuzzyg{\tau_{2}}{t_{2}}$ is true to a larger ``temporal
extent'' than the temporal triple~$\fuzzyg{\tau_{1}}{t_{1}}$. 

The partial order~$\preceq$ induces the following join ($\oplus$) operation on~$L$. Intuitively, if a triple is true
at~$t_{1}$ and also true at~$t_{2}$ then it will be true also for time points specified by~$t_{1} \oplus t_{2}$ (a kind
of union of time points).
%
As an example, if $\fuzzyg{\tau}{\{[2002,2005], [2008,2010]\}}$ and~$\fuzzyg{\tau}{\{[2004,2006], [2009,2012]\}}$ are
true then we expect that this is the same as saying that~$\fuzzyg{\tau}{\{[2002,2006], [2008,2012]\}}$ is true. 
%
The join operator will be defined in such a way that~$\{[2002,2005], [2008,2010]\} \oplus \{[2004,2006], [2009,2012]\} =
\{[2002,2006], [2008,2012]\}$.
%
Operationally, this means that~$t_{1} \oplus t_{2}$ will be obtained as follows: 
\begin{enumerate*}[label=(\roman*)]
\item take the union of the sets of intervals~$t = t_{1} \cup t_{2}$; and
\item join overlapping intervals in~$t$ until no more overlapping intervals can be obtained.
\end{enumerate*}
%
Formally,
%
\[
t_1 \oplus t_2 = \mathrm{infimum }\set{t \mid t \succeq t_{i}, i=1,2} \enspace .
\]
\nd It remains to define the~$\otimes$ over sets of intervals.
%
Intuitively, we would like to support inferences such that
%
from the triples~$\fuzzyg{\triple{a, \subclass, b}}{\{[2002,2005], [2008,2010]\}}$ and~$\fuzzyg{\triple{b, \subclass,
    c}}{\{[2004,2006], [2009,2012]\}}$, we can infer $\fuzzyg{\triple{a, \subclass, c}}{\{[2004,2005], [2009,2010]\}}$,
%
where~$\{[2002,2005], [2008,2010]\} \otimes \{[2004,2006]$, $[2009,2012]\} = \{[2004,2005], [2009,2010]\}$.
%
We get it by means of
% 
\[
t_1 \otimes t_2 = \mathrm{supremum }\set{t \mid t \preceq t_{i}, i=1,2 }  \enspace .
\]


\begin{example}[Temporal domain~$\otimes$]
  \label{exU1}
  %
  Using the following triples regarding another member of the Nightwish band
  %
  \[
  \begin{array}{l}
    \fuzzyg{\triple{\qname{}{NightwishMember}, \subclass, \qname{mo}{MusicArtist}}}{[1992,2012]}\\
    \fuzzyg{\triple{\qname{dbpedia}{Troy\mathunderscore{}Donockley}, \typeR, \qname{}{NightwishMember}}}{[1996, 1999]} \enspace ,
  \end{array} 
  \]
  % 
  we can infer that \vspace{-\abovedisplayskip}
  \[
    \fuzzyg{\triple{\qname{dbpedia}{Troy\mathunderscore{}Donockley}, \typeR, \qname{mo}{MusicArtist}}}{[1996, 1999]} \enspace ,
  \]
  where
  \(
  \{[1992,2012]\} \otimes \{[1996, 1999]\} = \{[1996, 1999]\}
  \).
  %
\end{example}




\subsubsection{The Fuzzy Domain}
To model fuzzy \ac{RDFS}~\cite{Straccia:2009aa} we may define the annotation domain as~$D_{[0,1]} = \tuple{[0,1], \max,
  \otimes, 0, 1}$ where~$\otimes$ is any continuous t-norm on~$[0,1]$.
%
\begin{example}[Fuzzy domain~$\otimes$]
  \label{exUSf}
  Adapting our running example to the fuzzy domain we can state the following:
  %
  Nightwish collaborators are partial members of the band (50\%), and since Troy is a Nightwish collaborator:
  %
  \[
  \begin{array}{l}
    \fuzzyg{\triple{\qname{}{NightwishCollaborator}, \subclass, \qname{}{NightwishMember}}}{0.5} \\
    \fuzzyg{\triple{\qname{dbpedia}{Troy\mathunderscore{}Donockley}, \typeR, \qname{}{NightwishCollaborator}}}{0.7}
  \end{array}
  \]
  \nd Then, \eg~under the product t-norm~$\otimes$, we can infer the following triple:
  \[
  \begin{array}{l}
    \fuzzyg{\triple{\qname{dbpedia}{Troy\mathunderscore{}Donockley}, \typeR, \qname{}{NightwishMember}}}{0.35}
  \end{array}
  \]
  %
\end{example}



\subsubsection{The Provenance domain}
\label{sec:provenance-domain}

Typically, provenance is identified by a URI, usually the URI of the document in which the triples are defined or
possibly a URI identifying a named graph. However, provenance of inferred triples is an issue that has been little
tackled in the literature~\cite{DelbruPolleresTummarello:2008aa,FlourisFundulakiPediaditis:2009aa}.
%
The intuition behind our approach is similar to the one of~\citet{DelbruPolleresTummarello:2008aa}
and~\citet{FlourisFundulakiPediaditis:2009aa} where provenance of an inferred triple is defined as the aggregation of
provenances of documents that allow to infer that triple. 


We start from a countably infinite set of \emph{atomic provenances} \AP, which in practice can be represented by
\acp{URI}. We consider the propositional formulae made from symbols in \AP (atomic propositions), logical \textsl{or}
($\lor$) and logical \textsl{and} ($\land$), for which we have the standard entailment~$\models$.
%
A \emph{provenance value} is an equivalent class for the logical equivalence relation, \ie~the set of annotation values
is the quotient set of~$\AP$ by the logical equivalence. The order relation is~$\models$,~$\otimes$ and~$\oplus$ are
$\land$ and~$\lor$, respectively. We set~$\top$ to \textsl{true} and~$\bot$ to \textsl{false}.
%

\begin{example}[Provenance domain~$\otimes$]
  % 
  Consider the following triples (numbered for easier reference below):
  %
  {\allowdisplaybreaks
  \begin{align}
    &\fuzzyg{\triple{\qname{dbpedia}{Marco\mathunderscore{}Hietala}, \qname{mo}{member\mathunderscore{}of}, \qname{}{NightwishMember}}}{\texttt{\footnotesize dbpedia}}\label{prov1}\\
    &\fuzzyg{\triple{\qname{dbpedia}{Marco\mathunderscore{}Hietala}, \typeR, \qname{foaf}{Person}}}{\texttt{\footnotesize dbpedia}}\label{prov2}\\
    &\fuzzyg{\triple{\qname{dbpedia}{Nightwish},\typeR, \qname{mo}{MusicGroup}}}{\texttt{\footnotesize dbpedia}}\label{prov3}\\
    &\fuzzyg{\triple{\qname{foaf}{Person}, \subclass, \qname{foaf}{Agent}}}{\texttt{\footnotesize foaf}} \label{prov4}\\
    &\fuzzyg{\triple{\qname{mo}{member\mathunderscore{}of}, \domR, \qname{foaf}{Person}}}{\texttt{\footnotesize mo}} \label{prov5}\\
    &\fuzzyg{\triple{\qname{mo}{member\mathunderscore{}of}, \range, \qname{mo}{MusicGroup}}}{\texttt{\footnotesize mo}}\label{prov6}
  \end{align}%
  }%
  % 
  We can deduce that \qname{dbpedia}{Marco\mathunderscore{}Hietala} is a \qname{foaf}{Agent} in two different ways:
  %
  \begin{enumerate*}[noitemsep,label=(\alph*)]
  \item using statements~\eqref{prov1},~\eqref{prov4}, and~\eqref{prov5}; or
  \item using the statements~\eqref{prov2} and~\eqref{prov4}.
  \end{enumerate*}
  %
  So, it is possible to infer the following annotated triple:
  % 
  \[
  \fuzzyg{\triple{\qname{dbpedia}{Marco\mathunderscore{}Hietala}, \typeR, \qname{foaf}{Agent}}}{(\texttt{\footnotesize
      dbpedia}\land\texttt{\footnotesize foaf}\land\texttt{\footnotesize mo})\lor(\texttt{\footnotesize
      foaf}\land\texttt{\footnotesize mo})}
  \]
  % 
  However, since~$(\texttt{\footnotesize dbpedia}\land\texttt{\footnotesize foaf}\land\texttt{\footnotesize
    mo})\lor(\texttt{\footnotesize foaf}\land\texttt{\footnotesize mo})$ is logically equivalent
  to~$\texttt{\footnotesize foaf}\land\texttt{\footnotesize mo}$, the aggregated inference can be collapsed into:
  \[
  \fuzzyg{\triple{\qname{dbpedia}{Marco\mathunderscore{}Hietala}, \typeR, \uri{Agent}}}{\texttt{\footnotesize
      foaf}\land\texttt{\footnotesize mo}}
  \]
  % 
\end{example}


%%% Local Variables:
%%% mode: latex
%%% TeX-master: "../thesis"
%%% End:



\subsection{Deductive system}
\label{sec:anql-deductive-system}
An important feature of our framework is that we are able to provide a deductive system in the style of the one for
classical \ac{RDFS}.  Moreover, \emph{the schemata of the rules are the same for any annotation domain} (only support for the
domain dependent~$\otimes$ and~$\oplus$ operations has to be provided).
%
The rules of our deductive system, as in \cref{sec:rdf-schema}, are arranged in groups that capture the semantic
conditions of models, where~$A,B,C,X$, and~$Y$ are meta-variables representing elements in~$\AUBL$ and~$D,E$ represent
elements in~$\AUL$.
%
The rule set contains two rules,~$(1a)$ and $(1b)$, that are the same as for the crisp case, while rules~$(2a)$
to~$(5b)$ are the annotated rules homologous to the crisp ones. 
%
\ifnormalisedardf%
\else%
Finally, rule~$(6)$ is specific to the annotated case.
\fi%
%
  \begin{description}
  \item[1. Simple:]~ \\[0.5em]
    \begin{tabular}{ll}
      $(a) \quad \frac{G}{G'}$ for an annotated map $\theta:G' \to G$ & 
      \ifnormalisedardf%
      $(b) \quad \frac{G}{G'}$ for $G' \preceq G$
      \else%
      $(b) \quad \frac{G}{G'}$ if $\forall \fuzzyg{\tau}{\lambda_{1}}
      \in G', \exists\fuzzyg{\tau}{\lambda_{2}} \in G$ such that~$\lambda_{1} \preceq \lambda_{2}$
      \fi
    \end{tabular}
  \item[2. Subproperty:]~ \\[0.5em]
    \begin{tabular}{ll}
      $(a) \quad \frac{\fuzzyg{\triple{A,  \spp, B}}{\lambda_{1}},  \fuzzyg{\triple{B,  \spp, C}}{\lambda_{2}}}{\fuzzyg{\triple{A,  \spp, C}}{\lambda_{1}\otimes \lambda_{2}}}$ &
      $(b) \quad \frac{\fuzzyg{\triple{D, \spp, E}}{\lambda_{1}},  \fuzzyg{\triple{X, D, Y}}{\lambda_{2}}}{\fuzzyg{\triple{X, E, Y}}{\lambda_{1} \otimes \lambda_{2}}}$
    \end{tabular}
  \item[3. Subclass:]~ \\[0.5em]
    \begin{tabular}{ll}
      $(a) \quad \frac{\fuzzyg{\triple{A, \subclass, B}}{\lambda_{1}},  \fuzzyg{\triple{B, \subclass, C}}{\lambda_{2}}}{\fuzzyg{\triple{A, \subclass, C}}{\lambda_{1} \otimes \lambda_{2}}}$ &
      $(b) \quad \frac{\fuzzyg{\triple{A, \subclass, B}}{\lambda_{1}},  \fuzzyg{\triple{X, \typeR, A}}{\lambda_{2}}}{\fuzzyg{\triple{X, \typeR, B}}{\lambda_{1} \otimes \lambda_{2}}}$
    \end{tabular}
  \item[4. Typing:]~ \\[0.5em]
    \begin{tabular}{ll}
      $(a) \quad \frac{\fuzzyg{\triple{D, \domR, B}}{\lambda_{1}},  \fuzzyg{\triple{X, D, Y}}{\lambda_{2}}}{\fuzzyg{\triple{X, \typeR, B}}{\lambda_{1} \otimes \lambda_{2}}}$ &
  $(b) \quad \frac{\fuzzyg{\triple{D, \range, B}}{\lambda_{1}},  \fuzzyg{\triple{X, D, Y}}{\lambda_{2}}}{\fuzzyg{\triple{Y, \typeR, B}}{\lambda_{1} \otimes \lambda_{2}}}$
    \end{tabular}
  \item[5. Implicit Typing:]~ \\[0.5em]
    \begin{tabular}{l}
      $(a) \quad \frac{\fuzzyg{\triple{A, \domR, B}}{\lambda_{1}},  \fuzzyg{\triple{D,  \spp, A}}{\lambda_{2}},
        \fuzzyg{\triple{X, D, Y}}{\lambda_{3}}}{\fuzzyg{\triple{X, \typeR, B}}{\lambda_{1} \otimes \lambda_{2} \otimes
          \lambda_{3}}}$ \\[1.5em]
      $(b) \quad \frac{\fuzzyg{\triple{A, \range, B}}{\lambda_{1}},  \fuzzyg{\triple{D,  \spp, A}}{\lambda_{2}}, \fuzzyg{\triple{X, D, Y}}{\lambda_{3}}}{\fuzzyg{\triple{Y, \typeR, B}}{\lambda_{1} \otimes \lambda_{2} \otimes \lambda_{3}}}$
    \end{tabular}
\ifnormalisedardf%
\else%
  \item[6. Generalisation:]~ \\[0.5em]
    \begin{tabular}{l}
      $\frac{\fuzzyg{\triple{X, A, Y}}{\lambda_{1}},  \fuzzyg{\triple{X, A, Y}}{\lambda_{2}}}{\fuzzyg{\triple{X, A, Y}}{\lambda_{1} \oplus \lambda_{2}}}$
    \end{tabular}
\fi%
\end{description}
%
\ifnormalisedardf%
Please note we assume that a rule is not applied if the consequence is of the form~$\fuzzyg{\tau}{\bot}$ (see
\cref{sec:anql-semantics}).
%  
\else%
Please note that rule~$(6)$ is destructive \ie~this rule removes the premises as the conclusion is inferred.
%
We also assume that a rule is not applied if the consequence is of the form~$\fuzzyg{\tau}{\bot}$ 
%
(see \cref{sec:anql-semantics}).
\fi%


\begin{comment}
%
Alternatively, rules~2--5 can be represented concisely using the following inference rule:
%
\begin{center}
  \begin{tabular}{cl}
    $(AG)$ & $\frac{
      \fuzzyg{\tau_{1}}{\lambda_{1}},\ \ldots,\ \fuzzyg{\tau_{n}}{\lambda_{n}, \{\tau_{1}, \ldots \tau_{n}\} \vdash_{\mathsf{RDFS}} \tau}
    }
    {
      \fuzzyg{\tau}{\bigotimes_{i} \lambda_{i}}
    }$
  \end{tabular}
\end{center}
%
\nd Essentially, this rule says that if a classical \ac{RDFS} triple~$\tau$ can be inferred by applying a classical
\ac{RDFS} inference rule to triples~$\tau_{1}, \ldots, \tau_{n}$ (denoted~$\{\tau_{1}, \ldots, \tau_{n}\}
\vdash_{\mathsf{RDFS}} \tau$), then the annotation term of~$\tau$ will be~$\bigotimes_{i} \lambda_{i}$,
where~$\lambda_{i}$ is the annotation of triple~$\tau_{i}$.  It follows immediately that, using rule~$(AG)$, in
addition to~\ifnormalisedardf rule~$1$~\else rules~$1$ and~$6$~\fi from the deductive system above, it is easy to
extend these rules to cover complete \ac{RDFS}.
%
\end{comment}

\ifnormalisedardf%
Like for the classical case, the \emph{closure} is defined as~$\fc{cl}{G} = \set{ \fuzzyg{\tau}{\lambda} \mid G
  \vdash^{*} \fuzzyg{\tau}{\lambda} }$, where~$\vdash^{*}$ is as~$\vdash$ without rule~$(1a)$.  Note again that the size
of the closure of~$G$ is polynomial in~$|G|$ and can be computed in polynomial time, provided that the computational
complexity of operations~$\otimes$ and~$\oplus$ are polynomially bounded (from a computational complexity point of view,
it is as for the classical case, plus the cost of the operations~$\otimes$ and~$\oplus$ in~$L$).
%
\fi


\ifnormalisedardf%
Furthermore note that~$\fc{cl}{G}$ is not guaranteed to be a normalised annotated \ac{RDF} graph.  In order to ensure a
normalised graph we can apply the following rule, denoted \emph{generalisation rule}:
%
$$\frac{\fuzzyg{\triple{X, A, Y}}{\lambda_{1}},  \fuzzyg{\triple{X, A, Y}}{\lambda_{2}}}{\fuzzyg{\triple{X, A,
      Y}}{\lambda_{1} \oplus \lambda_{2}}} \enspace ,$$
%
where each application of this rule removes the premises from the graph. 
%
\fi%
%
Let us show an example of the application of the generalisation rule\ifnormalisedardf\else~$(6)$\fi.
%
\begin{example}[Generalisation operation]
  %
  Consider the following triples along with our running example from \cref{fig:dataset-example}:
  %
    \[
    \begin{array}{l}
      %
      \fuzzyg{\triple{\qname{foaf}{name}, \domR, \qname{foaf}{Person}}}{[-\infty,+\infty]}\footnote{We assume this
        triple is valid to reduce the number of triples shown, the domain of \qname{foaf}{name} is in fact \qname{owl}{Thing}.} \\
      \fuzzyg{\triple{\qname{foaf}{Person}, \subclass, \qname{foaf}{Agent}}}{[-\infty,+\infty]} \\
      \fuzzyg{\triple{\qname{mo}{MusicArtist}, \subclass, \qname{foaf}{Agent}}}{[-\infty,+\infty]} \enspace ,
      % 
    \end{array}
    \]
  % 
  we infer the following triples:
  % 
    \[
    \begin{array}{l}
      \fuzzyg{\triple{\qname{dbpedia}{Marco\mathunderscore{}Hietala}, \typeR, \qname{foaf}{Agent}}}{[1984,2012]} \\
      \fuzzyg{\triple{\qname{dbpedia}{Marco\mathunderscore{}Hietala}, \typeR, \qname{foaf}{Agent}}}{[1966,2012]} \enspace .
    \end{array}
    \]
  % 
  The application of the ``Generalisation'' rule will collapse the triples:
  % 
    \[
    \begin{array}{l}
      \fuzzyg{\triple{\qname{dbpedia}{Marco\mathunderscore{}Hietala}, \typeR, \qname{foaf}{Agent}}}{[1966,2012]}  \enspace ,
    \end{array}
    \]
  % 
  since~$[1984,2012] \oplus [1966,2012] = [1966,2012]$.
  %
\end{example}

\ifnormalisedardf%
\else%
%
Finally, like for the classical case, the \emph{closure} is defined as~$\fc{cl}{G} = \{\fuzzyg{\tau}{\lambda} \mid G
\vdash^{*} \fuzzyg{\tau}{\lambda} \}$, where~$\vdash^{*}$ is as~$\vdash$ without rule~$(1a)$.  Note again that the size
of the closure of~$G$ is polynomial in~$|G|$ and can be computed in polynomial time, provided that the computational
complexity of operations~$\otimes$ and~$\oplus$ are polynomially bounded (from a computational complexity point of view,
it is as for the classical case, plus the cost of the operations~$\otimes$ and~$\oplus$ in~$L$).
%
\fi


\begin{proposition}[Soundness and completeness]
  For two annotated graphs~$G$ and~$G'$, the proof system~$\vdash$ is sound and complete for~$\models$, that is~$G
  \vdash G'$ \iff~$G \models G'$.
\end{proposition}


\begin{proof}[Extends~\citep{MunozPerezGutierrez:2009aa}]
  ($\Rightarrow$) 
  %
  Let~$\I=\tuple{\Delta_{R}, \Delta_{P}, \Delta_{C}, \Delta_{L}, \intP{\cdot}, \intC{\cdot},\int{\cdot}}$ be an
  interpretation such that~$\I \models G$ \ie~$\I$ satisfies all the conditions of \cref{def:annotated-model}.
  %
  Furthermore let~$\fuzzyg{\tau}{\lambda}$ be the result from an instantiation of a \ifnormalisedardf rule~$2-5$\else
  rule~$2-6$\fi, such that~$\frac{R}{\fuzzyg{\tau}{\lambda}}$, where~$R \subseteq G$ and
  let~$G'=G\cup\set{\fuzzyg{\tau}{\lambda}}$,
  then~$\I \models G'$.
  % 
  \begin{description}[nosep]
  \item[1. Simple:]~
    \begin{enumerate}[label=(\alph*),nosep]
    \item We show that if~$G' \to G$ then~$G \models G'$.  Let~$\theta$ be an annotated map such that~$\fc{\theta}{G'}
      \preceq G$. Consider the function~$A' \colon \AB \to \Delta_R$ such that~$$\fc{A'}{x} =
      \left\{\begin{array}{l}\fc{A}{\fc{\theta}{x}}~\mathrm{if}~\fc{\theta}{x} \in \AB\\
          \int{\fc{\theta}{x}}~\mathrm{if}~\fc{\theta}{x} \not\in \AB
        \end{array}\right. \enspace .$$
      %
      Note that:
      %
      \begin{enumerate*}
      \item if~$x\in\AB$ and~$\fc{\theta}{x} \in \AB$ we get that~$\intA{\fc{\theta}{x}} = \fc{A}{\fc{\theta}{x}} =
        \fc{A'}{x} = \intAP{x}$;
      \item if~$x\in\AB$ and~$\fc{\theta}{x} \not\in \AB$ we get that~$\intA{\fc{\theta}{x}} = \int{\fc{\theta}{x}} =
        \fc{A'}{x} = \intAP{x}$;
      \item if~$x\not\in\AB$ we get that~$\fc{\theta}{x} = x$ and~$\intA{\fc{\theta}{x}} = \int{x} = \fc{A'}{x} =
        \intAP{x}$.
      \end{enumerate*}
      %
      Thus we have that~$\intA{\fc{\theta}{x}} = \intAP{x}$ for all~$x\in\AUB$.
      %
      Let~$\fuzzyg{\triple{s,p,o}}{\lambda} \in G'$,
      then~$\fuzzyg{\triple{\fc{\theta}{s},\fc{\theta}{p},\fc{\theta}{o}}}{\lambda} =
      \fuzzyg{\triple{\fc{\theta}{s},p,\fc{\theta}{o}}}{\lambda} \in G$.
      %
      Since~$\I \models G$ we have that~$\int{p} \in \Delta_P$ and~$\intP{\intA{\fc{\theta}{s}},\intA{\fc{\theta}{s}}}
      \succeq \lambda$ and~$\intP{\intAP{\fc{\theta}{s}},\intAP{\fc{\theta}{s}}} \succeq \lambda$.  
      %
      Thus~$\I$ satisfies condition~$(1)$ from \cref{def:annotated-model} for~$G'$ (with function~$A'$) and satisfies
      all other conditions of \cref{def:annotated-model}, so~$\I \models G'$.

      \ifnormalisedardf%
    \item if~$G' \preceq G$ then~$G' \to G$ and thus~$G \models G'$.%
      \else%
    \item if $\forall \fuzzyg{\tau}{\lambda_{1}} \in G', \exists\fuzzyg{\tau}{\lambda_{2}} \in G$ such
      that~$\lambda_{1} \preceq \lambda_{2}$ then~$G' \to G$ and thus~$G \models G'$.%
      \fi
    \end{enumerate}
  \item[2. Subproperty:]~ 
    \begin{enumerate}[label=(\alph*),nosep]
    \item Let~$\I \models \fuzzyg{\triple{A, \spp, B}}{\lambda_{1}}$ and~$\I \models \fuzzyg{\triple{B, \spp,
          C}}{\lambda_{2}}$.  
      %
      It follows that~$\fc{\intP{\int{\spp}}}{\int{A}, \int{B}} \succeq \lambda_{1}$
      and~$\fc{\intP{\int{\spp}}}{\int{B}, \int{C}} \succeq \lambda_{2}$.
      % 
      According to condition ``Subproperty 1.'' from \cref{def:annotated-model}, we have
      that~$\fc{\intP{\int{\spp}}}{\int{A}, \int{B}} \otimes \fc{\intP{\int{ \spp}}}{\int{B}, \int{C}} \preceq
      \fc{\intP{\int{ \spp}}}{\int{A}, \int{C}}$ and thus~$\lambda_{1} \otimes \lambda_{2} \preceq
      \fc{\intP{\int{\spp}}}{\int{A}, \int{C}}$.
      %
      Therefore~$\I \models \fuzzyg{\triple{A, \spp, C}}{\lambda_{1}\otimes\lambda_{2}}$.

    \item Let~$\I \models \fuzzyg{\triple{D, \spp, E}}{\lambda_{1}}$ and~$\I \models \fuzzyg{\triple{X, D,
          Y}}{\lambda_{2}}$.
      %
      It follows that~$\fc{\intP{\int{\spp}}}{\int{D}, \int{E}} \succeq \lambda_{1}$
      and~$\fc{\intP{\int{D}}}{\int{X}, \int{Y}} \succeq \lambda_{2}$.
      % 
      According to condition ``Subproperty 2.'' from \cref{def:annotated-model}, we have
      that~$\fc{\intP{\int{\spp}}}{\int{D}, \int{E}} \otimes \fc{\intP{\int{D}}}{\int{X}, \int{Y}} \preceq
      \fc{\intP{\int{E}}}{\int{X}, \int{Y}}$ and thus~$\lambda_{1} \otimes \lambda_{2} \preceq
      \fc{\intP{\int{E}}}{\int{X}, \int{Y}}$.
      %
      Therefore~$\I \models \fuzzyg{\triple{X, E, Y}}{\lambda_{1}\otimes\lambda_{2}}$.
    \end{enumerate}
  \item[3. Subclass:]~ 
    \begin{enumerate}[label=(\alph*),nosep]
    % \item $\frac{\fuzzyg{\triple{A, \subclass, B}}{\lambda_{1}}, \fuzzyg{\triple{B, \subclass,
    %         C}}{\lambda_{2}}}{\fuzzyg{\triple{A, \subclass, C}}{\lambda_{1} \otimes \lambda_{2}}}$
    \item Let~$\I \models \fuzzyg{\triple{A, \subclass, B}}{\lambda_{1}}$ and~$\I \models \fuzzyg{\triple{B, \subclass,
          C}}{\lambda_{2}}$.  
      %
      It follows that~$\fc{\intP{\int{\subclass}}}{\int{A}, \int{B}} \succeq \lambda_{1}$
      and~$\fc{\intP{\int{\subclass}}}{\int{B}, \int{C}} \succeq \lambda_{2}$.
      % 
      According to condition ``Subclass 1.'' from \cref{def:annotated-model}, we have
      that~$\fc{\intP{\int{\subclass}}}{\int{A}, \int{B}} \otimes \fc{\intP{\int{ \subclass}}}{\int{B}, \int{C}} \preceq
      \fc{\intP{\int{\subclass}}}{\int{A}, \int{C}}$ and thus~$\lambda_{1} \otimes \lambda_{2} \preceq
      \fc{\intP{\int{\subclass}}}{\int{A}, \int{C}}$.
      %
      Therefore~$\I \models \fuzzyg{\triple{A, \subclass, C}}{\lambda_{1}\otimes\lambda_{2}}$.

    % \item $\frac{\fuzzyg{\triple{A, \subclass, B}}{\lambda_{1}}, \fuzzyg{\triple{X, \typeR,
    %         A}}{\lambda_{2}}}{\fuzzyg{\triple{X, \typeR, B}}{\lambda_{1} \otimes \lambda_{2}}}$
    \item Let~$\I \models \fuzzyg{\triple{A, \subclass, B}}{\lambda_{1}}$ and~$\I \models \fuzzyg{\triple{X, \typeR,
          A}}{\lambda_{2}}$.
      %
      It follows that~$\fc{\intP{\int{\subclass}}}{\int{A}, \int{B}} \succeq \lambda_{1}$
      and~$\fc{\intP{\int{\typeR}}}{\int{X}, \int{A}} \succeq \lambda_{2}$.
      % 
      From condition ``Typing I, 1.'' (\cref{def:annotated-model}), we have that~$\fc{\intC{\int{A}}}{\int{X}} \succeq
      \lambda_{2}$.
      % 
      According to condition ``Subclass 2.'' from \cref{def:annotated-model}, we have
      that~$\fc{\intP{\int{\subclass}}}{\int{A}, \int{B}} \otimes \fc{\intC{\int{A}}}{\int{X}} \preceq
      \fc{\intC{\int{B}}}{\int{X}}$ and thus~$\lambda_{1} \otimes \lambda_{2} \preceq \fc{\intC{\int{B}}}{\int{X}}$ and,
      again from condition ``Typing I, 1.'', $\lambda_{1} \otimes \lambda_{2} \preceq \fc{\intP{\int{\typeR}}}{\int{X}, \int{B}}$.
      %
      Therefore~$\I \models \fuzzyg{\triple{X, \typeR, B}}{\lambda_{1}\otimes\lambda_{2}}$.
      


    \end{enumerate}
  \item[4. Typing:]~ 
    \begin{enumerate}[label=(\alph*),nosep]
    \item Let~$\I \models \fuzzyg{\triple{D, \domR, B}}{\lambda_{1}}$ and~$\I \models \fuzzyg{\triple{X, D,
          Y}}{\lambda_{2}}$.
      %
      It follows that~$\fc{\intP{\int{\domR}}}{\int{D}, \int{B}} \succeq \lambda_{1}$
      and~$\fc{\intP{\int{D}}}{\int{X}, \int{Y}} \succeq \lambda_{2}$.
      % 
      From condition ``Typing I, 2.'' (\cref{def:annotated-model}), we have that~$\fc{\intP{\int{\domR}}}{\int{D},
        \int{B}} \otimes \fc{\intP{\int{D}}}{\int{X}, \int{Y}} \preceq \fc{\intC{\int{B}}}{\int{X}}$ and
      thus~$\lambda_{1} \otimes \lambda_{2} \preceq \fc{\intC{\int{B}}}{\int{X}}$.
      % 
      From condition ``Typing I, 1.'' (\cref{def:annotated-model}), we have that~$\lambda_{1} \otimes \lambda_{2}
      \preceq \fc{\intP{\int{\typeR}}}{\int{X},\int{B}}$.
      % 
      Therefore~$\I \models \fuzzyg{\triple{X, \typeR, B}}{\lambda_{1} \otimes \lambda_{2}}$.

    \item Let~$\I \models \fuzzyg{\triple{D, \range, B}}{\lambda_{1}}$ and~$\I \models \fuzzyg{\triple{X, D,
          Y}}{\lambda_{2}}$.
      %
      It follows that~$\fc{\intP{\int{\range}}}{\int{D}, \int{B}} \succeq \lambda_{1}$ and~$\fc{\intP{\int{D}}}{\int{X},
        \int{Y}} \succeq \lambda_{2}$.
      % 
      From condition ``Typing I, 3.'' (\cref{def:annotated-model}), we have that~$\fc{\intP{\int{\range}}}{\int{D},
        \int{B}} \otimes \fc{\intP{\int{D}}}{\int{X}, \int{Y}} \preceq \fc{\intC{\int{B}}}{\int{Y}}$ and
      thus~$\lambda_{1} \otimes \lambda_{2} \preceq \fc{\intC{\int{B}}}{\int{Y}}$.
      % 
      From condition ``Typing I, 1.'' (\cref{def:annotated-model}), we have that~$\lambda_{1} \otimes \lambda_{2}
      \preceq \fc{\intP{\int{\typeR}}}{\int{Y},\int{B}}$.
      % 
      Therefore~$\I \models \fuzzyg{\triple{Y, \typeR, B}}{\lambda_{1} \otimes \lambda_{2}}$.

    \end{enumerate}
  \item[5. Implicit Typing:]~ 
    \begin{enumerate}[label=(\alph*),nosep]
      %
    \item Let~$\I \models \fuzzyg{\triple{A, \domR, B}}{\lambda_{1}}$, $\I \models \fuzzyg{\triple{D, \spp,
          A}}{\lambda_{2}}$, and~$\I \models \fuzzyg{\triple{X, D, Y}}{\lambda_{3}}$.
      %
      It follows that~$\fc{\intP{\int{\domR}}}{\int{A}, \int{B}} \succeq \lambda_{1}$, $\fc{\intP{\int{\spp}}}{\int{D},
        \int{A}} \succeq \lambda_{2}$, and~$\fc{\intP{\int{D}}}{\int{X}, \int{Y}} \succeq \lambda_{3}$.
      % 
      According to condition ``Subproperty 2.'' from \cref{def:annotated-model}, we have
      that~$\fc{\intP{\int{\spp}}}{\int{D}, \int{A}} \otimes \fc{\intP{\int{D}}}{\int{X}, \int{Y}} \preceq
      \fc{\intP{\int{A}}}{\int{X}, \int{Y}}$ and thus~$\lambda_{1} \otimes \lambda_{2} \preceq
      \fc{\intP{\int{A}}}{\int{X}, \int{Y}}$.
      % 
      From condition ``Typing I, 2.'' (\cref{def:annotated-model}), we have that~$\fc{\intP{\int{\domR}}}{\int{A},
        \int{B}} \otimes \fc{\intP{\int{A}}}{\int{X}, \int{Y}} \preceq \fc{\intC{\int{B}}}{\int{X}}$ and
      thus~$\lambda_{1} \otimes \lambda_{2} \otimes \lambda_{3} \preceq \fc{\intC{\int{B}}}{\int{X}}$.
      % 
      From condition ``Typing I, 1.'' (\cref{def:annotated-model}), we have that~$\lambda_{1} \otimes \lambda_{2}
      \otimes \lambda_{3} \preceq \fc{\intP{\int{\typeR}}}{\int{X},\int{B}}$.
      % 
      Therefore~$\I \models \fuzzyg{\triple{X, \typeR, B}}{\lambda_{1} \otimes \lambda_{2} \otimes \lambda_{3}}$.
      %
    \item Let~$\I \models \fuzzyg{\triple{A, \range, B}}{\lambda_{1}}$, $\I \models \fuzzyg{\triple{D, \spp,
          A}}{\lambda_{2}}$, and~$\I \models \fuzzyg{\triple{X, D, Y}}{\lambda_{3}}$.
      %
      It follows that~$\fc{\intP{\int{\range}}}{\int{A}, \int{B}} \succeq \lambda_{1}$, $\fc{\intP{\int{\spp}}}{\int{D},
        \int{A}} \succeq \lambda_{2}$, and~$\fc{\intP{\int{D}}}{\int{X}, \int{Y}} \succeq \lambda_{3}$.
      % 
      According to condition ``Subproperty 2.'' from \cref{def:annotated-model}, we have
      that~$\fc{\intP{\int{\spp}}}{\int{D}, \int{A}} \otimes \fc{\intP{\int{D}}}{\int{X}, \int{Y}} \preceq
      \fc{\intP{\int{A}}}{\int{X}, \int{Y}}$ and thus~$\lambda_{1} \otimes \lambda_{2} \preceq
      \fc{\intP{\int{A}}}{\int{X}, \int{Y}}$.
      % 
      From condition ``Typing I, 3.'' (\cref{def:annotated-model}), we have that~$\fc{\intP{\int{\range}}}{\int{A},
        \int{B}} \otimes \fc{\intP{\int{A}}}{\int{X}, \int{Y}} \preceq \fc{\intC{\int{B}}}{\int{X}}$ and
      thus~$\lambda_{1} \otimes \lambda_{2} \otimes \lambda_{3} \preceq \fc{\intC{\int{B}}}{\int{X}}$.
      % 
      From condition ``Typing I, 1.'' (\cref{def:annotated-model}), we have that~$\lambda_{1} \otimes \lambda_{2}
      \otimes \lambda_{3} \preceq \fc{\intP{\int{\typeR}}}{\int{X},\int{B}}$.
      % 
      Therefore~$\I \models \fuzzyg{\triple{X, \typeR, B}}{\lambda_{1} \otimes \lambda_{2} \otimes \lambda_{3}}$.
    \end{enumerate}
    \ifnormalisedardf%
    \else%
  \item[6. Generalisation:]~ 
    Let~$\I \models \fuzzyg{\triple{X, A, Y}}{\lambda_{1}}$ and $\I \models \fuzzyg{\triple{X, A, Y}}{\lambda_{2}}$.
    % 
    It follows that~$\fc{\intP{\int{A}}}{\int{X}, \int{Y}} \succeq \lambda_{1}$ and $\fc{\intP{\int{A}}}{\int{X},
      \int{Y}} \succeq \lambda_{2}$.
    % 
    The~$\oplus$ operation, which corresponds to the least upper bound in the partial order induced over~$L$, guarantees
    that~$\lambda_{1} \oplus \lambda_{2} \succeq \lambda_{1}$ and~$\lambda_{1} \oplus \lambda_{2} \succeq \lambda_{2}$
    and thus
    % 
    we have that~$\fc{\intP{\int{A}}}{\int{X}, \int{Y}} \succeq \lambda_{1} \oplus \lambda_{2}$.
    % 
    Therefore~$\I \models \fuzzyg{\triple{X, A, Y}}{\lambda_{1} \oplus \lambda_{2}}$.
    % 
    Since~$\lambda_{1} \oplus \lambda_{2} \succeq \lambda_{1}$ and~$\lambda_{1} \oplus \lambda_{2} \succeq \lambda_{2}$,
    the triples~$\fuzzyg{\triple{X, A, Y}}{\lambda_{1}}$ and $\fuzzyg{\triple{X, A, Y}}{\lambda_{2}}$ are redundant and
    can be removed from~$G$.
    % 
    \fi
  \end{description}
  
  ($\Leftarrow$)
  %
  Given an annotated graph~$G$, let~$\I=\tuple{\Delta_{R}, \Delta_{P}, \Delta_{C}, \Delta_{L}, \intP{\cdot},
    \intC{\cdot},\int{\cdot}}$ be an interpretation defined as follows:
  %
  \begin{itemize}[nosep]
  \item $\Delta_{R} = \fc{\universe}{G} \cup \rhodf$;
  \item $\Delta_{P} = \set{p \in \fc{\voc}{G} \mid \fuzzyg{\triple{s,p,o}}{\lambda} \in \fc{cl}{G}} \cup $ \\ $ \rhodf \cup
    \set{p \in \fc{\universe}{G} \mid \fuzzyg{\triple{p,\spp,x}}{\lambda}, \fuzzyg{\triple{y,\spp,p}}{\lambda},
      \fuzzyg{\triple{p,\domR,z}}{\lambda}~\mathrm{or}~\fuzzyg{\triple{p,\range,v}}{\lambda} \in G}$;
  \item $\Delta_{C} = \set{c \in \fc{\universe}{G} \mid \fuzzyg{\triple{x,\typeR,c}}{\lambda} \in G} \cup $ \\ $\set{c
      \in \fc{\universe}{G} \mid \fuzzyg{\triple{c,\subclass,x}}{\lambda}, \fuzzyg{\triple{y,\subclass,c}}{\lambda},
        \fuzzyg{\triple{z,\domR,c}}{\lambda}~\mathrm{or}~\fuzzyg{\triple{v,\range,c}}{\lambda} \in G}$;
  \item $\Delta_{L} = \AL \cap \fc{\universe}{G}$;

  \item $\intP{\cdot}: \Delta_{P} \to 2^{\Delta_{R}\times\Delta_{R}}$ is an interpretation function such that:
    \begin{itemize}[nosep]
    \item if~$p \in \AU \cap \Delta_{P}~\mathrm{and}~\fuzzyg{\triple{x,p,y}}{\lambda} \in \fc{cl}{G}$
      then~$\fc{\intP{p}}{x,y} \succeq \lambda$;
    \item if~$p \in \AB \cap
      \Delta_{P}~\mathrm{and}~\fuzzyg{\triple{p,\spp,p'}}{\lambda_{1}},\fuzzyg{\triple{x,p',y}}{\lambda_{2}} \in
      \fc{cl}{G}$ then~$\fc{\intP{p}}{x,y} \succeq \lambda_{1} \otimes \lambda_{2}$ such that~$\lambda_{1} \otimes
      \lambda_{2} \neq \bot$.
    \end{itemize}

  \item $\intC{\cdot}: \Delta_{C} \to L$ is an interpretation function such that~$\fuzzyg{\triple{x,\typeR,c}}{\lambda}
    \in \fc{cl}{G}, \fc{\intC{c}}{x} \succeq \lambda$;

  \item $\int{\cdot}$ is the identity function over~$\fc{\universe}{G} \cup \rhodf$.
  \end{itemize}
  %
  We have that~$\I \models G$ if $\I$ satisfies all the conditions from \cref{def:annotated-model}:
  %
  \begin{description}[nosep]
  \item[Simple:]~
    \begin{enumerate}[label=(\alph*),nosep]
    %   
    \item First note that from the construction of~$\fc{cl}{G}$, $\fc{\universe}{\fc{cl}{G}} = \fc{\universe}{G} \cup
      \rhodf$.  
      % 
      Let~$\fuzzyg{\triple{s,p,o}}{\lambda} \in G$ then, from the construction of~$\I$, we have that~$\int{p} = p \in
      \Delta_P$ and~$\fc{\intP{p}}{\int{s},\int{o}} = \lambda$ and thus~$\I$ satisfies condition (i) for~$G$.

    \item We now show that if~$G \models G'$ then there is an annotated map~$G' \to \fc{cl}{G}$.  From the construction
      of~$\I$ we have that~$\I \models G$ and since~$G \models G'$, $\I \models G'$.  Since~$\I$ satisfies
      condition~$(i)$ there exists a function~$A \colon \AB \to \fc{\universe}{G}\cup\rhodf$ such that for
      each~$\fuzzyg{\triple{s,p,o}}{\lambda} \in G'$, $p\in\Delta_P$ and~$\fc{\intP{\int{p}}}{\intA{s}, \intA{o}} =
      \lambda$.  Since~$p\in\AU$, we have that~$\int{p} = \intA{p} = p$ and thus~$\fc{\intP{\int{p}}}{\intA{s},
        \intA{o}} = \fc{\intP{p}}{\intA{s}, \intA{o}} = \lambda$ for each~$\fuzzyg{\triple{s,p,o}}{\lambda} \in
      \fc{cl}{G}$.
      %
      Since~$\fc{\intP{\int{p}}}{\intA{s}, \intA{o}} = \lambda$, we have
      that~$\fuzzyg{\triple{\intA{s},\intA{p},\intA{o}}}{\lambda} \in\fc{cl}{G}$ for
      each~$\fuzzyg{\triple{s,p,o}}{\lambda} \in G'$.
      % 
      Thus~$\IA \colon G' \to \fc{cl}{G}$ is an annotated map~$G' \to \fc{cl}{G}$.
    \end{enumerate}
  \item[Subproperty:]~ 
    \begin{enumerate}[label=(\alph*),nosep]
    \item Let~$\fc{\intP{\int{\spp}}}{\int{A}, \int{B}} \succeq \lambda_{1}$ and~$\fc{\intP{\int{\spp}}}{\int{B},
        \int{C}} \succeq \lambda_{2}$.
      % 
      From the construction of~$\I$ we have that~$\fuzzyg{\triple{A,\spp,B}}{\lambda_{1}},
      \fuzzyg{\triple{B,\spp,C}}{\lambda_{2}} \in \fc{cl}{G}$ and~$A,B,C \in \Delta_P$.  Since~$\fc{cl}{G}$ is closed
      under application of rule~$(2a)$ we have that~$\fuzzyg{\triple{A,\spp,C}}{\lambda_{1} \otimes \lambda_{2}} \in
      \fc{cl}{G}$ and thus~$\fc{\intP{\int{\spp}}}{\int{A}, \int{B}} \succeq \lambda_{1} \otimes \lambda_{2}$.

    \item Let~$\fc{\intP{\int{D}}}{\int{X}, \int{Y}} \succeq \lambda_1$ and~$\fc{\intP{\int{\spp}}}{\int{D}, \int{E}}
      \succeq \lambda_2$, thus~$\fuzzyg{\triple{D,\spp,E}}{\lambda_2} \in \fc{cl}{G}$ and~$D,E\in\Delta_P$.
      %
      We must consider the following cases:
      \begin{itemize}[nosep]
      \item if~$D \in \AU$ then, from the construction of~$\I$ we have that~$\fuzzyg{\triple{X,D,Y}}{\lambda_1} \in
        \fc{cl}{G}$.
        %
        If~$E \in \AU$ and since~$\fc{cl}{G}$ is closed under the application of rule~$(2b)$, we also have
        that~$\fuzzyg{\triple{X,E,Y}}{\lambda_1 \otimes \lambda_2} \in \fc{cl}{G}$.
        % 
        Therefore~$\fc{\intP{\int{E}}}{\int{X}, \int{Y}} \succeq \lambda_1 \otimes \lambda_2$.
        %
        If~$E \in \AB$, then~$\fuzzyg{\triple{X,D,Y}}{\lambda_1}, \fuzzyg{\triple{D,\spp,E}}{\lambda_2} \in \fc{cl}{G}$,
        and from the construction of~$\I$ we have that~$\fc{\intP{\int{E}}}{\int{X}, \int{Y}} \succeq \lambda_1 \otimes
        \lambda_2$.
        %
      \item if~$D \in \AB$ by the construction of~$\I$ there exists~$D'$ such
        that~$\fuzzyg{\triple{D',\spp,D}}{\lambda_3}, \fuzzyg{\triple{X,D',Y}}{\lambda_4} \in \fc{cl}{G}$ and~$D' \in
        \Delta_{P}$.
        % 
        Since~$\fc{cl}{G}$ is closed under the application of rule~$(2a)$, we also have
        that~$\fuzzyg{\triple{D',\spp,E}}{\lambda_2 \otimes \lambda_3} \in \fc{cl}{G}$.
        %
        If~$E \in \AU$ as~$\fuzzyg{\triple{D',\spp,E}}{\lambda_2 \otimes \lambda_3}, \fuzzyg{\triple{X,D',Y}}{\lambda_4}
        \in \fc{cl}{G}$ and since~$\fc{cl}{G}$ is closed under the application of rule~$(2b)$, we also have
        that~$\fuzzyg{\triple{X,E,Y}}{\lambda_2 \otimes \lambda_3 \otimes \lambda_4} \in \fc{cl}{G}$.
        % 
        Therefore, from the construction of~$\I$, we have that~$\fc{\intP{\int{E}}}{\int{X}, \int{Y}} \succeq \lambda_2
        \otimes \lambda_3 \otimes \lambda_4$.
        %
        If~$E \in \AB$, then~$\fuzzyg{\triple{X,D,Y}}{\lambda_1}, \fuzzyg{\triple{D,\spp,E}}{\lambda_2} \in \fc{cl}{G}$,
        and from the construction of~$\I$ we have that~$\fc{\intP{\int{E}}}{\int{X}, \int{Y}} \succeq \lambda_1 \otimes
        \lambda_2$.
        %
      \end{itemize}
    \end{enumerate}
  \item[Subclass:]~ 
    \begin{enumerate}[label=(\alph*),nosep]
    \item Let~$\fc{\intP{\int{\subclass}}}{\int{A}, \int{B}} \succeq \lambda_{1}$
      and~$\fc{\intP{\int{\subclass}}}{\int{B}, \int{C}} \succeq \lambda_{2}$.
      % 
      From the construction of~$\I$ we have that~$\fuzzyg{\triple{A,\subclass,B}}{\lambda_{1}},
      \fuzzyg{\triple{B,\subclass,C}}{\lambda_{2}} \in \fc{cl}{G}$ and~$A,B,C \in \Delta_C$.  Since~$\fc{cl}{G}$ is
      closed under application of rule~$(3a)$ we have that~$\fuzzyg{\triple{A,\subclass,C}}{\lambda_{1} \otimes
        \lambda_{2}} \in \fc{cl}{G}$ and thus~$\fc{\intP{\int{\subclass}}}{\int{A}, \int{B}} \succeq \lambda_{1} \otimes
      \lambda_{2}$.

    \item Let~$\fc{\intC{\int{A}}}{\int{X}} \succeq \lambda_{1}$ and~$\fc{\intP{\int{\subclass}}}{\int{A}, \int{B}}
      \succeq \lambda_{2}$.
      %
      From condition ``Typing I, 1.'' (\cref{def:annotated-model}), we have that~$\fc{\intP{\int{\typeR}}}{\int{X},
        \int{A}} = \lambda_{2}$ and thus~$\I \models \fuzzyg{\triple{X, \typeR, A}}{\lambda_{2}}$.
      % 
      Since~$\fc{cl}{G}$ is closed under application of rule~$(3b)$ we have that~$\fuzzyg{\triple{X, \typeR, B}}{\lambda_1
        \otimes \lambda_2}$.
      %
      Then~$\fc{\intP{\int{\typeR}}}{\int{X}, \int{B}} \succeq \lambda_{1} \otimes \lambda_{2}$ and
      thus~$\fc{\intC{\int{B}}}{\int{X}} \succeq \lambda_{1} \otimes \lambda_{2}$.
      %
    \end{enumerate}
  \item[Typing I:]~
    \begin{enumerate}[label=(\alph*),nosep]
    \item Let~$\fc{\intP{\int{\typeR}}}{X,C} \succeq \lambda_1$, by construction of~$\I$ we have that~$C\in\Delta_C$
      and~$\fuzzyg{\triple{X,\typeR,C}}{\lambda_1} \in \fc{cl}{G}$.  Also by the construction of~$\intC{\cdot}$ we have
      that~$\fc{\intC{\int{C}}}{X} \succeq \lambda_1$.
      %
      On the other hand, if~$C\in\Delta_C$ and~$\fc{\intC{\int{C}}}{X} \succeq \lambda_1$, by construction
      of~$\intC{\cdot}$ we have that~$\fuzzyg{\triple{X, \typeR, C}}{\lambda_1} \in \fc{cl}{G}$ and
      so~$\fc{\intP{\int{\typeR}}}{X,C} \succeq \lambda_1$.
      %
    \item Let~$\fc{\intP{\int{\domR}}}{D, B} \succeq \lambda_1$ and~$\fc{\intP{D}}{X,Y} \succeq \lambda_2$.
      By construction of~$\I$ we have that~$D\in\Delta_P$ and~$B\in\Delta_C$. 
      % 
      Since~$\fc{cl}{G}$ is closed under application of rule~$(4a)$ we have that~$\fuzzyg{\triple{X, \typeR, B}}{\lambda_1
        \otimes \lambda_2} \in \fc{cl}{G}$.
      %
      Then~$\fc{\intP{\typeR}}{\int{X},\int{B}} \succeq \lambda_1 \otimes \lambda_2$ and~$\fc{\intC{\int{B}}}{\int{X}}
      \succeq \lambda_1 \otimes \lambda_2$.
      %
    \item Let~$\fc{\intP{\int{\range}}}{D, B} \succeq \lambda_1$ and~$\fc{\intP{D}}{X,Y} \succeq \lambda_2$.
      By construction of~$\I$ we have that~$D\in\Delta_P$ and~$B\in\Delta_C$. 
      % 
      Since~$\fc{cl}{G}$ is closed under application of rule~$(4b)$ we have that~$\fuzzyg{\triple{Y, \typeR, B}}{\lambda_1
        \otimes \lambda_2} \in \fc{cl}{G}$.
      %
      Then~$\fc{\intP{\typeR}}{\int{Y},\int{B}} \succeq \lambda_1 \otimes \lambda_2$ and~$\fc{\intC{\int{B}}}{\int{Y}}
      \succeq \lambda_1 \otimes \lambda_2$.
      %
    \end{enumerate}
  \item[Typing II:] The definition of~$\Delta_R$ and~$\Delta_P$ satisfy all of these conditions.
  \end{description}%
\end{proof}


%%% Local Variables:
%%% mode: latex
%%% TeX-master: "../thesis"
%%% End:



%%% Local Variables:
%%% mode: latex
%%% TeX-master: "../thesis"
%%% End:




\subsection{Query Answering} 
\label{aqa}
%
Informally, queries are as for the classical case where triples are replaced with annotated triples in which
\emph{annotation variables} (taken from an appropriate alphabet and denoted~$\Lambda$) may occur.  We allow built-in
triples of the form $\triple{s,p,o}$, where~$p$ is a built-in predicate taken from a reserved vocabulary and having a
\emph{fixed interpretation} on the annotation domain~$D$, such as~$\triple{\lambda, \preceq , l}$ stating that the value
of $\lambda$ has to be~$\preceq$ than the value~$l \in L$. We generalise the built-ins to any~$n$-ary predicate~$p$,
where $p$'s arguments may be annotation variables,~$\rhodf$ variables, domain values of~$D$, values from~$\AUL$, and~$p$
has a fixed interpretation. We will assume that the evaluation of the predicate can be decided in finite time. As for
the crisp case, for convenience, we write ``functional predicates'' as \emph{assignments} of the form~$x\assign
\funcCall{f}{\vec{z}}$ and assume that the function~$\funcCall{f}{\vec{z}}$ is safe. Furthermore, we also assume that
any non functional built-in predicate~$\funcCall{p}{\vec{z}}$ should be safe as well.

For instance, informally for a given time interval~$[t_{1}, t_{2}]$, we may define~$x\assign length([t_{1}, t_{2}])$ as
true \iff the value of~$x$ is~$t_{2} - t_{1}$.

\begin{example}[Annotated query]
  \label{exUs4}
  % 
  \noindent Considering our dataset from \cref{fig:dataset-example} as input and the query asking for artists that
  were members of the Nightwish band between 2000 and 2010 and the temporal term at which this was true:
  %
  \vspace{-\abovedisplayskip}
  %
  \begin{align*}
    q(x, \Lambda) \leftarrow \fuzzyg{\triple{\qname{dbpedia}{Nightwish}, \qname{foaf}{member}, x}}{\Lambda'},
    \Lambda\assign (\Lambda' \land [2000,2010]) 
  \end{align*}
  %
  \noindent will get the following answers:
  \[ \begin{array}{l}
    \tuple{\qname{dbpedia}{Marco\mathunderscore{}Hietala}, [2001, 2010]}\\
    \tuple{\qname{dbpedia}{Tarja\mathunderscore{}Turunen}, [2000, 2005]} \enspace .
  \end{array} \]
  %
\end{example}

\noindent Formally, an \emph{annotated query} is of the form
\[
q(\vec{x},\vec{\Lambda}) \leftarrow \exists \vec{y}\exists\mathbf{\Lambda}'.\varphi(\vec{x},\vec{\Lambda},\vec{y},\vec{\Lambda}')
\]
in which~$\varphi(\vec{x}, \vec{\Lambda},\vec{y},\vec{\Lambda}')$ is a conjunction (as for the crisp case, we use
\character{,} as conjunction symbol) of annotated triples and built-in predicates,~$\vec{x}$ and~$ \vec{\Lambda}$ are
the distinguished variables,~$\vec{y}$ and~$\vec{\Lambda}'$ are the \emph{non-distinguished variables} (existential
quantified variables), and~$\vec{x}$,~$\vec{\Lambda}$,~$\vec{y}$ and $\vec{\Lambda}'$ are pairwise disjoint.
%
Variables in~$\vec{\Lambda}$ and~$\vec{\Lambda}'$ can only appear in annotations or built-in predicates and furthermore,
the query head must contain at least one variable.

Given an annotated graph~$G$, a query~$q(\vec{x}, \vec{\Lambda}) \leftarrow \exists
\vec{y}\exists\mathbf{\Lambda}'.\varphi(\vec{x}, \vec{\Lambda}, \vec{y},\vec{\Lambda}')$, a vector~$\vec{t}$ of terms in
$uni\-verse(G)$ and a vector~$\vec{\lambda}$ of annotated terms in~$L$, we say that~$q(\vec{t}, \vec{\lambda})$ is
\emph{entailed} by~$G$, denoted~$G \models q(\vec{t}, \vec{\lambda})$, \iff in any model~$\I$ of~$G$, there is a vector
$\vec{t}'$ of terms in~$ \universe(G)$ and a vector~$\vec{\lambda}'$ of annotation values in~$L$ such that~$\I$ is a
model of~$\varphi(\vec{t}, \vec{\lambda}, \vec{t}', \vec{\lambda}')$. If~$G \models q(\vec{t}, \vec{\lambda})$ then
$\tuple{\vec{t}, \vec{\lambda}}$ is called an \emph{answer} to~$q$. The \emph{answer set} of~$q$ \wrt~$G$ is ($\preceq$
extends to vectors point-wise)
%
\[
ans(G, q) = \set{ \tuple{\vec{t}, \vec{\lambda}} \mid G \models  q(\vec{t}, \vec{\lambda}), \vec{\lambda} \neq \vec{\bot}
  \mbox { and for any }  \vec{\lambda}' \neq \vec{\lambda}
  \mbox { such that } G \models  q(\vec{t}, \vec{\lambda}'), \vec{\lambda}' \preceq \vec{\lambda}  \mbox { holds} } \enspace .
\]
%
\noindent That is, for any tuple~$\vec{t}$, the vector of annotation values~$\vec{\lambda}$ is as large as
possible. This is to avoid that redundant/subsumed answers occur in the answer set.  The following can be shown:
%
\begin{proposition}\label{propU2}
  Given a graph~$G$,~$\tuple{\vec{t}, \vec{\lambda}}$ is an \emph{answer} to~$q$ \iff~$\exists
  \vec{y}\exists\mathbf{\Lambda}'.\varphi(\vec{t}, \vec{\lambda}, \vec{y},\vec{\Lambda}')$ is true in the closure of~$G$
  and~$\lambda$ is~$\preceq$-maximal.\footnote{$\exists \vec{y}\exists\mathbf{\Lambda}'.\varphi(\vec{t}, \vec{\lambda},
    \vec{y},\vec{\Lambda}')$ is true in the closure of~$G$ \iff for some~$\vec{t}'$,~$\vec{\lambda}'$ for all triples
    in~$\varphi(\vec{t},\vec{\lambda},\vec{t}',\vec{\lambda}')$ there is a triple in~$cl(G)$ that subsumes it and the
    built-in predicates are true, where an annotated triple~$\fuzzyg{\tau}{\lambda_{1}}$
    subsumes~$\fuzzyg{\tau}{\lambda_{2}}$ \iff~$\lambda_{2} \preceq \lambda_{1}$.}
\end{proposition}



\subsubsection*{Queries with Aggregates} 
\label{sec:aggr}

Next we extend the query language by allowing so-called aggregates to occur in a query. Essentially, aggregates may be
like the usual SQL aggregate functions such as~$\keyword{SUM}, \keyword{AVG}, \keyword{MAX}, \keyword{MIN}$. But, we
have also domain specific aggregates such as~$\oplus$ and~$\otimes$.
%
The following examples present some queries that can be expressed with the use of built-in queries and aggregates.

\begin{example}[Assignment query]
  Using a built-in aggregate we can pose a query that, for each band member, retrieves his maximal time of employment
  for any band in the following way:
  \[
  \funcCall{q}{x, \mathit{maxL}} \leftarrow \fuzzyg{\triple{y, \qname{foaf}{member}, x}}{\lambda}, \mathit{maxL} \assign
  \funcCall{maxlength}{\lambda} \enspace .
  \]
  \nd Here, the~$\funcName{maxlength}$ built-in predicate returns, given a set of temporal intervals, the maximal interval
  in the set.
  % 
\end{example}

\begin{example}[Aggregation query]
  \label{exAA}
  Suppose we are looking for artists that are members of some \qname{mo}{MusicGroup} for a certain time period and we
  would like to know the average length of their membership. Then such a query will be expressed as
  \[
    \funcCall{q}{x, avgL} \leftarrow~  \fuzzyg{\triple{y, \qname{foaf}{member}, x}}{\lambda},\funcCall{\mathsf{GroupedBy}}{x},
                            avgL \assign \keyword{AVG}[\funcCall{length}{\lambda}] \enspace .
  \]
  \nd Essentially, we group by the artist, compute for each artist the time he was a member of the
  \qname{mo}{MusicGroup} (by means of the built-in function~\funcName{length}), and finally compute the average value
  for each group.
  %
  That is,~$g = \{\tuple{t, t_{1}},\ldots, \tuple{t, t_{n}}\}$ is a group of tuples with the same value~$t$ for
  artist~$x$, and value~$t_{i}$ for~$y$, where each length of membership for~$t_{i}$ is~$l_{i}$ (computed
  as~$\funcCall{length}{\cdot}$), then the value of~$avgL$ for the group~$g$ is~$(\sum_{i} l_{i})/n$.
  % 
\end{example}

\nd Formally, let~$\aggr$ be an aggregate function with~$\aggr \in \{\keyword{SUM}, \keyword{AVG}, \keyword{MAX},
\keyword{MIN}, \keyword{COUNT}, \oplus, \otimes\}$ then a query with aggregates is of the form
\begin{align*}
  \funcCall{q}{\vec{x}, \vec{\Lambda},\alpha} \leftarrow~ & \exists \vec{y}\exists\mathbf{\Lambda}'.\funcCall{\varphi}{\vec{x}, \vec{\Lambda}, \vec{y},\vec{\Lambda}'},\\
  & \funcCall{\mathsf{GroupedBy}}{\vec{w}},\\
  & \alpha \assign\aggr[\funcCall{f}{\vec{z}}]
\end{align*}

\nd where~$\vec{w}$ are variables in~$\vec{x}$,~$\vec{y}$ or~$\vec{\Lambda}$, each variable in~$\vec{x}$
and~$\vec{\Lambda}$ must occur in~$\vec{w}$ and any variable in~$\vec{z}$ occurs in~$\vec{y}$ or~$\vec{\Lambda'}$.
%
From a semantics point of view, we say that~$\I$ \emph{is a model of} (\emph{satisfies})~$q(\vec{t},\vec{\lambda}, a)$,
denoted $\I \models q(\vec{t},\vec{\lambda}, a)$ \iff
%
$a = \aggr [a_{1}, \ldots, a_{k}]$, where~$g = \{ \tuple{\vec{t}, \vec{\lambda}, \vec{t}'_{1},\vec{\lambda}'_{1}},
\ldots , \tuple{\vec{t}, \vec{\lambda}, \vec{t}'_{k},\vec{\lambda}_{k}'} \}$ is a group of~$k$ tuples with identical
projection on the variables in~$\vec{w}$,~$\varphi(\vec{t}, \vec{\lambda}, \vec{t}'_{r},\vec{\lambda}'_{r})$ is true in~\I
and~$a_{r} =f(\vec{\vec{t}})$ where~$\vec{\vec{t}}$ is the projection of~$\tuple{\vec{t}'_{r}, \vec{\lambda}'_{r}}$ on the
variables~$\vec{z}$.
%
Now, the notion of~$G \models q(\vec{t},\vec{\lambda}, a)$ is as usual, any model of~$G$ is a model
of~$q(\vec{t},\vec{\lambda}, a)$.

Eventually, we further allow to order answers according to some ordering functions.

\begin{example}[Ordering query]
  \label{exx} 
  % 
  Consider \cref{exAA}. We additionally would like to order the artists according to the average length of
  membership to a band.  Then such a query will be expressed as: \vspace{-\abovedisplayskip}
  %
  \begin{align*}
    \funcCall{q}{x,avgL} \leftarrow~& \fuzzyg{\triple{y, \qname{foaf}{member}, x}}{\lambda}, \funcCall{\mathsf{GroupedBy}}{x},\\
    & avgL \assign \keyword{AVG}[\funcCall{length}{\lambda}],\funcCall{\mathsf{OrderBy}}{avgL} \enspace .
  \end{align*}
  % 
\end{example}

\nd Formally, a query with ordering is of the form
\[
\begin{array}{lcl}
q(\vec{x}, \vec{\Lambda}, z) & \leftarrow & \exists \vec{y}\exists\mathbf{\Lambda}'.\varphi(\vec{x}, \vec{\Lambda}, \vec{y},\vec{\Lambda}'), \mathsf{OrderBy}(z)
\end{array}
\]


\nd or, in case grouping is allowed as well, it is of the form
\[
\begin{array}{lcl}
  q(\vec{x}, \vec{\Lambda},z, \alpha) & \leftarrow & \exists \vec{y}\exists\mathbf{\Lambda}'.\varphi(\vec{x}, \vec{\Lambda}, \vec{y},\vec{\Lambda}'),\\
                                      &            & \mathsf{GroupedBy(\vec{w})},\\
                                      &            & \alpha \assign\aggr[f(\vec{z})],\\
                                      &            & \mathsf{OrderBy}(z)
\end{array}
\]

\nd From a semantics point of view, the notion of~$G \models q(\vec{t},\vec{\lambda}, z, a)$ is as before, but the
notion of answer set has to be enforced with the fact that the answers are now ordered according to the assignment of
the variable~$z$. Of course, we require that the set of values over which~$z$ ranges can be ordered (like string,
integers, reals).
%
In case the variable~$z$ is an annotation variable, the order is induced by~$\preceq$.  



%%% Local Variables:
%%% mode: latex
%%% TeX-master: "../thesis"
%%% End:





%%% Local Variables:
%%% mode: latex
%%% TeX-master: "../thesis"
%%% End:
