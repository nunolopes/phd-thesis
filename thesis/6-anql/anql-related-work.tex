\section{Related Work}
\label{sec:related}

Adding annotations to logical statements was already proposed in the logic programming realm
by~\citet{KiferSubrahmanian:1992aa} who took a similar approach, where atomic formulas are annotated with a value taken
from a lattice of annotation values, an annotation variable or a complex annotation, \ie~a function applied to
annotation values or variables.
%
Similarly, we can relate our work to annotated relational databases, especially~\citet{GreenKarvounarakisTannen:2007aa}
who provides a similar framework for relational algebra.  After presenting a generic structure for annotations, they
focus more specifically on the provenance domain. 


Annotated RDF was first presented by~\citet{UdreaRecuperoSubrahmanian:2006aa,UdreaRecuperoSubrahmanian:2010aa}, where the
authors define triples annotated with values taken from a \emph{finite partial order}.
%
In their work, triples are of the form $\triple{s,p\!:\!\lambda,o}$, where the property, rather than the triple is
annotated and furthermore represent \ac{RDF} as a set of nodes and edges rather than extending the model theoretic
semantics followed by the \ac{W3C}.
%
In our work, we rely on a richer, not necessarily finite, structure and provide additional inference capabilities when
compared to~\citet{UdreaRecuperoSubrahmanian:2010aa}, such as a more involved propagation of annotation values through
schema triples.
%
Essentially,~\citeauthor{UdreaRecuperoSubrahmanian:2010aa} do not provide an operation to combine annotations in
\ac{RDFS} inferences. 
%
The query language presented by~\citet{UdreaRecuperoSubrahmanian:2010aa} consists of conjunctive queries and, while
SPARQL's \acp{BGP} are compared to their conjunctive queries, they do not consider extending SPARQL with the possibility
of querying annotations. Furthermore, $\keyword{optional}$, $\keyword{union}$ and $\keyword{filter}$ SPARQL queries are
not considered, which results in a subset of SPARQL that can be directly translated into their previously presented
conjunctive query system.

In our initial approach the structure for representing annotations was defined as a residuated
lattice~\cite{StracciaLopesLukacsy:2010aa,LopesPolleresStraccia:2010aa}, which was later extended to the more general
semiring structure by~\citet{BunemanKostylev:2010aa}.
%
Furthermore, \citet{BunemanKostylev:2010aa} also show that once the \ac{RDFS} inferences of an \ac{RDF} graph have been
computed for a specific domain, it is possible to reuse these inferences if the graph is annotated with a different
domain.  Based on this result, the authors define a universal domain, which is possible to transform to other domains by
applying the corresponding transformations.


For the Semantic Web, several extensions of RDF were proposed in order to deal with \emph{specific} domains such as
truth of imprecise
information~\cite{MazzieriDragoni:2008aa,MazzieriDragoni:2005aa,Mazzieri:2004aa,Straccia:2009aa,LvMaYan:2008aa},
time~\cite{GutierrezHurtadoVaisman:2007aa,PuglieseUdreaSubrahmanian:2008aa,TappoletBernstein:2009aa},
trust~\cite{Hartig:2009ab,Schenk:2008aa} and provenance~\cite{DividinoSizovStaab:2009aa}.  These approaches are detailed
in the following paragraphs.


\citet{Straccia:2009aa} presents Fuzzy RDF in a general setting where triples are annotated with a degree of truth in
$[0,1]$. For instance, ``Rome is a big city to degree 0.8'' can be represented with $\fuzzyg{\triple{\term{Rome},
    \typeR, \term{BigCity}}}{0.8}$; the annotation domain is $[0,1]$.  For the query language, it formalises conjunctive
queries.
%
Other similar approaches for Fuzzy RDF~\cite{MazzieriDragoni:2008aa,MazzieriDragoni:2005aa,Mazzieri:2004aa} provide the
syntax and semantics, along with RDF and \ac{RDFS} interpretations of the annotated triples.
%
\citet{Mazzieri:2004aa} describes an implementation strategy that relies on translating the Fuzzy triples into plain RDF
triples by using reification.  However these works focus mostly on the representation format and the query answering
problem is not addressed.

\citet{GutierrezHurtadoVaisman:2007aa} present the definitions of Temporal RDF, including reduction of the semantics of
Temporal RDF graphs to RDF graphs and a sound and complete inference system.  They show that entailment of Temporal
graphs does not yield extra complexity beyond RDF entailment.  Our Annotated RDFS framework encompasses this work by
defining the temporal domain.  The authors present conjunctive queries with built-in predicates as the query language
for Temporal RDF, although they do not consider full SPARQL.
%
\citet{GutierrezHurtadoVaisman:2007aa} describe some further features such as a ``Now'' time point (which is a defined
time point in the domain) and anonymous time points, allowing to state that a triple is true at some point. Adding
anonymous time points would require us to extend the semi-ring by appropriate operators, \eg~$[2004, T] \oplus [T, 2008]
= [2004,2008]$ (where~$T$ is an anonymous time point).
% 
\citet{PuglieseUdreaSubrahmanian:2008aa} presents an optimised indexing schema for Temporal RDF, along with the notion
of normalised Temporal RDF graph, and a query language for these graphs based on SPARQL.  The indexing scheme consists
of clustering the RDF data based on their temporal distance, for which several metrics are given.  For the query
language they only define conjunctive queries, thus ignoring some of the more advanced features of SPARQL.
%
\citet{TappoletBernstein:2009aa} present another approach to the implementation of Temporal RDF, where each temporal
interval is represented as a named graph~\citep{CarrollBizerHayes:2005aa} containing all triples valid in that time
period. Information about temporal intervals, such as their relative relations, start and end points, is asserted in the
default graph.  The $\tau$-SPARQL query language allows to query the temporal RDF representation using an extended
SPARQL syntax that can match the graph pattern against the snapshot of a temporal graph at any given time point and
allows to query the start and endpoints of a temporal interval, whose values can then be used in other parts of the
query.

SPARQL extensions towards querying trust have been presented by~\citet{Hartig:2009ab}, introducing a trust aware query
language, tSPARQL, that includes a new constructor to access the trust value of a graph pattern.  This value can then be
used in other statements such as $\keyword{FILTER{s}}$ or $\keyword{ORDER}$.  Although focusing on trust, the approach
is close to our general framework, introducing concepts similar to the ones presented in this chapter.  However, a
general framework was not presented.
%
Also in the setting of trust management, \citet{Schenk:2008aa} defines a \emph{bilattice} structure to model
\emph{trust} relying on the dimensions of knowledge and truth.  The defined knowledge about trust in information sources
can then be used to compute the trust of an inferred statement. An extension towards \ac{OWL} is presented but there is
no query language defined.  Finally, this approach is used to resolve inconsistencies in ontologies arising from
connecting multiple data sources.



Regarding provenance, in \citet{DelbruPolleresTummarello:2008aa}, the authors do not formalise the semantics and
properties of the aggregation operation (simply denoted by~$\land$) nor the exact rules that should be applied to
correctly reason with provenance. Query answering is not tackled either.
%
\citet{FlourisFundulakiPediaditis:2009aa} provides more insight into the formalisation and details the rules by reusing
(tacitly)~\citet{MunozPerezGutierrez:2007aa}. They also provide a formalisation of a simple query language. However, the
semantics they define is based on a strong restriction of \rhodf (which is already a restriction of \ac{RDFS}).
%
As an example, they define the answers to the query~$(\var{x}, \typeR, \var{y}, \var{c})$ as the tuples~$(X,Y,C)$ such
that there is a triple~$(X, \typeR, Y, C)$ which can be inferred from only the application of rules (3a) and (3b) from
the deductive system presented in \cref{sec:rdf-schema}.
%
This means that a domain or range assertion would not provide additional answers to that type of query.
%
Provenance also relates to the Named Graphs formalism~\cite{CarrollBizerHayes:2005aa} where one can identify distinct
graphs with a URI. The name can be seen as an atomic provenance annotation. However, Named Graphs do not provide
operations to combine the provenances. Yet, the formalism could be used as a possible syntactic solution for
representing annotated triples.


\citet{DividinoSizovStaab:2009aa} also present a generic extension of RDF to represent \emph{meta information}, mostly
focused on provenance and uncertainty.  Such meta information is stored using named graphs and their extended semantics
of RDF, denoted RDF$^+$, assumes a predefined vocabulary to be interpreted as meta information.  However they do not
provide an extension of the \ac{RDFS} inference rules or any operations for combining meta information.  The authors
also provide an extension of the SPARQL query language, considering an additional expression that enables querying the
RDF meta information.


\citet{BonattiHoganPolleres:2011aa} provide a framework for a specific combination of annotations (authoritativeness,
rank, blacklisting, and provenance) within \ac{RDFS} and (a variant of) \ac{OWL} 2 RL. This work is orthogonal to ours,
in that it does not focus on aspects of query answering, or providing a generic framework for combinations of
annotations, but rather on scalable and efficient algorithms for materialising inferences for the specific combined
annotations under consideration.




Different extensions of \ac{RDF} and SPARQL focused on modelling spatial and temporal data were presented, namely
stRDF~\cite{KoubarakisKyzirakos:2010aa} and SPARQL-ST~\cite{PerryJainSheth:2011aa}.
%
SPARQL-ST focuses on extending the SPARQL query language relying on previous proposals such as Temporal
\ac{RDF}~\cite{GutierrezHurtadoVaisman:2007aa} and proposes a modelling of two dimensional geometries to represent the
spatial coordinates in plain \ac{RDF}.
%
The extension of SPARQL is done by defining spatial and temporal variables and graph patterns and new filters and
built-in conditions that operate over the temporal and spatial variables.  Possible spatial filters allow to determine
wether specific relations (\eg~equal, contains) hold between different geometries or to determine the distance between
the geometries.
% 
Filtering the temporal variables is based on the Allen interval relations~\cite{Allen:1983aa}.
%
stRDF and stSPARQL focus especially on representing sensor data, introducing triple annotations capable of representing
moving trajectories of sensors and geometric areas where the sensors are deployed.  Spatial data is represented by
allowing \ac{RDF} objects to be of a custom representation for geometries, whereas the temporal data is represented as
an annotation over \ac{RDF} triples.  
%
The stSPARQL query language consists of an extension of SPARQL to consider the fourth element to query the temporal
annotations, while spatial querying is based on filter expressions.

%%% Local Variables:
%%% mode: latex
%%% TeX-master: "../thesis"
%%% End:
