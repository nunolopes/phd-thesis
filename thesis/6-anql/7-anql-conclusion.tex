\section{Conclusion}
\label{sec:conclusion}

In this chapter we have presented a generalised \ac{RDF} annotation framework that conservatively extends the \ac{RDFS}
semantics, along with an extension of the SPARQL query language to query annotated data.
%
The framework presented here is generic enough to cover other proposals for \ac{RDF} annotations and their query
languages.  Our approach extends the classical case of \ac{RDFS} reasoning with features of different annotation
domains, such as temporality, fuzzyness, or provenance. 
%
Furthermore, we presented a semantics for an extension of the SPARQL query language, AnQL, that enables querying
\ac{RDF} with annotations.


In the proposed data integration setting, this \ac{RDF} extension can be used as a target data model, allowing to
represent meta-information about the integrated data and thus allowing to resolve conflicts arising from the data
integration process.
%
In the next chapter we present a complete \usecase scenario where the defined language and data model are used to
integrate data from different enterprise sources that may be protected by access control information.  We also introduce
the access control annotation domain that allows us to represent such annotated data and to enable sharing and querying
only restricted sets of triples, on a per-user basis.


%%% Local Variables:
%%% mode: latex
%%% TeX-master: "../thesis"
%%% End:
