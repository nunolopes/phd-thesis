According to the prototype described in \cref{sec:implementation-notes}, the implementation of the access control
annotation domain consists of a Prolog module that is imported by the reasoner.
%
This module defines the domain operations $\otimes_{ac}$ and $\oplus_{ac}$, represented as the predicates
\texttt{infimum/3} and \texttt{supremum/3}, respectively.
%
The annotation values are represented simply by using \emph{lists}, in this case lists of lists, following the
definitions presented in the previous section. 
%



The implementation of the $\oplus_{ac}$ operation involves concatenating the list representation of both annotations and
then performing the normalisation operation.
%
As for the $\otimes_{ac}$ operation, we follow a similar procedure to the $\oplus_{ac}$ operation, with the additional
step of applying one of the previously presented \emph{brave} and \emph{safe} conflict resolution methods.
%
The evaluation of the \nrdn program can be performed based on the representation of the annotation values, by checking
if the list of credentials of a user is a superset of any of the positive literals of the statements of our annotation
values and also that it does not contain any of the negative literals of the statement.
%

An example of \ac{RDF} data annotated with Access Control information, where the salary information is only available to
the respective employee, is presented in \cref{fig:ardf-ac}.  In this figure we are representing the \ac{RDF} triples
and annotation element using the N-Quads \ac{RDF} serialisation~\cite{CyganiakHarthHogan:2009aa}.
%
\begin{data}[t]
\centering
\begin{lstlisting}[basicstyle=\tt\small]
@prefix : <http://urq.deri.org/enterprise#> . 

:westportCars rdf:type :Company  "[[jb]]".
:westportCars :netIncome 1000000 .
:joeBloggs :worksFor :westportCars .
:joeBloggs :salary 80000 "[[jb]]".
:johnSmith :worksFor :westportCars .
:johnSmith :salary 40000 "[[js]].
\end{lstlisting}
\caption{Access Control Annotated RDFS}
\label{fig:ardf-ac}
\end{data}
%
Using AnQL, the extension of the SPARQL query language described in \cref{sec:annotated-sparql}, it is possible
to perform queries that take into consideration the access control annotations.
%
An example of an AnQL query over this data is presented in the following example:
%
\begin{example}[AnQL Query Example]
  %
  \label{sec:anql-ac-query} This query specifies that we are interested in the salary of employees that someone with the
  permissions $[[jb,st,it]]$ is allowed to access.
  %
\begin{lstlisting}[basicstyle=\tt\small,frame=none,numbers=none]
SELECT * WHERE { ?p :salary ?s "[[jb, st, it]]" } 
\end{lstlisting}
  % 
  The answers for this query (when matched against \cref{fig:ardf-ac}) under SPARQL semantics,
  \ie~if the annotation would be omitted, would be:
  \[ \set{ 
    %
    \set{ \var{p} \to \uri{{:}joeBloggs}, \var{s} \to 80000 }, 
    % 
    \set{ \var{p} \to \uri{{:}johnSmith}, \var{s} \to 40000 }
    % 
  } \enspace.\]
  %
  However, with the inclusion of domain annotations, an AnQL query engine must also perform the following check: \(
  [[jb,st,it]]\) satisfies the \nrdn program \(\lambda\),
  % 
  where $\lambda$ is the program represented by the annotation of each matched triple, thus yielding only the following
  answer:
  %
  \[ \set{ 
    % 
    \set{ \var{p} \to \uri{{:}joeBloggs}, \var{s} \to 80000 }
    % 
  } \enspace. \]
\end{example}


%%% Local Variables:
%%% mode: latex
%%% mode: flyspell
%%% mode: reftex
%%% mode: visual-line
%%% TeX-master: "../thesis"
%%% End:
