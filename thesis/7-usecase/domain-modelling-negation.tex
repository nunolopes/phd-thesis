
For the modelling of the access control domain consider, in addition to the previously presented sets of URIs \AU, blank
nodes \AB, and literals \AL, a set of credential elements~\AC.
%
The elements of~\AC are used to represent \emph{users}, \emph{roles}, and \emph{groups}.
%
To cater for \emph{attribute based access control}, we consider a set~\AT of pairs of form $k=v$, to be considered as
attribute-value pairs, where $k,v \in \AL$.  For example ``$age=30$'' or ``$institute=DERI$'' are elements
of~$\aset{T}$.  We allow shortcuts to represent intervals of integers, for example ``$age=[25,30]$'' to indicate that
all entities with attribute~$age$ between~$25$ and~$30$ are allowed access to the triple.

Considering an element $e \in \ACT$, $e$ and $\neg e$ are \emph{access control elements}, where~$e$ is called a
\emph{positive} element and~$\neg e$ is called a \emph{negative} element.\footnote{Here we are using $\neg e$ to
  represent strong negation.  In our access control domain representation, $\neg e$ indicates that $e$ will be
  specifically \emph{denied} access.}
%
An \emph{access control statement}~$S$ consists of a set of access control elements and we further consider that~$S$ is
in \emph{consistent} iff for any element~$e \in \ACT$, only one among $e$ and~$\neg e$ may appear in~$S$.  This
restriction avoids \emph{conflicts}, where a statement is attempting to both \emph{grant} and \emph{deny} access to a
triple.
%
Furthermore, we can define a partial order between statements as~$S_{1} \leq S_{2} \mbox{ iff\ } S_{1} \subseteq S_{2}$
that can be used to eliminate \emph{redundant} access permissions: if a user is granted access by statement~$S_{2}$, he
will also be granted access by statement~$S_{1}$ (and thus $S_{2}$ can be removed). 
%
Finally, an \emph{\ac{ACL}} consists of a set of access control statements and an \ac{ACL} is considered
\emph{consistent} iff each statement it contains is consistent and not redundant.  In our domain representation, only
consistent \acp{ACL} are considered as annotation values.
%
Intuitively, an annotation value specifies which entities have read permission to the triple, or are denied access when
the annotation is preceded by~$\neg$.
%
\begin{example}[Access Control List]
  %
  \label{ex:ann-value}
  We are considering the following set of entities~$\AC = \{ jb,js,st,it \}$, where $jb$ and $js$ are employee usernames
  and~$st$ and~$it$ are shorthand for~$\mathit{softwareTester}$ and~$\mathit{informationTechnology}$, respectively.
  % 
  The following annotated triple:
  $$\fuzzyg{\tau}{[[it],[st, \neg js]]}$$
  % 
  states that the entities identified with $it$ or $st$ (except if the $js$ credential is also present) have read access
  to the triple $\tau$.
\end{example}
%
An \ac{ACL}~$A$ can be considered as a non-recursive Datalog with negation (\nrdn) program, where access control
statement~$s \in A$ corresponds to the body of a rule in the Datalog program.
%
The head of the Datalog rules is a reserved literal $access \not\in\ACT$ and the evaluation of the Datalog program
determines the access permission to a triple for a specific set of credentials.

The set of user credentials is assumed to be provided by an external authentication service and consists of elements
of~\ACT, which equate to a non-empty \ac{ACL} representing the entities associated with the user.  We further assume
that this \ac{ACL} consists of only one positive statement, \ie~the \ac{ACL} will contain only one statement with all
the entities associated with the user and does not contain any negative elements.
%
\vspace{-10pt}
%
\begin{example}[Datalog Representation of an ACL]
  %
  \label{ex:datalog-program}
  Consider the annotation example presented in \cref{ex:ann-value}. The \nrdn program corresponding to the
  \ac{ACL}~$[[it],[st, \neg js]]$ is:
  %
  \begin{align*}
    access \leftarrow &~it .\\
    access \leftarrow &~st, \neg js .  
  \end{align*}
  % 
  The set of credentials of the user \emph{session}, provided by the external authentication system eg.~$[[jb,it]]$, are
  considered the facts in the \nrdn program.
\end{example}
%
Further domain specific information, for example hierarchies between the access control entities, can be encoded as
extra rules within the \nrdn program.  These extra rules can be used to provide \emph{implicit} credentials to a user,
allowing the access control to be specified based on credentials that the authentication system does not necessarily
assign to a user.
%
\begin{example}[Credential Hierarchies]
  Considering that the entity~$emp$ represents all the employees within a specific company, and that~$jb$ and~$js$
  correspond to employee usernames (as presented in \cref{ex:ann-value}), the following rules can be added to the
  \nrdn program from \cref{ex:datalog-program}:
  %
  \begin{align*}
    emp \leftarrow &~ js .\\
    emp \leftarrow &~ jb .  
  \end{align*}
  % 
  These rules ensure that both~$jb$ and~$js$ are given access when the credential~$emp$ is required in an annotation
  value.
\end{example}


\subsection{Annotation Domain}
\label{sec:access-contr-annot}
%
We now turn to the annotation domain operations~$\otimes$ and~$\oplus$ that, as presented in \cref{sec:annotated-rdfs},
allow for the combination of annotation values when performing \ac{RDFS} inference.
%
A naive implementation of these domain operations may produce \acp{ACL} that are not consistent (and would not be
considered valid annotation values).  To avoid such invalid \acp{ACL}, we rely on a normalisation step that ensures the
result is a valid annotation value by checking for redundant statements and applying a conflict resolution policy
(described below) if necessary.
%
\begin{definition}[Normalise]
  Let~$A$ be an \ac{ACL}. We define the reduction of~$A$ into its consistent form, denoted~$\fc{norm}{A}$, as:
  \[
  \fc{norm}{A} = \set{ \fc{normalise}{s_{i}} \mid s_{i} \in A \mbox{ and } {\not{\exists}} s_{j} \in A, i \neq j \mbox{ such that } s_{i} \leq
  s_{j}}
  \]
  %
  where the normalisation of a statement~$s$, denoted~$\fc{normalise}{s}$ consists of applying the conflict resolution policy
  described below.
\end{definition}
%
We say that an access statement contains a \emph{conflict} if it contains a positive and negative access control element
of the same entity, \eg~$[jb, \neg jb]$.
%
There are different ways to resolve conflicts in the annotation statements: apply a
\begin{inparaenum}[(i)]
\item \emph{brave conflict resolution} (allow access); or
\item \emph{safe conflict resolution} (deny access).
\end{inparaenum}
%
This is achieved during the normalisation step, represented by the $normalise$ function, by removing the appropriate
element:~$\neg jb$ for brave or $jb$ for safe conflict resolution.  In our current modelling, we are assuming safe
conflict resolution.


The~$\oplus$ operation for the access control domain consists of the union of the annotations and then performing the
normalisation operation.  The intuitive behaviour is that of creating a new \nrdn program that consists of the union of
the rules of the programs of both original annotations. Formally,
\[
A_{1} \oplus_{ac} A_{2} = \fc{norm}{A_{1} \cup A_{2} } \enspace .
\]
%
In turn, the~$\otimes$ operation consists of merging the rules belonging to both annotation programs and then performing
the normalisation and conflict resolution.
%
This corresponds to further restricting the statements from both annotations to only those entities that are provided access
by both annotations. Formally, the~$\otimes$ operations corresponds to:
\[
A_{1} \otimes_{ac} A_{2} = \fc{norm}{\set{ s_{1} \cup s_{2} \mid s_{1} \in A_{1} \mbox{ and } s_{2} \in A_{2}}}
\enspace,
\]
%
where~$s_1 \cup s_2$ represents the set theoretical union.
%
Unlike the~$\oplus_{ac}$ operation, the~$\otimes_{ac}$ may produce conflicts in the annotation statements.

\begin{example}[Domain Operations]
  Consider the annotations~$A_{1} = [[jb],[js],[\neg it]]$ and~$A_{2} = [[it]]$.  
  % 
  The~$\otimes$ operation is used when inferring new triples, and thus the resulting annotation should provide access to
  the resulting triple only to entities that are allowed to access all the premisses:
  % 
  $$A_{1} \otimes_{ac} A_{2} = [[jb,it],[js,it], [\neg it]] \enspace . $$
  %   
  Please note that the aforementioned conflict resolution mechanism has simplified~$[\neg it,it]$ into~$[\neg it]$.
  %
  On the other hand the~$\oplus$ operation is used to combine annotations when the same triple is deduced from different
  inference steps.  Thus, combining annotations with the~$\oplus$ operations should result in providing access to all
  the entities with are allowed to access the premises:
  % 
  \[
  A_{1} \oplus_{ac} A_{2} = [[jb],[js],[\neg it],[it]] \enspace . 
  \]
  % 
\end{example}
%
Lastly, the smallest and largest annotation value in the access control domain~$\bot_{ac}$ and~$\top_{ac}$, respectively
correspond to an empty \nrdn program and another that provides access to all entities~$e\in\ACT$:~$\bot_{ac} = []$
and~$\top_{ac} = \set{ [e], [\neg e] \mid e \in \ACT }$.
%
The~$\bot_{ac}$ annotation value element indicates that the annotated triple is not accessible to any entity, since no
annotation statements will provide access to the triple, and an annotation value of~$\top_{ac}$ states that the triple
is considered \emph{public}, since any credential contained in the user session will provide access to the triple.
%
\begin{definition}[Access Control Annotation Domain]
  Let~$\aset{F}$ be the set of annotation values over~\ACT, \ie~consistent \acp{ACL}. The access control annotation
  domain is formally defined as:
  %
  $$D_{ac} = \tuple{\aset{F}, \oplus_{ac}, \otimes_{ac}, \bot_{ac}, \top_{ac}} \enspace .$$
  %
\end{definition}
%
For our access control domain model, the~$\bot_{ac}$ is considered the \emph{default} annotation for any non-annotated
triple, which implicitly denies access to the triple.



This modelling of the access control domain can be extended to consider other permissions, like $\mathit{update}$, and
$\mathit{delete}$ simply by extending the annotation to an $n$-tuple of propositional formul\ae\footnote{One formula,
  two formul\ae.}~$\tuple{P,Q,\dots}$, where $P$ specifies the formula for $\mathit{read}$ permission, $Q$ for
$\mathit{update}$ permission, etc.  This extension allows to use the defined domain operations simply extended to
operate over the corresponding components of the tuple.
%
A $\mathit{create}$ permission has a different behaviour as it would not be attached to any specific triple but rather
as a graph-wide permission and thus is not considered in this modelling.
%
In this chapter, we are focusing only on $\mathit{read}$ permissions in the description of the domain and thus restrict
the modelling to a single propositional formula.
%
It is worth noting that the support for $\mathit{create}$ and $\mathit{update}$ of \ac{RDF} is only included in the
forthcoming W3C SPARQL 1.1 Recommendation~\cite{HarrisSeaborne:2012aa}.


%%% Local Variables:
%%% mode: latex
%%% mode: flyspell
%%% mode: reftex
%%% mode: visual-line
%%% TeX-master: "../thesis"
%%% End:
