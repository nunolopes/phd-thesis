\subsection{Combining XSPARQL and AnQL}
\label{sec:comb-xsparql-anql}


Next we present the combination of the XSPARQL language, as presented in \cref{cha:xsparql}, with the AnQL query
language described in \cref{cha:anql}.
%
This combination caters for the creation and querying of Annotated RDF graphs using the XSPARQL language.  For the
purposes of this thesis, namely the data integration \usecase, we are mostly interested in creating the Annotated RDF
data.

In XSPARQL we extend the syntax of \SparqlForClause{s} and \ConstructClause{s} to cater for the fourth element (as
presented in \cref{cha:anql}), thus allowing us to create and query the Annotated RDF graphs, respectively.
%
Considering this extended expression syntax, the semantics of \SparqlForClause{s}, presented in \cref{sec:semantics},
can be changed to follow the AnQL semantics instead of the SPARQL semantics.  Conversely, in the XSPARQL implementation
(described in \cref{sec:implementation}) we can replace the ARQ SPARQL engine with our own AnQL prototype
implementation~(\cf~\cref{sec:implementation-notes}).
%

For the creation of \ac{RDF} graphs, the current implementation of the XSPARQL language (described in
\cref{sec:constructclause}) relies on creating a string representation of the \ac{RDF} graph in Turtle notation.
%
For the creation of Annotated RDF graphs, we similarly extend this string representation to cater for an Annotated
\ac{RDF} graph according to the N-Quads representation~\cite{CyganiakHarthHogan:2009aa}.
 
Following the N-Quads specification, we represent the annotation value as an \ac{RDF} literal, which also allows us to
implement the extension of XSPARQL independently of the annotation domain.
%
We also introduce a new generic type, which we call \type{AnnotationLiteral}, which will be the type of any annotation
values and variables.  
%
In XSPARQL we follow the restriction that annotation variables and non-annotation variables should be distinct in the
query and this newly introduced type ensures that we can enforce this restriction in nested XSPARQL queries.
% 
Similar to AnQL, we assume the sharing of variables is possible only by using domain specific functions that handle
the appropriate type conversions.

The combination of XSPARQL and Annotated \ac{RDFS} also introduces inferencing capabilities into XSPARQL by reusing the
annotated inference rules presented in \cref{sec:anql-deductive-system}.  Notably the classical domain
(\cf~\cref{sec:crisp}) caters for the classical \ac{RDFS} inferences.
%
A proper formalisation of this combination is beyond the scope of this thesis, however a possible starting point is the
SPARQL~1.1 Entailment Regimes specification~\cite{GlimmOgbuji:2012aa}, which introduces other entailment regimes (beyond
simple \ac{RDF} entailment) into the upcoming SPARQL~1.1 specification.
%




%%% Local Variables:
%%% mode: latex
%%% TeX-master: "../thesis"
%%% End:
