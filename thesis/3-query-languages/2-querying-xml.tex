
\section{Querying XML}
\label{sec:prelim-xquery}

As for querying \ac{XML} data, there are different alternatives, most notably \ac{XSLT} and XQuery.
%
Although these languages are very similar and both tackle similar problems, their fundamental difference is that XSLT
was designed to perform transformations between different \ac{XML} formats, mostly considered for styling and display of
information on the Web, while XQuery is focused on querying parts of an \ac{XML} document or
tree~\cite{KatzChamberlinKay:2003aa}.  Notably, XSLT uses an \ac{XML} syntax for specifying the transformations while
XQuery provides a non-\ac{XML} syntax that aims to be familiar for SQL users.
%
Both of these languages rely on a common core, the \ac{XPath}, that allows nodes of an \ac{XML} document to be selected.
Although \ac{XPath} was being designed at the same time as the XSLT language, it was published as a separate standard by
the \ac{W3C}, who envisioned its use in other languages or even as a single standalone language.
%

In this section, we start by explaining the \ac{XPath} language, followed by an high-level overview of the XSLT language.
Finally we present the XQuery language, that is used as a basis for the syntax and semantics of XSPARQL.



\subsection{XPath}
\label{sec:xpath}

\ac{XPath}~\cite{BerglundBoagChamberlin:2010aa} consists of a common core language that is reused by both \ac{XSLT} and
XQuery.  The main purpose of \ac{XPath} is to access specific nodes of an \ac{XML} document, which it does by providing
a non-\ac{XML} syntax for navigating though the structure of an \ac{XML} document and selecting the relevant nodes.
%
In \ac{XPath}, an \emph{expression} is the basic construct of the language and may consist, among other elements, of
variable references, function calls, or location paths.
%
Such expressions are evaluated with respect to an \emph{expression context}, which contains the necessary information to
determine the output of an expression, most notably the \emph{context node}: the \ac{XML} element over which the
expression will be evaluated.  The evaluation of an \ac{XPath} expression results in an \emph{object} that can consist
of a node-set, boolean, number, or a string.
%

\paragraph*{Location paths.} Location paths are an especially important form of \ac{XPath} expressions since they
specify how to navigate through the \ac{XML} document.  A location path consists of a sequence of \emph{location steps}
that are separated by the \character{/} character.
%
The result of the evaluation of each location step is the set of nodes and each step in the location path is applied to
the set of nodes resulting from the previous step.
% 
A location step can be composed of three parts: an \emph{axis}, a \emph{node test}, and any number of \emph{predicates},
which are described next.
%
\begin{description}
\item[Axis:] The axis is used to select nodes by specifying the relation that the selected nodes should have from the
  context node.
  % 
  Some of the available axes allow the \texttt{child}, \texttt{parent}, \texttt{following-sibling}, or
  \texttt{preceding-sibling} of the context node to be selected.  These axes correspond to selecting, relative to the
  context node, all the nodes that are one level below (\texttt{child}), one level up (\texttt{parent}), at the same
  level after (\texttt{following-sibling}) or before (\texttt{preceding-sibling}) the context node.
  % 
  Furthermore, the \texttt{attribute} axis can be used to select all the attributes of the context node,
  \stringValue{self} refers to the context node and \stringValue{descendant-or-self} to the context node and its
  descendants.
  % 

\item[Node Tests:] Node tests can be combined with axes to further restrict the selected set of nodes.  For example
  \stringValue{child::band} will select all the children of the context node with \stringValue{band} name.  Similarly,
  \character{*} can be used to select all elements: \stringValue{attribute::*} selects all the attributes of the context
  node.
  % 
  
\item[Predicates:] Predicates can be used to further filter a node-set and produce a new node-set: given a node-set, the predicate
  expression is evaluated by using each node in the node set as the new context node.  If the evaluation of the predicate
  yields true, then the node is included in the newly created node set.
  % 
\end{description}

The \ac{XPath} specification also defines an abbreviated syntax for representing location steps, where the expression
can be specified by either assuming a default axis or by specifying shortcuts.  For example, the \stringValue{child} axis
is the default and can thus be omitted.
%
The \character{@} abbreviation can be used for selecting the \emph{attribute} axis, thus, \stringValue{@name="Nightwish"}
is short for \stringValue{attribute::name="Nightwish"}.
%
Other available abbreviations are \character{//} for \stringValue{/descendant-or-self::node()/}, \character{.} for
\stringValue{self::node()} and \character{..} for \stringValue{parent::node()}.  
%


\begin{example}[XPath expression]
  The following XPath expression:

  \lstinline{//band[@name="Nightwish"]/members}
  
  \noindent which is an abbreviated form of the expression:

  \lstinline{/descendant-or-self::node()/child::band[attribute::name="Nightwish"]/child::members}
  %
  
  \noindent when executed over the \ac{XML} document presented in \cref{fig:bands-xml}, returns the
  \stringValue{band} \ac{XML} element whose value of the \stringValue{name} attribute is \stringValue{Nightwish}.
  %
\end{example}
%
Further expressions available in the XPath language include \FOR expressions that, in a somewhat similar fashion to
imperative programming languages,\footnote{We say somewhat similar since the evaluation order of the return expression
  is not imposed by XPath.} allow to repeat an expression (called \emph{return expression}) for different values of the
\emph{range variable} (detailed in \cref{sec:xquery}), conditional expressions and quantified expressions.
%
Conditional expressions allow to execute different expressions (\emph{then expression} or \emph{else expression})
depending on the result of the \emph{if expression} and quantified expressions can be used to test whether an expression
is true for all (\keyword{every}) or at least one (\keyword{some}) members of a sequence.


\subsection{XSLT}
\label{sec:xslt}

\acf{XSLT}~\cite{Kay:2007aa} is a transformation language for \ac{XML} that allows to manipulate \ac{XML}
documents by matching subsets of the \ac{XML} structure and specifying transformation rules for the matched elements.  
%
The syntax of \ac{XSLT} is \ac{XML}-based and defines a set of \ac{XML} elements that are interpreted as \ac{XSLT}
instructions, distinguished by using the namespace \stringValue{\url{http://www.w3.org/1999/XSL/Transform}}, commonly
abbreviated by the \qname{xsl}{} prefix.
%
As previously noted, \ac{XSLT} relies on the \ac{XPath} language to navigate and access the elements in the
\ac{XML} document.
%

\ac{XSLT} transformations are called \emph{stylesheets}, referring to the origins of the language, mostly used for
defining the style of an \ac{XML} or XHTML document for presentation in a web browser.
%
\ac{XSLT} follows the functional programming paradigm, where the stylesheet defines a set of rules that produce the output
tree as a function of the input tree.  
%
Such rules, called \emph{template rules}, are applied to the source or input tree in order to produce the result (or
output) tree.  The part of the rule that is matched against the \ac{XML} elements in the source is called the
\emph{pattern}, while the \emph{sequence constructor} part is instantiated by elements matched from the source tree in
order to produce the result tree.
%
Recalling the logic programming approach for querying relational databases (\cref{sec:querying-rdb}), the pattern
can be considered the body of the query, while the sequence constructor can be considered the head.
%


Template rules are defined using \ac{XML} elements named \qname{xsl}{template} (as presented in
\cref{ex:xslt-templates}).  
%
The \stringValue{match} attribute is used to specify the \ac{XML} elements to which the template will be applied, while
the body of the template defines the output.  The recursive application of template rules is selected by the
\stringValue{\qname{xsl}{apply{\mbox{-}}templates}} element, possibly specifying which \ac{XML} elements from the input
should be matched by providing an \ac{XPath} expression as the value of the \stringValue{select} attribute.  Whenever
this attribute is omitted, the default is to select all children of the context node.
%
\begin{example}[\acs{XSLT} template rules]
  %
  \label{ex:xslt-templates}
  %
  The following template rule selects the band element whose value for the name attribute is
  \stringValue{Nightwish}:
  % 
  \lstinputlisting[frame=none,numbers=none,firstline=7,lastline=9]{0-data+queries/members-therion.xsl}
  %
  \noindent While the next template rule simply outputs all members of the selected bands:
  \lstinputlisting[frame=none,numbers=none,firstline=3,lastline=5]{0-data+queries/members-therion.xsl}
  %
  \noindent Combining these two template rules in an \ac{XSLT} stylesheet will make the stylesheet output the names of
  members of the \stringValue{Nightwish} band, when applied to \cref{fig:bands-xml}.
  %
\end{example}
%
Similar to \ac{XPath}, an \ac{XSLT} stylesheet is evaluated with regards to an \emph{expression context} and relies on
the \ac{XPath} specification for defining the contents of each expression context.
%
Each template rule is evaluated by specifying the matched input \ac{XML} element as the context node.

The \ac{XSLT} specification further defines instructions for specifying repetition (\texttt{xsl:for-each}), conditional
processing (\texttt{xsl:if} and \texttt{xsl:choose}), variable declaration (\texttt{xsl:variable}), and function
declaration (\texttt{xsl:function}).  
%





\subsection{XQuery}
\label{sec:xquery}

XQuery~\cite{ChamberlinRobieBoag:2010aa} has been the \ac{W3C} recommended query language for \ac{XML} since early 2007.
%
There are several similarities between XQuery and \ac{XSLT} and both query languages can address the same \usecases.
Some of the most evident similarities include
\begin{enumerate*}[nosep,label=(\roman*), before=\unskip{: }, after=\unskip{.}, itemjoin={{; }}, itemjoin*={{; and }}]
\item declarative semantics supporting single-assignment variables
\item the use of \ac{XPath} for selecting input \ac{XML} elements
\item the construction of new \ac{XML} elements explicitly and at runtime
\item support for user-defined functions
\end{enumerate*}



As we have seen, \ac{XSLT} was designed as a transformation language for \ac{XML} documents, focusing on transformations
that facilitate displaying data for the user.  On the other hand, XQuery behaves more like a query language, aiming at
extracting data from collections or large individual \ac{XML} documents.
% 
These design choices are apparent even in the syntax of the languages, where XQuery follows a non-\ac{XML} syntax.
% 
However, XQuery reuses other \ac{XML}-based specifications such as the \ac{XDM} data model and \ac{XSD}.
%
Any input document for an XQuery query, commonly specified using the \funcName{\qname{fn}{doc}} function, is translated
into an \ac{XDM} instance and the respective query is executed over this abstract structure.
%

As in \ac{XPath} and \ac{XSLT}, also in XQuery the basic construct of a query is called an \emph{expression}.  In
XQuery, expressions are mostly composed of \FLWOR expressions.
%
This name stems from the available expressions: \FOR, \LET, \WHERE, \ORDERBY, and \RETURN.
%
\begin{definition}[Tuple Stream]
  In a \FLWOR expression, the result of the evaluation of ``\FOR~\var{v}'' and ``\LET~\var{v}'' clauses
  consists of an ordered sequence of elements that~\var{v} is bound to.  Following the XQuery specification, we refer to
  this sequence as \emph{tuple stream}.
\end{definition}
%
Optionally, this produced sequence can be filtered using the \WHERE clause or ordered with the \ORDERBY clause.
%
Finally, the \RETURN clause is evaluated for every element of the resulting sequence and each result is included in the
sequence produced by the \FLWOR expression.
%
Any XQuery variable is represented by using an expanded QName also making it possible to disambiguate variables based on
declared prefixes. Further details are available in~\citet[Section 3.1.1.1]{DraperFankhauserFernandez:2010aa}.
%
Another important feature is that \FOR clauses may optionally include a positional variable in the form of ``\FOR
\var{Var} \AT \var{PosVar}''.  In this case, for each evaluation of the \RETURN expression, \var{PosVar} is assigned an
integer corresponding to the position of \var{Var} in the tuple stream.
%


\begin{definition}[Expression Context]
  Similar to \ac{XPath} and \ac{XSLT}, any XQuery expression~$E$ is evaluated with regards to an \emph{Expression
    Context} that holds the static environment (\stat) and the dynamic environment (\dyn) up until the evaluation
  of~$E$.
\end{definition}
%
Environments include different components that hold the necessary information for the evaluation of any XQuery
expression: \stat{} holds the information available during static analysis, for example the \ecomp{varType} component
holds variable type information.
%
The \dyn{} environment contains information available during expression evaluation, like the value for variables, stored
in the \ecomp{varValue} component.
%
Given an expression context~$C$, we refer to the static environment of~$C$ as~$\funcCall{\stat}{C}$ and to the dynamic
environment as~$\funcCall{\dyn}{C}$.  Different components can be accessed via their
name:~$\funcCall{\stat}{C}.\ecomp{varType}$ and the specific value of the environment element~$var$ can be accessed
using~$\funcCall{\stat}{C}.\ecomp{varType}(var)$.  If the expression context~$C$ is not explicitly presented, \stat and
\dyn can be used in place of \funcCall{\stat}{C} and \funcCall{\dyn}{C}.



\begin{example}[XQuery query]
  %
  \label{ex:xquery-query}
  %
  The slightly verbose XQuery equivalent to the XSLT presented in \cref{ex:xslt-templates} is the following
  query:
  %
  \lstinputlisting[frame=none,numbers=none]{0-data+queries/members-therion.xquery}
  % 
  Executing this query over the \ac{XML} document presented in \cref{fig:bands-xml}, again returns the members of the
  \stringValue{band} \ac{XML} element whose value of the \stringValue{name} attribute is \stringValue{Nightwish}.
\end{example}
%
XQuery allows to write arbitrary queries, it is actually a Turing complete language~\cite{Kepser:2004aa}, for instance
the \RETURN part of a \FLWOR expression may contain other (nested) \FLWOR expressions.  
%
In such cases we commonly refer to the first \FLWOR expression as the \emph{outer} query, while the \FLWOR expression
that is containted inside the \RETURN is refered to as the \emph{inner} query.


\subsubsection*{Semantics}
%
The semantics of XQuery~\cite{DraperFankhauserFernandez:2010aa} is defined in terms of
%
\begin{inparaenum}[(i)]
\item \emph{normalisation rules},
\item \emph{static typing rules}, and
\item \emph{dynamic evaluation rules}.
\end{inparaenum}
%
Normalisation rules reduce the syntax of XQuery to an abstract syntax denoted XQuery Core: a subset of XQuery that,
while semantically equivalent, aims to be easier to define, implement and optimise~\cite{KatzChamberlinKay:2003aa}.
%
Static typing rules are applied over the XQuery Core language and are used to assign a type to each XQuery expression.
The dynamic evaluation rules are responsible for producing the results of each expression while guaranteeing that its
input is consistent with the previously determined typing information.

In this thesis we will use the term \emph{bound variable} to refer to a variable that has been previously declared in an
query, for example, \var{v} is considered bound if it has been previously declared by a \stringValue{\FOR \var{v}} or
\stringValue{\LET \var{v}} expression.


The complete semantics of XQuery is defined by specifying \emph{normalisation}, \emph{static} and \emph{dynamic
  evaluation} rules for each expression of the language and, as an example, we next present the rules of the XQuery \FOR
expression.

\paragraph*{Normalisation Rules.}
\emph{Normalisation rules} are represented using mapping rules, where \sem{\cdot}{Expr} represents the XQuery expression
to be matched, while the resulting XQuery Core expression is included after the~$==$ separator.
%
Furthermore, fixed-width font (like \FOR and \IN) refer to specific keywords, and \emph{italic} font refers to
productions in the XQuery Core grammar~\cite[Appendix~A]{DraperFankhauserFernandez:2010aa}.
%
The following example shows the application of the normalisation rules over consecutive \ForClause{s} -- considered a
shorthand syntax -- into nested \ForClause{s} in XQuery Core:
%
\begin{normalisationrule}
  \begin{prooftree}
    \alwaysNoLine \AxiomC{$\sem{\begin{array}{l}
          \FOR~\envElem{\var{VarName}}{1}~\envElem{OptTypeDeclaration}{1}~\envElem{OptPositionalVar}{1}~\IN~\envElem{Expr}{1} \\
          \quad,\cdots, \\
          \quad\var{VarName}_n~\envElem{OptTypeDeclaration}{n}~\envElem{OptPositionalVar}{n}~\IN~\envElem{Expr}{n}\\
          \quad\grammarRule{ReturnClause}
        \end{array}
      }{Expr}$}
    \UnaryInfC{$==$}
    \UnaryInfC{$\begin{array}{l}
        \FOR~\envElem{\var{VarName}}{1}~\envElem{OptTypeDeclaration}{1}~\envElem{OptPositionalVar}{1}~\IN~\sem{\envElem{Expr}{1}}{Expr}~\RETURN \\
        \quad\cdots \\
        \quad\FOR~\envElem{\var{VarName}}{n}~\envElem{OptTypeDeclaration}{n}~\envElem{OptPositionalVar}{n}~\IN~\sem{\envElem{Expr}{n}}{Expr}\\
        \quad\sem{\grammarRule{ReturnClause}}{Expr}
      \end{array}$}
    \label{for-normalisation}
  \end{prooftree}%
\end{normalisationrule}%
%
The normalisation process consists of the recursive application of the defined rules over each expression in the
language.
%

\paragraph*{Static and Dynamic Evaluation Rules.}
On the other hand, \emph{static type rules} and \emph{dynamic evaluation rules} are represented using inference rules
of the form:
%
\begin{prooftree}
  \AxiomC{$\textit{premise}_{1}~\dotsb~\textit{premise}_{n}$}
  \UnaryInfC{$\textit{conclusion}$}
\end{prooftree} 
%
Rule premises are composed of the so-called \emph{judgements} and such judgements are said to \emph{hold} if they are
considered true.  Some judgments used in this thesis are:
%
\begin{description}
\setlength\abovedisplayshortskip{1pt plus 2pt minus 0pt}
\setlength\belowdisplayshortskip{1pt plus 2pt minus 0pt}
\setlength\abovedisplayskip{1pt plus 2pt minus 0pt}
\setlength\belowdisplayskip{1pt plus 2pt minus 0pt}
\item[Type.] The judgment:
  %
  \[\statEnv{\envElem{Expr}{} \colon \envElem{Type}{}}\] 
  % 
  holds if, in the static environment~\stat, the expression \envElem{Expr}{} has the type \envElem{Type}{}.
  % 
  Also related to typing, the \funcName{prime} and \funcName{quantifier} are functions that extract all the item types
  of its parameter and try to estimate the number of items in a type (?, +, or *), respectively.
%
\item[Variable Expansion.] Similarly, \[\statEnv \envElem{VarName}{}~\textbf{of var expands to}~\envElem{Variable}{}\]
  holds if \envElem{Variable}{} corresponds to the expanded QName of \envElem{VarName}{}.
%
\item[Context Extension.] Contexts can be extended by using the~\character{$+$} notation, for
  example: \[\stat\envExtend{varType}{\envElem{Variable}{pos} \Rightarrow \qname{xs}{integer}}\] creates a new context
  based on~\stat by adding the information that \envElem{Variable}{pos} is of type \qname{xs}{integer} to the
  \ecomp{varType} component of~\stat.
%
\item[Expression Evaluation.] A commonly used judgment in dynamic evaluation rules is
  \[\dynEnv{\envElem{Expr}{} \Rightarrow \envElem{Value}{}}\] which holds if, in the environment~\dyn, the expression
  \envElem{Expr}{} evaluates to the value \envElem{Value}{}.
%
\end{description}
%
As an example of the use of these judgments, the following static type rule handles the typing of a \FOR clause with a
positional variable:
%
\begin{staticrule}
  \begin{prooftree}
    \def\ScoreOverhang{1pt}%
    \def\extraVskip{1pt}%
    \alwaysNoLine%
    %
    \AxiomC{$\statEnv \envElem{Expr}{1} \colon \envElem{Type}{1}$}
    %
    \UnaryInfC{$\statEnv \envElem{VarName}{}~\textbf{of var expands to}~\envElem{Variable}{}$}
    %
    \UnaryInfC{$\statEnv \envElem{VarName}{pos}~\textbf{of var expands to}~\envElem{Variable}{pos}$}
    %
    \UnaryInfC{$\stat\envExtend{varType}{\begin{array}{l} 
          \envElem{Variable}{} \Rightarrow \funcCall{prime}{\envElem{Type}{1}}; \\
          \envElem{Variable}{pos} \Rightarrow \qname{xs}{integer} 
        \end{array}}~\proofs  \envElem{Expr}{2} \colon \envElem{Type}{2}
      $}
    %
    \singleLine
    %
    \UnaryInfC{$\statEnv{
        \begin{array}{l}
          \FOR~\envElem{\var{VarName}}{}~\keyword{at}~\envElem{VarName}{pos}~\keyword{in}~\envElem{\grammarRule{Expr}}{1}\\
          \keyword{return}~\envElem{\grammarRule{Expr}}{2} \colon \envElem{Type}{2} \cdot \funcCall{quantifier}{\envElem{Type}{1}}
        \end{array}
      }$}
  \end{prooftree}%
  \label{eq:static-type-for}
\end{staticrule}%
%


The dynamic evaluation of \FOR expressions consists of first evaluating the expression specified by the \IN clause and,
for each element of the resulting sequence, assign it to the \FOR variable and then evaluating the \RETURN expression.
%
Hence, the semantics separates the dynamic evaluation rules of \FOR expressions into two cases, depending on whether the
\IN expression returns any elements.  If the \IN expression evaluates to an empty sequence, the \FOR expression also
evaluates to an empty sequence:
%
\begin{dynamicrule}
    %
    \AxiomC{$\dynEnv{\envElem{Expr}{1} \Rightarrow \left(\right)}$}
    %
    \singleLine \UnaryInfC{$\dynEnv{
        \FOR~\envElem{\var{VarName}}{}~\grammarRule{OptPositionalVar}~\keyword{in}~\envElem{\grammarRule{Expr}}{1}~\keyword{return}~\envElem{\grammarRule{Expr}}{2}
        \Rightarrow \left(\right) }$}
  \label{eq:dyn-ForClause-empty}
\end{dynamicrule}%
%
Otherwise, the dynamic evaluation rule of a \FOR clause with a positional variable is presented next:
%
\begin{dynamicrule}
    %
    \AxiomC{$\dynEnv{\envElem{Expr}{1} \Rightarrow \envElem{Item}{1}, \dotsb, \envElem{Item}{n}}$}
    %
    \UnaryInfC{$\statEnv \envElem{VarName}{}~\textbf{of var expands to}~\envElem{Variable}{}$}
    %
    \UnaryInfC{$\statEnv \envElem{VarName}{pos}~\textbf{of var expands to}~\envElem{Variable}{pos}$}
    %
    \UnaryInfC{$\dyn\envExtend{varValue}{\begin{array}{l}
            \envElem{Variable}{} \Rightarrow \envElem{Item}{1};\\
            \envElem{Variable}{pos} \Rightarrow 1 
          \end{array}}~\proofs \envElem{Expr}{2} \Rightarrow \envElem{Value}{1}$}
    \UnaryInfC{$\vdots$}
    \UnaryInfC{$\dyn\envExtend{varValue}{\begin{array}{l}
            \envElem{Variable}{} \Rightarrow \envElem{Item}{n};\\
            \envElem{Variable}{pos} \Rightarrow n 
          \end{array}}~\proofs \envElem{Expr}{2} \Rightarrow \envElem{Value}{n}$}
    \singleLine
    \UnaryInfC{$\dynEnv{
        \begin{array}{l}
          \FOR~\envElem{\var{VarName}}{}~\keyword{at}~\envElem{VarName}{pos}~\keyword{in}~\envElem{\grammarRule{Expr}}{1}\\
          \keyword{return}~\envElem{\grammarRule{Expr}}{2} \Rightarrow \envElem{Value}{1}, \dotsb, \envElem{Value}{n}
        \end{array}
      }$}
  \label{eq:dyn-ForClause}
\end{dynamicrule}%
%
Another judgement used for matching element values to types
is~\stringValue{\small{$\statEnv{\envElem{Value}{}~\textbf{matches}~\envElem{Type}{}}$}}, which holds when, in the
static environment~\stat, the type of \envElem{Value}{} is~\envElem{Type}{} or can be derived from~\envElem{Type}{} (as
presented in \cref{sec:xml-validation}).
%



%%% Local Variables:
%%% fill-column: 120
%%% TeX-PDF-mode: t
%%% TeX-debug-bad-boxes: t
%%% TeX-parse-self: t
%%% TeX-auto-save: t
%%% reftex-plug-into-AUCTeX: t
%%% mode: latex
%%% mode: flyspell
%%% mode: reftex
%%% TeX-master: "../thesis"
%%% End:
