


\section{Querying RDF with SPARQL}
\label{sec:sparql-preliminaries}


This section provides an overview of the SPARQL query language, which is the \ac{W3C} recommended query language for
\ac{RDF}.
%
We present the syntax and semantics of SPARQL and wrap-up with an overview of the new features introduced by the
forthcoming update to the SPARQL language, dubbed SPARQL~1.1.
%
The \ac{W3C} SPARQL specification consists of the following documents:
\begin{enumerate}[(i),noitemsep]
\item a \emph{query language} for \ac{RDF}~\cite{PrudhommeauxSeaborne:2008aa};
\item a protocol describing the interactions between a \emph{query engine} and \emph{query
    clients}~\cite{ClarkFeigenbaumTorres:2008aa}; and
\item the \ac{XML} serialisation of the results of a \SELECT and \ASK query~\cite{BeckettBroekstra:2008aa}. 
\end{enumerate}
% 
We will focus on the description of the SPARQL query language for \ac{RDF} by following the \ac{W3C} specification and the
semantics presented by~\citet{PerezArenasGutierrez:2009aa}.
%

\subsection*{Syntax}
\label{sec:sparql-syntax}

A \emph{SPARQL query} is defined by a triple~$Q=(P,G,V)$, where~$P$ is a \emph{graph pattern},~$G$ is an \ac{RDF}
\emph{dataset} and~$V$ is the \emph{result form}.
%
Considering a setting similar to rule-based query answering for relational databases, a SPARQL query can also be viewed
as:
%
$ V \leftarrow P$,
%
where~$V$ can be assumed as the \emph{head} of the query, while~$P$ is the
\emph{body}~\cite{PerezArenasGutierrez:2009aa}.
%
The next sections describe each component of SPARQL queries, namely \ac{RDF} datasets, the result form, and graph
patterns.


\paragraph*{RDF Dataset.}% 
%
An \ac{RDF} \emph{dataset} forms the input data provided to a SPARQL query and is composed of:
%
\begin{inparaenum}[(i)]
\item exactly one (unnamed) graph considered to be the \emph{default} graph; and
\item a set of named graphs of the form~$\tuple{n_{i},g_{i}}$, where~$n_{i}$ is a \ac{URI} corresponding to the
  \emph{name} of the graph and~$g_{i}$ is an \ac{RDF} graph.
\end{inparaenum}
%
In a SPARQL query, the default graph is specified using \FROM clauses, while the named graphs are indicated using
\FROMNAMED clauses.
%
Since a SPARQL query may contain several \FROM clauses, the default graph is taken as the \emph{RDF merge} of graphs
specified in all \FROM clauses (\cf~\cref{def:rdf-merge}).

The notion of \emph{active graph} is introduced in the evaluation semantics of SPARQL to distinguish which \ac{RDF}
graph the basic graph pattern is matched against.  At the start of a SPARQL query evaluation, the active graph is the
default graph and it is changed when a \GRAPH keyword is encountered in the graph pattern (as further explained below).
% 


\paragraph*{Result Form.}
%
The \emph{result form} specifies the output of a SPARQL query and may be one of the following four types:
%
\begin{description}[noitemsep]
\item[\SELECT:] returns the matched values (substitutions) for variables present in the query;
\item[\CONSTRUCT:] returns an \ac{RDF} graph that is created based on the specified \emph{template} and the
  substitutions obtained by executing the query;
\item[\ASK:] returns a boolean indicating if the graph pattern matches any of the data; and
\item[\DESCRIBE:] returns an \ac{RDF} graph that contains information regarding the resources contained in
  the query.
\end{description}
% 
For this thesis we focus primarily on \SELECT and \CONSTRUCT queries.\footnote{In \cref{cha:xsparql} we will refer to
  these result forms as \SparqlForClause and \ConstructClause, respectively.}
%
In the case of \SELECT queries, the result form is a set of variables and the result of the query consists of sequences
of variable bindings for these variables, determined according to the specified graph pattern.
%
In a \CONSTRUCT query, as presented in~\citet[Section 10.2]{PrudhommeauxSeaborne:2008aa}, the solutions of the graph
pattern are used to instantiate the \emph{template} provided. The result of a \CONSTRUCT query is an \ac{RDF} graph
obtained from the union of all instantiations of variables in the template that result in valid \ac{RDF} triples.
%
When a \CONSTRUCT template contains blank nodes, a different blank node label will be generated for each instantiation of
the template, \ie~blank nodes are only shared within the same solution.
%


\paragraph*{Graph Patterns.}% 
%
SPARQL is a graph-matching query language and its syntax directly reflects this.  The body (graph pattern) of a SPARQL
query consists primarily of \emph{triple patterns} that are matched against the \ac{RDF} data.  Triple patterns are
\ac{RDF} triples, possibly containing variables appearing in subject, predicate or object positions.
%
In the SPARQL syntax, a graph pattern follows the \WHERE keyword.

A simple form of graph pattern is a set of triple patterns, also called a \ac{BGP}.
%
Here, we present the syntax of SPARQL based on the definitions provided by~\citet{PerezArenasGutierrez:2009aa}, which
describes a normalised syntax based on 3-tuples:
%
\begin{definition}[Graph Patterns] 
  \label{def:graphpattern}
  Let~$\AU$,~$\AB$,~$\AL$ be defined as before. Furthermore, let~$\AV$ denote a set of variables disjoint from~$\AUBL$,
  \emph{graph patterns} are inductively defined as follows:
  % 
  \begin{itemize}[noitemsep]
  \item a tuple~$(s,p,o) \in \AULV \times \AUV \times \AULV$, called a \emph{triple pattern}, is a graph pattern;
  \item a set of triple patterns, called a \emph{\acf{BGP}}, is a graph pattern;
  \item if~$P$ and~$P'$ are graph patterns, then~$(P\ \AND\ P')$,~$(P\ \text{\OPTIONAL}\ P')$, and~$(P\ \text{\UNION}\ P')$ are graph
    patterns;
  \item if~$P$ is a graph pattern and~$i \in \AUV$, then~$(\text{\GRAPH}\ i\ P)$ is a graph pattern; and
  \item if~$P$ is a graph pattern and~$R$ is a \FILTER expression, then~$(P\ \text{\FILTER}\ R)$ is a graph pattern.
  \end{itemize}
  % 
  For any pattern~$P$, we write~$\vars{P}$ for the set of all variables occurring in~$P$.
  % 
  A \FILTER expression~$R$ can be composed from constants, elements of~$\AULV$, comparison operators~(\character{$=$},
  \character{$<$}, \character{$>$}, \character{$\leq$}, \character{$\geq$}), logical connectives~(\character{$\neg$},
  \character{$\land$}, \character{$\lor$}) and \emph{built-in} functions.
  % 
  Some of the available built-in functions include the unary functions: $\keyword{BOUND}$, $\keyword{isIRI}$,
  $\keyword{isURI}$, $\keyword{isBLANK}$, $\keyword{isLITERAL}$, $\keyword{STR}$, $\keyword{LANG}$, and
  $\keyword{DATATYPE}$.  A complete list of built-in functions is included
  in~\citet[Section~11]{PrudhommeauxSeaborne:2008aa}.
\end{definition}
%
\nd As is common practice in the definition of SPARQL queries, we do not consider blank nodes in graph patterns, and
thus do not include them in our definitions. However, this restriction does not affect the expressivity of SPARQL, since
blank nodes in query patterns can always be replaced equivalently with variables~\cite{PerezArenasGutierrez:2009aa}.
%
Although in definitions we rely on an algebraic formalism for the syntax of SPARQL, as
per~\citet{PerezArenasGutierrez:2009aa}, in the examples we follow the \ac{W3C} specification, which can be naturally
mapped to the algebraic form, where the \AND operator is represented by a dot (\character{.}).  The mapping between the
\ac{W3C} SPARQL syntax and the algebraic form we use is presented by~\citet{ArenasGutierrezPerez:2009ab}.
%
Thus, \cref{ex:sparql-query} presents a SPARQL query where the \PREFIX keyword declares a \ac{URI} prefix that is
used later in the query.
%
\begin{example}[SPARQL query] 
  \label{ex:sparql-query}
  %
  The following SPARQL query retrieves the names of persons that are members of the \stringValue{Nightwish} band:
  %
  \lstinputlisting[frame=none,numbers=none]{0-data+queries/members-nightwish.sparql}
  % 
\end{example}

\paragraph*{Solution Modifiers.}
\label{sec:solution-modifiers}

The evaluation of graph patterns generates a sequence of results initially with no specific \emph{order} (further
detailed in the following section).  Solution modifiers, such as \ORDERBY, \LIMIT, \OFFSET, and \DISTINCT can be applied
to this solution sequence.
%
The \ORDERBY modifier is used to specify an ordering for the sequence, specified as a list of variables present in the
solution sequence and the direction of the ordering (\keyword{ASC} or \keyword{DESC}).
%
Furthermore, the \DISTINCT modifier eliminates any duplicate solutions, while \LIMIT and \OFFSET are used to restrict
the number of solutions that are returned and to discard solutions from the beginning of the sequence, respectively.



\subsection*{Semantics}
\label{sec:sparql-semantics}


The semantics of SPARQL is defined based on the evaluation of \acp{BGP}, namely the \emph{matching} of the \acp{BGP}
against the supplied \ac{RDF} graph and the algebra that is built on top of this \ac{BGP} matching.
%
We start by presenting the notion of \emph{solution mappings}, which will be the results of the evaluation of \acp{BGP}
and then present how \emph{compatible} solution mappings can be combined in order to define the evaluation semantics of
SPARQL.
%
This evaluation algebra was presented by~\citet{Cyganiak:2005aa,PerezArenasGutierrez:2006aa} and later adapted to the
\ac{W3C} specification~\cite[Section~12.5]{PrudhommeauxSeaborne:2008aa}.

The matching of \acp{BGP} is performed against the previously mentioned \emph{active graph}, a specific \ac{RDF} graph
contained in the dataset of the query.  The active graph is initially set to the default graph of the dataset and is
changed whenever a \GRAPH keyword is processed.
%
This matching is represented by a function that maps query variables to \ac{RDF} terms present in the active graph and is
called a \emph{solution mapping}:
%
\begin{definition}[Solution Mapping~\cite{PrudhommeauxSeaborne:2008aa}]
  \label{def:solution-mapping}
  A \emph{solution mapping} is a partial function mapping SPARQL variables to \ac{RDF} terms.  The \emph{domain} of a
  solution mapping~$\mu$, $\dom{\mu}$, is the set of variables for which~$\mu$ is defined. We denote the value of
  variable~$v\in\AV$ according to solution~$\mu$ as~$\mu(v)$.
  % 
\end{definition}
% 
\nd The replacement of variables included in a graph pattern according to a solution mapping is defined next.
%
\begin{definition}[Variable Substitution]
  \label{def:variable-substitution}
  Let~$P$ be a graph pattern and~$\mu$ be a solution mapping.  
  %
  The \emph{variable substitution of~$P$ by~$\mu$}, denoted~$\mu(P)$, is the graph pattern~$P$ with all
  variables~$v\in\vars{P}\cap \dom{\mu}$ substituted by~$\mu(v)$.
  %
\end{definition}
%
\noindent It is worthy to note that if a solution mapping~$\mu$ contains bindings for all variables in a graph
pattern~$P$, \ie~$\dom{\mu} = \vars{P}$, and all triples in~$\funcCall{\mu}{P}$ are valid \ac{RDF} triples,
then~$\funcCall{\mu}{P}$ can be considered an \ac{RDF} graph.
%
If~$\mu$ provides bindings only for a subset of the variables present in the graph pattern~$P$,~$\mu(P)$ yields another
(more restrictive) graph pattern.
%
For the specification of the SPARQL algebra below, we introduce the notion of compatible solution mappings.
%
\begin{definition}[Compatible Mappings]
  \label{def:compatible-mappings}
  % 
  Let~$\mu_1$ and~$\mu_2$ be solution mappings,~$\mu_1$ and~$\mu_2$ are \emph{compatible} \iff for any~$v \in
  \dom{\mu_1} \cap \dom{\mu_2}$ it holds that~$\mu_1(v) = \mu_2(v)$.  The \emph{union} of two compatible
  mappings~$\mu_1$ and~$\mu_2$ consists of the standard set-theoretical union~$\mu_1 \cup \mu_2$.
  %
\end{definition}
%
\noindent The SPARQL relational
algebra~(see~\citet{Cyganiak:2005aa,PrudhommeauxSeaborne:2008aa,PerezArenasGutierrez:2009aa}) defines how to combine
solution mappings.
%
Our semantics of SPARQL is based on the semantics presented by~\citet{ArenasGutierrezPerez:2009ab}, where the SPARQL
algebra operators are extended to the multiset case by preserving the cardinality of solutions:\footnote{Following the
  notation of the operators presented by~\citet{ArenasGutierrezPerez:2009ab} we use the standard set operators.}
%
\begin{definition}[SPARQL Relational Algebra]
 \label{def:sparql-relalg}
 Let~$\Omega_1$ and~$\Omega_2$ be multisets of solution mappings:
 %
 \begin{tabbing}
   $\Omega_1 \leftouterjoin \Omega_2$\ \=$=$\ \=\kill
   $\Omega_1 \bowtie \Omega_2$\>$=$\ \>$\set{ \mu_1 \cup \mu_2 \mid \mu_1 \in \Omega_1, \mu_2 \in \Omega_2,\mu_1\textrm{ and }\mu_2\textrm{ compatible} }$\\
   $\Omega_1 \cup \Omega_2$\>$=$\>$\set{ \mu \mid \mu \in \Omega_1\textrm{ or } \mu \in \Omega_2 }$\\
   $\Omega_1 - \Omega_2$\>$=$\>$\set{ \mu_1 \in \Omega_1 \mid \textrm{ for all }\mu_2 \in \Omega_2,\mu_1\textrm{ and }\mu_2\textrm{ not compatible} }$\\
   $\Omega_1 \leftouterjoin \Omega_2$\>$=$\>$(\Omega_1 \bowtie \Omega_2) \cup (\Omega_1 - \Omega_2)$
   \end{tabbing}
\end{definition}


\noindent The definition of BGP matching from~\citet[Section~12.3]{PrudhommeauxSeaborne:2008aa} specifies the solutions
to a query.  We denote the evaluation of a \ac{BGP}~$P$ over a graph~$G$ as~$\eval{P}_{G}$:
%
\begin{definition}[Basic Graph Pattern Matching~{\cite[Section\ 12.3.1]{PrudhommeauxSeaborne:2008aa}}]
  \label{def:bgp-matching}
  Given a graph~$G$ and a \ac{BGP}~$P$, a \emph{solution}~$\mu$ for~$P$ over~$G$ is a mapping over~$V \subseteq
  \vars{P}$ such that~$G \models \funcCall{\mu}{P}$.  Following the definitions presented in \cref{sec:rdf},~$G \models
  \fc{\mu}{P}$, means that any triple in~$\fc{\mu}{P}$ is entailed by~$G$.
\end{definition}
%
\noindent This definition of \ac{BGP} matching relies on the underlying entailment notion, which according to the SPARQL
specification corresponds to simple graph entailment~\cite{Hayes:2004aa}.
%
Furthermore, in order to ensure the local scope of blank nodes, query solutions are taken from the \emph{scoping graph},
a graph that is equivalent to the active graph but does not share any blank nodes with it or any graph pattern within
the query.
%

% ----------------------------------------





%-----------------------------------
%
The evaluation semantics of more complex patterns including~\FILTER{\keyword{s}},~\OPTIONAL patterns,~\AND
patterns,~\UNION patterns is built on top of this \emph{basic graph pattern matching}, where each SPARQL operator is
mapped to an algebra expression:
%
\begin{definition}[Evaluation {\cite[Definition 2.2]{PerezArenasGutierrez:2009aa}}]
\label{def:semantics-sparql}
Let~$\tau = \triple{s,p,o}$ be a triple pattern,~$P, P_1, P_2$ graph patterns and~$G$ an \ac{RDF} graph, then the
evaluation~$\eval{\cdot}_{G}$ is recursively defined as follows: {
%
\begin{tabbing}
  $\eval{\tau}_{G}\ =\ \set{ \mu \mid \mathit{dom}(\mu)=\mathit{var}(P)\textrm{ and }G \models \mu(\tau) }$\\
  $\eval{P_1\ \FILTER\ P_2}_{G}\quad\quad$\=$=$\ \=$\eval{P_1}_{G}$\ \=$\leftouterjoin$\ \=$\eval{P_2}_{G}$\kill
  $\eval{P_1\ \AND\ P_2}_{G}$\>$=$\>$\eval{P_1}_{G}$\>$\ \bowtie$\>$\eval{P_2}_{G}$\\
  $\eval{P_1\ \UNION\ P_2}_{G}$\>$=$\>$\eval{P_1}_{G}$\>$\ \cup$\>$\eval{P_2}_{G}$\\
  $\eval{P_1\ \OPTIONAL\ P_2}_{G}$\>$=$\>$\eval{P_1}_{G}$\>$\leftouterjoin$\>$\eval{P_2}_{G}$\\
  $\eval{P\ \FILTER\ R}_{G}$\>$=$\>$\set{ \mu \in \eval{P}_{G} \mid R\mu \mbox{ is true } }$
\end{tabbing}}
%
\nd where~$R$ is a~\FILTER\footnote{For simplicity, we will omit from the presentation~\FILTER{\keyword{s}} such as
  comparison operators (\character{$<$}, \character{$>$}, \character{$\leq$}, \character{$\geq$}), data type conversion
  and string functions.  Further details are presented in~\citet[Section 11.3]{PrudhommeauxSeaborne:2008aa}.}
expression,~$u,v \in \AUBLV$. The valuation of~$R$ on a substitution~$\mu$, written~$R\mu$, is \emph{true} if:
%
{
\begin{tabbing}
(1) $R = \keyword{BOUND}(v)$ with~$v \in \mathit{dom}(\mu)$;\\
(2) $R = \keyword{isBLANK}(v)$ with~$v \in \mathit{dom}(\mu)$ and~$\mu(v) \in \AB$;\\
(3) $R = \keyword{isIRI}(v)$ with~$v \in \mathit{dom}(\mu)$ and~$\mu(v) \in \AU$;\\
(4) $R = \keyword{isLITERAL}(v)$ with~$v \in \mathit{dom}(\mu)$ and~$\mu(v) \in \AL$;\\
(5) $R = (u = v)$ with~$u,v \in \mathit{dom}(\mu)\cup \AUBL \wedge\ \mu(u) = \mu(v)$;\\
(6) $R = (\neg R_1)$ with~$R_1\mu\mbox{ is false}$;\\
(7) $R = (R_1 \vee R_2 )$ with~$R_1\mu\mbox{ is true}$ or~$R_2\mu\mbox{ is true}$;\\
(8) $R = (R_1 \wedge R_2)$ with~$R_1\mu\mbox{ is true}$ and~$R_2\mu\mbox{ is true}$.
\end{tabbing}}

\noindent$R\mu$ yields an error (denoted~$\varepsilon$), if: 
{
\begin{enumerate}[(1)]
\setlength{\itemsep}{-1.5mm}
%
\item $R = \keyword{isBLANK}(v)$,~$R = \keyword{isIRI}(v)$, or~$R = \keyword{isLITERAL}(v)$ 
  and~$v \not\in \mathit{dom}(\mu)$;
\item $R = (u = v)$ with~$u \not\in \mathit{dom}(\mu)\cup T$ or~$v\not\in \mathit{dom}(\mu)$;
\item $R = (\neg R_1)$ and~$R_1\mu=\varepsilon$;
\item $R = (R_1 \vee R_2 )$ and~$(R_1\mu\not=\mathrm{true}$ and~$R_2\mu\not=\mathrm{true})$ and~$(R_1\mu=\varepsilon$ or~$R_2\mu=\varepsilon)$;
\item $R = (R1 \wedge R2)$ and~$R_1\mu=\varepsilon$ or~$R_2\mu=\varepsilon$.
\end{enumerate}}
\noindent Otherwise~$R\mu$ is \emph{false}.
\end{definition}
%
\noindent The presented definition considers only safe~\FILTER{\keyword{s}} where, for a pattern~\inlineExpr{$P\
  \FILTER\ R$}, the filter~$R$ is said to be \emph{safe} if~$\vars{R} \subseteq \vars{P}$.
%
However, the SPARQL specification defines that in~\OPTIONAL{\keyword{s}}, any filter is scoped to the Group Graph
Pattern that contains the~\OPTIONAL.
%
As such, we include the definition that caters for unsafe~\FILTER{\keyword{s}}, introduced
by~\citet{AnglesGutierrez:2008aa}:
%
\begin{definition}[\OPTIONAL with \FILTER evaluation]
  \label{def:optsemantics}
  Let~$P_1, P_2$ be graph patterns and~$R$ a \FILTER expression. A mapping~$\mu$ is in~$\eval{P_1\ \OPTIONAL\ (P_2\
    \FILTER\ R)}_{G}$ \iff:
  \begin{itemize}[nosep]
  \item $\mu = \mu_1\cup \mu_2$, s.t.~$\mu_1 \in \eval{P_1}_{G}$,~$\mu_2 \in \eval{P_2 }_{G}$ are compatible and~$R\mu$
    is true, or
  \item $\mu \in \eval{P_1}_{G}$ and~$\forall \mu_2 \in \eval{P_2}_{G}$,~$\mu$ and~$\mu_2$ are not compatible, or
  \item $\mu \in \eval{P_1}_{G}$ and~$\forall \mu_2 \in \eval{P_2}_{G}$ s.t.~$\mu$ and~$\mu_2$ are compatible,
    and~$R\mu_{3}$ is false for~$\mu_{3} = \mu \cup \mu_2$.
  \end{itemize}
\end{definition}


%
Finally, the evaluation semantics of SPARQL consists of computing a \emph{sequence of solution mappings}, where any
existing solution modifiers are applied to the multiset of results.  If no solution modifiers are specified a default
ordering is assumed.
\begin{definition}[Solution sequences]
  Sequences of solution mappings are simply referred to as \emph{solution sequences}, often denoted by~$\omg{}{}$.
\end{definition}
%


%
These conditions of SPARQL \CONSTRUCT queries, informally specified in \cref{sec:sparql-syntax}, are reflected in
the following definition. 
%
Later, we will rely on this definition to show the equivalence of the newly introduced XSPARQL \CONSTRUCT expressions
and SPARQL \CONSTRUCT expressions.
%
\begin{definition}[SPARQL \CONSTRUCT semantics]
  %
  \label{def:sparql-construct}
  %
  Let~$C$ be a \ConstructTemplate and~$\Omega$ a \emph{solution sequence}. The SPARQL \CONSTRUCT returns an RDF graph
  generated by the set-theoretical union of the triples obtained from substituting variables in~$C$ with their bindings
  from~$\Omega$ and satisfying the following conditions:
 %
  \begin{enumerate}[label=(\arabic*),noitemsep]
  \item \label{def:sparql-construct-1} any invalid RDF triples that may be produced by the instantiation of the
    \ConstructTemplate are ignored; and
  \item \label{def:sparql-construct-2} blank node labels within the \ConstructTemplate are considered scoped to the
    template for each solution, \ie~if the same label occurs twice in a template, then there will be one blank node
    created for each solution in $\Omega$, but there will be different blank nodes for triples generated by different query
    solutions.  Blank nodes in the graph template be shared only within the same query solution~$\mu_i \in \Omega$.
  \end{enumerate}
  % 
\end{definition}
% 




\subsection*{Query Answering}
\label{sec:query-answering}

The SPARQL query language presented in the previous section can be viewed in a similar setting to the rule based
conjunctive queries presented for relational databases in \cref{sec:querying-rdb}.
%
Also inspired by~\citet{GutierrezHurtadoMendelzon:2004aa}, we assume that an RDF graph~$G$ is \emph{ground}, where all
blank nodes have been skolemised, \ie~consistently replaced with terms in~$\AUL$.
%
A \emph{query} is of the rule-like form:
%
$$q(\vec{x}) \leftarrow \exists \vec{y}.\varphi(\vec{x},\vec{y})$$
%
where~$q(\vec{x})$ is the \emph{head} and~$\exists \vec{y}.\varphi(\vec{x},\vec{y})$ is the \emph{body} of the query.
%
The body of the query is a conjunction of triples~$\tau_{i}$ ($1 \leq i \leq n$) and, similar to
\cref{sec:querying-rdb}, we use the symbol~\character{$,$} to denote conjunction in the rule body.
%
The vectors~$\vec{x}$ and~$\vec{y}$ are vectors of variables occurring in the body of the rule called the
\emph{distinguished variables} and \emph{non-distinguished variables}, respectively.  The variables in~$\vec{x}$ and~$\vec{y}$ are
disjoint and each variable occurring in~$\tau_{i}$ must be either distinguished or non-distinguished.
%

In a query, we allow \emph{built-in triples} of the form~$\triple{s,p,o}$, where~$p$ is a \emph{built-in predicate}
taken from a reserved vocabulary and having a \emph{fixed interpretation}. 
%
We generalise the built-ins to any~$n$-ary predicate~$p$, where~$p$'s arguments may be variables from~$\AV$
%
and values from~$\AUL$.  We will assume that the evaluation of the predicate can be decided in finite time.
%
For convenience, we write functional predicates\footnote{A predicate~$p(\vec{x},y)$ is functional if for any~$\vec{t}$
  there is a \emph{unique}~$t'$ for which~$p(\vec{t},t')$ is true.} as \emph{assignments} of the form~$x\assign
f(\vec{z})$ and assume that the function~$f(\vec{z})$ is safe (according to
\cref{def:rule-based-conjunctive-query}). We also assume that a non functional built-in predicate~$p(\vec{z})$ should
be safe as well.

\begin{example}[RDF conjunctive query]
  An example query is:
  \[q(n) \leftarrow \triple{x, \term{ex{:}memberOf}, y}, \triple{x, \term{foaf{:}name}, n}, \triple{y, \typeR,
    \term{mo{:}Band}}, \triple{y, \term{foaf{:}Name}, \stringValue{\term{Nightwish}}}\] \nd which intends to retrieve
  all persons names~$n$ that are members of a band~$y$ with the name \stringValue{\term{Nightwish}}.
\end{example}
In order to define an \emph{answer} to a query we introduce the following:

\begin{definition}[Query instantiation]
Given a vector~$\vec{x} = \tuple{x_1, \dots, x_k}$ of variables, a \emph{substitution} over~$\vec{x}$ is a vector of terms~$\vec{t}$ replacing variables in~$\vec{x}$ with terms of~$\AUBL$. Then,
given a query~$q(\vec{x}) \leftarrow \exists \vec{y}.\varphi(\vec{x},\vec{y})$, and two substitutions~$\vec{t},
  \vec{t'}$ over~$\vec{x}$ and~$\vec{y}$, respectively, the \emph{query instantiation}~$\varphi(\vec{t}, \vec{t}')$ is derived from~$\varphi(\vec{x},
  \vec{y})$ by replacing~$\vec{x}$ and~$\vec{y}$ with~$\vec{t}$ and~$\vec{t}'$, respectively.
\end{definition}

\nd Note that, similar to the variable substitution of a solution mapping in SPARQL
(\cf~\cref{def:variable-substitution}), if all tripes in a query instantiation are valid \ac{RDF} triples, the
query instantiation can be considered an RDF graph.

\begin{definition}[Entailment]
  \label{def:entailment}
  Given a graph~$G$, a query~$q(\vec{x}) \leftarrow \exists \vec{y}.\varphi(\vec{x},\vec{y})$, and a vector~$\vec{t}$ of
  terms in~$\universe(G)$, we say that~$q(\vec{t})$ is \emph{entailed} by~$G$, denoted~$G \models q(\vec{t})$, \iff in
  any model~$\I$ of~$G$, there is a vector~$\vec{t}'$ of terms in~$\universe(G)$ such that~$\I$ is a model of the query
  instantiation~$\varphi(\vec{t}, \vec{t}')$.
\end{definition}

\begin{definition}[Query Answers]
  If~$G \models q(\vec{t})$ then~$\vec{t}$ is called an \emph{answer} to~$q$. The \emph{answer set} of~$q$ \wrt~$G$ is
  defined as~$ans(G,q) = \set{ \vec{t} \mid G \models q(\vec{t}) }$.
\end{definition}
%
\noindent The notion of a solution for \acp{BGP} in SPARQL is the same as the notion of answers for conjunctive queries:
%
\begin{proposition} 
  \label{pp}
  Given a graph~$G$ and a \ac{BGP}~$P$, the solutions of~$P$ are the same as the answers of the query~$q(var(P))
  \leftarrow P$, \ie~$ans(G,q) = \eval{P}_{G}$.
\end{proposition}



%%% Local Variables:
%%% fill-column: 120
%%% TeX-PDF-mode: t
%%% TeX-debug-bad-boxes: t
%%% TeX-parse-self: t
%%% TeX-auto-save: t
%%% reftex-plug-into-AUCTeX: t
%%% mode: latex
%%% mode: flyspell
%%% mode: reftex
%%% TeX-master: "../thesis"
%%% End:



\subsection*{SPARQL~1.1}
\label{sec:sparql-11}

A new version of SPARQL, called SPARQL~1.1~\cite{HarrisSeaborne:2012aa}, is in the process of being proposed as a \ac{W3C} recommendation.  
%
This new version is composed of several documents specifying the updated query language and introduces new features that
were already used in practice by several SPARQL engines, such as
%
\begin{enumerate*}[nosep,label=(\roman*), before=\unskip{: }, after=\unskip{.}, itemjoin={{; }}, itemjoin*={{; and }}]
\item aggregates
\item subqueries
\item negation
\item assignment
\item property paths
\end{enumerate*}
% 
Other documents included in this new version, but not detailed in this section, specify an Update
language~\cite{GearonPassantPolleres:2012aa} and extensions for federated
querying~\cite{PrudhommeauxBuil-Aranda:2011aa}.


Aggregates allow expressions to be applied over groups of solutions to obtain a single value, for example determining
the minimum (\keyword{min}) value of the group.  Other aggregator functions included in the standard are
\keyword{count}, \keyword{sum}, \keyword{max}, \keyword{avg}, and \keyword{group\mathunderscore{}concat}.
%
Although the use of aggregate functions was already available in several SPARQL engines, it will only be introduced into
the official \ac{W3C} specification with SPARQL~1.1.
%


In SPARQL~1.1, nested \SELECT queries are allowed to be used in graph patterns and the projected variables of the
subquery are then joined with the results of the outer query.  These nested \SELECT queries are however not allowed to
specify a dataset and are restricted to be executed over the same dataset as the outer query.  SPARQL follows a
bottom-up query evaluation and thus the inner queries are evaluated first and its results made available to the outer
query.
%
A proposal for subqueries in SPARQL was previously presented by~\citet{AnglesGutierrez:2010aa} and later the same authors
compared different forms of subqueries to the W3C semantics~\cite{AnglesGutierrez:2011aa}.
%



Although negation was already permitted in SPARQL by using a combination of the \FILTER and \BOUND operators, this is
made explicit in SPARQL~1.1 by allowing two forms of negation: the \keyword{exists} and \keyword{minus}.
%
The \keyword{exists} (and \keyword{not exists}) \FILTER expression allows to test if a graph pattern matches (or does
not match) the dataset and consequently remove such solutions from the results.
%
The other form of negation uses the \keyword{minus} operator that, when applied to two graph patterns, removes solutions
from the left-hand side compatible with any solution from the right-hand side.
%
Since the \keyword{minus} operator relies on the notion of compatible solutions, it will only remove solutions if there
are shared variables between the solution sequences it is applied to.
%
This causes different results between the two forms of negation when the provided graph patterns do not share
variables:\footnote{A special case of graph patterns that do not share any variables is when the pattern to be removed
  contains no variables, \ie~is a fixed pattern.}
%
since no two solutions are compatible, the \keyword{minus} operator does not remove any solutions from the resulting
sequence.  However, the \keyword{exists} operator will remove the respective solutions from the final sequence.
%

%
SPARQL~1.1 includes a basic query federation by means of the \keyword{SERVICE} keyword, which specifies that the
following subquery will be executed in a remote SPARQL endpoint.

Other features include assignment of variables in the graph pattern (using the \keyword{bind} operator), in the \SELECT
clause, and in the \GROUPBY clause.  All assignments are of the form
%
\inlineExpr{\lstinline[basicstyle=\small\ttfamily]{(expression AS $var)}},
% 
where \lstinline[basicstyle=\small\ttfamily]{expression} is the expression to be evaluated and
%
\lstinline[basicstyle=\small\ttfamily]{$var}
%
is the variable name the result of the expression is assigned to.
%
Another form of assignment is using the \keyword{bindings} clause,\footnote{Note that the SPARQL~1.1 syntax is still
  under development and the \keyword{\footnotesize{bindings}} clause may be changed to \keyword{\footnotesize{values}}.}
which allows to specify a solution sequence that is to be joined with the results of the graph pattern.  The values for
variables in the provided solution sequence must be \ac{RDF} terms, \ie~no variables can be specified.  The
\keyword{bindings} clause is envisioned to be used with the \keyword{service} keyword to specify values for federated
querying.


Property paths are used to specify a connection between two \ac{RDF} nodes.  An extended graph pattern syntax is defined
that allows for a concise pattern matching, for example specifying alternative routes for connecting the nodes, or to
match paths of arbitrary or specific lengths.





%%% Local Variables:
%%% fill-column: 120
%%% TeX-PDF-mode: t
%%% TeX-debug-bad-boxes: t
%%% TeX-parse-self: t
%%% TeX-auto-save: t
%%% reftex-plug-into-AUCTeX: t
%%% mode: latex
%%% mode: flyspell
%%% mode: reftex
%%% TeX-master: "../thesis"
%%% End:
