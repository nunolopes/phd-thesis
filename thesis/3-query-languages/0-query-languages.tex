\chapter{Query Languages}
\label{cha:query-languages}

Query languages allow users to select and transform data from large sources.  The ability to select only relevant data
is an essential feature to minimise serialisation and communication overheads, especially when we consider the
transmission of data over the Web.
%

Due to the specific characteristics of each data model, query languages are usually tailored to work with a single data
model.
%
For the data models presented in the previous section, the respective query languages are \acs{SQL}, XQuery and SPARQL,
for which we will give an overview next.
%
We also present the closely related \acs{XSLT} transformation language for \ac{XML} data.
%

In \cref{fig:data-models-overview} we presented a high-level overview of the available languages for each data model.
Data and schema manipulation languages are widely available for relational data, for \ac{XML} an update language has
been recently standardised~\cite{RobieChamberlinDyck:2011aa} while for \ac{RDF} data this feature is only included in
the upcoming version of SPARQL~1.1~\cite{GearonPassantPolleres:2012aa}.

In the following sections we start by presenting a short overview of the possible forms of querying relational
databases, including the \ac{SQL} query language, before turning to the different query languages for \ac{XML} in
\cref{sec:prelim-xquery}.  In this section we again present a short overview of the XPath and \ac{XSLT} languages and
then focus in more detail on the XQuery language, which will be the basis for the XSPARQL language in
\cref{cha:xsparql}.
%
Finally, \cref{sec:sparql-preliminaries} provides a detailed description of the SPARQL query language for
\ac{RDF} (which we will extend in \cref{cha:anql}).



\section{Querying Relational Databases}
\label{sec:querying-rdb}

In this section we give an overview of conjunctive queries which, according to~\citet{AbiteboulHullVianu:1995aa},
represent the vast majority of relational database queries that are relevant in practice.  Later we present the \ac{SQL}
query language, which is the most used query language for relational databases.

\subsection{Conjunctive queries}
\label{sec:conjunctive-queries}
%
In line with the different views on relational data (presented in \cref{sec:relational-model}), conjunctive queries can
be formalised under different, although equivalent, perspectives: logic programming and the relational algebra.
%
The logic programming approach follows the corresponding view on relational data presented in
\cref{sec:relational-model}, while the relational algebra approach relies on the conventional view.
%
We then present the \acs{SQL} query language and provide an overview of its mapping into relational algebra.

Under the logic programming approach, in addition to the sets of relations \AR, attributes \AAs, and values \AD, we rely
on the set of variables \AV that range over elements of \AD.
%
We can now extend the notion of fact to \emph{atom}: an atom over a relation~$r \in \AR$ is an
expression~$\fact{r}{e_{1}, \dots, e_{n}}$ where~$n$ is the arity of~$r$ and each~$e_{i} \in \ADV$ is called a
\emph{term}.  A fact can also be referred to as a \emph{ground atom}.  The notion of \emph{query} can then be defined
as:
%
\begin{definition}[Rule-based conjunctive query~\cite{AbiteboulHullVianu:1995aa}]
  \label{def:rule-based-conjunctive-query}
  Given a database schema~$S$, a \emph{rule-based conjunctive query}~$q$ over~$S$ is an expression of the form:
  %
  \begin{equation}
    \fact{q}{u} \leftarrow \fact{r_{1}}{u_{1}}, \dots, \fact{r_{n}}{u_{n}}\label{eq:1}
  \end{equation}
  % 
  where each~$r_{i}, i \in [1,n]$ is a relation name from~$S$ and each~$\fact{r_{i}}{u_{i}}$, is an atom over~$r_{i}$.
  Any variable occurring in~$u$ must be \emph{safe}, \ie~it must also occur at least once in any~$u_{1}, \dots, u_{n}$.
  Furthermore, we denote the set of variables present in~$q$ as \vars{q}.
\end{definition}
%
\noindent A rule-based conjunctive query can be referred to simply as a \emph{rule} where the lefthand side
of~\character{$\leftarrow$} is called the \emph{head} and the righthand side is called the \emph{body} of the rule.  For
example, in Rule~\eqref{eq:1}, the head is~$q(u)$ and the body corresponds to~$\fact{r_{1}}{u_{1}}, \dots,
\fact{r_{n}}{u_{n}}$.
% 
A rule can be interpreted as: the head atom can be deduced if there are values for the variables in the rule that make
the body hold.
%
Given a set of variables~$V \subset \AV$, a \emph{mapping} (or \emph{valuation}) over~$V$ is a function~$v : V \to \AD$.
This function can be extended to represent the identity over any element of~$\AD$ and thus can map any atom with
variables to a fact by applying it to all elements of the atom.  For any atom~$t$, the mapping of~$v$ is denoted as
\funcCall{v}{t}.
%
Based on this notion of mapping, the \emph{answers} to a query can be defined as:

\begin{definition}[Answers of a query]
  Let~$q = \fact{q}{u} \leftarrow r_{1}, \dots, r_{n}$ be a rule-based conjunctive query and~$I$ be a database instance.
  An \emph{answer} of~$I$ under~$q$ is:
  \[
  \funcCall{q}{I} = \set{ \funcCall{v}{r} \mid v\mbox{ is a mapping over}~\vars{q}~\mbox{and }\funcCall{v}{r_i} \in I
  \mbox{ for each }i \in [1, n] }
  \]
\end{definition}
%
As we will see in the next section, since duplicate removal is a computational expensive operation, \ac{SQL} maintains
duplicates in the answers of a query (thus represented as a multiset) unless otherwise instructed.

Another paradigm for relational queries is the algebraic paradigm, which is defined by specifying operations on relation
instances, called the \emph{relational algebra}.
%
The three primitive algebra operators are the \emph{selection}~($\sigma$), \emph{projection}~($\pi$) and the
\emph{cartesian product}~($\times$) operators.
%
The full set of operators that form the \emph{relational algebra} also include the \emph{join}~($\bowtie$),
\emph{union}~($\cup$) and \emph{difference}~($-$) operators.  
%
The selection operator consists of restricting the tuples present in a relation according to a specified condition.
%
The projection operator is used to discard attributes of a relation while the cartesian product combines any two
relations and produces a new relation that includes all the attributes of both relations.
%
The join operator consists of combining the projection and cartesian product operators: the result of this operator
consists of all the tuples from both relations that have a common value on any common attributes. 
%
The union~($\cup$) and difference~($-$) algebra operators are defined as the standard set-theoretical operators.
% 

Considering two relational instances~$I$ and~$J$ (with sorts~$U$ and~$W$, respectively), the relation attributes~$A$
and~$B$, and a constant~$c \in \AD$, the relational algebra operators are defined as:
% 
\begin{description}
\item[selection:] The two forms of the selection operator,~$\sigma_{A=c}$ and~$\sigma_{A=B}$ select tuples that match a
  constant or where the value of two attributes is the same, respectively:
  %
  \begin{align*}
    \sigma_{A=c}(I) &= \set{ t \in I \mid \funcCall{t}{A} = c } \\
    \sigma_{A=B}(I) &= \set{ t \in I \mid \funcCall{t}{A} = \funcCall{t}{B} }
  \end{align*}
  %
\item[projection:] This operator consists of restricting the attributes present in a relation. Given a set of
  attributes~$X \subseteq U$, the projection operator returns:
  \[
  \pi_{X}(I) = \set{ t[X] \mid t \in I }
  \]
\item[join:] The join operator between~$I$ and~$J$ produces a relation with sort~$U \cup W$, such that:
  \[
  I \bowtie J = \set{ t \mid \funcCall{sort}{t} = U \cup W\mbox{ and }t[U] = u\mbox{ and }t[W] = w\mbox{ for some }u \in I\mbox{ and }w \in J }
  \]
\end{description}
% 


\subsection{SQL}
\label{sec:sql}

The \ac{SQL} is the most widely available query language for relational data, supported by most commercial relational
database management systems~\cite{AbiteboulHullVianu:1995aa} and is an American National Standards Institute (ANSI)
standard.
%
The core of SQL queries consist of the commonly known \emph{select-from-where} queries, which are equivalent in
expressivity to conjunctive queries.  An example of a \ac{SQL} query is shown in \cref{ex:sql-query}.
%
\begin{example}[SQL query]
  %
  \label{ex:sql-query}
  The following \ac{SQL} query, when executed against a database instance following the schema presented in
  \cref{ex:bands-schema}, extracts the names of the artists that are in the \stringValue{Nightwish} band:
  %
\begin{lstlisting}[language=SQL,frame=none,numbers=none]
SELECT persons.personName
FROM persons, bands
WHERE persons.bandId = bands.bandId
      AND bands.bandName = 'Nightwish'
\end{lstlisting}
\end{example}
%
The \SELECT keyword specifies the attributes that should be present in the query results, while \FROM specifies the
relations names over which the query will be evaluated.  It is possible to write \character{*} in place of attribute
names in a \SELECT clause if all the attributes in the relations specified in the \FROM clause are to be returned.
%
In SQL, relation names are considered as variables that range over tuples occurring in the corresponding relation and,
as shown in \cref{ex:sql-query}, can be used in the \WHERE clause to specify the relations and attributes.
%
If a query requires more than one variable ranging over the same relation, they can be specified in the \FROM clause and
assigned different \emph{aliases} for the relation, \eg~\lstinline{FROM person p1, person p2}.  Furthermore, if the
attributes are distinct, it is possible to omit the relation name from the \SELECT query.
%
Finally, the \WHERE keyword specifies the conditions that any result of the query must satisfy in order to be included
in the result set and can express conjunction, disjunction, negation, and nesting.  It is possible to completely omit
the \WHERE clause and, in this case, all tuples of the cartesian product of the relations specified in the query are
returned.


It is also possible to represent nested queries by using the keywords \IN and \NOTIN.  These keywords behave as
operators over sets, testing the inclusion (or not) of an element in the set resulting from the nested query.
%

% 
The result of a \ac{SQL} \emph{select-from-where} query evaluation consists of a multiset of tuples, \ie~there may be
repeated answers in the results.  The explicit use of the \DISTINCT keyword after the \SELECT keyword removes any
duplicate answers from the resulting set.
%
\ac{SQL} also includes several aggregate functions, such as \keyword{count}, \keyword{sum}, and \keyword{average}, which
perform the specified function over the resulting multisets, such as counting number of elements in the collection, or
adding the elements of a multiset composed of numeric elements.
%
Furthermore grouping of tuples, by means of the \keyword{group by} operator, also allows to create collections of tuples
over which aggregates can be applied.
%

%
Included in the SQL specification is also the definition of a DML.  This part of the SQL language allows the schema of a
relational database to be manipulated, for instance creating new relations, altering the structure of existing ones, or
removing existing relations.


\subsubsection{Translation to Relational Algebra}
\label{sec:sql-semantics}

A translation from a subset of \ac{SQL} into relational algebra was presented by~\citet{CeriGottlob:1985aa}, while the 
%
semantics catering for the full syntax of \ac{SQL} and the three-valued logic inherent with \NULL{s}, was presented
by~\citet{NegriPelagattiSbattella:1991aa}.
%
In this thesis, we follow the translation of \ac{SQL} \emph{select-from-where} queries into relational algebra as
presented by~\citet{AbiteboulHullVianu:1995aa}:
\begin{inparaenum}[(i)]
\item the \SELECT keyword behaves as the projection operator~$\pi$;
\item the \FROM keyword corresponds to the cartesian product operator~$\times$; and
\item the \WHERE keyword specifies a selection operation~$\sigma$.
\end{inparaenum}
%
In the rest of the thesis, we denote this translation of a \ac{SQL} query~$S$ into relational algebra
as~$\RASQL{S}$.
%
\begin{example}[SQL translation into Relational Algebra]
  The translation into relational algebra of the query in \cref{ex:sql-query} is:
  %
  \[
  \pi_{\mathit{persons.personName}}(\sigma_{\mathit{persons.bandId} = \mathit{bands.bandId} \land \mathit{bands.bandId} = \mathit{'Nightwish'}}(\mathit{persons} \times \mathit{bands}))
  \]
\end{example}
%


\subsubsection{Mapping SQL Results to XML}
\label{sec:mapping-xml-types}
%
\begin{table}[t]\centering
  \caption{Mapping from SQL to XML datatypes}
  \label{fig:SQL2XML}
  { \scriptsize
    \begin{tabular}{l|l}
      \toprule
      \textbf{SQL datatype} & \textbf{XML datatype}\\
      \midrule
      character string & xs:string\\
      numeric, decimal& xs:decimal\\
      boolean & xs:boolean \\
      smallint, integer, bigint & xsd:integer\\
      float, real, double precision & xsd:double\\
      date & xsd:date\\
      time & xsd:time\\
      timestamp & xsd:dateTime\\
      \bottomrule
    \end{tabular}
  }
\end{table}
%
The mapping from SQL datatypes into \ac{XML} Schema datatypes is defined in the \ac{SQL} specification and was presented
in~\citet{EisenbergMelton:2001aa}.  
%
An overview of this mapping is presented in \cref{fig:SQL2XML}.
%
Since \ac{XML} datatypes generically allow a wider range of valid values, it is common for concrete mappings to impose
further restrictions on \ac{XML} datatypes, however, for this thesis we will omit these restrictions on the \ac{XML}
datatypes.
%
Later, in \cref{sec:semantics}, we will rely on this mapping of datatypes for the translation of \ac{SQL} to \ac{XML}
and we refer to the \ac{XML} representation of \ac{SQL} values as~\funcCall{sql2xml}{\mathit{SQLValue}} and vice-versa
as~\funcCall{xml2sql}{\mathit{XMLValue}}.
%



%%% Local Variables:
%%% fill-column: 120
%%% TeX-PDF-mode: t
%%% TeX-debug-bad-boxes: t
%%% TeX-parse-self: t
%%% TeX-auto-save: t
%%% reftex-plug-into-AUCTeX: t
%%% mode: latex
%%% mode: flyspell
%%% mode: reftex
%%% TeX-master: "../thesis"
%%% End:



\section{Querying XML}
\label{sec:prelim-xquery}

As for querying \ac{XML} data, there are different alternatives, most notably \ac{XSLT} and XQuery.
%
Although these languages are very similar and both tackle similar problems, their fundamental difference is that XSLT
was designed to perform transformations between different \ac{XML} formats, mostly considered for styling and display of
information on the Web, while XQuery is focused on querying parts of an \ac{XML} document or
tree~\cite{KatzChamberlinKay:2003aa}.  Notably, XSLT uses an \ac{XML} syntax for specifying the transformations while
XQuery provides a non-\ac{XML} syntax that aims to be familiar for SQL users.
%
Both of these languages rely on a common core, the \ac{XPath}, that allows nodes of an \ac{XML} document to be selected.
Although \ac{XPath} was being designed at the same time as the XSLT language, it was published as a separate standard by
the \ac{W3C}, who envisioned its use in other languages or even as a single standalone language.
%

In this section, we start by explaining the \ac{XPath} language, followed by an high-level overview of the XSLT language.
Finally we present the XQuery language, that is used as a basis for the syntax and semantics of XSPARQL.



\subsection{XPath}
\label{sec:xpath}

\ac{XPath}~\cite{BerglundBoagChamberlin:2010aa} consists of a common core language that is reused by both \ac{XSLT} and
XQuery.  The main purpose of \ac{XPath} is to access specific nodes of an \ac{XML} document, which it does by providing
a non-\ac{XML} syntax for navigating though the structure of an \ac{XML} document and selecting the relevant nodes.
%
In \ac{XPath}, an \emph{expression} is the basic construct of the language and may consist, among other elements, of
variable references, function calls, or location paths.
%
Such expressions are evaluated with respect to an \emph{expression context}, which contains the necessary information to
determine the output of an expression, most notably the \emph{context node}: the \ac{XML} element over which the
expression will be evaluated.  The evaluation of an \ac{XPath} expression results in an \emph{object} that can consist
of a node-set, boolean, number, or a string.
%

\paragraph*{Location paths.} Location paths are an especially important form of \ac{XPath} expressions since they
specify how to navigate through the \ac{XML} document.  A location path consists of a sequence of \emph{location steps}
that are separated by the \character{/} character.
%
The result of the evaluation of each location step is the set of nodes and each step in the location path is applied to
the set of nodes resulting from the previous step.
% 
A location step can be composed of three parts: an \emph{axis}, a \emph{node test}, and any number of \emph{predicates},
which are described next.
%
\begin{description}
\item[Axis:] The axis is used to select nodes by specifying the relation that the selected nodes should have from the
  context node.
  % 
  Some of the available axes allow the \texttt{child}, \texttt{parent}, \texttt{following-sibling}, or
  \texttt{preceding-sibling} of the context node to be selected.  These axes correspond to selecting, relative to the
  context node, all the nodes that are one level below (\texttt{child}), one level up (\texttt{parent}), at the same
  level after (\texttt{following-sibling}) or before (\texttt{preceding-sibling}) the context node.
  % 
  Furthermore, the \texttt{attribute} axis can be used to select all the attributes of the context node,
  \stringValue{self} refers to the context node and \stringValue{descendant-or-self} to the context node and its
  descendants.
  % 

\item[Node Tests:] Node tests can be combined with axes to further restrict the selected set of nodes.  For example
  \stringValue{child::band} will select all the children of the context node with \stringValue{band} name.  Similarly,
  \character{*} can be used to select all elements: \stringValue{attribute::*} selects all the attributes of the context
  node.
  % 
  
\item[Predicates:] Predicates can be used to further filter a node-set and produce a new node-set: given a node-set, the predicate
  expression is evaluated by using each node in the node set as the new context node.  If the evaluation of the predicate
  yields true, then the node is included in the newly created node set.
  % 
\end{description}

The \ac{XPath} specification also defines an abbreviated syntax for representing location steps, where the expression
can be specified by either assuming a default axis or by specifying shortcuts.  For example, the \stringValue{child} axis
is the default and can thus be omitted.
%
The \character{@} abbreviation can be used for selecting the \emph{attribute} axis, thus, \stringValue{@name="Nightwish"}
is short for \stringValue{attribute::name="Nightwish"}.
%
Other available abbreviations are \character{//} for \stringValue{/descendant-or-self::node()/}, \character{.} for
\stringValue{self::node()} and \character{..} for \stringValue{parent::node()}.  
%


\begin{example}[XPath expression]
  The following XPath expression:

  \lstinline{//band[@name="Nightwish"]/members}
  
  \noindent which is an abbreviated form of the expression:

  \lstinline{/descendant-or-self::node()/child::band[attribute::name="Nightwish"]/child::members}
  %
  
  \noindent when executed over the \ac{XML} document presented in \cref{fig:bands-xml}, returns the
  \stringValue{band} \ac{XML} element whose value of the \stringValue{name} attribute is \stringValue{Nightwish}.
  %
\end{example}
%
Further expressions available in the XPath language include \FOR expressions that, in a somewhat similar fashion to
imperative programming languages,\footnote{We say somewhat similar since the evaluation order of the return expression
  is not imposed by XPath.} allow to repeat an expression (called \emph{return expression}) for different values of the
\emph{range variable} (detailed in \cref{sec:xquery}), conditional expressions and quantified expressions.
%
Conditional expressions allow to execute different expressions (\emph{then expression} or \emph{else expression})
depending on the result of the \emph{if expression} and quantified expressions can be used to test whether an expression
is true for all (\keyword{every}) or at least one (\keyword{some}) members of a sequence.


\subsection{XSLT}
\label{sec:xslt}

\acf{XSLT}~\cite{Kay:2007aa} is a transformation language for \ac{XML} that allows to manipulate \ac{XML}
documents by matching subsets of the \ac{XML} structure and specifying transformation rules for the matched elements.  
%
The syntax of \ac{XSLT} is \ac{XML}-based and defines a set of \ac{XML} elements that are interpreted as \ac{XSLT}
instructions, distinguished by using the namespace \stringValue{\url{http://www.w3.org/1999/XSL/Transform}}, commonly
abbreviated by the \qname{xsl}{} prefix.
%
As previously noted, \ac{XSLT} relies on the \ac{XPath} language to navigate and access the elements in the
\ac{XML} document.
%

\ac{XSLT} transformations are called \emph{stylesheets}, referring to the origins of the language, mostly used for
defining the style of an \ac{XML} or XHTML document for presentation in a web browser.
%
\ac{XSLT} follows the functional programming paradigm, where the stylesheet defines a set of rules that produce the output
tree as a function of the input tree.  
%
Such rules, called \emph{template rules}, are applied to the source or input tree in order to produce the result (or
output) tree.  The part of the rule that is matched against the \ac{XML} elements in the source is called the
\emph{pattern}, while the \emph{sequence constructor} part is instantiated by elements matched from the source tree in
order to produce the result tree.
%
Recalling the logic programming approach for querying relational databases (\cref{sec:querying-rdb}), the pattern
can be considered the body of the query, while the sequence constructor can be considered the head.
%


Template rules are defined using \ac{XML} elements named \qname{xsl}{template} (as presented in
\cref{ex:xslt-templates}).  
%
The \stringValue{match} attribute is used to specify the \ac{XML} elements to which the template will be applied, while
the body of the template defines the output.  The recursive application of template rules is selected by the
\stringValue{\qname{xsl}{apply{\mbox{-}}templates}} element, possibly specifying which \ac{XML} elements from the input
should be matched by providing an \ac{XPath} expression as the value of the \stringValue{select} attribute.  Whenever
this attribute is omitted, the default is to select all children of the context node.
%
\begin{example}[\acs{XSLT} template rules]
  %
  \label{ex:xslt-templates}
  %
  The following template rule selects the band element whose value for the name attribute is
  \stringValue{Nightwish}:
  % 
  \lstinputlisting[frame=none,numbers=none,firstline=7,lastline=9]{0-data+queries/members-therion.xsl}
  %
  \noindent While the next template rule simply outputs all members of the selected bands:
  \lstinputlisting[frame=none,numbers=none,firstline=3,lastline=5]{0-data+queries/members-therion.xsl}
  %
  \noindent Combining these two template rules in an \ac{XSLT} stylesheet will make the stylesheet output the names of
  members of the \stringValue{Nightwish} band, when applied to \cref{fig:bands-xml}.
  %
\end{example}
%
Similar to \ac{XPath}, an \ac{XSLT} stylesheet is evaluated with regards to an \emph{expression context} and relies on
the \ac{XPath} specification for defining the contents of each expression context.
%
Each template rule is evaluated by specifying the matched input \ac{XML} element as the context node.

The \ac{XSLT} specification further defines instructions for specifying repetition (\texttt{xsl:for-each}), conditional
processing (\texttt{xsl:if} and \texttt{xsl:choose}), variable declaration (\texttt{xsl:variable}), and function
declaration (\texttt{xsl:function}).  
%





\subsection{XQuery}
\label{sec:xquery}

XQuery~\cite{ChamberlinRobieBoag:2010aa} has been the \ac{W3C} recommended query language for \ac{XML} since early 2007.
%
There are several similarities between XQuery and \ac{XSLT} and both query languages can address the same \usecases.
Some of the most evident similarities include
\begin{enumerate*}[nosep,label=(\roman*), before=\unskip{: }, after=\unskip{.}, itemjoin={{; }}, itemjoin*={{; and }}]
\item declarative semantics supporting single-assignment variables
\item the use of \ac{XPath} for selecting input \ac{XML} elements
\item the construction of new \ac{XML} elements explicitly and at runtime
\item support for user-defined functions
\end{enumerate*}



As we have seen, \ac{XSLT} was designed as a transformation language for \ac{XML} documents, focusing on transformations
that facilitate displaying data for the user.  On the other hand, XQuery behaves more like a query language, aiming at
extracting data from collections or large individual \ac{XML} documents.
% 
These design choices are apparent even in the syntax of the languages, where XQuery follows a non-\ac{XML} syntax.
% 
However, XQuery reuses other \ac{XML}-based specifications such as the \ac{XDM} data model and \ac{XSD}.
%
Any input document for an XQuery query, commonly specified using the \funcName{\qname{fn}{doc}} function, is translated
into an \ac{XDM} instance and the respective query is executed over this abstract structure.
%

As in \ac{XPath} and \ac{XSLT}, also in XQuery the basic construct of a query is called an \emph{expression}.  In
XQuery, expressions are mostly composed of \FLWOR expressions.
%
This name stems from the available expressions: \FOR, \LET, \WHERE, \ORDERBY, and \RETURN.
%
\begin{definition}[Tuple Stream]
  In a \FLWOR expression, the result of the evaluation of ``\FOR~\var{v}'' and ``\LET~\var{v}'' clauses
  consists of an ordered sequence of elements that~\var{v} is bound to.  Following the XQuery specification, we refer to
  this sequence as \emph{tuple stream}.
\end{definition}
%
Optionally, this produced sequence can be filtered using the \WHERE clause or ordered with the \ORDERBY clause.
%
Finally, the \RETURN clause is evaluated for every element of the resulting sequence and each result is included in the
sequence produced by the \FLWOR expression.
%
Any XQuery variable is represented by using an expanded QName also making it possible to disambiguate variables based on
declared prefixes. Further details are available in~\citet[Section 3.1.1.1]{DraperFankhauserFernandez:2010aa}.
%
Another important feature is that \FOR clauses may optionally include a positional variable in the form of ``\FOR
\var{Var} \AT \var{PosVar}''.  In this case, for each evaluation of the \RETURN expression, \var{PosVar} is assigned an
integer corresponding to the position of \var{Var} in the tuple stream.
%


\begin{definition}[Expression Context]
  Similar to \ac{XPath} and \ac{XSLT}, any XQuery expression~$E$ is evaluated with regards to an \emph{Expression
    Context} that holds the static environment (\stat) and the dynamic environment (\dyn) up until the evaluation
  of~$E$.
\end{definition}
%
Environments include different components that hold the necessary information for the evaluation of any XQuery
expression: \stat{} holds the information available during static analysis, for example the \ecomp{varType} component
holds variable type information.
%
The \dyn{} environment contains information available during expression evaluation, like the value for variables, stored
in the \ecomp{varValue} component.
%
Given an expression context~$C$, we refer to the static environment of~$C$ as~$\funcCall{\stat}{C}$ and to the dynamic
environment as~$\funcCall{\dyn}{C}$.  Different components can be accessed via their
name:~$\funcCall{\stat}{C}.\ecomp{varType}$ and the specific value of the environment element~$var$ can be accessed
using~$\funcCall{\stat}{C}.\ecomp{varType}(var)$.  If the expression context~$C$ is not explicitly presented, \stat and
\dyn can be used in place of \funcCall{\stat}{C} and \funcCall{\dyn}{C}.



\begin{example}[XQuery query]
  %
  \label{ex:xquery-query}
  %
  The slightly verbose XQuery equivalent to the XSLT presented in \cref{ex:xslt-templates} is the following
  query:
  %
  \lstinputlisting[frame=none,numbers=none]{0-data+queries/members-therion.xquery}
  % 
  Executing this query over the \ac{XML} document presented in \cref{fig:bands-xml}, again returns the members of the
  \stringValue{band} \ac{XML} element whose value of the \stringValue{name} attribute is \stringValue{Nightwish}.
\end{example}
%
XQuery allows to write arbitrary queries, it is actually a Turing complete language~\cite{Kepser:2004aa}, for instance
the \RETURN part of a \FLWOR expression may contain other (nested) \FLWOR expressions.  
%
In such cases we commonly refer to the first \FLWOR expression as the \emph{outer} query, while the \FLWOR expression
that is containted inside the \RETURN is refered to as the \emph{inner} query.


\subsubsection*{Semantics}
%
The semantics of XQuery~\cite{DraperFankhauserFernandez:2010aa} is defined in terms of
%
\begin{inparaenum}[(i)]
\item \emph{normalisation rules},
\item \emph{static typing rules}, and
\item \emph{dynamic evaluation rules}.
\end{inparaenum}
%
Normalisation rules reduce the syntax of XQuery to an abstract syntax denoted XQuery Core: a subset of XQuery that,
while semantically equivalent, aims to be easier to define, implement and optimise~\cite{KatzChamberlinKay:2003aa}.
%
Static typing rules are applied over the XQuery Core language and are used to assign a type to each XQuery expression.
The dynamic evaluation rules are responsible for producing the results of each expression while guaranteeing that its
input is consistent with the previously determined typing information.

In this thesis we will use the term \emph{bound variable} to refer to a variable that has been previously declared in an
query, for example, \var{v} is considered bound if it has been previously declared by a \stringValue{\FOR \var{v}} or
\stringValue{\LET \var{v}} expression.


The complete semantics of XQuery is defined by specifying \emph{normalisation}, \emph{static} and \emph{dynamic
  evaluation} rules for each expression of the language and, as an example, we next present the rules of the XQuery \FOR
expression.

\paragraph*{Normalisation Rules.}
\emph{Normalisation rules} are represented using mapping rules, where \sem{\cdot}{Expr} represents the XQuery expression
to be matched, while the resulting XQuery Core expression is included after the~$==$ separator.
%
Furthermore, fixed-width font (like \FOR and \IN) refer to specific keywords, and \emph{italic} font refers to
productions in the XQuery Core grammar~\cite[Appendix~A]{DraperFankhauserFernandez:2010aa}.
%
The following example shows the application of the normalisation rules over consecutive \ForClause{s} -- considered a
shorthand syntax -- into nested \ForClause{s} in XQuery Core:
%
\begin{normalisationrule}
  \begin{prooftree}
    \alwaysNoLine \AxiomC{$\sem{\begin{array}{l}
          \FOR~\envElem{\var{VarName}}{1}~\envElem{OptTypeDeclaration}{1}~\envElem{OptPositionalVar}{1}~\IN~\envElem{Expr}{1} \\
          \quad,\cdots, \\
          \quad\var{VarName}_n~\envElem{OptTypeDeclaration}{n}~\envElem{OptPositionalVar}{n}~\IN~\envElem{Expr}{n}\\
          \quad\grammarRule{ReturnClause}
        \end{array}
      }{Expr}$}
    \UnaryInfC{$==$}
    \UnaryInfC{$\begin{array}{l}
        \FOR~\envElem{\var{VarName}}{1}~\envElem{OptTypeDeclaration}{1}~\envElem{OptPositionalVar}{1}~\IN~\sem{\envElem{Expr}{1}}{Expr}~\RETURN \\
        \quad\cdots \\
        \quad\FOR~\envElem{\var{VarName}}{n}~\envElem{OptTypeDeclaration}{n}~\envElem{OptPositionalVar}{n}~\IN~\sem{\envElem{Expr}{n}}{Expr}\\
        \quad\sem{\grammarRule{ReturnClause}}{Expr}
      \end{array}$}
    \label{for-normalisation}
  \end{prooftree}%
\end{normalisationrule}%
%
The normalisation process consists of the recursive application of the defined rules over each expression in the
language.
%

\paragraph*{Static and Dynamic Evaluation Rules.}
On the other hand, \emph{static type rules} and \emph{dynamic evaluation rules} are represented using inference rules
of the form:
%
\begin{prooftree}
  \AxiomC{$\textit{premise}_{1}~\dotsb~\textit{premise}_{n}$}
  \UnaryInfC{$\textit{conclusion}$}
\end{prooftree} 
%
Rule premises are composed of the so-called \emph{judgements} and such judgements are said to \emph{hold} if they are
considered true.  Some judgments used in this thesis are:
%
\begin{description}
\setlength\abovedisplayshortskip{1pt plus 2pt minus 0pt}
\setlength\belowdisplayshortskip{1pt plus 2pt minus 0pt}
\setlength\abovedisplayskip{1pt plus 2pt minus 0pt}
\setlength\belowdisplayskip{1pt plus 2pt minus 0pt}
\item[Type.] The judgment:
  %
  \[\statEnv{\envElem{Expr}{} \colon \envElem{Type}{}}\] 
  % 
  holds if, in the static environment~\stat, the expression \envElem{Expr}{} has the type \envElem{Type}{}.
  % 
  Also related to typing, the \funcName{prime} and \funcName{quantifier} are functions that extract all the item types
  of its parameter and try to estimate the number of items in a type (?, +, or *), respectively.
%
\item[Variable Expansion.] Similarly, \[\statEnv \envElem{VarName}{}~\textbf{of var expands to}~\envElem{Variable}{}\]
  holds if \envElem{Variable}{} corresponds to the expanded QName of \envElem{VarName}{}.
%
\item[Context Extension.] Contexts can be extended by using the~\character{$+$} notation, for
  example: \[\stat\envExtend{varType}{\envElem{Variable}{pos} \Rightarrow \qname{xs}{integer}}\] creates a new context
  based on~\stat by adding the information that \envElem{Variable}{pos} is of type \qname{xs}{integer} to the
  \ecomp{varType} component of~\stat.
%
\item[Expression Evaluation.] A commonly used judgment in dynamic evaluation rules is
  \[\dynEnv{\envElem{Expr}{} \Rightarrow \envElem{Value}{}}\] which holds if, in the environment~\dyn, the expression
  \envElem{Expr}{} evaluates to the value \envElem{Value}{}.
%
\end{description}
%
As an example of the use of these judgments, the following static type rule handles the typing of a \FOR clause with a
positional variable:
%
\begin{staticrule}
  \begin{prooftree}
    \def\ScoreOverhang{1pt}%
    \def\extraVskip{1pt}%
    \alwaysNoLine%
    %
    \AxiomC{$\statEnv \envElem{Expr}{1} \colon \envElem{Type}{1}$}
    %
    \UnaryInfC{$\statEnv \envElem{VarName}{}~\textbf{of var expands to}~\envElem{Variable}{}$}
    %
    \UnaryInfC{$\statEnv \envElem{VarName}{pos}~\textbf{of var expands to}~\envElem{Variable}{pos}$}
    %
    \UnaryInfC{$\stat\envExtend{varType}{\begin{array}{l} 
          \envElem{Variable}{} \Rightarrow \funcCall{prime}{\envElem{Type}{1}}; \\
          \envElem{Variable}{pos} \Rightarrow \qname{xs}{integer} 
        \end{array}}~\proofs  \envElem{Expr}{2} \colon \envElem{Type}{2}
      $}
    %
    \singleLine
    %
    \UnaryInfC{$\statEnv{
        \begin{array}{l}
          \FOR~\envElem{\var{VarName}}{}~\keyword{at}~\envElem{VarName}{pos}~\keyword{in}~\envElem{\grammarRule{Expr}}{1}\\
          \keyword{return}~\envElem{\grammarRule{Expr}}{2} \colon \envElem{Type}{2} \cdot \funcCall{quantifier}{\envElem{Type}{1}}
        \end{array}
      }$}
  \end{prooftree}%
  \label{eq:static-type-for}
\end{staticrule}%
%


The dynamic evaluation of \FOR expressions consists of first evaluating the expression specified by the \IN clause and,
for each element of the resulting sequence, assign it to the \FOR variable and then evaluating the \RETURN expression.
%
Hence, the semantics separates the dynamic evaluation rules of \FOR expressions into two cases, depending on whether the
\IN expression returns any elements.  If the \IN expression evaluates to an empty sequence, the \FOR expression also
evaluates to an empty sequence:
%
\begin{dynamicrule}
    %
    \AxiomC{$\dynEnv{\envElem{Expr}{1} \Rightarrow \left(\right)}$}
    %
    \singleLine \UnaryInfC{$\dynEnv{
        \FOR~\envElem{\var{VarName}}{}~\grammarRule{OptPositionalVar}~\keyword{in}~\envElem{\grammarRule{Expr}}{1}~\keyword{return}~\envElem{\grammarRule{Expr}}{2}
        \Rightarrow \left(\right) }$}
  \label{eq:dyn-ForClause-empty}
\end{dynamicrule}%
%
Otherwise, the dynamic evaluation rule of a \FOR clause with a positional variable is presented next:
%
\begin{dynamicrule}
    %
    \AxiomC{$\dynEnv{\envElem{Expr}{1} \Rightarrow \envElem{Item}{1}, \dotsb, \envElem{Item}{n}}$}
    %
    \UnaryInfC{$\statEnv \envElem{VarName}{}~\textbf{of var expands to}~\envElem{Variable}{}$}
    %
    \UnaryInfC{$\statEnv \envElem{VarName}{pos}~\textbf{of var expands to}~\envElem{Variable}{pos}$}
    %
    \UnaryInfC{$\dyn\envExtend{varValue}{\begin{array}{l}
            \envElem{Variable}{} \Rightarrow \envElem{Item}{1};\\
            \envElem{Variable}{pos} \Rightarrow 1 
          \end{array}}~\proofs \envElem{Expr}{2} \Rightarrow \envElem{Value}{1}$}
    \UnaryInfC{$\vdots$}
    \UnaryInfC{$\dyn\envExtend{varValue}{\begin{array}{l}
            \envElem{Variable}{} \Rightarrow \envElem{Item}{n};\\
            \envElem{Variable}{pos} \Rightarrow n 
          \end{array}}~\proofs \envElem{Expr}{2} \Rightarrow \envElem{Value}{n}$}
    \singleLine
    \UnaryInfC{$\dynEnv{
        \begin{array}{l}
          \FOR~\envElem{\var{VarName}}{}~\keyword{at}~\envElem{VarName}{pos}~\keyword{in}~\envElem{\grammarRule{Expr}}{1}\\
          \keyword{return}~\envElem{\grammarRule{Expr}}{2} \Rightarrow \envElem{Value}{1}, \dotsb, \envElem{Value}{n}
        \end{array}
      }$}
  \label{eq:dyn-ForClause}
\end{dynamicrule}%
%
Another judgement used for matching element values to types
is~\stringValue{\small{$\statEnv{\envElem{Value}{}~\textbf{matches}~\envElem{Type}{}}$}}, which holds when, in the
static environment~\stat, the type of \envElem{Value}{} is~\envElem{Type}{} or can be derived from~\envElem{Type}{} (as
presented in \cref{sec:xml-validation}).
%



%%% Local Variables:
%%% fill-column: 120
%%% TeX-PDF-mode: t
%%% TeX-debug-bad-boxes: t
%%% TeX-parse-self: t
%%% TeX-auto-save: t
%%% reftex-plug-into-AUCTeX: t
%%% mode: latex
%%% mode: flyspell
%%% mode: reftex
%%% TeX-master: "../thesis"
%%% End:





\section{Querying RDF with SPARQL}
\label{sec:sparql-preliminaries}


This section provides an overview of the SPARQL query language, which is the \ac{W3C} recommended query language for
\ac{RDF}.
%
We present the syntax and semantics of SPARQL and wrap-up with an overview of the new features introduced by the
forthcoming update to the SPARQL language, dubbed SPARQL~1.1.
%
The \ac{W3C} SPARQL specification consists of the following documents:
\begin{enumerate}[(i),noitemsep]
\item a \emph{query language} for \ac{RDF}~\cite{PrudhommeauxSeaborne:2008aa};
\item a protocol describing the interactions between a \emph{query engine} and \emph{query
    clients}~\cite{ClarkFeigenbaumTorres:2008aa}; and
\item the \ac{XML} serialisation of the results of a \SELECT and \ASK query~\cite{BeckettBroekstra:2008aa}. 
\end{enumerate}
% 
We will focus on the description of the SPARQL query language for \ac{RDF} by following the \ac{W3C} specification and the
semantics presented by~\citet{PerezArenasGutierrez:2009aa}.
%

\subsection*{Syntax}
\label{sec:sparql-syntax}

A \emph{SPARQL query} is defined by a triple~$Q=(P,G,V)$, where~$P$ is a \emph{graph pattern},~$G$ is an \ac{RDF}
\emph{dataset} and~$V$ is the \emph{result form}.
%
Considering a setting similar to rule-based query answering for relational databases, a SPARQL query can also be viewed
as:
%
$ V \leftarrow P$,
%
where~$V$ can be assumed as the \emph{head} of the query, while~$P$ is the
\emph{body}~\cite{PerezArenasGutierrez:2009aa}.
%
The next sections describe each component of SPARQL queries, namely \ac{RDF} datasets, the result form, and graph
patterns.


\paragraph*{RDF Dataset.}% 
%
An \ac{RDF} \emph{dataset} forms the input data provided to a SPARQL query and is composed of:
%
\begin{inparaenum}[(i)]
\item exactly one (unnamed) graph considered to be the \emph{default} graph; and
\item a set of named graphs of the form~$\tuple{n_{i},g_{i}}$, where~$n_{i}$ is a \ac{URI} corresponding to the
  \emph{name} of the graph and~$g_{i}$ is an \ac{RDF} graph.
\end{inparaenum}
%
In a SPARQL query, the default graph is specified using \FROM clauses, while the named graphs are indicated using
\FROMNAMED clauses.
%
Since a SPARQL query may contain several \FROM clauses, the default graph is taken as the \emph{RDF merge} of graphs
specified in all \FROM clauses (\cf~\cref{def:rdf-merge}).

The notion of \emph{active graph} is introduced in the evaluation semantics of SPARQL to distinguish which \ac{RDF}
graph the basic graph pattern is matched against.  At the start of a SPARQL query evaluation, the active graph is the
default graph and it is changed when a \GRAPH keyword is encountered in the graph pattern (as further explained below).
% 


\paragraph*{Result Form.}
%
The \emph{result form} specifies the output of a SPARQL query and may be one of the following four types:
%
\begin{description}[noitemsep]
\item[\SELECT:] returns the matched values (substitutions) for variables present in the query;
\item[\CONSTRUCT:] returns an \ac{RDF} graph that is created based on the specified \emph{template} and the
  substitutions obtained by executing the query;
\item[\ASK:] returns a boolean indicating if the graph pattern matches any of the data; and
\item[\DESCRIBE:] returns an \ac{RDF} graph that contains information regarding the resources contained in
  the query.
\end{description}
% 
For this thesis we focus primarily on \SELECT and \CONSTRUCT queries.\footnote{In \cref{cha:xsparql} we will refer to
  these result forms as \SparqlForClause and \ConstructClause, respectively.}
%
In the case of \SELECT queries, the result form is a set of variables and the result of the query consists of sequences
of variable bindings for these variables, determined according to the specified graph pattern.
%
In a \CONSTRUCT query, as presented in~\citet[Section 10.2]{PrudhommeauxSeaborne:2008aa}, the solutions of the graph
pattern are used to instantiate the \emph{template} provided. The result of a \CONSTRUCT query is an \ac{RDF} graph
obtained from the union of all instantiations of variables in the template that result in valid \ac{RDF} triples.
%
When a \CONSTRUCT template contains blank nodes, a different blank node label will be generated for each instantiation of
the template, \ie~blank nodes are only shared within the same solution.
%


\paragraph*{Graph Patterns.}% 
%
SPARQL is a graph-matching query language and its syntax directly reflects this.  The body (graph pattern) of a SPARQL
query consists primarily of \emph{triple patterns} that are matched against the \ac{RDF} data.  Triple patterns are
\ac{RDF} triples, possibly containing variables appearing in subject, predicate or object positions.
%
In the SPARQL syntax, a graph pattern follows the \WHERE keyword.

A simple form of graph pattern is a set of triple patterns, also called a \ac{BGP}.
%
Here, we present the syntax of SPARQL based on the definitions provided by~\citet{PerezArenasGutierrez:2009aa}, which
describes a normalised syntax based on 3-tuples:
%
\begin{definition}[Graph Patterns] 
  \label{def:graphpattern}
  Let~$\AU$,~$\AB$,~$\AL$ be defined as before. Furthermore, let~$\AV$ denote a set of variables disjoint from~$\AUBL$,
  \emph{graph patterns} are inductively defined as follows:
  % 
  \begin{itemize}[noitemsep]
  \item a tuple~$(s,p,o) \in \AULV \times \AUV \times \AULV$, called a \emph{triple pattern}, is a graph pattern;
  \item a set of triple patterns, called a \emph{\acf{BGP}}, is a graph pattern;
  \item if~$P$ and~$P'$ are graph patterns, then~$(P\ \AND\ P')$,~$(P\ \text{\OPTIONAL}\ P')$, and~$(P\ \text{\UNION}\ P')$ are graph
    patterns;
  \item if~$P$ is a graph pattern and~$i \in \AUV$, then~$(\text{\GRAPH}\ i\ P)$ is a graph pattern; and
  \item if~$P$ is a graph pattern and~$R$ is a \FILTER expression, then~$(P\ \text{\FILTER}\ R)$ is a graph pattern.
  \end{itemize}
  % 
  For any pattern~$P$, we write~$\vars{P}$ for the set of all variables occurring in~$P$.
  % 
  A \FILTER expression~$R$ can be composed from constants, elements of~$\AULV$, comparison operators~(\character{$=$},
  \character{$<$}, \character{$>$}, \character{$\leq$}, \character{$\geq$}), logical connectives~(\character{$\neg$},
  \character{$\land$}, \character{$\lor$}) and \emph{built-in} functions.
  % 
  Some of the available built-in functions include the unary functions: $\keyword{BOUND}$, $\keyword{isIRI}$,
  $\keyword{isURI}$, $\keyword{isBLANK}$, $\keyword{isLITERAL}$, $\keyword{STR}$, $\keyword{LANG}$, and
  $\keyword{DATATYPE}$.  A complete list of built-in functions is included
  in~\citet[Section~11]{PrudhommeauxSeaborne:2008aa}.
\end{definition}
%
\nd As is common practice in the definition of SPARQL queries, we do not consider blank nodes in graph patterns, and
thus do not include them in our definitions. However, this restriction does not affect the expressivity of SPARQL, since
blank nodes in query patterns can always be replaced equivalently with variables~\cite{PerezArenasGutierrez:2009aa}.
%
Although in definitions we rely on an algebraic formalism for the syntax of SPARQL, as
per~\citet{PerezArenasGutierrez:2009aa}, in the examples we follow the \ac{W3C} specification, which can be naturally
mapped to the algebraic form, where the \AND operator is represented by a dot (\character{.}).  The mapping between the
\ac{W3C} SPARQL syntax and the algebraic form we use is presented by~\citet{ArenasGutierrezPerez:2009ab}.
%
Thus, \cref{ex:sparql-query} presents a SPARQL query where the \PREFIX keyword declares a \ac{URI} prefix that is
used later in the query.
%
\begin{example}[SPARQL query] 
  \label{ex:sparql-query}
  %
  The following SPARQL query retrieves the names of persons that are members of the \stringValue{Nightwish} band:
  %
  \lstinputlisting[frame=none,numbers=none]{0-data+queries/members-nightwish.sparql}
  % 
\end{example}

\paragraph*{Solution Modifiers.}
\label{sec:solution-modifiers}

The evaluation of graph patterns generates a sequence of results initially with no specific \emph{order} (further
detailed in the following section).  Solution modifiers, such as \ORDERBY, \LIMIT, \OFFSET, and \DISTINCT can be applied
to this solution sequence.
%
The \ORDERBY modifier is used to specify an ordering for the sequence, specified as a list of variables present in the
solution sequence and the direction of the ordering (\keyword{ASC} or \keyword{DESC}).
%
Furthermore, the \DISTINCT modifier eliminates any duplicate solutions, while \LIMIT and \OFFSET are used to restrict
the number of solutions that are returned and to discard solutions from the beginning of the sequence, respectively.



\subsection*{Semantics}
\label{sec:sparql-semantics}


The semantics of SPARQL is defined based on the evaluation of \acp{BGP}, namely the \emph{matching} of the \acp{BGP}
against the supplied \ac{RDF} graph and the algebra that is built on top of this \ac{BGP} matching.
%
We start by presenting the notion of \emph{solution mappings}, which will be the results of the evaluation of \acp{BGP}
and then present how \emph{compatible} solution mappings can be combined in order to define the evaluation semantics of
SPARQL.
%
This evaluation algebra was presented by~\citet{Cyganiak:2005aa,PerezArenasGutierrez:2006aa} and later adapted to the
\ac{W3C} specification~\cite[Section~12.5]{PrudhommeauxSeaborne:2008aa}.

The matching of \acp{BGP} is performed against the previously mentioned \emph{active graph}, a specific \ac{RDF} graph
contained in the dataset of the query.  The active graph is initially set to the default graph of the dataset and is
changed whenever a \GRAPH keyword is processed.
%
This matching is represented by a function that maps query variables to \ac{RDF} terms present in the active graph and is
called a \emph{solution mapping}:
%
\begin{definition}[Solution Mapping~\cite{PrudhommeauxSeaborne:2008aa}]
  \label{def:solution-mapping}
  A \emph{solution mapping} is a partial function mapping SPARQL variables to \ac{RDF} terms.  The \emph{domain} of a
  solution mapping~$\mu$, $\dom{\mu}$, is the set of variables for which~$\mu$ is defined. We denote the value of
  variable~$v\in\AV$ according to solution~$\mu$ as~$\mu(v)$.
  % 
\end{definition}
% 
\nd The replacement of variables included in a graph pattern according to a solution mapping is defined next.
%
\begin{definition}[Variable Substitution]
  \label{def:variable-substitution}
  Let~$P$ be a graph pattern and~$\mu$ be a solution mapping.  
  %
  The \emph{variable substitution of~$P$ by~$\mu$}, denoted~$\mu(P)$, is the graph pattern~$P$ with all
  variables~$v\in\vars{P}\cap \dom{\mu}$ substituted by~$\mu(v)$.
  %
\end{definition}
%
\noindent It is worthy to note that if a solution mapping~$\mu$ contains bindings for all variables in a graph
pattern~$P$, \ie~$\dom{\mu} = \vars{P}$, and all triples in~$\funcCall{\mu}{P}$ are valid \ac{RDF} triples,
then~$\funcCall{\mu}{P}$ can be considered an \ac{RDF} graph.
%
If~$\mu$ provides bindings only for a subset of the variables present in the graph pattern~$P$,~$\mu(P)$ yields another
(more restrictive) graph pattern.
%
For the specification of the SPARQL algebra below, we introduce the notion of compatible solution mappings.
%
\begin{definition}[Compatible Mappings]
  \label{def:compatible-mappings}
  % 
  Let~$\mu_1$ and~$\mu_2$ be solution mappings,~$\mu_1$ and~$\mu_2$ are \emph{compatible} \iff for any~$v \in
  \dom{\mu_1} \cap \dom{\mu_2}$ it holds that~$\mu_1(v) = \mu_2(v)$.  The \emph{union} of two compatible
  mappings~$\mu_1$ and~$\mu_2$ consists of the standard set-theoretical union~$\mu_1 \cup \mu_2$.
  %
\end{definition}
%
\noindent The SPARQL relational
algebra~(see~\citet{Cyganiak:2005aa,PrudhommeauxSeaborne:2008aa,PerezArenasGutierrez:2009aa}) defines how to combine
solution mappings.
%
Our semantics of SPARQL is based on the semantics presented by~\citet{ArenasGutierrezPerez:2009ab}, where the SPARQL
algebra operators are extended to the multiset case by preserving the cardinality of solutions:\footnote{Following the
  notation of the operators presented by~\citet{ArenasGutierrezPerez:2009ab} we use the standard set operators.}
%
\begin{definition}[SPARQL Relational Algebra]
 \label{def:sparql-relalg}
 Let~$\Omega_1$ and~$\Omega_2$ be multisets of solution mappings:
 %
 \begin{tabbing}
   $\Omega_1 \leftouterjoin \Omega_2$\ \=$=$\ \=\kill
   $\Omega_1 \bowtie \Omega_2$\>$=$\ \>$\set{ \mu_1 \cup \mu_2 \mid \mu_1 \in \Omega_1, \mu_2 \in \Omega_2,\mu_1\textrm{ and }\mu_2\textrm{ compatible} }$\\
   $\Omega_1 \cup \Omega_2$\>$=$\>$\set{ \mu \mid \mu \in \Omega_1\textrm{ or } \mu \in \Omega_2 }$\\
   $\Omega_1 - \Omega_2$\>$=$\>$\set{ \mu_1 \in \Omega_1 \mid \textrm{ for all }\mu_2 \in \Omega_2,\mu_1\textrm{ and }\mu_2\textrm{ not compatible} }$\\
   $\Omega_1 \leftouterjoin \Omega_2$\>$=$\>$(\Omega_1 \bowtie \Omega_2) \cup (\Omega_1 - \Omega_2)$
   \end{tabbing}
\end{definition}


\noindent The definition of BGP matching from~\citet[Section~12.3]{PrudhommeauxSeaborne:2008aa} specifies the solutions
to a query.  We denote the evaluation of a \ac{BGP}~$P$ over a graph~$G$ as~$\eval{P}_{G}$:
%
\begin{definition}[Basic Graph Pattern Matching~{\cite[Section\ 12.3.1]{PrudhommeauxSeaborne:2008aa}}]
  \label{def:bgp-matching}
  Given a graph~$G$ and a \ac{BGP}~$P$, a \emph{solution}~$\mu$ for~$P$ over~$G$ is a mapping over~$V \subseteq
  \vars{P}$ such that~$G \models \funcCall{\mu}{P}$.  Following the definitions presented in \cref{sec:rdf},~$G \models
  \fc{\mu}{P}$, means that any triple in~$\fc{\mu}{P}$ is entailed by~$G$.
\end{definition}
%
\noindent This definition of \ac{BGP} matching relies on the underlying entailment notion, which according to the SPARQL
specification corresponds to simple graph entailment~\cite{Hayes:2004aa}.
%
Furthermore, in order to ensure the local scope of blank nodes, query solutions are taken from the \emph{scoping graph},
a graph that is equivalent to the active graph but does not share any blank nodes with it or any graph pattern within
the query.
%

% ----------------------------------------





%-----------------------------------
%
The evaluation semantics of more complex patterns including~\FILTER{\keyword{s}},~\OPTIONAL patterns,~\AND
patterns,~\UNION patterns is built on top of this \emph{basic graph pattern matching}, where each SPARQL operator is
mapped to an algebra expression:
%
\begin{definition}[Evaluation {\cite[Definition 2.2]{PerezArenasGutierrez:2009aa}}]
\label{def:semantics-sparql}
Let~$\tau = \triple{s,p,o}$ be a triple pattern,~$P, P_1, P_2$ graph patterns and~$G$ an \ac{RDF} graph, then the
evaluation~$\eval{\cdot}_{G}$ is recursively defined as follows: {
%
\begin{tabbing}
  $\eval{\tau}_{G}\ =\ \set{ \mu \mid \mathit{dom}(\mu)=\mathit{var}(P)\textrm{ and }G \models \mu(\tau) }$\\
  $\eval{P_1\ \FILTER\ P_2}_{G}\quad\quad$\=$=$\ \=$\eval{P_1}_{G}$\ \=$\leftouterjoin$\ \=$\eval{P_2}_{G}$\kill
  $\eval{P_1\ \AND\ P_2}_{G}$\>$=$\>$\eval{P_1}_{G}$\>$\ \bowtie$\>$\eval{P_2}_{G}$\\
  $\eval{P_1\ \UNION\ P_2}_{G}$\>$=$\>$\eval{P_1}_{G}$\>$\ \cup$\>$\eval{P_2}_{G}$\\
  $\eval{P_1\ \OPTIONAL\ P_2}_{G}$\>$=$\>$\eval{P_1}_{G}$\>$\leftouterjoin$\>$\eval{P_2}_{G}$\\
  $\eval{P\ \FILTER\ R}_{G}$\>$=$\>$\set{ \mu \in \eval{P}_{G} \mid R\mu \mbox{ is true } }$
\end{tabbing}}
%
\nd where~$R$ is a~\FILTER\footnote{For simplicity, we will omit from the presentation~\FILTER{\keyword{s}} such as
  comparison operators (\character{$<$}, \character{$>$}, \character{$\leq$}, \character{$\geq$}), data type conversion
  and string functions.  Further details are presented in~\citet[Section 11.3]{PrudhommeauxSeaborne:2008aa}.}
expression,~$u,v \in \AUBLV$. The valuation of~$R$ on a substitution~$\mu$, written~$R\mu$, is \emph{true} if:
%
{
\begin{tabbing}
(1) $R = \keyword{BOUND}(v)$ with~$v \in \mathit{dom}(\mu)$;\\
(2) $R = \keyword{isBLANK}(v)$ with~$v \in \mathit{dom}(\mu)$ and~$\mu(v) \in \AB$;\\
(3) $R = \keyword{isIRI}(v)$ with~$v \in \mathit{dom}(\mu)$ and~$\mu(v) \in \AU$;\\
(4) $R = \keyword{isLITERAL}(v)$ with~$v \in \mathit{dom}(\mu)$ and~$\mu(v) \in \AL$;\\
(5) $R = (u = v)$ with~$u,v \in \mathit{dom}(\mu)\cup \AUBL \wedge\ \mu(u) = \mu(v)$;\\
(6) $R = (\neg R_1)$ with~$R_1\mu\mbox{ is false}$;\\
(7) $R = (R_1 \vee R_2 )$ with~$R_1\mu\mbox{ is true}$ or~$R_2\mu\mbox{ is true}$;\\
(8) $R = (R_1 \wedge R_2)$ with~$R_1\mu\mbox{ is true}$ and~$R_2\mu\mbox{ is true}$.
\end{tabbing}}

\noindent$R\mu$ yields an error (denoted~$\varepsilon$), if: 
{
\begin{enumerate}[(1)]
\setlength{\itemsep}{-1.5mm}
%
\item $R = \keyword{isBLANK}(v)$,~$R = \keyword{isIRI}(v)$, or~$R = \keyword{isLITERAL}(v)$ 
  and~$v \not\in \mathit{dom}(\mu)$;
\item $R = (u = v)$ with~$u \not\in \mathit{dom}(\mu)\cup T$ or~$v\not\in \mathit{dom}(\mu)$;
\item $R = (\neg R_1)$ and~$R_1\mu=\varepsilon$;
\item $R = (R_1 \vee R_2 )$ and~$(R_1\mu\not=\mathrm{true}$ and~$R_2\mu\not=\mathrm{true})$ and~$(R_1\mu=\varepsilon$ or~$R_2\mu=\varepsilon)$;
\item $R = (R1 \wedge R2)$ and~$R_1\mu=\varepsilon$ or~$R_2\mu=\varepsilon$.
\end{enumerate}}
\noindent Otherwise~$R\mu$ is \emph{false}.
\end{definition}
%
\noindent The presented definition considers only safe~\FILTER{\keyword{s}} where, for a pattern~\inlineExpr{$P\
  \FILTER\ R$}, the filter~$R$ is said to be \emph{safe} if~$\vars{R} \subseteq \vars{P}$.
%
However, the SPARQL specification defines that in~\OPTIONAL{\keyword{s}}, any filter is scoped to the Group Graph
Pattern that contains the~\OPTIONAL.
%
As such, we include the definition that caters for unsafe~\FILTER{\keyword{s}}, introduced
by~\citet{AnglesGutierrez:2008aa}:
%
\begin{definition}[\OPTIONAL with \FILTER evaluation]
  \label{def:optsemantics}
  Let~$P_1, P_2$ be graph patterns and~$R$ a \FILTER expression. A mapping~$\mu$ is in~$\eval{P_1\ \OPTIONAL\ (P_2\
    \FILTER\ R)}_{G}$ \iff:
  \begin{itemize}[nosep]
  \item $\mu = \mu_1\cup \mu_2$, s.t.~$\mu_1 \in \eval{P_1}_{G}$,~$\mu_2 \in \eval{P_2 }_{G}$ are compatible and~$R\mu$
    is true, or
  \item $\mu \in \eval{P_1}_{G}$ and~$\forall \mu_2 \in \eval{P_2}_{G}$,~$\mu$ and~$\mu_2$ are not compatible, or
  \item $\mu \in \eval{P_1}_{G}$ and~$\forall \mu_2 \in \eval{P_2}_{G}$ s.t.~$\mu$ and~$\mu_2$ are compatible,
    and~$R\mu_{3}$ is false for~$\mu_{3} = \mu \cup \mu_2$.
  \end{itemize}
\end{definition}


%
Finally, the evaluation semantics of SPARQL consists of computing a \emph{sequence of solution mappings}, where any
existing solution modifiers are applied to the multiset of results.  If no solution modifiers are specified a default
ordering is assumed.
\begin{definition}[Solution sequences]
  Sequences of solution mappings are simply referred to as \emph{solution sequences}, often denoted by~$\omg{}{}$.
\end{definition}
%


%
These conditions of SPARQL \CONSTRUCT queries, informally specified in \cref{sec:sparql-syntax}, are reflected in
the following definition. 
%
Later, we will rely on this definition to show the equivalence of the newly introduced XSPARQL \CONSTRUCT expressions
and SPARQL \CONSTRUCT expressions.
%
\begin{definition}[SPARQL \CONSTRUCT semantics]
  %
  \label{def:sparql-construct}
  %
  Let~$C$ be a \ConstructTemplate and~$\Omega$ a \emph{solution sequence}. The SPARQL \CONSTRUCT returns an RDF graph
  generated by the set-theoretical union of the triples obtained from substituting variables in~$C$ with their bindings
  from~$\Omega$ and satisfying the following conditions:
 %
  \begin{enumerate}[label=(\arabic*),noitemsep]
  \item \label{def:sparql-construct-1} any invalid RDF triples that may be produced by the instantiation of the
    \ConstructTemplate are ignored; and
  \item \label{def:sparql-construct-2} blank node labels within the \ConstructTemplate are considered scoped to the
    template for each solution, \ie~if the same label occurs twice in a template, then there will be one blank node
    created for each solution in $\Omega$, but there will be different blank nodes for triples generated by different query
    solutions.  Blank nodes in the graph template be shared only within the same query solution~$\mu_i \in \Omega$.
  \end{enumerate}
  % 
\end{definition}
% 




\subsection*{Query Answering}
\label{sec:query-answering}

The SPARQL query language presented in the previous section can be viewed in a similar setting to the rule based
conjunctive queries presented for relational databases in \cref{sec:querying-rdb}.
%
Also inspired by~\citet{GutierrezHurtadoMendelzon:2004aa}, we assume that an RDF graph~$G$ is \emph{ground}, where all
blank nodes have been skolemised, \ie~consistently replaced with terms in~$\AUL$.
%
A \emph{query} is of the rule-like form:
%
$$q(\vec{x}) \leftarrow \exists \vec{y}.\varphi(\vec{x},\vec{y})$$
%
where~$q(\vec{x})$ is the \emph{head} and~$\exists \vec{y}.\varphi(\vec{x},\vec{y})$ is the \emph{body} of the query.
%
The body of the query is a conjunction of triples~$\tau_{i}$ ($1 \leq i \leq n$) and, similar to
\cref{sec:querying-rdb}, we use the symbol~\character{$,$} to denote conjunction in the rule body.
%
The vectors~$\vec{x}$ and~$\vec{y}$ are vectors of variables occurring in the body of the rule called the
\emph{distinguished variables} and \emph{non-distinguished variables}, respectively.  The variables in~$\vec{x}$ and~$\vec{y}$ are
disjoint and each variable occurring in~$\tau_{i}$ must be either distinguished or non-distinguished.
%

In a query, we allow \emph{built-in triples} of the form~$\triple{s,p,o}$, where~$p$ is a \emph{built-in predicate}
taken from a reserved vocabulary and having a \emph{fixed interpretation}. 
%
We generalise the built-ins to any~$n$-ary predicate~$p$, where~$p$'s arguments may be variables from~$\AV$
%
and values from~$\AUL$.  We will assume that the evaluation of the predicate can be decided in finite time.
%
For convenience, we write functional predicates\footnote{A predicate~$p(\vec{x},y)$ is functional if for any~$\vec{t}$
  there is a \emph{unique}~$t'$ for which~$p(\vec{t},t')$ is true.} as \emph{assignments} of the form~$x\assign
f(\vec{z})$ and assume that the function~$f(\vec{z})$ is safe (according to
\cref{def:rule-based-conjunctive-query}). We also assume that a non functional built-in predicate~$p(\vec{z})$ should
be safe as well.

\begin{example}[RDF conjunctive query]
  An example query is:
  \[q(n) \leftarrow \triple{x, \term{ex{:}memberOf}, y}, \triple{x, \term{foaf{:}name}, n}, \triple{y, \typeR,
    \term{mo{:}Band}}, \triple{y, \term{foaf{:}Name}, \stringValue{\term{Nightwish}}}\] \nd which intends to retrieve
  all persons names~$n$ that are members of a band~$y$ with the name \stringValue{\term{Nightwish}}.
\end{example}
In order to define an \emph{answer} to a query we introduce the following:

\begin{definition}[Query instantiation]
Given a vector~$\vec{x} = \tuple{x_1, \dots, x_k}$ of variables, a \emph{substitution} over~$\vec{x}$ is a vector of terms~$\vec{t}$ replacing variables in~$\vec{x}$ with terms of~$\AUBL$. Then,
given a query~$q(\vec{x}) \leftarrow \exists \vec{y}.\varphi(\vec{x},\vec{y})$, and two substitutions~$\vec{t},
  \vec{t'}$ over~$\vec{x}$ and~$\vec{y}$, respectively, the \emph{query instantiation}~$\varphi(\vec{t}, \vec{t}')$ is derived from~$\varphi(\vec{x},
  \vec{y})$ by replacing~$\vec{x}$ and~$\vec{y}$ with~$\vec{t}$ and~$\vec{t}'$, respectively.
\end{definition}

\nd Note that, similar to the variable substitution of a solution mapping in SPARQL
(\cf~\cref{def:variable-substitution}), if all tripes in a query instantiation are valid \ac{RDF} triples, the
query instantiation can be considered an RDF graph.

\begin{definition}[Entailment]
  \label{def:entailment}
  Given a graph~$G$, a query~$q(\vec{x}) \leftarrow \exists \vec{y}.\varphi(\vec{x},\vec{y})$, and a vector~$\vec{t}$ of
  terms in~$\universe(G)$, we say that~$q(\vec{t})$ is \emph{entailed} by~$G$, denoted~$G \models q(\vec{t})$, \iff in
  any model~$\I$ of~$G$, there is a vector~$\vec{t}'$ of terms in~$\universe(G)$ such that~$\I$ is a model of the query
  instantiation~$\varphi(\vec{t}, \vec{t}')$.
\end{definition}

\begin{definition}[Query Answers]
  If~$G \models q(\vec{t})$ then~$\vec{t}$ is called an \emph{answer} to~$q$. The \emph{answer set} of~$q$ \wrt~$G$ is
  defined as~$ans(G,q) = \set{ \vec{t} \mid G \models q(\vec{t}) }$.
\end{definition}
%
\noindent The notion of a solution for \acp{BGP} in SPARQL is the same as the notion of answers for conjunctive queries:
%
\begin{proposition} 
  \label{pp}
  Given a graph~$G$ and a \ac{BGP}~$P$, the solutions of~$P$ are the same as the answers of the query~$q(var(P))
  \leftarrow P$, \ie~$ans(G,q) = \eval{P}_{G}$.
\end{proposition}



%%% Local Variables:
%%% fill-column: 120
%%% TeX-PDF-mode: t
%%% TeX-debug-bad-boxes: t
%%% TeX-parse-self: t
%%% TeX-auto-save: t
%%% reftex-plug-into-AUCTeX: t
%%% mode: latex
%%% mode: flyspell
%%% mode: reftex
%%% TeX-master: "../thesis"
%%% End:



\subsection*{SPARQL~1.1}
\label{sec:sparql-11}

A new version of SPARQL, called SPARQL~1.1~\cite{HarrisSeaborne:2012aa}, is in the process of being proposed as a \ac{W3C} recommendation.  
%
This new version is composed of several documents specifying the updated query language and introduces new features that
were already used in practice by several SPARQL engines, such as
%
\begin{enumerate*}[nosep,label=(\roman*), before=\unskip{: }, after=\unskip{.}, itemjoin={{; }}, itemjoin*={{; and }}]
\item aggregates
\item subqueries
\item negation
\item assignment
\item property paths
\end{enumerate*}
% 
Other documents included in this new version, but not detailed in this section, specify an Update
language~\cite{GearonPassantPolleres:2012aa} and extensions for federated
querying~\cite{PrudhommeauxBuil-Aranda:2011aa}.


Aggregates allow expressions to be applied over groups of solutions to obtain a single value, for example determining
the minimum (\keyword{min}) value of the group.  Other aggregator functions included in the standard are
\keyword{count}, \keyword{sum}, \keyword{max}, \keyword{avg}, and \keyword{group\mathunderscore{}concat}.
%
Although the use of aggregate functions was already available in several SPARQL engines, it will only be introduced into
the official \ac{W3C} specification with SPARQL~1.1.
%


In SPARQL~1.1, nested \SELECT queries are allowed to be used in graph patterns and the projected variables of the
subquery are then joined with the results of the outer query.  These nested \SELECT queries are however not allowed to
specify a dataset and are restricted to be executed over the same dataset as the outer query.  SPARQL follows a
bottom-up query evaluation and thus the inner queries are evaluated first and its results made available to the outer
query.
%
A proposal for subqueries in SPARQL was previously presented by~\citet{AnglesGutierrez:2010aa} and later the same authors
compared different forms of subqueries to the W3C semantics~\cite{AnglesGutierrez:2011aa}.
%



Although negation was already permitted in SPARQL by using a combination of the \FILTER and \BOUND operators, this is
made explicit in SPARQL~1.1 by allowing two forms of negation: the \keyword{exists} and \keyword{minus}.
%
The \keyword{exists} (and \keyword{not exists}) \FILTER expression allows to test if a graph pattern matches (or does
not match) the dataset and consequently remove such solutions from the results.
%
The other form of negation uses the \keyword{minus} operator that, when applied to two graph patterns, removes solutions
from the left-hand side compatible with any solution from the right-hand side.
%
Since the \keyword{minus} operator relies on the notion of compatible solutions, it will only remove solutions if there
are shared variables between the solution sequences it is applied to.
%
This causes different results between the two forms of negation when the provided graph patterns do not share
variables:\footnote{A special case of graph patterns that do not share any variables is when the pattern to be removed
  contains no variables, \ie~is a fixed pattern.}
%
since no two solutions are compatible, the \keyword{minus} operator does not remove any solutions from the resulting
sequence.  However, the \keyword{exists} operator will remove the respective solutions from the final sequence.
%

%
SPARQL~1.1 includes a basic query federation by means of the \keyword{SERVICE} keyword, which specifies that the
following subquery will be executed in a remote SPARQL endpoint.

Other features include assignment of variables in the graph pattern (using the \keyword{bind} operator), in the \SELECT
clause, and in the \GROUPBY clause.  All assignments are of the form
%
\inlineExpr{\lstinline[basicstyle=\small\ttfamily]{(expression AS $var)}},
% 
where \lstinline[basicstyle=\small\ttfamily]{expression} is the expression to be evaluated and
%
\lstinline[basicstyle=\small\ttfamily]{$var}
%
is the variable name the result of the expression is assigned to.
%
Another form of assignment is using the \keyword{bindings} clause,\footnote{Note that the SPARQL~1.1 syntax is still
  under development and the \keyword{\footnotesize{bindings}} clause may be changed to \keyword{\footnotesize{values}}.}
which allows to specify a solution sequence that is to be joined with the results of the graph pattern.  The values for
variables in the provided solution sequence must be \ac{RDF} terms, \ie~no variables can be specified.  The
\keyword{bindings} clause is envisioned to be used with the \keyword{service} keyword to specify values for federated
querying.


Property paths are used to specify a connection between two \ac{RDF} nodes.  An extended graph pattern syntax is defined
that allows for a concise pattern matching, for example specifying alternative routes for connecting the nodes, or to
match paths of arbitrary or specific lengths.





%%% Local Variables:
%%% fill-column: 120
%%% TeX-PDF-mode: t
%%% TeX-debug-bad-boxes: t
%%% TeX-parse-self: t
%%% TeX-auto-save: t
%%% reftex-plug-into-AUCTeX: t
%%% mode: latex
%%% mode: flyspell
%%% mode: reftex
%%% TeX-master: "../thesis"
%%% End:




\section{Conclusion}
\label{sec:conclusion-query-languages}

This chapter introduced the \ac{SQL}, XQuery, and SPARQL query languages that allow to access data in the formats
presented in \cref{cha:data-models}.
%
Each query language focuses on a specific data model, namely \ac{SQL} for relational data, XQuery for \ac{XML} data, and
SPARQL for \ac{RDF} data.  For \ac{XML}, we briefly presented the \ac{XPath} and \ac{XSLT} languages, which are closely
related to XQuery.

We briefly introduced the syntax and semantics of each language, with special focus on XQuery and SPARQL, which will be
used in the next chapters to define the novel transformation language, called XSPARQL, and the extension of SPARQL
towards querying meta-information.
%
XSPARQL integrates the \ac{SQL}, XQuery, and SPARQL query languages presented in this chapter, thus allowing to combine
data from the different data models.


%%% Local Variables:
%%% mode: latex
%%% mode: flyspell
%%% mode: reftex
%%% TeX-master: "../thesis"
%%% End:
