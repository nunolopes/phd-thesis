
\section{Querying Relational Databases}
\label{sec:querying-rdb}

In this section we give an overview of conjunctive queries which, according to~\citet{AbiteboulHullVianu:1995aa},
represent the vast majority of relational database queries that are relevant in practice.  Later we present the \ac{SQL}
query language, which is the most used query language for relational databases.

\subsection{Conjunctive queries}
\label{sec:conjunctive-queries}
%
In line with the different views on relational data (presented in \cref{sec:relational-model}), conjunctive queries can
be formalised under different, although equivalent, perspectives: logic programming and the relational algebra.
%
The logic programming approach follows the corresponding view on relational data presented in
\cref{sec:relational-model}, while the relational algebra approach relies on the conventional view.
%
We then present the \acs{SQL} query language and provide an overview of its mapping into relational algebra.

Under the logic programming approach, in addition to the sets of relations \AR, attributes \AAs, and values \AD, we rely
on the set of variables \AV that range over elements of \AD.
%
We can now extend the notion of fact to \emph{atom}: an atom over a relation~$r \in \AR$ is an
expression~$\fact{r}{e_{1}, \dots, e_{n}}$ where~$n$ is the arity of~$r$ and each~$e_{i} \in \ADV$ is called a
\emph{term}.  A fact can also be referred to as a \emph{ground atom}.  The notion of \emph{query} can then be defined
as:
%
\begin{definition}[Rule-based conjunctive query~\cite{AbiteboulHullVianu:1995aa}]
  \label{def:rule-based-conjunctive-query}
  Given a database schema~$S$, a \emph{rule-based conjunctive query}~$q$ over~$S$ is an expression of the form:
  %
  \begin{equation}
    \fact{q}{u} \leftarrow \fact{r_{1}}{u_{1}}, \dots, \fact{r_{n}}{u_{n}}\label{eq:1}
  \end{equation}
  % 
  where each~$r_{i}, i \in [1,n]$ is a relation name from~$S$ and each~$\fact{r_{i}}{u_{i}}$, is an atom over~$r_{i}$.
  Any variable occurring in~$u$ must be \emph{safe}, \ie~it must also occur at least once in any~$u_{1}, \dots, u_{n}$.
  Furthermore, we denote the set of variables present in~$q$ as \vars{q}.
\end{definition}
%
\noindent A rule-based conjunctive query can be referred to simply as a \emph{rule} where the lefthand side
of~\character{$\leftarrow$} is called the \emph{head} and the righthand side is called the \emph{body} of the rule.  For
example, in Rule~\eqref{eq:1}, the head is~$q(u)$ and the body corresponds to~$\fact{r_{1}}{u_{1}}, \dots,
\fact{r_{n}}{u_{n}}$.
% 
A rule can be interpreted as: the head atom can be deduced if there are values for the variables in the rule that make
the body hold.
%
Given a set of variables~$V \subset \AV$, a \emph{mapping} (or \emph{valuation}) over~$V$ is a function~$v : V \to \AD$.
This function can be extended to represent the identity over any element of~$\AD$ and thus can map any atom with
variables to a fact by applying it to all elements of the atom.  For any atom~$t$, the mapping of~$v$ is denoted as
\funcCall{v}{t}.
%
Based on this notion of mapping, the \emph{answers} to a query can be defined as:

\begin{definition}[Answers of a query]
  Let~$q = \fact{q}{u} \leftarrow r_{1}, \dots, r_{n}$ be a rule-based conjunctive query and~$I$ be a database instance.
  An \emph{answer} of~$I$ under~$q$ is:
  \[
  \funcCall{q}{I} = \set{ \funcCall{v}{r} \mid v\mbox{ is a mapping over}~\vars{q}~\mbox{and }\funcCall{v}{r_i} \in I
  \mbox{ for each }i \in [1, n] }
  \]
\end{definition}
%
As we will see in the next section, since duplicate removal is a computational expensive operation, \ac{SQL} maintains
duplicates in the answers of a query (thus represented as a multiset) unless otherwise instructed.

Another paradigm for relational queries is the algebraic paradigm, which is defined by specifying operations on relation
instances, called the \emph{relational algebra}.
%
The three primitive algebra operators are the \emph{selection}~($\sigma$), \emph{projection}~($\pi$) and the
\emph{cartesian product}~($\times$) operators.
%
The full set of operators that form the \emph{relational algebra} also include the \emph{join}~($\bowtie$),
\emph{union}~($\cup$) and \emph{difference}~($-$) operators.  
%
The selection operator consists of restricting the tuples present in a relation according to a specified condition.
%
The projection operator is used to discard attributes of a relation while the cartesian product combines any two
relations and produces a new relation that includes all the attributes of both relations.
%
The join operator consists of combining the projection and cartesian product operators: the result of this operator
consists of all the tuples from both relations that have a common value on any common attributes. 
%
The union~($\cup$) and difference~($-$) algebra operators are defined as the standard set-theoretical operators.
% 

Considering two relational instances~$I$ and~$J$ (with sorts~$U$ and~$W$, respectively), the relation attributes~$A$
and~$B$, and a constant~$c \in \AD$, the relational algebra operators are defined as:
% 
\begin{description}
\item[selection:] The two forms of the selection operator,~$\sigma_{A=c}$ and~$\sigma_{A=B}$ select tuples that match a
  constant or where the value of two attributes is the same, respectively:
  %
  \begin{align*}
    \sigma_{A=c}(I) &= \set{ t \in I \mid \funcCall{t}{A} = c } \\
    \sigma_{A=B}(I) &= \set{ t \in I \mid \funcCall{t}{A} = \funcCall{t}{B} }
  \end{align*}
  %
\item[projection:] This operator consists of restricting the attributes present in a relation. Given a set of
  attributes~$X \subseteq U$, the projection operator returns:
  \[
  \pi_{X}(I) = \set{ t[X] \mid t \in I }
  \]
\item[join:] The join operator between~$I$ and~$J$ produces a relation with sort~$U \cup W$, such that:
  \[
  I \bowtie J = \set{ t \mid \funcCall{sort}{t} = U \cup W\mbox{ and }t[U] = u\mbox{ and }t[W] = w\mbox{ for some }u \in I\mbox{ and }w \in J }
  \]
\end{description}
% 


\subsection{SQL}
\label{sec:sql}

The \ac{SQL} is the most widely available query language for relational data, supported by most commercial relational
database management systems~\cite{AbiteboulHullVianu:1995aa} and is an American National Standards Institute (ANSI)
standard.
%
The core of SQL queries consist of the commonly known \emph{select-from-where} queries, which are equivalent in
expressivity to conjunctive queries.  An example of a \ac{SQL} query is shown in \cref{ex:sql-query}.
%
\begin{example}[SQL query]
  %
  \label{ex:sql-query}
  The following \ac{SQL} query, when executed against a database instance following the schema presented in
  \cref{ex:bands-schema}, extracts the names of the artists that are in the \stringValue{Nightwish} band:
  %
\begin{lstlisting}[language=SQL,frame=none,numbers=none]
SELECT persons.personName
FROM persons, bands
WHERE persons.bandId = bands.bandId
      AND bands.bandName = 'Nightwish'
\end{lstlisting}
\end{example}
%
The \SELECT keyword specifies the attributes that should be present in the query results, while \FROM specifies the
relations names over which the query will be evaluated.  It is possible to write \character{*} in place of attribute
names in a \SELECT clause if all the attributes in the relations specified in the \FROM clause are to be returned.
%
In SQL, relation names are considered as variables that range over tuples occurring in the corresponding relation and,
as shown in \cref{ex:sql-query}, can be used in the \WHERE clause to specify the relations and attributes.
%
If a query requires more than one variable ranging over the same relation, they can be specified in the \FROM clause and
assigned different \emph{aliases} for the relation, \eg~\lstinline{FROM person p1, person p2}.  Furthermore, if the
attributes are distinct, it is possible to omit the relation name from the \SELECT query.
%
Finally, the \WHERE keyword specifies the conditions that any result of the query must satisfy in order to be included
in the result set and can express conjunction, disjunction, negation, and nesting.  It is possible to completely omit
the \WHERE clause and, in this case, all tuples of the cartesian product of the relations specified in the query are
returned.


It is also possible to represent nested queries by using the keywords \IN and \NOTIN.  These keywords behave as
operators over sets, testing the inclusion (or not) of an element in the set resulting from the nested query.
%

% 
The result of a \ac{SQL} \emph{select-from-where} query evaluation consists of a multiset of tuples, \ie~there may be
repeated answers in the results.  The explicit use of the \DISTINCT keyword after the \SELECT keyword removes any
duplicate answers from the resulting set.
%
\ac{SQL} also includes several aggregate functions, such as \keyword{count}, \keyword{sum}, and \keyword{average}, which
perform the specified function over the resulting multisets, such as counting number of elements in the collection, or
adding the elements of a multiset composed of numeric elements.
%
Furthermore grouping of tuples, by means of the \keyword{group by} operator, also allows to create collections of tuples
over which aggregates can be applied.
%

%
Included in the SQL specification is also the definition of a DML.  This part of the SQL language allows the schema of a
relational database to be manipulated, for instance creating new relations, altering the structure of existing ones, or
removing existing relations.


\subsubsection{Translation to Relational Algebra}
\label{sec:sql-semantics}

A translation from a subset of \ac{SQL} into relational algebra was presented by~\citet{CeriGottlob:1985aa}, while the 
%
semantics catering for the full syntax of \ac{SQL} and the three-valued logic inherent with \NULL{s}, was presented
by~\citet{NegriPelagattiSbattella:1991aa}.
%
In this thesis, we follow the translation of \ac{SQL} \emph{select-from-where} queries into relational algebra as
presented by~\citet{AbiteboulHullVianu:1995aa}:
\begin{inparaenum}[(i)]
\item the \SELECT keyword behaves as the projection operator~$\pi$;
\item the \FROM keyword corresponds to the cartesian product operator~$\times$; and
\item the \WHERE keyword specifies a selection operation~$\sigma$.
\end{inparaenum}
%
In the rest of the thesis, we denote this translation of a \ac{SQL} query~$S$ into relational algebra
as~$\RASQL{S}$.
%
\begin{example}[SQL translation into Relational Algebra]
  The translation into relational algebra of the query in \cref{ex:sql-query} is:
  %
  \[
  \pi_{\mathit{persons.personName}}(\sigma_{\mathit{persons.bandId} = \mathit{bands.bandId} \land \mathit{bands.bandId} = \mathit{'Nightwish'}}(\mathit{persons} \times \mathit{bands}))
  \]
\end{example}
%


\subsubsection{Mapping SQL Results to XML}
\label{sec:mapping-xml-types}
%
\begin{table}[t]\centering
  \caption{Mapping from SQL to XML datatypes}
  \label{fig:SQL2XML}
  { \scriptsize
    \begin{tabular}{l|l}
      \toprule
      \textbf{SQL datatype} & \textbf{XML datatype}\\
      \midrule
      character string & xs:string\\
      numeric, decimal& xs:decimal\\
      boolean & xs:boolean \\
      smallint, integer, bigint & xsd:integer\\
      float, real, double precision & xsd:double\\
      date & xsd:date\\
      time & xsd:time\\
      timestamp & xsd:dateTime\\
      \bottomrule
    \end{tabular}
  }
\end{table}
%
The mapping from SQL datatypes into \ac{XML} Schema datatypes is defined in the \ac{SQL} specification and was presented
in~\citet{EisenbergMelton:2001aa}.  
%
An overview of this mapping is presented in \cref{fig:SQL2XML}.
%
Since \ac{XML} datatypes generically allow a wider range of valid values, it is common for concrete mappings to impose
further restrictions on \ac{XML} datatypes, however, for this thesis we will omit these restrictions on the \ac{XML}
datatypes.
%
Later, in \cref{sec:semantics}, we will rely on this mapping of datatypes for the translation of \ac{SQL} to \ac{XML}
and we refer to the \ac{XML} representation of \ac{SQL} values as~\funcCall{sql2xml}{\mathit{SQLValue}} and vice-versa
as~\funcCall{xml2sql}{\mathit{XMLValue}}.
%



%%% Local Variables:
%%% fill-column: 120
%%% TeX-PDF-mode: t
%%% TeX-debug-bad-boxes: t
%%% TeX-parse-self: t
%%% TeX-auto-save: t
%%% reftex-plug-into-AUCTeX: t
%%% mode: latex
%%% mode: flyspell
%%% mode: reftex
%%% TeX-master: "../thesis"
%%% End:
