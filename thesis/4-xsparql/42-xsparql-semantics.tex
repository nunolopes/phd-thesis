\section{Semantics}
\label{sec:semantics}

Next we define the semantics of XSPARQL by reusing the semantics of \ac{SQL} and SPARQL. 
%
We start by defining how we ensure that the semantics of \ac{SQL} and SPARQL queries respects any existing XSPARQL
variable bindings.
%
In \cref{sec:chang-xquerys-semant}, we present the extensions to the \ac{W3C} XQuery's
semantics~\cite{DraperFankhauserFernandez:2010aa}, namely the new types we use, an extension to the normalisation rules
of XQuery \ForClause{s}, and necessary additional environment components.
%
\cref{sec:sparqlforclause-sem} presents the semantics of the newly introduced expressions: \SparqlForClause,
\ConstructClause, and \SQLForClause, based on XQuery's formal semantics~\cite{DraperFankhauserFernandez:2010aa}, by
defining normalisation, static type and dynamic evaluation rules for each of the new expressions.



\subsection{XSPARQL Types}
\label{sec:xsparql-types}
%
We extend the \acl{XDM} (described in \cref{sec:xpath-data-model}) with the following new types to accommodate
for \ac{SQL} and SPARQL specific parts of XSPARQL:
%
\begin{enumerate}[(1),noitemsep]
\item\label{enum:t0} the \type{SQLTerm} is an extension of \qname{xs}{anyAtomicType} (as presented in
  \cref{sec:mapping-xml-types});
\item\label{enum:t1} the \type{RDFTerm} type further consists of the subtypes \type{uri}, \type{bnode} and
  \type{literal} and is used as the type of SPARQL variables;
\item\label{enum:t2} the \type{PatternSolution} type consists of a sequence of pairs \seq{\grammarRule{variableName},
  \type{RDFTerm}}, representing \ac{SQL} or SPARQL variable bindings;
\item\label{enum:t3} the \type{RDFGraph} is the type resulting from the evaluation of \CONSTRUCT expressions; and
\item\label{enum:t4} the \type{RDFDataset} is the type used for representing \ac{RDF} datasets, which is further
  constituted by one default \type{RDFGraph} and a sequence of named graphs (\type{RDFNamedGraph}).
\end{enumerate}
%
\cref{verb:types} presents the formal definition of \ref{enum:t0}--\ref{enum:t4} following the notation for \ac{XML}
Schema datatypes (presented in \cref{sec:xml-schema}).
%
The \type{RDFTerm} type is used to represent RDF terms (composed of \acp{URI}, blank nodes or literals).  The type of
SPARQL variables is represented by the \type{Binding} type, that consists of the variable name and the RDF term that is
assigned to it.
%
Finally, sequences of SPARQL variable bindings are represented by the type \type{PatternSolution}.  

\begin{figure}
  \centering
\begin{tabular}{|cc|}
\hline
\begin{minipage}{.45\linewidth}
\begin{Verbatim}[framesep=2mm,frame=none,fontsize=\scriptsize,commandchars=\\\(\)]

define type \textbf(URI-reference) restricts xs:anyURI;

define type \textbf(Literal) extends xs:string {
      attribute datatype of type URI-reference?, 
      attribute lang of type xml:lang? };

define type \textbf(RDFTerm) { 
      element uri of type URI-reference |
      element bnode of type xs:string |
      element literal of type Literal };

define type \textbf(SPARQLBinding) extends RDFTerm { 
      attribute name of type xs:string };

define type \textbf(SPARQLResult) { 
      element binding of type SPARQLBinding* };

define type \textbf(SQLTerm) extends xs:anyAtomicType ;

define type \textbf(SQLBinding) extends SQLTerm { 
      attribute name of type xs:string };

define type \textbf(SQLResult) { 
      element binding of type SQLBinding* };

define type \textbf(SQLAttribute) extends xs:string ;

  \end{Verbatim}
\end{minipage}
&
\begin{minipage}{.45\linewidth}
\begin{Verbatim}[frame=none,fontsize=\scriptsize,commandchars=\\\(\)]

define type \textbf(PatternSolution) { 
      element result of type SPARQLResult |
      element result of type SQLResult };

define type \textbf(RDFGraph) { 
      element triple of type RDFTriple* };

define type \textbf(RDFTriple) { 
      element subject of type RDFTerm,
      element predicate of type RDFTerm,
      element object of type RDFTerm };

define type \textbf(RDFDataset) { 
      element defaultGraph of type RDFGraph,
      element namedGraphs of type RDFNamedGraphs };

define type \textbf(RDFNamedGraphs) { 
      element namedGraph of type RDFNamedGraph* };

define type \textbf(RDFNamedGraph) { 
      attribute name of type xs:string,
      element graph of type RDFGraph };
  \end{Verbatim}
\end{minipage}%
\\
\hline
\end{tabular}
\caption{XSPARQL Type Definitions}
\label{verb:types}
\end{figure}


Similarly for \ac{SQL} results, sequences of \ac{SQL} variable bindings are also represented by the type
\type{PatternSolution}.
%
Analogously, we define the types \type{SQLResult} and \type{SQLBinding} for representing \ac{SQL} results.  The
\type{SQLBinding} type is defined as an extension of \qname{xs}{anyAtomicType}, and we follow the mapping from \ac{SQL}
types into \ac{XML} types presented in \cref{fig:SQL2XML}.


The \type{RDFGraph} type corresponds to a sequence of \type{RDF\-Triple}{\type{s}}, which are in turn a complex type
composed of \type{sub\-ject}, \type{predicate} and \type{object}.
%
The \type{RDFDataset} type is defined as an \type{RDFGraph} that is considered the default graph and a sequence of
\type{RDFNamed\-Graph}{\type{s}} represented by the \type{name} of the graph and the corresponding \type{RDFGraph}.




\subsubsection{Translating SQL and SPARQL Solutions into the \type{PatternSolution} Type}
\label{sec:transl-sql-sparql}

The next definition presents the translation between a SPARQL solution sequence and a sequence of \type{SPARQLResult}
type elements that we implement in XSPARQL.
%
This serialisation of SPARQL results mimics the SPARQL Query Results XML Format~\cite{BeckettBroekstra:2008aa}, defined
by the XML Schema available at {\url{http://www.w3.org/2007/SPARQL/result.xsd}}.
%
\begin{definition}[Serialisation of Solution Sequences]
  \label{lem:solution2types}
  Given a SPARQL solution sequence $\Omega = (\mu_1, \dotsc, \mu_n)$ a serialisation of $\Omega$ into a sequence of
  \type{PatternSolution} is defined as follows:
  \begin{itemize}[noitemsep]
  \item $\funcCall{serialise}{\Omega} \Rightarrow \funcCall{serialise}{\mu_1}, \ldots, \funcCall{serialise}{\mu_n}$
  \item $\funcCall{serialise}{\mu} \Rightarrow 
    \stt{<result>} 
    \{\forall x \in \dom{\mu}, \funcCall{serialise}{\mu, x}\}
    \stt{</result>}
    $
  \item $\funcCall{serialise}{\mu, x} \Rightarrow  
    \stt{<binding name="}x\stt{">}
    \{\funcCall{term}{\funcCall{\mu}{x}}\}
    \stt{</binding>}
    $,
    %
    where \funcCall{term}{\mu(x)} is 
    \begin{itemize}[noitemsep]
    \item $\stt{<uri>}\{\mu(x)\}\stt{</uri>} \hfill \textrm{if}~\mu(x) \in \AU$
    \item $\stt{<bnode>}\{\mu(x)\}\stt{</bnode>} \hfill\textrm{if}~\mu(x) \in \AB$
    \item $\stt{<literal>}\{\mu(x)\}\stt{</literal>}\hfill\textrm{if}~\mu(x) \in \AL$
    \end{itemize}

  \end{itemize}
  %
\end{definition}
%
\noindent Following the definition of the \funcName{serialise} function, in evaluation rules, we will refer to sequences
of elements of type \type{PatternSolution} as \omg{}{} and to elements of type \type{SPARQLResult} as \sm{}{}.

For the representation of \ac{SQL} results we follow a similar approach:
%
\begin{definition}[Serialisation of SQL Relation Instances]
  \label{lem:relation2types}
  % 
  The serialisation of a relation instance~$I = \seq{I_1, \ldots, I_n}$ of relation~$R$ with~$\funcCall{sort}{R} = U$,
  into \type{PatternSolution} is:
  % 
  \begin{itemize}[noitemsep]
  \item $\funcCall{serialise}{I} \Rightarrow \funcCall{serialise}{I_1}, \ldots, \funcCall{serialise}{I_n}$
  \item $\funcCall{serialise}{I_i} \Rightarrow
    \stt{<result>} \{\forall x \in U, \funcCall{serialise}{I_i, x}\} \stt{</result>}
    $
  \item $\funcCall{serialise}{I_i, x} \Rightarrow  
    \stt{<binding name="}x\stt{">} \{ \funcCall{sql2xml}{\funcCall{I_i}{x}} \} \stt{</binding>}
    $.
  \end{itemize}
  %
\end{definition}



\subsubsection{Serialisation into SQL and SPARQL Representations}
\label{sec:sql-sparql-serialisation}

%
The following definitions present the \funcName{SQLTerm} and \funcName{RDFTerm} functions that, when applied to an
\ac{XSD} datatype, return their representation in \ac{SQL} or SPARQL syntax, respectively.  We first present the
serialisation into \ac{SQL}:
%
\begin{definition}[SQL representation]
  %
  \label{def:sqlTerm}
  %
  Let~$C$ be an expression context with static environment~$T_C = \funcCall{\stat}{C}$ and dynamic environment~$D_C =
  \funcCall{\dyn}{C}$, and~$x \in \funcCall{dom}{T_C.varType}$ an XSPARQL variable name.
  %
  The \ac{SQL} representation of~$x$ according to~$C$, denoted~$\funcCall{SQLTerm_C}{x}$ is:
  %
  \begin{itemize}[noitemsep]
  \item~$\funcCall{data}{D_C.\funcCall{\ecomp{varValue}}{x}}$ \hfill if $T_C.\funcCall{\ecomp{varType}}{x} =
    (\type{SQLTerm} \textrm{ or } \type{SQLAttribute} \textrm{ or } \type{RDFTerm}  \textrm{ or } \type{node()})$; and
  \item~$\funcCall{xml2sql}{D_C.\funcCall{\ecomp{varValue}}{x}}$ \hfill otherwise, 
  \end{itemize}
  % 
  where \funcName{xml2sql} is the value conversion function presented in \cref{sec:mapping-xml-types}.
  %
\end{definition}
%
Similarly, we next present the serialisation of SPARQL terms:
%
\begin{definition}[\funcName{RDFTerm}]
  % 
  \label{def:sparqlTerm}
  %
  Let~$C$ be an expression context with static environment~$T_C = \funcCall{\stat}{C}$ and dynamic environment~$D_C =
  \funcCall{\dyn}{C}$, and~$x \in \funcCall{dom}{T_C.varType}$ an XSPARQL variable name.
  %
  The \ac{RDF} representation of~$x$ according to~$C$, denoted~$\funcCall{RDFTerm_C}{x}$ is:
  % 
  \begin{itemize}[noitemsep]
  \item~$D_C.\funcCall{\ecomp{varValue}}{x} \hfill \textrm{if } T_C.\funcCall{\ecomp{varType}}{x} = \type{RDFTerm}$,
  \item~$\texttt{"}D_C.\funcCall{\ecomp{varValue}}{x}\texttt{"} \hfill\textrm{if } T_C.\funcCall{\ecomp{varType}}{x} = \type{xsd:string}$,
  \item~$\dt{\funcCall{D_C.\ecomp{varValue}}{\ensuremath{x}}}{\qname{rdf}{XMLLiteral}} \hfill \textrm{if }
    T_C.\funcCall{\ecomp{varType}}{x} = \type{element()}$,
  \item~$\texttt{"}data(D_C.\funcCall{\ecomp{varValue}}{x})\texttt{"} \hfill \textrm{if } T_C.\funcCall{\ecomp{varType}}{x} =
    (\type{attribute()} \textrm{ or } \type{SQLTerm} \textrm{ or } \type{SQLAttribute})$, and
  \item~$\texttt{"}D_C.\funcCall{\ecomp{varValue}}{x}\texttt{"\^{}\^{}}T_C.\funcCall{\ecomp{varType}}{x}$ \hfill
    otherwise.
  \end{itemize}
\end{definition}

%%% Local Variables:
%%% fill-column: 120
%%% TeX-master: t
%%% TeX-PDF-mode: t
%%% TeX-debug-bad-boxes: t
%%% TeX-parse-self: t
%%% TeX-auto-save: t
%%% reftex-plug-into-AUCTeX: t
%%% mode: latex
%%% mode: flyspell
%%% mode: reftex
%%% TeX-master: "../thesis"
%%% End:



\subsection{XSPARQL Semantics for Querying Relational and RDF data}
\label{sec:query-heter-data}

We now define the semantics of \SQLForClause{s} and \SparqlForClause{s} by relying on the evaluation semantics of their
original query languages, namely \ac{SQL} (presented in \cref{sec:sql}) and SPARQL (presented in
\cref{sec:sparql-preliminaries}).
%
The approach we take is to rely on the translation of each language into their respective algebra expressions and
further combine these algebra expressions with any existing XSPARQL variable bindings.
%
Since XSPARQL is based on the semantics of XQuery, variable bindings are stored in the \ecomp{varValue} environment
component of the dynamic environment (\cf~\cref{sec:xquery}), that maps variable names to their value.
%
Next we present how we interpret these variable mappings as a relation and as a solution sequence, thus allowing to
combine the results of \ac{SQL} and SPARQL queries with the existing variable bindings.


\subsubsection{Querying Relational Data}
\label{sec:extending-sql}


In order to reuse the semantics of \ac{SQL} for defining the semantics of XSPARQL \SQLForClause{s} we transform the
\ecomp{varValue} component of the dynamic environment in which the \SQLForClause is executed into a relation (which we
call the \emph{XSPARQL instance relation}).
%
The following definition presents this translation:
%
\begin{definition}[XSPARQL instance relation]
  \label{def:xml2sqlterm}
  Let the set of relation names (\AR) be defined as in \cref{sec:relational-model}, and let~$C$ be an expression
  context.
  %
  The \emph{XSPARQL instance relation} of~$C$ is a relation instance named \lit{$xir_C$}, where $xir_C$ is a reserved
  relation name, \ie~$xir_C \not\in \AR$, and~$\funcCall{sort}{xir_C} = \dom{\funcCall{\dyn}{C}.\ecomp{varValue}}$.
  %
  For each mapping~$v_{i} \rightarrow x_{i} \in \funcCall{\dyn}{C}.\ecomp{varValue}$, the value of~$xir_C$ for
  attribute~$v_i$, denoted~$\funcCall{xir_C}{v_{i}}$, is defined as:
  %
  \begin{itemize}[noitemsep]
  \item if~$x_{i} = \seq{}$ is an empty sequence then~$\funcCall{xir_C}{x_{i}} = \NULL$;
  \item if~$x_{i} = \seq{e_{1}, \dotsc, e_{n}}$ is a sequence, then~$\funcCall{xir_C}{x_{i}} =
    \funcCall{fn{:}concat}{\funcCall{SQLTerm_C}{e_{1}}, \dotsb, \funcCall{SQLTerm_C}{e_{n}}}$.\footnote{Since the values
      of any relation attribute must be atomic, in the case of a variable being assigned to an XQuery sequence we assume
      the concatenation of each element of the sequence.}
    %
  \end{itemize}
  %
   For a \synt{SQLWhereClause}~$S$, we call the XSPARQL instance relation of the expression context in which~$S$ is
  executed the \emph{XSPARQL instance relation of~$S$}.
\end{definition}
%

%
Another necessary step to enable the reuse of \ac{SQL} evaluation semantics is to convert our extended syntax (that
allows for \var{}-prefixed variable names) into valid \ac{SQL} syntax: each \synt{WhereSpec} in a \SQLForClause that
contains an XSPARQL variable is removed from the normalised \ac{SQL} query (by replacing it with \lit{true}) and is
stored for a later evaluation by the XSPARQL semantics.  For this we rely on the following normalisation function:
%
\begin{definition}[\ac{SQL} Representation of \synt{SQLWhereClauses}]
  %
  \label{def:normalisation-where}
  %
  Let~$S = \lit{where}~\synt{WhereSpecList}$ be a \synt{SQLWhereClause}.  The normalisation of~$S$,
  $\funcCall{normaliseSQL}{S} = \lit{where}~\funcCall{normaliseSQL}{\synt{WhereSpecList}}$, where
  \funcCall{normaliseSQL}{\synt{WhereSpecList}} is defined as:
  % 
  \begin{itemize}
  \item if \synt{WhereSpecList} is of form~$\lit{(}\ \synt{WhereSpecList}_{1}\ \synt{Op}\ \synt{WhereSpecList}_2\
    \lit{)}$ then
    %
    \[
    \lit{(}\ \funcCall{normaliseSQL}{\synt{WhereSpecList}_1}\ \synt{Op}\ \funcCall{normaliseSQL}{\synt{WhereSpecList}_2}\ \lit{)}
    \]
  \item if~\synt{WhereSpecList} is of form~$Attr_1~\synt{Op}~Attr_2$ then~$\funcCall{normaliseSQL}{Attr_1~Op~Attr_2}$
    is:
    \[
    \begin{dcases*}
      \lit{true}        & if~$Attr_1$  or~$Attr_2$ is an XSPARQL variable \\
      Attr_1~Op~Attr_2  & otherwise.
    \end{dcases*}
    \]
  \end{itemize}
  %
  Furthermore we denote the set of \synt{WhereSpec} of~$S$ in which an attribute is an XSPARQL variable as
  \funcCall{whereSpecVars}{S}.
  % 
\end{definition}
%
\noindent
%
The normalisation of complete \SQLForClause{s} consists also of the normalisation of the syntactical elements
\synt{AttrSpecList} and \synt{TableSelector} presented in \cref{sec:syntax-rdb}.
%
In the normalisation of~\synt{AttrSpecList} we remove any existing \synt{AttrNameSpec} component, since they reflect
only the name of the corresponding XSPARQL variable.
%
However, the normalisation of the \synt{TableSelector} can only be performed during the dynamic evaluation of the
XSPARQL query since any variables present in the \synt{TableSelector} must be evaluated to determine the corresponding
relation name.
%
With the restriction of performing the substitution at evaluation time, we can reuse the standard translation of a
\ac{SQL} query into relational algebra as presented in \cref{sec:sql-semantics}.

Next we present how XSPARQL combines the results of a \ac{SQL} query with an XSPARQL instance mapping.
%
For this we rely on the standard relational selection~($\sigma$) and cross-product~($\times$) algebra operators
presented in \cref{sec:querying-rdb} and on the~$xir_C$ relation instance from \cref{def:xml2sqlterm}.
%
Firstly, we present the construction of the relational algebra select expression that, based on the provided
\SQLForClause~$S$ and the XSPARQL instance mapping of~$S$, makes the connection between the results of the \ac{SQL}
query and the existing XSPARQL variable bindings:
%
\begin{definition}[XSPARQL $\sigma$ expression]
  \label{def:xsparql-join}
  %
  Let~$S$ be a \SQLForClause with expression context~$C$ and~$V = \funcCall{whereSpecVars}{S}$ the attribute
  specifications that contain XSPARQL variables in~$S$.
  %
  The \emph{XSPARQL $\sigma$ expression} of~$S$, denoted~$\funcCall{\sigma_{xs}}{S}$, is a relational algebra~$\sigma$
  expression that, for each $Att_1~Op~Att_2 \in V$ is~$\funcCall{trans}{Att_1}~Op~\funcCall{trans}{Att_2}$,
  where~$\funcCall{trans}{Att}$ is defined as:
  % 
  \begin{itemize}[noitemsep]
  \item $Attr$ \hfill if~$Attr$ is not an XSPARQL variable;
  \item if~$Attr = \lit{\$}AttrName$ is an XSPARQL variable then 
    %
    \[
    \funcCall{trans}{Attr} = \begin{dcases*}
        \dyn.\funcCall{\ecomp{varValue}}{AttrName} & if~\stat.\funcCall{\ecomp{varType}}{Attr} = \type{SQLAttribute}\\
        \textrm{`}xir_C.AttrName\textrm{'} & otherwise.
      \end{dcases*}
    \]
  \end{itemize}
\end{definition}
%
\noindent
%
This definition creates a relational algebra expression from the extended XSPARQL \SQLForClause syntax, which can then
be used to further restrict the results of the normalised \ac{SQL} expression.
%
Based on these definitions we can introduce the translation of \SQLForClause{s} into relational algebra.
%
\begin{definition}[XSPARQL relational algebra expression]
  \label{def:xsparql-sql-answers}
  %
  Let~$Q$ be a \SQLForClause,~$Q' = \funcCall{normaliseSQL}{Q}$ the \ac{SQL} rewriting of~$Q$,~$E =
  \funcCall{\sigma_{xs}}{S}$ the XSPARQL~$\sigma$ expression of~$S$, and~$\RASQL{Q'}$ the relational algebra expression
  obtained from the standard \ac{SQL} translation into relational algebra.
  %
  The \emph{XSPARQL relational algebra expression} of~$Q$, denoted~$\RAXSP{Q}$, combines the relational algebra
  expression of the \ac{SQL} query and restricts its results to the existing bindings for XSPARQL variables as follows:
  %
  \[
  \funcCall{\sigma_{E}}{\RASQL{Q'} \times xir_C} \enspace .
  \]
  %
\end{definition}
%
The following example illustrates the translation of XSPARQL \SQLForClause{s} into XSPARQL relational algebra
expressions.
%
\begin{figure}[t]
  \subfloat[Value Matching]{\label{ex:sql-for-ex1}
    \begin{minipage}{.5\linewidth}
      \lstinputlisting{0-data+queries/sql-for-clause-ex1.xsparql}
    \end{minipage}
  }
  % 
  \subfloat[Attribute Matching]{\label{ex:sql-for-ex2}
    \begin{minipage}{.5\linewidth}
      \lstinputlisting{0-data+queries/sql-for-clause-ex2.xsparql}
    \end{minipage}
  }
  \caption{XSPARQL \SQLForClause examples}%
  \label{fig:sql-for-ex}%
\end{figure}
%
\begin{example}[Translation of \SQLForClause{s} into Relational Algebra]
  %
  \cref{fig:sql-for-ex} presents two XSPARQL queries including \SQLForClause{s}.
  %
  The query in \cref{ex:sql-for-ex1} illustrates the syntax for querying values of a relation.  First the
  normalisation function drops the restriction in line~3, which is incorporated into the relational algebra~$\sigma$
  expression:
  % 
  \[
  \funcCall{\sigma_{band.bandId = xir_C.x}}{\funcCall{\sigma_{band.bandName = 'Nightwish'}}{band} \times xir_C} \enspace ,
  \]
  where~$\funcCall{sort}{xir_C} = \set{x}$ and~$\funcCall{xir_C}{x} = 1$.

  \medskip

  On the other hand, the query in \cref{ex:sql-for-ex2} shows how to match attribute names.
  %
  The query in this figure is converted into the following relational algebra expression:
  % 
  \[
  \funcCall{\sigma_{band.bandName = 'Nightwish'}}{\funcCall{\sigma_{band.bandId = 1}}{band} \times xir_C} \enspace ,
  \]
  %
  where~$\funcCall{sort}{xir_C} = \set{x}$ and~$\funcCall{xir_C}{x} = \textrm{`bandName'}$.  
  %
\end{example}







%%% Local Variables:
%%% fill-column: 120
%%% TeX-master: t
%%% TeX-PDF-mode: t
%%% TeX-debug-bad-boxes: t
%%% TeX-parse-self: t
%%% TeX-auto-save: t
%%% reftex-plug-into-AUCTeX: t
%%% mode: latex
%%% mode: flyspell
%%% mode: reftex
%%% TeX-master: "../thesis"
%%% End:




\subsubsection{Querying RDF Data}
\label{sec:bf-extended-bgp}
%
For querying \ac{RDF} data, we extend the notion of SPARQL \ac{BGP} (\cref{def:bgp-matching}) in order to provide SPARQL
with the variable bindings from XQuery.
%
For this we interpret the XQuery \ecomp{varValue} dynamic environment component as a set of bindings in the spirit of
SPARQL solution mappings (as presented in \cref{def:solution-mapping}).
%
Along these lines, we will regard the \ecomp{varValue} component of the dynamic environment in which a SPARQL graph
pattern~$P$ is executed as the basis for the \emph{XSPARQL instance mapping} of~$P$.  The transformation from the
\dyn.\ecomp{varValue} into the XSPARQL instance mapping is defined next:
%
\begin{definition}[XSPARQL instance mapping]
  \label{def:xml2rdfterm}
  Let~$C$ be an expression context, and furthermore let~$D_C = \funcCall{\dyn}{C}.\ecomp{varValue}$ and~$T_C =
  \funcCall{\stat}{C}.\ecomp{varType}$ the \ecomp{varValue} component of the dynamic environment of~$C$ and be the
  \ecomp{varType} component of the static environment of~$C$, respectively.
  %
  The \emph{XSPARQL instance mapping}~$\mu_C$ is a solution mapping where, for each mapping~$v_{i} \rightarrow x_{i} \in
  D_C$,~$x_{i}$ is converted into an instance of type \type{RDFTerm} or an \ac{RDF} \emph{Collection} according to the
  following conditions:
  %
  \begin{itemize}[noitemsep]
  \item if~$\funcCall{D_C}{v_{i}} = \seq{}$ and~$\funcCall{T_C}{v_{i}} = \type{RDFTerm}$ or~$\funcCall{T_C}{v_{i}} =
    \type{SQLTerm}$ then~$\funcCall{\mu_C}{\funcCall{D_C}{v_{i}}}$ is undefined;
  \item if~$\funcCall{D_C}{v_{i}} = \seq{}$ and~$\funcCall{T_C}{v_{i}} \neq \type{RDFTerm}$ and~$\funcCall{T_C}{v_{i}}
    \neq \type{SQLTerm}$ then~$\funcCall{\mu_C}{\funcCall{D_C}{v_{i}}} = \seq{}$ is an empty \ac{RDF} Collection;
  \item if~$\funcCall{D_C}{v_{i}}$ is a singleton sequence then~$\funcCall{\mu_C}{\funcCall{D_C}{v_{i}}} =
    \funcCall{RDFTerm}{\funcCall{D_C}{v_{i}}}$;
  \item if~$\funcCall{D_C}{v_{i}} = \seq{e_{1}, \dotsc, e_{n}}$,~$n > 1$, is a sequence
    then~$\funcCall{\mu_C}{\funcCall{D_C}{v_{i}}} = \seq{\funcCall{RDFTerm}{e_{1}} \dotsb \funcCall{RDFTerm}{e_{n}}}$ to
    be read as an \ac{RDF} Collection in Turtle notation (\cf~\cref{sec:turtle}).
  \end{itemize}
  % 
  For a graph pattern~$P$, we call the XSPARQL instance mapping of the expression context in which~$P$ is executed the
  \emph{XSPARQL instance mapping of~$P$}.
\end{definition}
%
Next we define the notion of XSPARQL \ac{BGP} matching based on the semantics of SPARQL \ac{BGP} matching presented in
\cref{sec:sparql-semantics}.
%
\begin{definition}[Extended solution mapping]\label{def:extended-solution-mapping}
  Let~$C$ be an expression context. An \emph{extended solution mapping} of a graph pattern~$P$ in~$C$ is a solution
  mapping \emph{compatible} with the \emph{XSPARQL instance mapping} of~$C$.
\end{definition}
% 
\noindent Accordingly, XSPARQL \ac{BGP} matching is defined analogously to the SPARQL \ac{BGP} matching with the
exception that we consider only extended solution mappings:
%
\begin{definition}[XSPARQL \ac{BGP} matching]\label{def:xsparql-bgp-matching}
  Let~$P$ be a \ac{BGP} with expression context~$C$, and~$G$ be an \ac{RDF} graph. We say that~$\mu$ is a
  \emph{solution} for~$P$ with respect to active graph~$G$, if there exists an extended solution mapping~$\mu'$ of~$C$
  such that~$\funcCall{\mu'}{P}$ is a subgraph of~$G$ and~$\mu$ is the restriction of~$\mu'$ to the variables
  in~$\vars{P}$.
\end{definition}
%
\noindent This definition quasi injects the variable bindings inherited from XQuery into SPARQL patterns occurring
within XSPARQL.  By considering \emph{extended solution mappings} the bindings returned for a \ac{BGP}~$P$ will not only
match the input graph~$G$ but also respect any bindings for variables in the dynamic environment.
%
We can extend the XSPARQL \ac{BGP} matching to generic graph patterns by following the SPARQL evaluation semantics
(presented in \cref{sec:sparql-semantics}).
%
Considering a graph pattern~$P$ with XSPARQL instance mapping~$\mu_C$, we denote by~$\evalXS{D}{P}{\mu_C}$ the
evaluation of~$P$ over dataset~$D$ following XSPARQL \ac{BGP} matching.
 
\subsubsection{Matching Blank Nodes in Nested Queries}
\label{sec:bgpmatch}
%
Although in XSPARQL, similar to SPARQL, we are not considering blank nodes in the semantics definitions of graph
patterns, in the case of nested \SparqlForClause{s} XSPARQL instance mappings may in fact contain assignments of
variables to blank nodes, injected from the outer \SparqlForClause into the inner \SparqlForClause.
%
\begin{query}
  \lstinputlisting[language=XSPARQL]{0-data+queries/nested-bnodes.xsparql}%
  \caption{Nested XSPARQL query}%
  \label{qr:nested-bnodes}%
\end{query}
%
\begin{example}[Blank node injection in XSPARQL nested queries]
  %
  For example, in \cref{qr:nested-bnodes}, blank nodes bound in the outer \SparqlForClause
  (\crefrange{qr:nested-bnodes_out-s}{qr:nested-bnodes_out-e}) to the variable~\var{song} will be injected into
  the inner \SparqlForClause expression (\crefrange{qr:nested-bnodes_in-s}{qr:nested-bnodes_in-e}).
  %
  If we would consider both \SparqlForClause{s} as distinct SPARQL queries, the blank nodes in the inner
  \SparqlForClause would be matched as variables.
\end{example}
%
However in XSPARQL, we want to enable coreference within nested queries over the same dataset and thus such injected
blank nodes should be matched like constants against the blank nodes present in the input \ac{RDF} data (rather than
being treated as variables).
%
To ensure this behaviour, we introduce the notion of \emph{active dataset} (similar to the concept of active graph in
SPARQL), where nested queries over the same active dataset keep the same the scoping graphs
(\cf~\cref{sec:sparql-semantics}).
%
Any \SparqlForClause with an \emph{explicit} \DatasetClause causes the \emph{active dataset} to change, \ie~new scoping
graphs (with fresh blank nodes) for each graph within it are created.  On the other hand, if no \DatasetClause is
present in a nested \SparqlForClause (implicit dataset), the active dataset remains
unchanged.
%
To ensure this behaviour in the dynamic evaluation we have to introduce a new dynamic environment component called
\ecomp{activeDataset}, that will be used to evaluate \SparqlWhereClause{s}.  Initially, this component is empty (or set
to a system default) and is changed by a \DatasetClause appearing in a \SparqlForClause{}, as defined in the next
section.



%%% Local Variables:
%%% fill-column: 120
%%% TeX-master: t
%%% TeX-PDF-mode: t
%%% TeX-debug-bad-boxes: t
%%% TeX-parse-self: t
%%% TeX-auto-save: t
%%% reftex-plug-into-AUCTeX: t
%%% mode: latex
%%% mode: flyspell
%%% mode: reftex
%%% TeX-master: "../thesis"
%%% End:




%%% Local Variables:
%%% fill-column: 120
%%% TeX-master: t
%%% TeX-PDF-mode: t
%%% TeX-debug-bad-boxes: t
%%% TeX-parse-self: t
%%% TeX-auto-save: t
%%% reftex-plug-into-AUCTeX: t
%%% mode: latex
%%% mode: flyspell
%%% mode: reftex
%%% TeX-master: "../thesis"
%%% End:




\subsection{Extensions to the XQuery Semantics}
\label{sec:chang-xquerys-semant}

In order to define the XSPARQL semantics according to XQuery's semantics we need to introduce new environment components
and extend the dynamic evaluation rules of XQuery \ForClause{s} to populate these new components.
%
We also introduce the functions that we will use in the dynamic evaluation rules presented in the next section.


\subsubsection{New Environment Components}
\label{sec:new-envir-comp}

For the definition of the XSPARQL semantics we add the following components to the dynamic environment:
%
\begin{itemize}[(i),noitemsep]
\item \ecomp{activeDataset}; and
\item \ecomp{globalPosition}.
\end{itemize}

The \dyn.\ecomp{activeDataset} is used to store the dataset over which \SparqlForClause{s} are evaluated in order to be
accessible when a nested \SparqlForClause without a \DatasetClause is specified.

The other introduced environment component, \dyn.\ecomp{globalPosition}, stores all the positions in the tuple streams.
%
The standard XQuery dynamic evaluation rules can only access the position of the current tuple stream, however, in order
to generate distinct blank node labels for each \ConstructClause, we need to guarantee that the labels are also distinct
in case of nested queries.
%
To ensure this, we store not only the position in the current tuple stream but also the positions of all previous ones.


Both environment components are populated in the dynamic evaluation rules introduced in
\cref{sec:new-semantics-rules}.  For the \dyn.\ecomp{globalPosition} we also need to adapt the evaluation rules
of XQuery \ForClause{s} to correctly populate this component.  These updated rules are presented next.


% using globalPosition
\subsubsection{XQuery \FOR Dynamic Evaluation}
\label{sec:norm-for}
%
In order to correctly generate blank node identifiers in \ConstructClause{s}, XSPARQL relies on the
\dyn.\ecomp{globalPosition} environment component to store the positions.
%
As such, we adapt the XQuery \FOR dynamic evaluation rules, presented in \cref{sec:xquery}, to populate the
\dyn.\ecomp{globalPosition} component and also make sure that the newly introduced XSPARQL \SQLForClause{s} and
\SparqlForClause{s} populate this component.  The case of these newly XSPARQL expressions is detailed later in
\cref{sec:new-semantics-rules}.


We show here only the adapted rule for \ForClause{s} with position variables and without type declaration.  The rules
that handle \FOR expressions without position variables and possibly containing type declarations are adapted
analogously, adding the \dyn.\ecomp{globalPosition} premisses to the rules presented
in~\citet[Section~4.8.2]{DraperFankhauserFernandez:2010aa}.
%
\begin{dynamicrule}
    % 
    \AxiomC{$\dyn.\ecomp{globalPosition} = \seq{ \envElem{Pos}{1}, \cdots, \envElem{Pos}{m} } $}
    %
    \UnaryInfC{$\dynEnv{\envElem{Expr}{1} \Rightarrow \envElem{Item}{1}, \dotsb, \envElem{Item}{n}}$}
    %
    \UnaryInfC{$\statEnv \envElem{VarName}{}~\textbf{of var expands to}~\envElem{Variable}{}$}
    %
    \UnaryInfC{$\statEnv \envElem{VarName}{pos}~\textbf{of var expands to}~\envElem{Variable}{pos}$}
    %
    \UnaryInfC{$\begin{array}{r@{\hspace{-1pt}}l}%
        \dyn & \envExtend{globalPosition}{\seq{ \envElem{Pos}{1}, \cdots, \envElem{Pos}{m}, 1 }} \\
        & \envExtend{varValue}{\begin{array}{l}
            \envElem{Variable}{} \Rightarrow \envElem{Item}{1};\\
            \envElem{Variable}{pos} \Rightarrow 1
          \end{array}}
      \end{array}~\proofs \envElem{Expr}{2} \Rightarrow \envElem{Value}{1}$}
    %
    \UnaryInfC{$\vdots$}
    %
    \UnaryInfC{$\begin{array}{r@{\hspace{-1pt}}l}%
        \dyn & \envExtend{globalPosition}{\seq{ \envElem{Pos}{1}, \cdots, \envElem{Pos}{m}, n }} \\
        &\envExtend{varValue}{\begin{array}{l}
            \envElem{Variable}{} \Rightarrow \envElem{Item}{n};\\
            \envElem{Variable}{pos} \Rightarrow n 
          \end{array}}
      \end{array}~\proofs \envElem{Expr}{2} \Rightarrow \envElem{Value}{n}$}
    %
    \singleLine
    %
    \UnaryInfC{$\dynEnv{
        \begin{array}{l}
          \FOR~\envElem{\var{VarName}}{}~\keyword{at}~\envElem{\var{VarName}}{pos}~\keyword{in}~\envElem{\grammarRule{Expr}}{1}\\
          \keyword{return}~\envElem{\grammarRule{Expr}}{2}
        \end{array} \Rightarrow \envElem{Value}{1}, \dotsb, \envElem{Value}{n}
      }$}
  \label{eq:dyn-ForClause-adapt}
\end{dynamicrule}%
%


%%% Local Variables:
%%% fill-column: 120
%%% TeX-master: t
%%% TeX-PDF-mode: t
%%% TeX-debug-bad-boxes: t
%%% TeX-parse-self: t
%%% TeX-auto-save: t
%%% reftex-plug-into-AUCTeX: t
%%% mode: latex
%%% mode: flyspell
%%% mode: reftex
%%% TeX-master: "../thesis"
%%% End:




\subsubsection{New Formal Semantics Functions}
\label{sec:form-semant-funct}

Next we will introduce the new XQuery formal semantics functions that we use in the static and dynamic evaluation rules
presented in the next section.  These functions are specified in an informal style, in a similar fashion to formal
semantics functions presented in~\citet[Section~7.1]{DraperFankhauserFernandez:2010aa} and the XQuery~1.0 and XPath~2.0
Functions and Operators specifications~\cite{MalhotraMeltonWalsh:2010aa}.  For each function, we present its signature,
consisting of the function name, the function parameters, and the return type, and include a textual description of the
semantics of the function.

The first introduced functions, \funcName{fs{:}sql} and \funcName{fs{:}sparql}, represent the extended \ac{SQL} and
SPARQL querying facilities implemented in XSPARQL (described in \cref{sec:query-heter-data}).
%
We further introduce two auxiliary functions~$\funcName{fs{:}value}$,~$\funcName{fs{:}dataset}$,
and~$\funcName{fs{:}evalCT}$.  These functions are used to access the value of a variable in a \type{PatternSolution},
to determine the dataset over which a \SparqlForClause is evaluated, and to evaluate a \CONSTRUCT query, \ie~a
\ConstructTemplate, respectively.
%


\paragraph{\funcName{\bf fs{:}sql}}
%
This function is responsible for executing the extended XSPARQL \ac{SQL} querying presented in
\cref{sec:extending-sql}.
%
In our semantics this function also implements the normalisation of $\grammarRule{SQLWhereClauses}$ (presented in
\cref{def:normalisation-where}) by receiving two parameters:~$\grammarRule{RelationList}$ and
$\grammarRule{SQLWhereClause}$ representing the list of relations involved in the query and the \ac{SQL} \emph{pattern}
to be executed, respectively.
%
The static type signature of this function is defined as:
%
{\small
\begin{verbatim}
fs:sql($SparqlWhere as xs:string)
  as PatternSolution*
\end{verbatim}
}%
%
The replacement of variables in \texttt{SQLWhereClauses} represented by \cref{def:normalisation-where} (that this
function implements), produces a valid \ac{SQL} query, that can be evaluated directly by the relational engine.
%
The results of this query are then translated into an instance of \type{PatternSolution} (according to
\cref{lem:relation2types}).


\paragraph{\funcName{\bf fs{:}sparql}.}
\label{sec:fs:sparql}
%
The \funcName{fs{:}sparql} function corresponds to the implementation of the \funcName{\e{eval}{xs}{}} function, that
evaluates SPARQL graph patterns and implements the extended notion of BGP Matching presented
in~\cref{def:xsparql-bgp-matching}.
%
The static type signature of this function is defined as:
%
{\small
\begin{verbatim}
fs:sparql($dataset as RDFDataset, $SparqlWhere as xs:string, $solutionModifiers as xs:string)
  as PatternSolution*
\end{verbatim}
}%
% 
\noindent The result of this function consists of a solution sequence, which can be translated directly into an XQuery
sequence of elements of type~\type{PatternSolution} by applying the~$\funcName{serialise}$ function
(\cf~\cref{lem:solution2types}).





\paragraph{\funcName{\bf fs{:}value}.}
\label{sec:fs:value}
%
The $\funcCall{fs{:}value}{PS, var}$ function returns the value of the specified variable $var$ in a
\type{Pattern\-Solution} specified by~$PS$.  If $var$ is not present in $PS$, the empty sequence is returned. 
%
The static type signature of this function is:
%
{\small
\begin{verbatim}
fs:value($ps as PatternSolution, $variable as xs:string)
  as (RDFTerm | SQLTerm)?
\end{verbatim}
}%
%
\noindent This function returns the respective \type{Binding} for the variable, which is an element of type
\type{SQLTerm} or \type{RDFTerm}, depending on whether the pattern solution was the result of a \SQLForClause or a
\SparqlForClause.


\paragraph{\funcName{\bf fs{:}dataset}.}
\label{sec:fs:dataset}
%
The $\funcCall{fs{:}dataset}{\DatasetClause}$ auxiliary function returns an element of type \type{RDFData\-set} based on
the evaluation of its argument.  
%
This conversion is performed according to the SPARQL semantics presented in \cref{sec:sparql-preliminaries}.  The result
of this function is stored (by dynamic evaluation rules) in the newly introduced \ecomp{activeDataset} dynamic
environment component and can be retrieved when a \SparqlForClause without an explicit \DatasetClause is found.
%
The static type signature of this function is:
%
\begin{center}
\begin{minipage}{0.9\linewidth}
{\small\centering
\begin{verbatim}
fs:dataset($datasetClause as xs:string)
  as RDFDataset
\end{verbatim}
}
\end{minipage}
\end{center}
%

\paragraph{\funcName{\bf fs{:}evalCT}.}
\label{sec:fs:evalCT}
%
The $\funcName{fs{:}evalCT}$ function ensures the created RDF graph is valid and rewrites any blank nodes inside of
\ConstructTemplate{s} to comply with the SPARQL semantics (as described in \cref{sec:bgpmatch}).  The auxiliary
$\funcName{fs{:}validTriple}$ function checks if each triple is valid according to the RDF semantics and is defined by
rules~\eqref{validTriple1} and~\eqref{validTriple2} presented in the next section.
%
The \funcName{fs{:}evalCT} function is further detailed in the following section by presenting specific rules that
ensure the generated \ac{RDF} graph is valid and to guarantee the generation of new blank node labels for each pattern
solution.
%
The static type signatures of these functions are defined as:
%
{\small
\begin{verbatim}
fs:evalCT($template as RDFTerm*)
  as RDFGraph

fs:validTriple($subject as RDFTerm, $predicate as RDFTerm,  $object as RDFTerm)
  as RDFTriple?
\end{verbatim}
}
%
\noindent The \funcName{fs{:}evalCT} function, and hence \CONSTRUCT expressions, return elements of type
\type{RDFGraph}, thus allowing the result of \CONSTRUCT expressions to be used in a \DatasetClause of a subsequent
\SparqlForClause.

\subsection{Semantics Rules for XSPARQL Expressions}
\label{sec:new-semantics-rules}

We are now ready to present the normalisation, static, and dynamic evaluation rules for the newly defined XSPARQL
expressions.
%
As presented in \cref{sec:syntax}, XQuery and SPARQL namespace declarations can be used interchangeably in the
prolog of an XSPARQL query and thus we start by presenting the translation of the query prolog into XQuery namespace
declarations via normalisations rules.
%
We then present the necessary normalisation, static, and dynamic evaluation rules for \SQLForClause{s},
\SparqlForClause{s}, and \ConstructClause{s}.
%
Please note that, since the variables included in \SQLForClause{s} and \SparqlForClause{s} are not allowed to contain a
namespace prefix, we omit the rules handling the namespace expansion for the respective variables.


\subsubsection{Query Prolog Normalisation}
\label{sec:query-prol-norm}

%
In order to follow the XQuery semantics, we convert any SPARQL syntax prefix declaration into XQuery namespace
declarations by the following normalisation rules: %
%
\begin{normalisationrule}
  \mapping{%
    \sem{\PREFIX~\grammarRule{NCName} \keyword{:}~\keyword{<}\grammarRule{IRI}\keyword{>}}{Expr}%
  }{%
    \sem{\keyword{declare~namespace}~\grammarRule{NCName} \keyword{ = "} \grammarRule{IRI} \keyword{" ;}}{Expr}%
  }
  \label{eq:2}
\end{normalisationrule}%
%
The empty \PREFIX declaration is converted into the default namespace for \ac{XML} elements: 
%
\begin{normalisationrule}
  \mapping{%
    \sem{\PREFIX ~ \keyword{:}~\keyword{<}\grammarRule{IRI}\keyword{>}}{Expr}%
  }{%
    \sem{\keyword{declare~default~element~namespace} \keyword{ = "} \grammarRule{IRI} \keyword{" ;}}{Expr}%
  }
  \label{eq:3}
\end{normalisationrule}%
%
Furthermore the SPARQL \keyword{base} declaration is considered equivalent to the XQuery~\keyword{base-uri} declaration:
%
\begin{normalisationrule}
  \mapping{%
    \sem{%
      \keyword{base} ~ \keyword{<}\grammarRule{IRI}\keyword{>}}{Expr}%
  }{%
    \sem{\keyword{declare~base{-}uri "}\grammarRule{IRI} \keyword{" ;}}{Expr}%
  }
  \label{eq:4}
\end{normalisationrule}%




\subsubsection{\SQLForClause}
\label{sec:semantics-rdb}

In this section we define the semantics of the newly introduced \SQLForClause by means of the normalisation rules,
static type analysis rules, and dynamic evaluation rules.

\paragraph{Normalisation rules.}
%
Let us start by presenting the normalisation rule that handles the syntactic shortcut~\lit{\FOR *}.
%
\begin{normalisationrule}%
  \mapping{%
    \sem{%
      \FOR~*~\envElem{RelationList}{}~\grammarRule{SQLWhereClause}~\grammarRule{ReturnClause}%
    }{\grammarRule{Expr}}%
  }{%
    \sema{%
      \FOR~\sem{%
        \envElem{RelationList}{}~\envElem{SQLWhereClause}{}
      }{attrs}~\envElem{RelationList}{}\\
      \grammarRule{SQLWhereClause}~\grammarRule{ReturnClause}%
    }{Expr}%
  }%
  \label{for_star-rdb}%
\end{normalisationrule}%
% 
The normalisation rule $\sem{\cdot}{attrs}$ returns a comma separated list of variables representing all the attributes
from each relation from $\envElem{RelationList}{}$.  As described in \cref{sec:syntax-rdb}, these generated
variables are of the form: $\var{\grammarRule{relationName}.\grammarRule{attributeName}}$.
%
Furthermore, the next normalisation rule guarantees that each variable in a \SQLForClause contains a variable alias:
%
\begin{normalisationrule}
  \mapping{%
    \sema{%
      \FOR~\envElem{AttrSpec}{1}, \dotsb, \envElem{AttrSpec}{n} \\
      \grammarRule{RelationList}~\grammarRule{SQLWhereClause}\\
      \grammarRule{ReturnClause}%
    }{\grammarRule{Expr}}
  }{%
    \begin{array}{l}
      \FOR~\sem{\envElem{AttrSpec}{1}}{Alias}, \dotsb, \sem{\envElem{AttrSpec}{n}}{Alias} \\
      \grammarRule{RelationList}~\grammarRule{SQLWhereClause}\\
      \sem{\grammarRule{ReturnClause}}{\grammarRule{Expr}}
    \end{array}}%
  \label{posvar-sql-normalisation1}%
\end{normalisationrule}%
%
A new normalisation rule~$\sem{\cdot}{\envElem{Alias}{}}$ takes care of introducing the variable alias when necessary,
where the variable alias will be the same as the attribute specification.
%
{\small
  \[
  \sem{\envElem{AttrSpec}{}}{\envElem{Alias}{}} == \envElem{AttrSpec}{}~\keyword{as}~\envElem{\var{AttrSpec}}{} \enspace
  .
\]}%
%
In case a variable alias is already present it is reused: 
%
{\small
  \[
  \sem{\envElem{AttrSpec}{}~\keyword{as}~\envElem{\var{VarRef}}{}}{\envElem{Alias}{}} ==
  \envElem{AttrSpec}{}~\keyword{as}~\envElem{\var{VarRef}}{} \enspace .
  \]}%


\paragraph{Static type analysis.}
\label{sec:static-type-analysis-rdb}

The following static type rule defines the type of each variable in an \SQLForClause as \type{SQLTerm} and infers the
static type of whole expression.
%
%
This rule, based on the static environment~$\stat$, creates a new environment with the added information that each of
the variables in the \SQLForClause ($\envElem{\var{Var}}{1} \dots \envElem{\var{Var}}{n}$) is of type
$\type{\qname{xs}{anySimpleType}}$.
%
Given this new extended environment the type of $\envElem{ReturnExpr}{}$ can be inferred to be $\envElem{Type}{}$,
making the type of the overall \SQLForClause a sequence of elements of inferred type $\envElem{Type}{}$.
%
\begin{staticrule}
  \begin{prooftree}
    \def\ScoreOverhang{1pt}%
    \def\extraVskip{1pt}%
    \alwaysNoLine%
    %
    \AxiomC{$\stat\envExtend{varType}{
          \begin{array}{l}
            \envElem{Var}{1} \Rightarrow \type{SQLTerm};\\
            \dots;\\
            \envElem{Var}{n} \Rightarrow \type{SQLTerm}
          \end{array}
        } 
      \proofs \grammarRule{ReturnExpr} \colon \envElem{Type}{}
      $}
    %
    \singleLine
    %
    \UnaryInfC{$\statEnv{
        \begin{array}{l}
          \FOR~\envElem{AttrSpec}{1}~\keyword{as}~\envElem{\var{Var}}{1}, \dotsb, \envElem{AttrSpec}{n} \texttt{ as } \envElem{\var{Var}}{n}\\
          \grammarRule{RelationList}~\grammarRule{SQLWhereClause}~
          \RETURN~\grammarRule{ReturnExpr} 
        \end{array} : \envElem{Type}{}*
      }$}
  \end{prooftree}
  \label{eq:static-type-rdb}
\end{staticrule}%



\paragraph{Dynamic Evaluation.}
\label{sec:dynamic-evaluation-rdb}

The dynamic evaluation rules for \SQLForClause{s} ensures that the return expression~($\grammarRule{ReturnExpr}$) is
executed for each \type{SQLResult} that is returned by the evaluation of the \ac{SQL} expression.
%
If the evaluation of the SQL expression does not yield any solutions, \ie~evaluates to an empty sequence, then the
overall result will also be the empty sequence:
%
\begin{dynamicrule}
  \AxiomC{$\dynEnv \funcCall{fs{:}sql}{\envElem{RelationList}{}, \grammarRule{SQLWhereClause}} \Rightarrow \seq{}$}
  % 
  \singleLine
  % 
  \UnaryInfC{$\dynEnv{\begin{array}{l}
        \FOR~\envElem{\var{Var}}{1}~\envElem{OptVarAlias}{1},  \dots, \envElem{\var{Var}}{n}~\envElem{OptVarAlias}{n}\\
        \grammarRule{RelationList}~\grammarRule{SQLWhereClause}~
        \RETURN~\grammarRule{ReturnExpr}
      \end{array}} \Rightarrow \seq{}
    $}%
  % 
\end{dynamicrule}%
%
Otherwise, for each solution, the respective value in the pattern solution is accessed and assigned to the respective
variable name in the \dyn.\ecomp{varValue} component.  The results of evaluating~$\grammarRule{ReturnExpr}$ in this
extended environment are then collected into the final sequence.  Please note that this rule also populates the
\dyn.\ecomp{globalPosition} environment.
%
\begin{dynamicrule}
  % 
  \AxiomC{$\dyn.\ecomp{globalPosition} = \seq{ \envElem{Pos}{1}, \cdots, \envElem{Pos}{j} } $}
  % 
  \UnaryInfC{$\dynEnv{\funcCall{fs{:}sql}{\envElem{RelationList}{}, \grammarRule{SQLWhereClause}} \Rightarrow \envElem{SR}{1}, \dots, \envElem{SR}{m}}$}
  % 
  \UnaryInfC{$        
    \begin{array}{r@{\hspace{-1pt}}l}
      \dyn & \envExtend{globalPosition}{\seq{ \envElem{Pos}{1}, \cdots, \envElem{Pos}{j}, 1 }} \\
      & \envExtend{varValue}{
        \begin{array}{l}
          \envElem{Var}{1} \Rightarrow \funcCall{fs{:}value}{\envElem{SR}{1}, \envElem{Var}{1}};\\
          \dots;\\
          \envElem{Var}{n} \Rightarrow \funcCall{fs{:}value}{\envElem{SR}{1},\envElem{Var}{n}}
        \end{array}
      } \proofs \grammarRule{ReturnExpr} \Rightarrow \envElem{Value}{1}
    \end{array}
    $}
  % 
  \UnaryInfC{$\vdots$}
  % 
  \UnaryInfC{$
    \begin{array}{r@{\hspace{-1pt}}l}
      \dyn & \envExtend{globalPosition}{\seq{ \envElem{Pos}{1}, \cdots, \envElem{Pos}{j}, m }} \\
      &  \envExtend{varValue}{
        \begin{array}{l}
          \envElem{Var}{1} \Rightarrow \funcCall{fs{:}value}{\envElem{SR}{m}, \envElem{Var}{1}};\\
          \dots;\\
          \envElem{Var}{n} \Rightarrow \funcCall{fs{:}value}{\envElem{SR}{m},\envElem{Var}{n}} \\
        \end{array}} \proofs \grammarRule{ReturnExpr} \Rightarrow \envElem{Value}{m}
    \end{array}
    $}
  % 
  \singleLine
  % 
  \UnaryInfC{$\dynEnv{\begin{array}{l}%
        \FOR~\envElem{AttrSpec}{1} \texttt{ as } \envElem{\var{Var}}{1} \dots \envElem{AttrSpec}{n} \texttt{ as } \envElem{\var{Var}}{n}\\
        \grammarRule{RelationList}~\grammarRule{SQLWhereClause}~
        \RETURN~\grammarRule{ReturnExpr} 
      \end{array}  \Rightarrow \envElem{Value}{1}, \dots, \envElem{Value}{m}
    }$}
  % 
\end{dynamicrule}%



\subsubsection{\SparqlForClause}
\label{sec:sparqlforclause-sem}
%
The semantics of the \SparqlForClause expression (\cref{fig:xsparql-flwor}) is defined by the following normalisation
rules, static type analysis rules and dynamic evaluation rules.
%
Again, we start by presenting the normalisation rules for \SparqlForClause{s} with implicit variable selection (by means
of ``\FOR~*''), which are translated into explicitly stated variables:
%
\begin{normalisationrule}
  \mapping{%
    \sema{%
      \FOR~\keyword{*}~\OptDatasetClause~\SparqlWhereClause\\
      \SolutionModifier~\ReturnExpr }{ \grammarRule{Expr} }%
  }%
  {%
    \sema{%
      \FOR~\sem{\SparqlWhereClause}{vars}\\
      \OptDatasetClause~\SparqlWhereClause\\
      \SolutionModifier~\ReturnExpr }{Expr}%
  }%
  \label{for_star}%
\end{normalisationrule}%
%
The normalisation rule $\sem{\grammarRule{WhereClause}}{vars}$ determines all statically \emph{unbound variables}
present in the \SparqlWhereClause, \ie~returns a whitespace separated list of all variables in the \SparqlWhereClause
that are not present in the $\stat{}.\ecomp{varType}$ environment component.


\paragraph{Static type analysis.}
\label{sec:static-type-analysis}
%
The following static rule takes care of defining the types of variables present in a \FOR expression as \type{RDFTerm}
and infers the static type of the \SparqlForClause expression:\footnote{Similar to the XQuery Core
  \grammarRule{OptPositionalVar}, the \grammarRule{OptDatasetClause} covers both cases when a \SparqlForClause contains
  (or does not contain) a \DatasetClause.}
%
\begin{staticrule}
  \begin{prooftree}
    \def\ScoreOverhang{1pt}%
    \def\extraVskip{1pt}%
    \alwaysNoLine%
    %
    \AxiomC{$\stat\envExtend{varType}{\begin{array}{l}
          \envElem{Var}{1} \Rightarrow \type{RDFTerm};\\
            \dotsb; \\
            \envElem{Var}{n} \Rightarrow \type{RDFTerm}  
          \end{array}}~\proofs  \grammarRule{ExprSingle} \colon \envElem{Type}{}
      $}
    %
    \singleLine
    %
    \UnaryInfC{$\statEnv{
        \begin{array}{l}
          \FOR~\envElem{\var{Var}}{1} \dotsb \envElem{\var{Var}}{n}~\grammarRule{OptDatasetClause}\\
          \SparqlWhereClause~\SolutionModifier ~
          \RETURN~\grammarRule{ExprSingle}  \colon \envElem{Type}{}*
        \end{array}
      }$}
  \end{prooftree}%
  \label{eq:static-type}
\end{staticrule}%



\paragraph{Dynamic Evaluation.}
\label{sec:dynamic-evaluation}


We can now define the dynamic evaluation rules for the \SparqlForClause expression.  Intuitively these rules state that
the return expression \grammarRule{ExprSingle} will be executed for each \type{Pattern\-Solution} that is returned from
the evaluation of the \funcName{fs{:}sparql} function. The following two dynamic rules specify the evaluation of the
\SparqlForClause with an explicit \DatasetClause.  These rules use the \funcName{fs{:}dataset} function to parse the
\DatasetClause into an element of type \type{RDFDataset}, which will be stored in the \dyn.\ecomp{activeDataset}
component: If the evaluation of the \funcName{fs{:}sparql} function does not yield any solutions, \ie~evaluates to an
empty sequence, the overall result will also be the empty sequence:
%
\begin{dynamicrule}
  % 
  \AxiomC{$\dynEnv{\funcCall{fs{:}dataset}{\DatasetClause} \Rightarrow \envElem{Dataset}{}}$}
  % 
  \UnaryInfC{$\dynEnv{%
      \funcCall{fs{:}sparql}{%
        \begin{array}{l}
          \envElem{Dataset}{},\SparqlWhereClause,\\
          \SolutionModifier
        \end{array}
      }%
    } \Rightarrow \seq{}$}
  % 
  \singleLine
  % 
  \UnaryInfC{$\dynEnv{\begin{array}{l}
        \FOR~\envElem{\var{Var}}{1}\dotsb\envElem{\var{Var}}{n}~\DatasetClause\\
        \SparqlWhereClause~\SolutionModifier\\
        \ReturnExpr
      \end{array} \Rightarrow \seq{} }$}
  \label{dyn:empty-sparqlfor}
\end{dynamicrule}%
%
Otherwise, \grammarRule{ExprSingle} is evaluated for each solution in the results of the SPARQL query:
%
\begin{dynamicrule}
    %
    \AxiomC{$\dyn.\ecomp{globalPosition} = \seq{ \envElem{Pos}{1}, \cdots, \envElem{Pos}{j} } $}
    % 
    \UnaryInfC{$\dynEnv{\funcCall{fs{:}dataset}{\DatasetClause} \Rightarrow \envElem{Dataset}{}}$}
    % 
    \UnaryInfC{$\begin{array}{r@{\hspace{-1pt}}l}%
        \dyn &~\proofs \funcCall{fs{:}sparql}{%
          \begin{array}{l}%
            \envElem{Dataset}{}, \SparqlWhereClause,\\
            \SolutionModifier
            \end{array}
          } \Rightarrow \sm{1}{}, \dots, \sm{m}{}
      \end{array}
      $}
    %
    \UnaryInfC{$\begin{array}{r@{\hspace{-1pt}}l}%
        \dyn & \envExtend{globalPosition}{\seq{ \envElem{Pos}{1}, \cdots, \envElem{Pos}{j}, 1 }} ~\envExtend{activeDataset}{\grammarRule{Dataset}}\\
        &\envExtend{varValue}{%
          \begin{array}{l}%
            \envElem{Var}{1} \Rightarrow \funcCall{fs{:}value}{\sm{1}{},\envElem{Var}{1}};\\
            \dotsc;\\
            \envElem{Var}{n} \Rightarrow \funcCall{fs{:}value}{\sm{1}{},\envElem{Var}{n}}
          \end{array}%
        }~\proofs \grammarRule{ExprSingle} \Rightarrow
        \envElem{Value}{1}
      \end{array}$}
    %
    \UnaryInfC{$\vdots$}
    %
    \UnaryInfC{$\begin{array}{r@{\hspace{-1pt}}l}
        \dyn & \envExtend{globalPosition}{\seq{ \envElem{Pos}{1}, \cdots, \envElem{Pos}{j}, m }} ~\envExtend{activeDataset}{\grammarRule{Dataset}}\\
        &\envExtend{varValue}{\begin{array}{l}
            \envElem{Var}{1} \Rightarrow \funcCall{fs{:}value}{\sm{m}{},\envElem{Var}{1}}; \\
            \dotsc; \\
            \envElem{Var}{n} \Rightarrow \funcCall{fs{:}value}{\sm{m}{},\envElem{Var}{n}} 
          \end{array}%
        }~\proofs \grammarRule{ExprSingle} \Rightarrow \envElem{Value}{m}
      \end{array}$}
    %
    \singleLine
    %
    \UnaryInfC{$\dynEnv{\begin{array}{l}
          \FOR~\envElem{\var{Var}}{1}\dotsb\envElem{\var{Var}}{n}~\DatasetClause\\
          \SparqlWhereClause~\SolutionModifier\\
          \RETURN~\grammarRule{ExprSingle}
        \end{array}} \Rightarrow \envElem{Value}{1}, \dots, \envElem{Value}{m}$}
  \label{eq:sparqlForClause}
\end{dynamicrule}%
%
This rule ensures that the \ecomp{activeDataset} component of the dynamic environment is updated to reflect the explicit
\DatasetClause of the \SparqlForClause and that the \ecomp{globalPosition} environment contains all the positions in the
previous tuple streams.

The rule that handles the \SparqlForClause without an explicit \DatasetClause is presented next.  These rules are very
similar, with the exception that in following rules, the dataset over which the \SparqlForClause is evaluated is read
from the \dyn.\ecomp{activeDataset} component.
%
\begin{dynamicrule}
    % 
    \AxiomC{$\dyn.\ecomp{globalPosition} = \seq{ \envElem{Pos}{1}, \cdots, \envElem{Pos}{j} } $}
    %
    \UnaryInfC{$\dyn.\ecomp{activeDataset} \Rightarrow \envElem{Dataset}{}$}
    % 
    \UnaryInfC{$\dynEnv{%
        \funcCall{fs{:}sparql}{%
          \begin{array}{l}
            \envElem{Dataset}{},\SparqlWhereClause,\\
            \SolutionModifier
          \end{array}
        } \Rightarrow \sm{1}{}, \dots, \sm{m}{}%
      } $}
    % 
    \UnaryInfC{$\begin{array}{r@{\hspace{-1pt}}l}
        \dyn & \envExtend{globalPosition}{\seq{ \envElem{Pos}{1}, \cdots, \envElem{Pos}{j}, 1 }}\\
        &\envExtend{varValue}{\begin{array}{l}
            \envElem{Var}{1} \Rightarrow \funcCall{fs{:}value}{\sm{1}{},\envElem{Var}{1}}; \\
            \dotsc; \\
            \envElem{Var}{n} \Rightarrow \funcCall{fs{:}value}{\sm{1}{},\envElem{Var}{n}} 
          \end{array}%
        }~\proofs \grammarRule{ExprSingle} \Rightarrow \envElem{Value}{1}
      \end{array}$}
    % 
    \UnaryInfC{$\vdots$}
    % 
    \UnaryInfC{$\begin{array}{r@{\hspace{-1pt}}l}
        \dyn & \envExtend{globalPosition}{\seq{ \envElem{Pos}{1}, \cdots, \envElem{Pos}{j}, m }}\\
        &\envExtend{varValue}{%
        \begin{array}{l}
          \envElem{Var}{1} \Rightarrow \funcCall{fs{:}value}{\sm{m}{}, \envElem{Var}{1}};\\
          \dots;\\
          \envElem{Var}{n} \Rightarrow \funcCall{fs{:}value}{\sm{m}{},\envElem{Var}{n}} 
        \end{array}}~\proofs \grammarRule{ExprSingle} \Rightarrow \envElem{Value}{m}
    \end{array}
    $}
    % 
    \singleLine
    % 
    \UnaryInfC{$\dynEnv{
        \begin{array}{l}
          \FOR~\envElem{\var{Var}}{1}\dotsb\envElem{\var{Var}}{n}\\
          \SparqlWhereClause~\SolutionModifier\\
          \ReturnExpr 
        \end{array}}  \Rightarrow \envElem{Value}{1}, \dots, \envElem{Value}{m}$}
\label{xsparql.new.fssparql}
\end{dynamicrule}%
%
Analogously to the \SparqlForClause with an explicit dataset (Rule~\ref{dyn:empty-sparqlfor}), whenever the
$\funcName{fs{:}sparql}$ function evaluates to an empty sequence, the result will also be an empty sequence.



\subsubsection{\ConstructClause}
\label{sec:constructsem}
%
XSPARQL normalises \ConstructClause{s} into standard XQuery \RETURN expressions with the necessary mechanisms for
validation of the returned \ac{RDF} graph and as such, we define the semantics of \ConstructClause{s}
(\cref{fig:xsparql-flwor}) by means of normalisation rules.
%
One valid syntax for XSPARQL is a SPARQL stand-alone \CONSTRUCT query (as described in \cref{sec:syntax}).  These
queries are normalised into \CONSTRUCT queries with a surrounding \SparqlForClause by the following rule:
%
\begin{normalisationrule}
  \mapping{%
    \sema{%
      \CONSTRUCT~\ConstructTemplate\\
      \DatasetClause~\SparqlWhereClause\\
      \SolutionModifier%
    }{Expr}%
  }{%
    \sema{%
      \FOR~*~\DatasetClause\\
      \SparqlWhereClause~\SolutionModifier\\
      \CONSTRUCT~\ConstructTemplate%
    }{Expr}}
\label{eq:construct2forclause}
\end{normalisationrule}%
%
The recursive call to \sem{\cdot}{Expr} ensures that the resulting query will be further rewritten according to
normalisation Rule~\eqref{for_star} presented above, in order to explicitly state the variables present in the
\SparqlWhereClause.
%

Similar to the normalisation rule for stand-alone \textit{ReturnClauses} presented in~\citet[Section
4.8.1]{DraperFankhauserFernandez:2010aa}, the following normalisation rule transforms \CONSTRUCT clauses into XQuery
\ReturnClause{s}.
%
\begin{normalisationrule}%
  \mapping{%
    \sem{\mathtt{construct}~\ConstructTemplate}{Expr}%
  }{%
    \RETURN~\funcCall{fs{:}evalCT}{\sem{\ConstructTemplate}{normCT}}%
  }
  \label{eq:construct2forclause2}
\end{normalisationrule}%
%
In the following we assume that \ConstructTemplate is a simple \character{.} separated list of \grammarRule{Subject},
\grammarRule{Predicate} and \grammarRule{Object}. The~$\sem{\cdot}{normCT}$ rule transforms any Turtle shortcut notation
used in \ConstructTemplate to these simple lists.
%
As an example of this rule, we present the rule for normalising Turtle~\character{;} abbreviations (previously described
in \cref{sec:turtle}):
%
\begin{normalisationrule}%
  \mapping{%
    \sem{\grammarRule{Subject}~\envElem{Pred}{1}~\envElem{Obj}{1};~\dotsb;
      ~\envElem{Pred}{n}~\envElem{Obj}{n}~.}{normCT}%
  } {%
    \grammarRule{Subject}~\envElem{Pred}{1}~\envElem{Obj}{1}~.~\dotsb~
    \grammarRule{Subject}~\envElem{Pred}{n}~\envElem{Obj}{n}~.%
  }
  \label{eq:flatten-triples}
\end{normalisationrule}%
%
The normalisation rules for the other Turtle shortcuts that are allowed in the SPARQL \ConstructTemplate syntax are
similar to this one and are not presented here. 

Since anonymous blank nodes can be written in numerous ways in Turtle, the~$\sem{\cdot}{normCT}$ normalisation rule also
transforms each anonymous blank node into a labelled blank node where the identifier/label is distinct from any other
blank node labels present in the \ConstructTemplate.  This label will then be used by the skolemisation function to
generate the distinct blank node label for each position in the tuple stream.


In more detail, the \funcName{fs{:}evalCT} function checks the constructed RDF graph for validity (according to the
conditions described in \cref{sec:sparql-syntax}), filtering out any non-valid RDF triples where \emph{subjects} are
literals or \emph{predicates} are literals or blank nodes.  This is illustrated by the following dynamic evaluation
rules.
%
\begin{dynamicrule}
    %
    \AxiomC{$\dynEnv{\funcCall{fs{:}validTriple}{\mathit{Subj_1}, \mathit{Pred_1}, \mathit{Obj_1}}} \Rightarrow \envElem{Triple}{1}$}
    %
    \UnaryInfC{$\vdots$}
    %
    \UnaryInfC{$\dynEnv{\funcCall{fs{:}validTriple}{\envElem{Subj}{n}, \envElem{Pred}{n}, \envElem{Obj}{n}}} \Rightarrow \envElem{Triple}{n}$}
    %
    \singleLine
    %
    \UnaryInfC{$\begin{array}{r@{\hspace{-1pt}}l}
        \dyn &~\proofs \funcCall{fs{:}evalCT}{\begin{array}{c}
            Subj_1~Pred_1~Obj_1\\
            \dots\\
            Subj_n~Pred_n~Obj_n
          \end{array}}~\Rightarrow  \envElem{Triple}{1}, ~ \dotsb, ~ \envElem{Triple}{n}
      \end{array}
      $}
\end{dynamicrule}%
%
The following dynamic evaluation rule for the $\funcName{fs{:}validTriple}$ function checks, relying on the
\funcName{fs{:}bnode} function defined below, if a triple is valid according to the RDF semantics.
%
\begin{dynamicrule}
    % 
    \AxiomC{$\dynEnv{\funcCall{fs{:}bnode}{\grammarRule{Subject}} \Rightarrow \envElem{ValS}{}}$}
    % 
    \UnaryInfC{$\statEnv{\envElem{ValS}{}\ \mathbf{matches}\ (\type{uri} \mid \type{bnode})}$}
    % 
    \UnaryInfC{$\dynEnv{\grammarRule{Predicate} \Rightarrow \envElem{ValP}{}}$}
    % 
    \UnaryInfC{$\statEnv{\envElem{ValP}{}\ \mathbf{matches}\ \type{uri}}$}
    % 
    \UnaryInfC{$\dynEnv{\funcCall{fs{:}bnode}{\envElem{Object}{}} \Rightarrow \envElem{ValO}{}}$}
    % 
    \UnaryInfC{$\dynEnv{\envElem{ValO}{}\ \mathbf{matches}\ (\type{uri} \mid \type{bnode} \mid \type{literal})}$}
    % 
    \singleLine
    % 
    \UnaryInfC{$\begin{array}{r@{\hspace{-1pt}}l} \dyn &~\proofs
        \funcCall{fs{:}validTriple}{\begin{array}{l}
            \grammarRule{Subject},\\
            \grammarRule{Predicate},\\
            \grammarRule{Object}
          \end{array}}~\Rightarrow \begin{array}{l} 
          \keyword{element}~\stt{triple}~\keyword{of}~\keyword{type}~\stt{RDFTriple}~\{ \\
          \qquad\keyword{element}~\stt{subject}~\keyword{of}~\keyword{type}~\stt{RDFTerm}~\{ \envElem{ValS}{} \}\\
          \qquad\keyword{element}~\stt{predicate}~\keyword{of}~\keyword{type}~\stt{RDFTerm}~\{ \envElem{ValP}{} \}\\
          \qquad\keyword{element}~\stt{object}~\keyword{of}~\keyword{type}~\stt{RDFTerm}~\{ \envElem{ValO}{} \}\\
          \}\!\!
        \end{array}\\
        \end{array}
    $}%
\label{validTriple1}
\end{dynamicrule}%
%
In case any of the subject, predicate or object do not match an allowed type, the empty sequence is
returned. Effectively this suppresses any invalid RDF triples from the output graph.
%
\begin{dynamicrule}
  % 
    \AxiomC{$\dynEnv{\funcCall{fs{:}bnode}{\grammarRule{Subject}} \Rightarrow \mathit{ValueS}}$}
    \UnaryInfC{$\dynEnv{\grammarRule{Predicate} \Rightarrow \envElem{ValueP}{}}$}
    \UnaryInfC{$\dynEnv{\funcCall{fs{:}bnode}{\envElem{Object}{}} \Rightarrow \envElem{ValueO}{}}$}
    \UnaryInfC{$\dynEnv{\textbf{not}\left(\begin{array}{@{}l@{}}
              \envElem{ValueS}{}\ \mathbf{matches}\ \left(\type{uri} \mid \type{bnode}\right)\ \textbf{and}\\
              \envElem{ValueP}{}\ \mathbf{matches}\ \type{uri}\ \textbf{and}\\
              \envElem{ValueO}{}\ \mathbf{matches}~\left(\begin{array}{l}\type{uri} \mid \type{bnode} \mid  \type{literal}\end{array}\right)
            \end{array}\right)}$}
    \singleLine
    \UnaryInfC{$\dynEnv{\funcCall{fs{:}validTriple}{\grammarRule{Subject},\grammarRule{Predicate},\grammarRule{Object}}} \Rightarrow \left(\right)$}
\label{validTriple2}
\end{dynamicrule}%

\paragraph{Blank Node Skolemisation.}
%
In order to comply with the SPARQL \CONSTRUCT semantics, all blank nodes inside a \ConstructTemplate need
to be \emph{skolemised}, \ie~for each solution a new distinct blank node identifier needs to be generated.
%
Since we keep all the positions in the tuple streams, we can rely on the blank node label and these position values to
generate a unique blank node label (represented by the~$\funcName{fs{:}skolemConstant}$ function).
%
This skolemisation of blank nodes is performed by the \funcName{fs{:}bnode} function.  If the argument of this function
is of type \type{bnode} the skolemised label is calculated.  
%
\begin{dynamicrule}%
    % 
    \AxiomC{$\dynEnv{\envElem{ValueR}{}\ \mathbf{matches}\ \type{bnode}}$}
    % 
    \UnaryInfC{$\dyn.\ecomp{globalPosition} = \seq{ \envElem{PosValue}{1}, \dotsb, \envElem{PosValue}{n} }$}
    % 
    \UnaryInfC{$\begin{array}{r@{\hspace{-1pt}}l}
        \dyn &~\proofs \funcCall{fs{:}skolemConstant}{
          \begin{array}{l}ValueR, \\
            \envElem{PosValue}{1},\\
            \dotsc,\\
            \envElem{PosValue}{n}
          \end{array}}~\Rightarrow \envElem{ValueRS}{}\\
      \end{array}$}
    % 
    \singleLine
    % 
    \UnaryInfC{$\dyn~\proofs\funcCall{fs{:}bnode}{\envElem{ValueR}{}}~\Rightarrow~\keyword{element}~\stt{bnode}~\keyword{of}~\keyword{type}~\stt{xs:string}~\{ \envElem{ValueRS}{} \}$}
  \label{eq:bnode-skolem}
\end{dynamicrule}%
%
Otherwise, \funcName{fs{:}bnode} returns its argument unchanged:
%
\begin{dynamicrule}
  %
  \AxiomC{$\dynEnv{\envElem{Value}{}\ \mathbf{matches}\ (\type{uri} \mid \type{literal})}$}
  %
  \singleLine
  % 
  \UnaryInfC{$\dynEnv{\funcCall{fs{:}bnode}{\envElem{Value}{}} \Rightarrow \envElem{Value}{}}$}
  \label{eq:bnode-skolem-default}
\end{dynamicrule}%



%%% Local Variables:
%%% fill-column: 120
%%% TeX-master: t
%%% TeX-PDF-mode: t
%%% TeX-debug-bad-boxes: t
%%% TeX-parse-self: t
%%% TeX-auto-save: t
%%% reftex-plug-into-AUCTeX: t
%%% mode: latex
%%% mode: flyspell
%%% mode: reftex
%%% TeX-master: "../thesis"
%%% End:
