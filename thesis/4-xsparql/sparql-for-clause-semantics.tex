
\subsubsection{Querying RDF Data}
\label{sec:bf-extended-bgp}
%
For querying \ac{RDF} data, we extend the notion of SPARQL \ac{BGP} (\cref{def:bgp-matching}) in order to provide SPARQL
with the variable bindings from XQuery.
%
For this we interpret the XQuery \ecomp{varValue} dynamic environment component as a set of bindings in the spirit of
SPARQL solution mappings (as presented in \cref{def:solution-mapping}).
%
Along these lines, we will regard the \ecomp{varValue} component of the dynamic environment in which a SPARQL graph
pattern~$P$ is executed as the basis for the \emph{XSPARQL instance mapping} of~$P$.  The transformation from the
\dyn.\ecomp{varValue} into the XSPARQL instance mapping is defined next:
%
\begin{definition}[XSPARQL instance mapping]
  \label{def:xml2rdfterm}
  Let~$C$ be an expression context, and furthermore let~$D_C = \funcCall{\dyn}{C}.\ecomp{varValue}$ and~$T_C =
  \funcCall{\stat}{C}.\ecomp{varType}$ the \ecomp{varValue} component of the dynamic environment of~$C$ and be the
  \ecomp{varType} component of the static environment of~$C$, respectively.
  %
  The \emph{XSPARQL instance mapping}~$\mu_C$ is a solution mapping where, for each mapping~$v_{i} \rightarrow x_{i} \in
  D_C$,~$x_{i}$ is converted into an instance of type \type{RDFTerm} or an \ac{RDF} \emph{Collection} according to the
  following conditions:
  %
  \begin{itemize}[noitemsep]
  \item if~$\funcCall{D_C}{v_{i}} = \seq{}$ and~$\funcCall{T_C}{v_{i}} = \type{RDFTerm}$ or~$\funcCall{T_C}{v_{i}} =
    \type{SQLTerm}$ then~$\funcCall{\mu_C}{\funcCall{D_C}{v_{i}}}$ is undefined;
  \item if~$\funcCall{D_C}{v_{i}} = \seq{}$ and~$\funcCall{T_C}{v_{i}} \neq \type{RDFTerm}$ and~$\funcCall{T_C}{v_{i}}
    \neq \type{SQLTerm}$ then~$\funcCall{\mu_C}{\funcCall{D_C}{v_{i}}} = \seq{}$ is an empty \ac{RDF} Collection;
  \item if~$\funcCall{D_C}{v_{i}}$ is a singleton sequence then~$\funcCall{\mu_C}{\funcCall{D_C}{v_{i}}} =
    \funcCall{RDFTerm}{\funcCall{D_C}{v_{i}}}$;
  \item if~$\funcCall{D_C}{v_{i}} = \seq{e_{1}, \dotsc, e_{n}}$,~$n > 1$, is a sequence
    then~$\funcCall{\mu_C}{\funcCall{D_C}{v_{i}}} = \seq{\funcCall{RDFTerm}{e_{1}} \dotsb \funcCall{RDFTerm}{e_{n}}}$ to
    be read as an \ac{RDF} Collection in Turtle notation (\cf~\cref{sec:turtle}).
  \end{itemize}
  % 
  For a graph pattern~$P$, we call the XSPARQL instance mapping of the expression context in which~$P$ is executed the
  \emph{XSPARQL instance mapping of~$P$}.
\end{definition}
%
Next we define the notion of XSPARQL \ac{BGP} matching based on the semantics of SPARQL \ac{BGP} matching presented in
\cref{sec:sparql-semantics}.
%
\begin{definition}[Extended solution mapping]\label{def:extended-solution-mapping}
  Let~$C$ be an expression context. An \emph{extended solution mapping} of a graph pattern~$P$ in~$C$ is a solution
  mapping \emph{compatible} with the \emph{XSPARQL instance mapping} of~$C$.
\end{definition}
% 
\noindent Accordingly, XSPARQL \ac{BGP} matching is defined analogously to the SPARQL \ac{BGP} matching with the
exception that we consider only extended solution mappings:
%
\begin{definition}[XSPARQL \ac{BGP} matching]\label{def:xsparql-bgp-matching}
  Let~$P$ be a \ac{BGP} with expression context~$C$, and~$G$ be an \ac{RDF} graph. We say that~$\mu$ is a
  \emph{solution} for~$P$ with respect to active graph~$G$, if there exists an extended solution mapping~$\mu'$ of~$C$
  such that~$\funcCall{\mu'}{P}$ is a subgraph of~$G$ and~$\mu$ is the restriction of~$\mu'$ to the variables
  in~$\vars{P}$.
\end{definition}
%
\noindent This definition quasi injects the variable bindings inherited from XQuery into SPARQL patterns occurring
within XSPARQL.  By considering \emph{extended solution mappings} the bindings returned for a \ac{BGP}~$P$ will not only
match the input graph~$G$ but also respect any bindings for variables in the dynamic environment.
%
We can extend the XSPARQL \ac{BGP} matching to generic graph patterns by following the SPARQL evaluation semantics
(presented in \cref{sec:sparql-semantics}).
%
Considering a graph pattern~$P$ with XSPARQL instance mapping~$\mu_C$, we denote by~$\evalXS{D}{P}{\mu_C}$ the
evaluation of~$P$ over dataset~$D$ following XSPARQL \ac{BGP} matching.
 
\subsubsection{Matching Blank Nodes in Nested Queries}
\label{sec:bgpmatch}
%
Although in XSPARQL, similar to SPARQL, we are not considering blank nodes in the semantics definitions of graph
patterns, in the case of nested \SparqlForClause{s} XSPARQL instance mappings may in fact contain assignments of
variables to blank nodes, injected from the outer \SparqlForClause into the inner \SparqlForClause.
%
\begin{query}
  \lstinputlisting[language=XSPARQL]{0-data+queries/nested-bnodes.xsparql}%
  \caption{Nested XSPARQL query}%
  \label{qr:nested-bnodes}%
\end{query}
%
\begin{example}[Blank node injection in XSPARQL nested queries]
  %
  For example, in \cref{qr:nested-bnodes}, blank nodes bound in the outer \SparqlForClause
  (\crefrange{qr:nested-bnodes_out-s}{qr:nested-bnodes_out-e}) to the variable~\var{song} will be injected into
  the inner \SparqlForClause expression (\crefrange{qr:nested-bnodes_in-s}{qr:nested-bnodes_in-e}).
  %
  If we would consider both \SparqlForClause{s} as distinct SPARQL queries, the blank nodes in the inner
  \SparqlForClause would be matched as variables.
\end{example}
%
However in XSPARQL, we want to enable coreference within nested queries over the same dataset and thus such injected
blank nodes should be matched like constants against the blank nodes present in the input \ac{RDF} data (rather than
being treated as variables).
%
To ensure this behaviour, we introduce the notion of \emph{active dataset} (similar to the concept of active graph in
SPARQL), where nested queries over the same active dataset keep the same the scoping graphs
(\cf~\cref{sec:sparql-semantics}).
%
Any \SparqlForClause with an \emph{explicit} \DatasetClause causes the \emph{active dataset} to change, \ie~new scoping
graphs (with fresh blank nodes) for each graph within it are created.  On the other hand, if no \DatasetClause is
present in a nested \SparqlForClause (implicit dataset), the active dataset remains
unchanged.
%
To ensure this behaviour in the dynamic evaluation we have to introduce a new dynamic environment component called
\ecomp{activeDataset}, that will be used to evaluate \SparqlWhereClause{s}.  Initially, this component is empty (or set
to a system default) and is changed by a \DatasetClause appearing in a \SparqlForClause{}, as defined in the next
section.



%%% Local Variables:
%%% fill-column: 120
%%% TeX-master: t
%%% TeX-PDF-mode: t
%%% TeX-debug-bad-boxes: t
%%% TeX-parse-self: t
%%% TeX-auto-save: t
%%% reftex-plug-into-AUCTeX: t
%%% mode: latex
%%% mode: flyspell
%%% mode: reftex
%%% TeX-master: "../thesis"
%%% End:
