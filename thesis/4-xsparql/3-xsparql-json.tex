
\section{Consuming JSON Data}
\label{sec:xsparql-json}

Due to the similarity between \ac{JSON} and \ac{XML}, in XSPARQL we incorporate \ac{JSON} data by translating the
\ac{JSON} objects into \ac{XML} data.
%
Furthermore \ac{JSON} does not specify a query language (this representation format is meant to be incorporated directly
into the JavaScript scripting language).
%
As presented in \cref{sec:json}, \ac{XML} is more flexible than \ac{JSON} and it is possible to convert \ac{JSON} into
\ac{XML} but not so easy in the opposite direction.  


This translation of \ac{JSON} to \ac{XML} enables access to the \ac{JSON} data using standard \ac{XPath}.  The following
definition presents the translation we use in XSPARQL.
% 
\begin{definition}[Translation from JSON to XML]
  % 
  \label{def:json2xml}
  %
  Let~$J$ be a \ac{JSON} object.  The translation of~$J$ to \ac{XML}, denoted \funcCall{translateXML}{J}, is an \ac{XML}
  document \xml{jsonObject}{\funcCall{translateMembers}{J}}, where \funcCall{translateMembers}{J} is defined as follows:
  % 
  \begin{itemize}[noitemsep]
  \item if~$J$ is an empty \ac{JSON} object or empty array, then \seq{};
  \item if~$J$ is a \ac{JSON} object, then for each~$K_i: V_i \in J$, \xml{K_i}{\funcCall{translateMembers}{V_i}};
  \item if~$J$ is a \ac{JSON} array, then for each~$E_i \in J$, \newline ~~~~~~\xml{arrayElement}{\funcCall{translateMembers}{E_i}};
  \item otherwise~$J$.
  \end{itemize}
\end{definition}
% 
\noindent For example the \ac{JSON} from \cref{fig:bands-json} translated into \ac{XML} according to
\cref{def:json2xml} is presented in \cref{fig:bands-json-xml}.
%
\begin{data}[t]
  \centering
  \lstinputlisting[language=XML,basicstyle=\small\ttfamily]{0-data+queries/usecaseData-json.xml}
  \caption{XML representation of JSON data}
  \label{fig:bands-json-xml}
\end{data}


\subsection*{Querying the XML representation of JSON}
\label{sec:query-jsonxml-repr}

\begin{query}[t]
  \centering
  \lstinputlisting[language=XSPARQL,basicstyle=\small\ttfamily]{0-data+queries/xsparql-json.xsparql}
  \caption{Querying JSON using XSPARQL}
  \label{qr:xsparql-json-xml}
\end{query}
%
\ac{JSON} data can be manipulated directly in JavaScript, where accessing members of objects can be done using the
\lit{.} separator, while accessing array elements is done using the standard bracket notation:~\lit{[} and~\lit{]}.
%
For example, if the \ac{JSON} object in \cref{fig:bands-json} is assigned to a JavaScript variable named \lit{b}, we
can access the member \lit{bands} by using \lit{b.bands} and accessing the second member of the Nightwish band can be
done with \lit{b.bands.Nightwish.members[1]}.\footnote{Please note that in JavaScript the first element of an array is
  at position 0, while the first element of XPath sequences is 1.}
%
In XSPARQL, querying the \ac{XML} representation of \ac{JSON} data can be done using an XPath expression, where,
assuming \funcCall{translateXML}{b} is assigned to an XSPARQL variable~\var{b}:
%
\begin{enumerate}[(i),noitemsep]
\item accessing members of an object can is done using the \lit{child} XPath axis, for example to access the
  representation of member \lit{bands} we write \lit{\var{b}/bands}; and
\item accessing specific elements of an array can be done using XPath predicates, \eg~to access the second member of the
  Nightwish band can be done with \lit{\var{b}/bands/Nightwish/members/*[2]}.\footnote{We can also use
    \lit{arrayElement} instead of \lit{*} in the XPath expression.}
\end{enumerate}
%
\begin{example}[Querying JSON using XSPARQL]
  %
  \cref{qr:xsparql-json-xml} presents the XSPARQL query that returns all members of the \stringValue{Nightwish}
  band from the (translated) \ac{JSON} \cref{fig:bands-json}.
  %
  In an XSPARQL query the transformation from \ac{JSON} into \ac{XML} is implemented using the
  \funcName{\qname{xsparql}{json\text{-}doc}} function (as shown in \cref{qr:xsparql-json-xml-line1} of
  \cref{qr:xsparql-json-xml}).
\end{example}
%
The implementation of the translation in \cref{def:json2xml} currently translates the complete \ac{JSON}
provided as input.  One possible optimisation for this implementation is to make it aware of the \ac{XPath} expression
and perform a selective translation of the input \ac{JSON} data.



%%% Local Variables:
%%% mode: latex
%%% mode: flyspell
%%% mode: reftex
%%% TeX-master: "../thesis"
%%% End:
