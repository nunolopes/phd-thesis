\section{Related Work}
\label{sec:relworks}

\begin{table}[t]
\caption{Overview of Related Work}
\label{fig:related-work-overview}
\centering
{\small
\begin{tabular}{r|>{\centering\arraybackslash}m{.63cm}>{\centering\arraybackslash}m{.66cm}>{\centering\arraybackslash}m{.63cm}|c|cc|c}
  \toprule
                                               &         \multicolumn{3}{c|}{Input Format}    & Target   &   \multicolumn{2}{c|}{Query Language}       & Ontology\\
  System                                       & {\footnotesize RDB} & {\footnotesize XML} & {\footnotesize RDF}        & Model    &   Surface      &  Target            & Generation\\
  \midrule
  Gloze                                        &      & \yes &      & RDF & --- & ---  & partial \\
  \citet{DroopFlarerGroppe:2008aa}             &      & \yes &      & RDF & SPARQL+XSLT & SPARQL & \no \\
  \citet{VrandecicDenglerRudolph:2005aa}       &      &      & \yes & --- & --- & SPARQL &  \no \\
  \citet{BohringAuer:2005aa}                   &      & \yes &      & RDF & --- & --- &  \yes \\
  \citet{RodriguesRosaCardoso:2008aa}          &      & \yes &      & RDF &--- & --- &  \yes \\
  \citet{FischerFlorescuKaufmann:2011aa}       & \yes & \yes & \yes & --- &--- & XQuery &  \no \\
  \citet{Walsh:2003aa}                         &      & \yes & \yes & --- &--- & --- &  \no \\
  \midrule
  \citet{BerruetaLabraHerman:2008aa}           &      & \yes & \yes & RDF & XSLT+SPARQL & --- &  \yes \\
  \citet{BikakisGioldasisTsinaraki:2009aa}     &      & \yes & \yes & --- &  SPARQL & XQuery & \yes \\
  \citet{GroppeGroppeLinnemann:2008aa}         &      & \yes & \yes & XML & SPARQL & XQuery &  \yes \\
  MarkLogic Server                             &      & \yes & \yes & XML &SPARQL & XQuery &  \yes \\ 
  \citet{CorbyKefi-KhelifCherfi:2009aa}        & \yes & \yes & \yes &  --- & SPARQL & --- & \no \\
  RDB2RDF                                      & \yes &      & \yes & RDF &--- & --- &  \no \\
  XSPARQL                                      & \yes & \yes & \yes & \begin{minipage}{.85cm}XML/ RDF\end{minipage} & XSPARQL & XQuery &  \no \\
  \bottomrule
\end{tabular}%
}
\end{table}
%
Several proposals for integrating data from relational databases, \ac{XML}, and \ac{RDF} were presented before. 
%
On one hand, converting between relational databases and \ac{XML} has been long studied, either by the integration of
\ac{SQL} and \ac{XML}~\cite{EisenbergMelton:2001aa,EisenbergMelton:2004aa} or the specification of the representation of
database instances in \ac{XML}.\footnote{\url{http://www.w3.org/XML/RDB.html}, retrieved on 2012/07/17.}  
%
In practice, most relational database management systems include a datatype for storing \ac{XML} data while other works
focus on the implementation of the XQuery language over a relational database
backend~\cite{GrustSakrTeubner:2004aa,GrustRittingerTeubner:2008aa}.
%
As such, this
section focuses on the integration of \ac{XML} and \ac{RDF} or relational databases and \ac{RDF} data.

To begin, an analysis of tools for converting between relational databases and \ac{RDF} is presented
by~\citet{GrayGrayOunis:2009aa}, which also aims at studying the expressivity of SPARQL to represent scientific queries,
namely in the astronomy domain.
%
Although, as stated by the authors, data and queries were mostly numeric and thus biased towards relational data and
SQL, the comparison gives a good overview of how the tested tools perform in comparison to relational databases.  
%
Some of the conclusions indicate that these tools are still not able to compete with relational databases in terms of
performance and that SPARQL is also not yet expressive enough to pose the necessary queries.


\citet{Patel-SchneiderSimeon:2003aa} present a proposal to integrate the semantics of \ac{XML} and \ac{RDF} by defining
a model-theory that encapsulates both the \ac{XDM} and \ac{RDF} data models.
%
This proposal has not been applied in practice and most of the existing proposals to merge \ac{XML} and \ac{RDF} rely on
translating the data from different formats and/or translating the queries from different languages.
%
With this in mind, we divided the proposals into the following categories:
\begin{enumerate}[label=(\arabic*),noitemsep]
\item \label{item:rel-1} \textbf{Normalised Representations:} include proposals that suggest using a normalised format for representing
  \ac{RDF} in \ac{XML}.  Although similar to the next proposals, in these systems the translation can usually be
  automated and they do not address querying, simply reusing standardised languages.
\item \label{item:rel-2} \textbf{Translation of data:} these tools aim at integrating the heterogeneous data by translating between
  different formats, usually relying on user predefined mappings.
\item \label{item:rel-3} \textbf{Integration of query languages:} this category of approaches (where XSPARQL is also included) considers
  the integration and/or expansion of query languages to allow querying different formats without requiring the
  translation of data from the original formats.
\end{enumerate}
%
\cref{fig:related-work-overview} presents an overview of systems considered in~\ref{item:rel-2} and~\ref{item:rel-3}.
This table classifies the different systems according to whether they support input from relational databases (RDB),
\ac{XML}, or \ac{RDF}.  The target model indicates, if there exists a data translation step, what is the format used for
the integrated representation.  The Surface and Target query languages state, if available, the language in which the
system accepts queries and if they are translated into a different query language.  Finally, the Ontology Generation
column specifies if the system generates an ontology description based an input \ac{XML} Schema or relational database
structure.
%
We next give a short description of some of the tools and proposals available grouped by the presented categories.
%
\subsection*{Normalised Representations}
\label{sec:normalised-representations}
%
The following proposals specify a normalised syntax for \ac{RDF} in \ac{XML}.
%
The TriX format~\cite{CarrollStickler:2004aa} consists of an alternative normalised serialisation for \ac{RDF} in \ac{XML}, with the aim of
being compatible with standard \ac{XML} tools.  
%
In this serialisation, each \ac{RDF} triple is represented as a \texttt{triple} \ac{XML} element with three children
elements representing the subject, predicate, and object of the triple.
%
It uses \ac{XSLT} as an extensibility mechanism, allowing syntactic extensions to be specified and macros to be defined.

Also in 2003, \emph{TreeHugger}\footnote{\scriptsize\url{http://rdfweb.org/people/damian/treehugger/index.html},
  retrieved on 2012/07/17.}  defines abstraction functions (implemented as extensions of the Saxon XQuery engine) that
enable the navigation of an \ac{RDF} graph structure in both \ac{XSLT} and XQuery.
%
This navigation is specified using XPath-like expressions that specify the \ac{RDF} class and property that users want
to query, which are in turn translated into SPARQL queries.

Well known parsers for \ac{RDF}, such as the Redland \ac{RDF} Libraries\footnote{\url{http://librdf.org/}, retrieved on
  2012/07/17.} also provide canonical formats of RDF/XML.
%
\emph{R3X}\footnote{\url{http://wasab.dk/morten/blog/archives/2004/05/30/transforming-rdfxml-with-xslt}, retrieved on
  2012/07/17.} takes this representation one step further by grouping the canonical RDF/XML output of the Redland parser
and grouping the triples by subject.  The aim of this grouping by subject is to make the canonical format easier to
process with \ac{XSLT}.
%
Very similar to R3X, Grit\footnote{\url{http://code.google.com/p/oort/wiki/Grit}, retrieved on 2012/07/17.} also defines
a normalised format of RDF/XML where triples are grouped by their subject to facilitate processing in \ac{XSLT} and
improve the triple access evaluation times.  Furthermore Grit normalises \acp{URI} to make lookups easier in
\ac{XSLT}. 


\subsection*{Data Translation}
\label{sec:data-translation}
%
We now present the proposals that rely on a user-specified normalised format for \ac{RDF}.
%
Gloze~\cite{Battle:2006aa} aims at interpreting an \ac{XML} document as \ac{RDF} data based on the \ac{XML} Schema
definition.
%
\ac{XML} elements and attributes are mapped to \ac{RDF} object or datatype properties, depending on wether they are
described as complex or simple types in the XML Schema (complex types are mapped to object properties and simple types
are mapped to datatype properties). 

\citet{DroopFlarerGroppe:2008aa} translate the \ac{XML} document into RDF, annotating it with necessary information to
answer XPath queries, namely the ordering, axes relations between XML elements, and attributes of elements.
%
The authors then propose to integrate XPath queries into SPARQL as subqueries in \acp{BGP}, where the result of the
subexpressions is assigned to SPARQL variables.
%
These XPath subexpressions are in turn translated into SPARQL queries that, using the introduced annotations, allow the
preservation of the semantics of the original queries and ordering of solutions.

\citet{DeursenPoppeMartens:2008aa} presents an approach for the transformation between \ac{XML} and \ac{RDF} by
specifying mappings between an XML Schema and an \ac{OWL} ontology.
%
The authors introduce a language for the mapping specification, relying on XPath expressions for selecting the \ac{XML}
elements, and defining with \ac{OWL} classes the elements are mapped to.  The target \ac{RDF} data is generated by
processing these input mappings.

\citet{VrandecicDenglerRudolph:2005aa} suggests using a normalised form of RDF/XML by specifying a restricted form of
\acp{DTD} that generate normalised \ac{XML} format and again relying on standard \ac{XML} processing tools for
subsequent transformations.  The provided \ac{DTD} is used to generate SPARQL queries that access the \ac{RDF} data and
the system then relies on post-processing of the SPARQL query results to generate the desired output.  The use of
\acp{DTD} and automatic generation of SPARQL queries allows to leverage the existing \ac{XML} users that are not
familiar with \ac{RDF} technologies.


Also catering for \ac{SQL} queries, \citet{FischerFlorescuKaufmann:2011aa} present a translation of both \ac{SQL} and
SPARQL queries into XQuery.  Again the translation of SPARQL to XQuery operates on a normalised form of RDF/XML and thus
a data translation step is required.  A similar approach is taken for translating relational data into \ac{XML} and then
rewriting \ac{SQL} to XQuery.  In this paper the authors do not present an extended syntax language for the combination
of data in the different formats and rather rely on the translation of data into \ac{XML}.


\emph{RDF Twig}~\cite{Walsh:2003aa} suggests \ac{XSLT} extension functions that provide views on the sub-trees of an RDF
graph. 
%
The main idea of \ac{RDF} Twig is that while RDF/XML is hard to navigate using XPath, a subtree of an \ac{RDF} graph can
be serialised in more useful forms that facilitate navigation.  As such the authors provide \ac{XSLT} extension
functions that create different views of parts of the input \ac{RDF}.



Several other approaches aim at automatically translating an \ac{XML} Schema into an equivalent \ac{OWL}
ontology~\cite{BohringAuer:2005aa,RodriguesRosaCardoso:2008aa}, focusing on mapping \ac{XML} elements to \ac{OWL}
classes and properties.
%
However in XSPARQL, we are focusing on translation and integration of instance data, rather than aiming to provide a
semantic interpretation for \ac{XML} data.


While not catering for the integration of \ac{XML}, several other approaches focus on mapping relational data to
\ac{RDF}.
%
For instance, D2R Server~\cite{Bizer:2003aa} and D2R Map or Triplify~\cite{AuerDietzoldLehmann:2009aa} enable the
conversion between RDB data and \ac{RDF}.
%
Large commercial database companies are also providing solutions for \ac{RDF} triple stores, such as
Oracle~\cite{DasSrinivasan:2009aa} and Virtuoso~\cite{ErlingMikhailov:2007aa}.  Most of these projects assume a fixed
translation schema where, for instance, database tables are translated into \ac{RDFS} classes and table attributes are
represented as properties.



\subsection*{Language Integration}
\label{sec:language-integration}
%
In this category of proposals we include the systems that consider the integration and/or expansion of query languages
that allow the querying different formats without requiring the translation of data from the original formats.

\citet{BerruetaLabraHerman:2008aa} presents a framework that facilitates SPARQL queries to be performed from \ac{XSLT}:
XSLT+SPARQL.  It adds functions to \ac{XSLT} that provide the ability to query SPARQL endpoints and uses standard
\ac{XSLT} to process the SPARQL \ac{XML} results format.
%
Similar to our current implementation, this relies on a clear separation between the SPARQL query and \ac{XSLT} parts of the
query.



Some proposals suggest compiling a SPARQL query to \ac{XSLT} or XQuery. \citet{BikakisGioldasisTsinaraki:2009aa}
translate each SPARQL query into an XQuery using a mapping from \ac{OWL} to \ac{XML} Schema.  The translation from
SPARQL to XQuery is guided by the provided mapping (which can be automatically generated by a separate system) and thus
allows the use of the SPARQL query language to access legacy \ac{XML} data without the need to perform data translation.

Similarly~\citet{GroppeGroppeLinnemann:2008aa} proposes to embed SPARQL into \ac{XSLT} or XQuery, by presenting
extensions to these languages that enable to query \ac{RDF} data.  In this proposal each SPARQL query is also translated
into an equivalent XQuery.
%
This language is very close to the XSPARQL language but however it requires converting the \ac{RDF} data to \ac{XML}
according to a predefined schema.
%
Assuming the queried dataset is available beforehand, this translation introduces an overhead to the query and, in case
the dataset is not available for example due to being stored behind a SPARQL endpoint, such translation is not possible.
%
In \cref{cha:optimisations}, we present some benchmark comparisons between an implementation of this language
(provided to us by the authors) and our implementation of the XSPARQL language.

Ding and Buxton presented another approach to translate SPARQL into XQuery at the 2011 Semantic Technology
Conference.\footnote{\url{http://semtech2011.semanticweb.com/sessionPop.cfm?confid=62&proposalid=4015}, retrieved on
  2012/07/17.} 
%
This rewriting generates XQuery specifically tailored for the Marklogic Server \ac{XML} database
engine,\footnote{\url{http://www.marklogic.com/products-and-services/marklogic-5/}, retrieved on 2012/07/17.} which
incorporates \ac{RDF} triples by using an internal \ac{XML} representation.


Part of the CORESE Semantic Web framework,\footnote{\url{http://wimmics.inria.fr/corese}, retrieved on 2012/07/17.}
\citet{CorbyKefi-KhelifCherfi:2009aa} provides extensions of SPARQL to process SQL, XPath, and \ac{XSLT} in SPARQL queries.  
%
The authors also define an \ac{XSLT} extension function that allows to evaluate SPARQL queries and integrate the query
result into the \ac{XSLT} processing.
%
The implementation of these extensions is based on the CORESE framework, which employs caching mechanisms for the input
\ac{XML} and \ac{RDF} documents.
%
This approach is again similar to XSPARQL however the choice here was to extend SPARQL and \ac{XSLT}, opposed to XSPARQL's extension
of XQuery.  


The Saxon XQuery engine (which we are using in our implementation) provides extension functions that allow to execute
\ac{SQL} queries and represent the results of the query as \ac{XML}, easily incorporating them into the XQuery or
\ac{XSLT} query.
%
This feature follows a similar implementation as XSPARQL but however does not provide the extend syntax as XSPARQL.  The
extension function executes a \ac{SQL} query, although the functionality of injecting variable values provided by
XSPARQL can be done, this task is left in charge of the query writer.

The nSPARQL query language~\cite{PerezArenasGutierrez:2008aa} proposes to extend SPARQL with navigational capabilities
using nested regular expressions.  With this addition, the language is sufficiently expressive to capture the semantics
of \ac{RDFS}.  In addition to this, it introduces a number of graph navigation operators and adds the ability to selectively
traverse the graph.  This work is different than our current proposed approach for XSPARQL, but one of the possibilities
for extending XSPARQL is to enable it to perform XQuery enriched SPARQL queries.



%%% Local Variables:
%%% fill-column: 120
%%% TeX-master: t
%%% TeX-PDF-mode: t
%%% TeX-debug-bad-boxes: t
%%% TeX-parse-self: t
%%% TeX-auto-save: t
%%% reftex-plug-into-AUCTeX: t
%%% mode: latex
%%% mode: flyspell
%%% mode: reftex
%%% TeX-master: "../thesis"
%%% End:
