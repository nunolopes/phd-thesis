\section{Semantic Correspondence between XSPARQL, SQL, XQuery, and SPARQL}
\label{sec:correspondence}
%
Since XSPARQL syntactically extends XQuery, and also any SPARQL \CONSTRUCT query is syntactically valid in XSPARQL,
these queries are considered semantically equivalent to the semantics in their base languages.
%
Regarding \ac{SQL} and SPARQL \SELECT expressions, we can show that their results remain the same under XSPARQL extended
semantics.
%
The next propositions formally establish this intuitive correspondence.

\begin{proposition}
\label{prop:xsparql}
  XSPARQL is a conservative extension of XQ\-uery.
\end{proposition}
%
\begin{proof}
  %
  The additional rules introduced in \cref{sec:semantics} do not modify the semantics of any native XQuery: the
  XSPARQL semantics -- expressed in terms of normalisation rules, static typing rules and dynamic evaluation rules --
  strictly extends the native semantics of XQuery.
  %
  In the semantics definition we also define new environment components, namely $\stat.\ecomp{globalPosition}$ and
  $\dyn.\textrm{activeDataset}$, which are not used in the XQuery semantics and thus do not interfere with query
  evaluation.
  %
  The only rules that use these newly created environments are the evaluation rules of \SparqlForClause{s}
  (\dyn.\ecomp{activeDataset}) and the dynamic evaluation rule~\eqref{eq:bnode-skolem} (\dyn.\ecomp{globalPosition}),
  which generates Skolem-identifiers for blank nodes in \CONSTRUCT parts.
  % 
  However, all these rules only apply to XSPARQL queries, which fall outside the native XQuery fragment, whereas the
  semantics of native XQuery queries remains untouched and independent of the extra environment components in XSPARQL.
  %
\end{proof}


We can also show that the answers of an XSPARQL \SQLForClause without any previously bound XSPARQL variables are the
same as the answers of the normalised query under \ac{SQL} semantics:
%
\begin{lemma}
  \label{lemma:xsparql-relation-instance}
  Let~$S$ be a \SQLForClause,~$xir_C$ the XSPARQL instance relation of~$S$, and~$S' = \funcCall{normaliseSQL}{S}$ the
  \ac{SQL} normalised query of~$S$. 
  %
  Furthermore, let~$R_1 = \RAXSP{S}$ and~$R_2 = \RASQL{S'}$, where~$\funcCall{sort}{R_2} = U$ be the relation instances
  resulting from evaluating~$S$ according to the XSPARQL semantics and the \ac{SQL} semantics, respectively.
  %
  If~$S$ does not contain any XSPARQL variables, \ie~$\vars{S} = \emptyset$, then~$R_1[U] =
  R_2$.
\end{lemma}
%
\begin{proof}
  %
  Following \cref{def:xsparql-sql-answers} we have that the answers of~$S$ under XSPARQL semantics are given
  by~$\funcCall{\sigma_{E}}{R_1 \times xir_C}$, where~$E$ is the XSPARQL select expression of~$S$.  
  % 
  Since~$\vars{S} = \emptyset$, according to \cref{def:xsparql-join},~$E$ will also be empty and we can
  simplify the expression that produces~$R_1$ to~$R_2 \times xir_C$.
  % 
  According to the definition of XSPARQL instance relation (\cref{def:xml2sqlterm})~$xir_C$ has
  cardinality~$1$ and thus the cross product does not change the cardinality of $R_2$, simply extending each solution
  in~$R_2$ with the attributes from the~$xir_C$ relation.  
  % 
  Since the cardinality of~$R_1$ and~$R_2$ is the same and the~$\times$ operation does not change any existing
  attributes of~$R_1$, we have that~$R_1[U] = R_2$.
\end{proof}

Similarly for SPARQL, we show the equivalence between SPARQL \ac{BGP} Matching~\cite[Section
12.3.1]{PrudhommeauxSeaborne:2008aa} and XSPARQL \ac{BGP} Matching (presented in \cref{sec:bf-extended-bgp}).
%
Based on this, we can then prove the equivalence between the XSPARQL and SPARQL semantics for \CONSTRUCT queries.
%
\begin{lemma}
  \label{lemma:xsparql-instance-mapping}
  Given a graph pattern~$P$, a dataset~$D$ and~$\mu_C$ the XSPARQL instance mapping of~$P$. Furthermore, let $\Omega_1 =
  \evalXS{D}{P}{\mu_C}$ and $\Omega_2 = \evalS{D}{P}$ be solution mappings.  If $\vars{P} \cap \dom{\mu_C} = \emptyset$,
  then $\Omega_1 = \Omega_2 \bowtie \set{\mu_C}$.
\end{lemma}
%
\begin{proof}
  %
  The XSPARQL BGP matching, $\evalXS{D}{P}{\mu_C}$, extends SPARQL's BGP matching, $\evalS{D}{P}$, by defining that the
  solutions of the BGP are the ones \emph{compatible} with the \emph{XSPARQL instance mapping} $\mu_C$.
  %
  Since the evaluation of graph patterns (such as \UNION, \OPTIONAL, \GRAPH and \FILTER) remains unchanged from the
  SPARQL semantics let us focus on the evaluation of a BGP $P$. 
  %
  If there are no shared values between the graph pattern and the XSPARQL instance mapping, as is the case
  when~$\vars{P} \cap \dom{\mu_C} = \emptyset$, then each solution $\mu \in \Omega_2$ returned by the SPARQL BGP
  evaluation semantics is trivially compatible with $\mu_C$ and the result of the XSPARQL BGP matching is $\mu \cup
  \mu_C$.  Extending this result to all solution mappings in $\Omega_2$, we obtain that $\Omega_1 = \Omega_2 \bowtie
  \set{\mu_C}$.
  % 
\end{proof}


\noindent
%
Finally, for SPARQL \CONSTRUCT queries we can state the following:
%
\begin{proposition}
\label{prop:xsparqlsparql}
  XSPARQL is a conservative extension of SP\-ARQL \CONSTRUCT queries.
\end{proposition}
%
\begin{proof}
  % 
  For XSPARQL queries consisting of a standalone SPARQL \CONSTRUCT query, there cannot exist any previous bindings for
  variables in XSPARQL and thus the XSPARQL instance mapping $\mu_C$ over which the \CONSTRUCT query will be executed is
  empty.  Let $P$ represent the graph pattern of the \CONSTRUCT query and $D$ the dataset, since~$\mu_C$ is empty,
  trivially there are no shared variables between $\mu_C$ and $P$.  Thus, following
  \cref{lemma:xsparql-instance-mapping} the bindings for XSPARQL BGP matching (say $\Omega_1$) are the same
  bindings as SPARQL BGP matching ($\Omega_2$), since $\Omega_1 = \Omega_2 \cup \{\emptyset\}$ and hence $\Omega_1 =
  \Omega_2$.
  % 
  Furthermore the formal semantics function $\funcName{fs{:}evalTemplate}$ returns an RDF graph satisfying all the
  conditions of \cref{def:sparql-construct}:
  \begin{inparaenum}[(i)]
  \item ignoring invalid RDF triples -- \cref{def:sparql-construct-1} -- is guaranteed by
    Rules~\eqref{validTriple1} and~\eqref{validTriple2}; and
  \item the generation of distinct blank nodes for each solution sequence -- \cref{def:sparql-construct-2} -- is
    enforced by the blank node skolemisation rules, Rules~\eqref{eq:bnode-skolem} and~\eqref{eq:bnode-skolem-default}.
  \end{inparaenum}
  %
\end{proof}





%%% Local Variables:
%%% fill-column: 120
%%% TeX-master: "../thesis" 
%%% TeX-PDF-mode: t
%%% TeX-debug-bad-boxes: t
%%% TeX-parse-self: t
%%% TeX-auto-save: t
%%% reftex-plug-into-AUCTeX: t
%%% End:
