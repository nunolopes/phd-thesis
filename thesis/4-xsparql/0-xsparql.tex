\chapter{The XSPARQL Language}
\label{cha:xsparql}


This chapter introduces a language that is capable of querying, transforming, and exposing data from heterogeneous
sources, namely sources adhering to the data models presented in \cref{cha:data-models}: the relational model,
the tree-based \ac{XML} and \ac{JSON} formats, and the graph-based \ac{RDF} model.
%
This language, called XSPARQL, is based on the existing standard query languages described in
\cref{cha:query-languages}: \ac{SQL}, XQuery, and SPARQL, that are used to query the heterogeneous input sources.
%
Since \ac{JSON} does not specify a query language XSPARQL, automatically converts \ac{JSON} into a predefined \ac{XML}
representation over which it is possible to use XQuery and XPath (this approach is detailed in \cref{sec:xsparql-json}).
%
XSPARQL consists of an extension of the XQuery language with syntactical constructs from both \ac{SQL} and SPARQL and as
such XSPARQL is an XQuery-flavoured language, whose semantics is defined as an extension of the XQuery
semantics.
%
As a first example we can use this language to expose data in relational databases as \ac{RDF} or \ac{XML} data, in a
similar approach to current proposals for translating relational data to \ac{RDF} (RDB2RDF).  But furthermore a common
language including \ac{SQL}, XQuery and SPARQL can support more involved transformations between different formats, for
instance, enabling the integration of enterprise legacy data into \ac{LOD} as described in \cref{cha:introduction}.
%
The importance of converting data between these data models has been acknowledged within the \ac{W3C} in several
standardisation efforts: \ac{GRDDL}~\cite{Connolly:2007aa}, \ac{SAWSDL}~\cite{FarrellLausen:2007aa}, and more recently
RDB2RDF~\cite{ArenasPrudhommeauxSequeda:2011aa,DasSundaraCyganiak:2011aa}.


In data integration scenarios, such as the one described in \cref{cha:introduction}, we often call the transformations
from the different formats into \ac{RDF}~\emph{lifting} and the transformations in the opposite direction
\emph{lowering}.  The names derive from the fact that \ac{RDF} is classified as having a higher abstraction level when
compared to relational data or even semi-structured \ac{XML} data.

\subsubsection*{Lifting: Transforming Heterogeneous Sources into RDF}
%
Within the \ac{W3C}, the \ac{GRDDL} working group addressed the lifting task by allowing \ac{RDF} data to be extracted
from existing \ac{XML} and (X)HTML Web pages.  The \ac{XML} or \ac{HTML} document can link (by means of a specialised
vocabulary) to \ac{XSLT} transformations that, when applied to the original document, produce the \ac{RDF} data.
%
In the Web Services community, the Web Services Description Language (WSDL)~\cite{ChinniciMoreauRyman:2007aa} is an
\ac{XML}-based language for describing the messages that a web service accepts (sends and receives).  The \ac{SAWSDL}
working group focused on defining mechanisms to add annotations to WSDL documents that allow the \ac{XML} messages of a
web service to be transformed into \ac{RDF} (adhering to a specified schema) and, vice versa, enable the lowering of
data stored in \ac{RDF} and the creation of target \ac{XML} messages.
%
The ongoing RDB2RDF Working Group focuses on transforming data between the relational model and \ac{RDF}, enabling the
vast amounts of data contained in relational databases to be exposed as \ac{RDF}, for example most enterprise data (as
discussed in \cref{cha:introduction}).
%
The RDB2RDF Working Group has defined a mapping vocabulary that specifies how existing relational data can be converted
into \ac{RDF}.  In \cref{sec:examples-rdb2rdf} we will look at how the XSPARQL language implements this
specification.


As described in \cref{sec:rdfxml}, RDF/XML~\cite{BeckettMcBride:2004aa} is the recommended syntax for RDF, using
XML as the underlying representation model and, based on this format, it is conceivable to use \ac{XML}-based tools,
such as \ac{XSLT} or XQuery, to produce \ac{RDF} data.
%
Both the \ac{GRDDL} and \ac{SAWSDL} specifications use \ac{XSLT} to perform lifting and lowering, however, as we will
show, approaches that rely on RDF/XML for transformations between \ac{RDF} and \ac{XML} have several disadvantages.
%
In the following examples we are using XQuery to perform the different transformations, similar transformations can also
be achieved using \ac{XSLT} but this does not invalidate any of the drawbacks we present.
%
\begin{query}[t]
  \lstinputlisting{0-data+queries/lifting-bands.xquery}
  \caption{Lifting using XQuery}
  \label{qr:lifting_xq}
\end{query}
%
\begin{example}[Lifting in XQuery]
  As an example of the lifting transformation, \cref{qr:lifting_xq} presents the XQuery that converts the \ac{XML}
  data from \cref{fig:bands-xml} into \ac{RDF}.
  % 
  This query produces an \ac{RDF} graph similar to the one presented in \cref{fig:bands-rdfxml-abbrev} with the
  exception that it uses blank nodes as idenfiers for all entities, while the graph from \cref{fig:bands-rdfxml-abbrev}
  uses DBpedia \acp{URI} as identifiers.
  % 
  The blank node labels assigned to each entity are generated by using a prefix for each type of entity:
  \textbf{(b)}ands, \textbf{(m)}usic artists, \textbf{(a)}lbums, and \textbf{(s)}ongs, followed by a sequencial
  identifier (\cf~\qname{rdf}{nodeID} in \cref{nodeid}).
  %
  Having determined all the identifiers (in \crefrange{pos-start}{pos-end}), the query produces the required RDF/XML
  structure: the triples refering to bands, their name and its members are generated in
  \crefrange{bands-start}{bands-end}.
  %
  A similar process is then repeated for artists (\crefrange{artists-start}{artists-end}), albums
  (\crefrange{albums-start}{albums-end}), and songs (\crefrange{songs-start}{songs-end}).
  % 
\end{example}
% 
While this example presents a valid solution for lifting, we can observe the following drawbacks:
%
\begin{itemize}[noitemsep]
\item we have to build RDF/XML manually and cannot use the more readable and concise Turtle syntax; and
\item the resulting \ac{RDF} data is not  guaranteed to be valid (according to \cref{def:rdf-triple}).
\end{itemize}
%
The task of lifting data from relational databases can be performed in a similar fashion by relying on
SQL/XML~\cite{EisenbergMelton:2001aa} however this would introduce an indirection step by first having to transform data
into \ac{XML} and then into \ac{RDF}.
%
Combining \ac{SQL}, XQuery, and SPARQL in XSPARQL simplifies the lifting process, allowing to use SPARQL
\ConstructClause{s} to generate \ac{RDF} in Turtle format (directly from relational data or \ac{XML}) and performing
automatic validation of the generated \ac{RDF}.



\subsubsection*{Lowering: Transforming from RDF into the Legacy Formats}
\label{sec:lowering}

As we have seen the lifting task can be accomplished (with some drawbacks) by using \ac{XSLT} or XQuery.
%
On the other hand, converting from \ac{RDF} data back into the legacy data models using \ac{XML} tools poses obstacles
that are even harder to overcome, namely:
%
\begin{itemize}[noitemsep]
\item the flexibility of the RDF/XML format (and the lack of a canonical format) makes writing transformations
  difficult;
\item merging different \ac{RDF} graphs may involve complex processing (\eg~renaming of blank nodes); and
\item possibly handling the interplay with inference mechanisms \eg~\ac{RDFS} would require custom-built code.
\end{itemize}


As we have presented in \cref{sec:rdfxml}, the RDF/XML serialisation format is very flexible, as it includes several
shortcuts and allows for different representations of the same \ac{RDF} graph.  For example, we have shown two
equivalent serialisations for the same \ac{RDF} graph in \cref{fig:bands-rdfxml,fig:bands-rdfxml-abbrev}, however both
serialisations are very different when we focus on their \ac{XML} structure.
%
Since using \ac{XML} tools requires handling RDF/XML as \ac{XML} data, all the possible different serialisations for an
\ac{RDF} graph would need to be taken into account.


\begin{query}[t]
  \lstinputlisting[]{0-data+queries/lowering-bands.xquery}
  \caption{Lowering using XQuery}
  \label{qr:xquery_lowering}
\end{query}
%
\begin{example}[Lowering in XQuery]
  \cref{qr:xquery_lowering} performs the lowering task directly from RDF/XML using XQuery. This query first
  retrieves all the \qname{mo}{Band} \ac{XML} elements (\crefrange{low-bands-start}{low-bands-end}) and, for each
  band, retrives the names of the artists (\crefrange{low-members-start}{low-members-end}) and albums
  (\crefrange{low-albums-start}{low-albums-end}) of the band.
  % 
  Furthermore, all the song names of each album are collected in \crefrange{low-songs-start}{low-songs-end}.
  % 
  This process creates the desired nested structure of the \ac{XML} file presented in \cref{fig:bands-xml}.
\end{example}
% 
One issue is that \cref{qr:xquery_lowering} is tailiored specifically to the RDF/XML serialisation from
\cref{fig:bands-rdfxml-abbrev} and will not produce the desired results if the serialisation changes.
%
Although creating a query capable of handling any RDF/XML serialisation would be possible, this would be a cumbersome
and error-prone task.  Futhermore, if the \ac{RDF} data is stored in the Turtle serialisation it is not possible to use
\ac{XML} tools.

On the other hand, SPARQL is agnostic to the actual XML representation of the underlying source graphs, which alleviates
the pain of having to deal with different RDF/XML representations of the graphs. 
%
Also merging several RDF source graphs specified in consecutive \FROM clauses (as described in
\cref{sec:sparql-syntax}), which could involve renaming of blank nodes at the pure XML level, comes for free in SPARQL.
%
However, we cannot use SPARQL alone for the lowering transformations since SPARQL does not provide the possibility of
handling XML data.



Apart from its syntactic ambiguities, processing RDF/XML via XQuery also loses another feature of RDF, namely its
interplay with ontological information, \eg~\ac{RDFS}.
%
Since XML tools do not support ontological inference, we would need to implement an \ac{RDFS} inference engine within
\ac{XSLT} or XQuery, to cater for a lowering mechanism that also works for this kind of \ac{RDF} data. Given the
availability of \ac{RDF} tools and engines that readily offer \ac{RDFS} support via materialising inferences, this is a
dispensable exercise.  Furthermore, in \cref{cha:anql} we present an extension of the \ac{RDFS} inference rules (called
Annotated RDFS) and of the SPARQL query language towards meta-information and in \cref{cha:usecase} we present the
combination of XSPARQL and Annotated RDFS, which also introduces inferencing capabilities in XSPARQL.


\subsubsection*{Benefits of an Integrated Language}
%
In recognition of the above problems the \ac{SAWSDL} specification contains a non-normative example, which performs a
lowering transformation by applying an \ac{XSLT} transformation to the \ac{XML} representation of the results of a
SPARQL query~\cite{ClarkFeigenbaumTorres:2008aa}.
%
Such a two-step approach alleviates the issues described:
%
first, since SPARQL works on the RDF data model the different RDF/XML serialisations are considered to be equivalent,
and second, \ac{RDFS} inferences can also be catered for in the SPARQL engine.
%
Although the approach proposed by the \ac{SAWSDL} Working Group provides a good starting point, it can still be improved
on several points: firstly, the detour through SPARQL's XML query results format is an unnecessary burden.
%
Secondly, a more tightly-coupled integration of the different query languages can provide a more expressive language,
beyond the capabilities of using different languages sequentially, and directly amenable to query optimisations.
%
The proposed language, XSPARQL, aims to provide exactly this: use cases that otherwise would require interleaved calls
to SPARQL (typically requiring an implementation using an external programming framework) can be solved in XSPARQL
directly, \cf~the lowering example in \cref{qr:lowering_xsparql}.


In fact, the current version of SPARQL~\cite{PrudhommeauxSeaborne:2008aa} is still preliminary in terms of expressivity
when compared to SQL or XQuery.
%
As a side effect of this integration, XSPARQL also extends SPARQL's expressiveness for pure \ac{RDF} transformations by
allowing, for instance, nested XSPARQL queries in the graph construction step.
%
The SPARQL 1.1 query language (\cf~\cref{sec:sparql-11}), currently under development by the W3C SPARQL Working
Group, gives a leap forward in terms of expressivity, however, providing mechanisms to convert data back into the native
legacy data models of \ac{XML} or SQL databases is beyond the scope of the Working Group.





%%% Local Variables:
%%% fill-column: 120
%%% TeX-master: t
%%% TeX-PDF-mode: t
%%% TeX-debug-bad-boxes: t
%%% TeX-parse-self: t
%%% TeX-auto-save: t
%%% reftex-plug-into-AUCTeX: t
%%% mode: latex
%%% mode: flyspell
%%% mode: reftex
%%% TeX-master: "../thesis"
%%% End:


\bigskip 
%
We next present the syntax and semantics of the XSPARQL language in \cref{sec:syntax,sec:semantics}, respectively.  We
continue by presenting properties relating the novel language with its constituent languages
(\cref{sec:correspondence}), introduce our handling of \ac{JSON} data in \cref{sec:xsparql-json}, and presenting the
processing of RDB2RDF mappings within XSPARQL (\cref{sec:examples-rdb2rdf}).
%
Finally, we introduce a discussion on related works in \cref{sec:relworks}.

\section{Syntax}
\label{sec:syntax}



\begin{figure}[t]
  \centering
  { \scriptsize
    \setlength{\extrarowheight}{2pt}
    \begin{tabular}{llll}
      \cline{2-2}
      \multicolumn{1}{l|}{Prolog:} & \DECLARE \emph{prefix}=\texttt{"}\emph{namespace-URI}\texttt{"}&\multicolumn{1}{|c}{\multirow{2}{*}{\textbf{or}}}&\\
      \cline{2-2}
      \multicolumn{1}{l|}{} & \PREFIX \emph{prefix}: \texttt{<}\emph{namespace-URI}\texttt{>}&\multicolumn{1}{|c}{}&\\
      \cline{2-2}\\
      \cline{2-2}
      \multicolumn{1}{l|}{Body:} & \FOR \emph{var} \IN \emph{\XSPARQLFLWORExpr expression}&\multicolumn{1}{|c}{}&\multirow{4}{*}{\ForClause}\\
      \multicolumn{1}{l|}{} & \LET \emph{var} := \emph{\XSPARQLFLWORExpr expression}&\multicolumn{1}{|c}{}&\\
      \multicolumn{1}{l|}{}& \WHERE \emph{\XSPARQLFLWORExpr expression} &\multicolumn{1}{|c}{}&\\
      \multicolumn{1}{l|}{}& \ORDERBY \emph{\XSPARQLFLWORExpr expression}&\multicolumn{1}{|c}{\multirow{2}{*}{\textbf{or}}}&\\
      \cline{2-2}
      \multicolumn{1}{l|}{}& \FOR \emph{SelectSpec} &\multicolumn{1}{|c}{}&\multirow{3}{*}{\SQLForClause}\\
      \multicolumn{1}{l|}{}& \FROM \emph{RelationList} &\multicolumn{1}{|c}{}&\\
      \multicolumn{1}{l|}{}& \WHERE \emph{WhereSpecList } &\multicolumn{1}{|c}{\multirow{2}{*}{\textbf{or}}}&\\
      \cline{2-2}
      \multicolumn{1}{l|}{}& \FOR \emph{varlist} &\multicolumn{1}{|c}{}&\multirow{6}{*}{\SparqlForClause}\\
      \multicolumn{1}{l|}{}& \FROM\,/\,\FROMNAMED~\DatasetClause &\multicolumn{1}{|c}{}&\\
      \multicolumn{1}{l|}{}& \WHERE \{\emph{ pattern } \}&\multicolumn{1}{|c}{}&\\
      \multicolumn{1}{l|}{}& \ORDERBY \emph{expression}&\multicolumn{1}{|c}{}&\\
      \multicolumn{1}{l|}{}& \LIMIT \emph{integer $> 0$} &\multicolumn{1}{|c}{}&\\
      \multicolumn{1}{l|}{}& \OFFSET \emph{integer $> 0$}&\multicolumn{1}{|c}{}&\\
      \cline{2-2}\\
      \cline{2-2}
      \multicolumn{1}{c|}{Head:}& \CONSTRUCT \{ \emph{ConstructTemplate (with nested  \XSPARQLFLWORExpr expressions)} \} &\multicolumn{1}{|c}{\multirow{2}{*}{\textbf{or}}}&\multirow{1}{*}{\ConstructClause}\\
      \cline{2-2}
      \multicolumn{1}{c|}{}& \RETURN \emph{XML+ nested  \XSPARQLFLWORExpr expressions}&\multicolumn{1}{|c}{}&\multirow{1}{*}{\ReturnClause}\\
      \cline{2-2}
    \end{tabular}}
  \caption{Schematic view of XSPARQL}
  \label{fig:XSPARQL}
\end{figure}


%%% Local Variables:
%%% fill-column: 120
%%% TeX-master: t
%%% TeX-PDF-mode: t
%%% TeX-debug-bad-boxes: t
%%% TeX-parse-self: t
%%% TeX-auto-save: t
%%% reftex-plug-into-AUCTeX: t
%%% mode: latex
%%% mode: flyspell
%%% mode: reftex
%%% TeX-master: "../thesis"
%%% End:

%
Conceptually, XSPARQL is a merge of XQuery, SPARQL \CONSTRUCT and \SELECT queries, and \ac{SQL} \SELECT queries, as
presented schematically in \cref{fig:XSPARQL}.
%
This re-use of different query languages allows us to benefit from their facilities for retrieving data in the different
models, while also allowing us to use Turtle-like syntax for constructing RDF graphs (inherited from the SPARQL
language).
%
Since XSPARQL is based on XQuery, we allow any native XQuery expression and further extend XQuery's syntax with the following
expressions:
%
\begin{enumerate}[label=(\roman*),noitemsep]
\item XQuery and SPARQL namespace declarations in the Prolog may be used interchangeably;
\item in the Body, we allow the existing XQuery \ForClause{s} and also
  SPARQL \SELECT queries (\SparqlForClause) and
  SQL \SELECT queries (\SQLForClause); and

\item in addition to XQuery's native \ReturnClause, in the head we allow RDF graphs to be created directly using
  \CONSTRUCT templates (\ConstructClause).
\end{enumerate}
%
In XSPARQL we also allow different forms of nesting:
%
\begin{inparaenum}[(i)]
\item \LET assignments can contain the result of subqueries that construct \ac{RDF} graphs, and the assigned variable
  can later be used in SPARQL-style \FROM clauses, or
\item nesting can also be used for value construction within SPARQL-style \CONSTRUCT templates.
\end{inparaenum}
%
Since the new \SQLForClause and \SparqlForClause expressions stand at the same level as XQuery's \FOR and \LET
expressions, such clauses are allowed to start new \synt{\XSPARQLFLWORExpr} expressions and may also occur inside deeply
nested XSPARQL queries.
%
The main difference between these new expressions and \ac{SQL} and SPARQL \SELECT expressions is that while the latter
expressions return bindings for variables (as described in \cref{sec:sql,sec:sparql-preliminaries}), the new expressions
follow an approach similar to XQuery's \ForClause by adding new variables to the scope of query and as such we choose a
syntax also inspired by the XQuery \ForClause.


An overview of the grammar productions for these newly introduced expressions (\SQLForClause, \SparqlForClause, and
\ConstructClause) is presented in \cref{fig:xsparql-flwor}.  Notably, when compared to the XQuery grammar, we
introduced a new production (\synt{\XSPARQLFLWORExpr}) that changes the XQuery \FLWORExpr to include the new
expressions.
%
\setlength{\grammarindent}{90pt}%

\begin{figure}[t]
  \centering
  \begin{boxedminipage}{0.935\linewidth}
    {\scriptsize
      \vspace{-\grammarparsep}
      \begin{grammar}
        <\textbf{\XSPARQLFLWORExpr}>       ::= ( <FLWOExpr> | <\SQLForClause> | <\SparqlForClause> ) \\
        ( <\ReturnClause> | <\ConstructClause> )
        
        <FLWOExpr>        ::=  ( <\ForClause> | <LetClause> )+ <XQWhereClause>? <OrderByClause>? 
        
        <\ReturnClause>     ::=  `return' <ExprSingle> 
        
        <\textbf{\SQLForClause}>    ::=  `for' <SelectSpec> <RelationList> <SQLWhereClause>?
        
        <\textbf{\SparqlForClause}>  ::=  `for' ( <VarName>+ | `*' ) <DatasetClause>? <WhereClause>? <SolutionModifier>
        
        <\textbf{\ConstructClause}>  ::=  `construct' <ConstructTemplate'>
      \end{grammar}
    }
    \end{boxedminipage}
  \caption{\XSPARQLFLWORExpr syntax overview}
  \label{fig:xsparql-flwor}
\end{figure}


We next look at the syntax of each newly introduced expression in more detail while presenting some XSPARQL query
examples that allow us to perform the lifting and lowering tasks in a straightforward fashion.


\subsection{\SparqlForClause}
\label{sec:sparqlforclause}
%
{\scriptsize 
  \begin{grammar}
    <\textbf{\SparqlForClause}> ::= `for' (<VarRef>+ | `*') <DatasetClause>? <WhereClause>? <SolutionModifier>
  \end{grammar}
  \vspace{-.5\grammarparsep}
}

\noindent The newly introduced \SparqlForClause is similar to an XQuery \FOR expression that returns a sequence of
SPARQL results.
%
In this grammar production, the \SparqlWhereClause and \SolutionModifier correspond to rules~[13] and~[14] from the
SPARQL grammar, respectively \cf~\citet[Appendix~A.8]{PrudhommeauxSeaborne:2008aa}.
%
Similar to SPARQL's and \ac{SQL}'s \lit{\SELECT *} shortcut, we allow to write \lit{\FOR *} in place of \lit{\FOR
  [\emph{\it list of all unbound variables appearing in the \grammarRule{WhereClause}}]} in \SparqlForClause{s}, which
effectively avoids listing the distinguished variables of the query.

We also extended the rules for the SPARQL \stt{SourceSelector} grammar expression (rule~[12] of the SPARQL grammar) in
order to allow graphs in a dataset to be specified by a variable:

{\scriptsize
  \begin{grammar}
    <SourceSelector>   ::=  <IRIref> | <VarRef>
  \end{grammar}
}


The variables used here must contain an \ac{RDF} graph, resulting from a \ConstructClause (as described in the next
section and further detailed in \cref{sec:semantics}).




Regarding the syntax for variables in XQuery and SPARQL, we restrict the use of SPARQL \lit{?}-prefixed variables and
allow only \lit{\var{}}-prefixed variables that are compatible with XQuery's variable specifications.
%
On the other hand, as mentioned in \cref{sec:xquery}, XQuery also allows to specify variables as QNames, allowing the
disambiguation of variables based on their namespace.  However, since such variable names are not allowed in SPARQL we
further assume that only unprefixed variables are shared between the XQuery and SPARQL expressions of XSPARQL.

\begin{query}[t]
  \lstinputlisting[language=XSPARQL]{0-data+queries/lowering-bands.xsparql}%
  \caption{Lowering using XSPARQL}%
  \label{qr:lowering_xsparql}%
\end{query}
%
The lowering transformation can also be rewritten using XSPARQL.  These are the kind of transformations that present
extra problems for the \ac{XSLT} and XQuery languages and where we can see the advantages of using XSPARQL.  By using the
introduced \SparqlForClause{s} for accessing the \ac{RDF} graph, XSPARQL avoids handling \ac{RDF} as \ac{XML} data,
along with all the encapsulated issues.
%
\begin{example}[Lowering RDF data with XSPARQL]
  % 
  The lowering XSPARQL query for our running example is shown in \cref{qr:lowering_xsparql}. Here we can note the
  inclusion of \SparqlForClause{s}, for instance in \cref{qr:lowering_xsparql_bands_start}, to retrieve all the
  bands (\qname{mo}{Band}) contained in the \ac{RDF} data.
  %
  Furthermore, nested \SparqlForClause{s} can be used for further processing of the input data: the \SparqlForClause in
  \crefrange{qr:lowering_xsparql_members_start}{qr:lowering_xsparql_members_end} is responsible for retrieving all
  the members of the respective band, where the \litVar{band} variable is instantiated with the band identifier
  currently being processed.
  % 
  A similar structure is repeated for converting the corresponding albums of the band
  (\crefrange{qr:lowering_xsparql_albums_start}{qr:lowering_xsparql_albums_end}) and songs of each album
  (\crefrange{qr:lowering_xsparql_songs_start}{qr:lowering_xsparql_songs_end}).
  % 
\end{example}




\subsection{\ConstructClause}
\label{sec:constructclause-syntax}
%
{\scriptsize 
  \begin{grammar}
    <\textbf{\ConstructClause}> ::= `construct' <ConstructTemplate>
  \end{grammar}
  \vspace{-.5\grammarparsep}%
}
%
\noindent The \ConstructClause allows XSPARQL to produce \ac{RDF} graphs and, by following SPARQL's restrictions on the
generated \ac{RDF} triples (\cf~\cref{sec:sparql-syntax}), we also ensure that the resulting graph is valid
\ac{RDF}.
%
The XSPARQL \synt{\ConstructTemplate} expression is defined in the same way as the production \synt{\ConstructTemplate}
in SPARQL~\cite{PrudhommeauxSeaborne:2008aa}, but we additionally allow nested \synt{\XSPARQLFLWORExpr} expressions in
subject, predicate, and object positions.
%
\noindent We allow three types of nested expressions, identified by the shortcuts
\lit{\textbraceleft}~\synt{\XSPARQLFLWORExpr}~\lit{\textbraceright},
\lit{<\textbraceleft}~\synt{\XSPARQLFLWORExpr}~\lit{\textbraceright>}, and
\lit{\_:\textbraceleft}~\synt{\XSPARQLFLWORExpr}~\lit{\textbraceright} that construct \emph{literals}, \emph{URIs}, and
\emph{blank nodes}, respectively.
%
This syntax is used during static analysis to correctly determine the type of each element:~\type{lit\-eral},
\type{uri}, and \type{bnode} (\cf~\cref{sec:xsparql-types} below).

Additionally, we also allow SPARQL-style \ConstructClause{s} to appear before the body part of queries, and as such
XSPARQL becomes a syntactic superset of native SPARQL \CONSTRUCT queries (with the minor exception being the restriction
on \lit{?}-prefixed variables).


\begin{query}
  \lstinputlisting[language=XSPARQL]{0-data+queries/lifting-bands.xsparql}%
  \caption{Lifting in XSPARQL}%
  \label{qr:lifting_xs}%
\end{query}
%
The following lifting query shows the use of the \ConstructClause expression.
%
\begin{example}[Lifting XML data with XSPARQL]
  \cref{qr:lifting_xq} can be reformulated into its slightly more concise XSPARQL version in \cref{qr:lifting_xs}.
  %
  This query behaves in a similar way to \cref{qr:lifting_xq}, creating the \ac{RDF} triples for each entity in the
  input \ac{XML} data.  The difference is that we are using nested SPARQL-like \CONSTRUCT clauses for creating the
  \ac{RDF} triples (\cf~\crefrange{lift-xs-construct-bands-start}{lift-xs-bands-end}).
  %
  In \cref{lift-xs-songs-end} we use the different XSPARQL shortcuts, in this case to create \acp{URI} and literals.
  %
  The result of this query is also guaranteed to be valid \ac{RDF} as explained in \cref{sec:constructsem}.
\end{example}


\subsection{\SQLForClause}
\label{sec:syntax-rdb}
%
{\scriptsize 
  \begin{grammar}
    <\textbf{\SQLForClause}>    ::=  `for' <SelectSpec> <RelationList> <SQLWhereClause>?
  \end{grammar}
  \vspace{-.5\grammarparsep}%
}


\noindent The \SQLForClause element represents an \ac{SQL} \SELECT query that can be evaluated against the underlying
database.  Similar to XQuery's \FOR clause, the \SQLForClause expression represents the results of the execution of a
SQL query and exposes the result values to other subsequent expressions in the query.
%
\noindent The additional \SQLForClause syntax rules are presented next, where \synt{VarRef} corresponds to an XSPARQL
variable (\lit{\var{}}-prefixed), \synt{TableAlias} represents a string used as an alternative name for the relation,
and \synt{Constant} represents an integer or string:

{\scriptsize
  \begin{grammar}
    <SelectSpec>      ::=  <AttrSpecList> | `*'  | `row' <VarRef>

    <AttrSpecList>    ::=  <AttrSpec> <AttrNameSpec>? (`,' <AttrSpec> <AttrNameSpec>?)*

    <AttrSpec>        ::=       <attrName> 
                           \alt <relationName> `.' <attrName> 
                           \alt <VarRef> 

    <AttrNameSpec>    ::=  `as' <VarRef>

    <RelationList>    ::=  `from' <RelationSelector> (`,' <RelationSelector>)*

    <RelationSelector> ::=      <RelationName> (`as' <RelationAlias>)? 
                           \alt <VarRef> (`as' <RelationAlias>)?

    <SQLWhereClause>  ::=  `where' <WhereSpecList>

    <WhereSpecList>   ::=  `(' <WhereSpecList> <BooleanOp> <WhereSpecList> `)' 
                            \alt <WhereSpec> 

    <WhereSpec>       ::=        <WhereAttrSpec> <ComparisonOp> <WhereAttrSpec> 
                            \alt <WhereAttrSpec> <ComparisonOp> <Constant>
                            \alt <Constant> <ComparisonOp> <WhereAttrSpec>

    <WhereAttrSpec>   ::=        <AttrSpec> 
                            \alt `\textbraceleft' <VarRef> `\textbraceright'

    <BooleanOp>       ::=  `and' | `or'

    <ComparisonOp>    ::=  `=' | `!=' | `!=' | `<' | `<=' | `>' | `=>'
  \end{grammar}
}%
%
When comparing the XQuery and \ac{SQL} languages we find a syntactical mismatch between the representation
of variables: while \ac{SQL} considers the relation names specified in \synt{RelationSelector} as variables (as
described in \cref{sec:sql}), XQuery assumes \lit{\$}-prefixed variable names.
%
XSPARQL provides ways of overcoming this mismatch, allowing to specify variable names for the results of an
\SQLForClause, by:
%
\begin{enumerate}[label=(\roman*), noitemsep]
\item explicitly specifying a variable name for each attribute -- represented by the syntax rule \synt{AttrNameSpec},
  where \synt{VarRef} is the variable name to which the attribute value is assigned: \eg~\lit{for bands.bandId as
    \$bandId};
\item \label{forStar} implicitly by omitting the variable name or using \lit{\FOR *}; and
\item \label{forRow} using the \lit{row} keyword instantiates the specified variable with each \emph{result row} the
  query produces.
\end{enumerate}
%
For~\ref{forStar}, each attribute in the result set is assigned a variable name automatically with the same name as the
attribute name, of the format: \litVar{relationName.attributeName}.
%
\begin{example}[Variable Name Generation]
  %
  Consider the relational schema presented in \cref{ex:bands-schema}.
  %
  If we specify a \SQLForClause in the form of \lit{for * from person}, the variable names that will be available for
  the query will be \litVar{person.personId}, \litVar{person.personName}, and \litVar{person.bandId}.
  % 
\end{example}

If the relation attributes are not known beforehand, \eg~if the relation is specified as a variable, it is not possible
to generate the variable names as described in~\ref{forStar}.
%
In this case, we can use \lit{row \$r} in place of the variable names specification, and at execution time, \lit{\$r}
will be instantiated with an \ac{XML} representation containing all the attributes in the queried relations.
%
It is then possible to access all the attributes or to retrieve (if known) a specific attribute.
%
This form of selecting attributes is necessary for processing RDB2RDF mappings (presented in \cref{sec:r2rml-xsparql})
since the queried relations and attributes are read from a user-specified \ac{RDF} graph and thus the attributes of the
relation cannot be determined during syntactical analysis of the query.

In \ac{SQL}, \WHERE clauses indicate specific values of an attribute to be matched or that the value of two attributes
must be the same.  When we introduce the extended XSPARQL syntax (which allows to use \var{}-prefixed variables) we need
a way to specify if the variable represents an attribute name or an attribute value.  We make this distinction in the
syntax of XSPARQL: a \var{}-prefixed variable represents an attribute value, in case we want a variable to represent an
attribute name of a relation we use the \lit{\textbraceleft} \synt{VarRef} \lit{\textbraceright} syntax.  Further
details on how XSPARQL handles this distinction are presented in \cref{sec:semantics}.

In a similar fashion to the lifting query from \ac{XML} data (\cref{qr:lifting_xs}), we can use \SQLForClause{s} to
access relational data and convert it to \ac{RDF}, as presented in the following example.
%
\begin{query}[t]
  \centering
  \lstinputlisting{0-data+queries/xsparql-rdb-bands.xsparql}%
  \caption{Lifting from relational database}%
  \label{qr:xsparql-db-example}%
\end{query}
%
\begin{example}[Lifting Relational data with XSPARQL]
  \label{ex:xsparql-rdb}
  %
  \cref{qr:xsparql-db-example} shows an XSPARQL query that performs the lifting task over the relational schema
  described in \cref{ex:bands-schema}.
  %
  In this query we are using the primary key (generated identifier) of each relation for generating the blank node label
  of each entity (\cf~\cref{qr:xsparql-db-band-end}).
  %
  The rest of the query consists of creating the respective \ac{RDF} triples for the other relations: person
  (\crefrange{qr:xsparql-db-person-start}{qr:xsparql-db-person-end}), album
  (\crefrange{qr:xsparql-db-album-start}{qr:xsparql-db-album-end}), and song
  (\crefrange{qr:xsparql-db-song-start}{qr:xsparql-db-song-end}).
\end{example}






%%% Local Variables:
%%% fill-column: 120
%%% TeX-master: t
%%% TeX-PDF-mode: t
%%% TeX-debug-bad-boxes: t
%%% TeX-parse-self: t
%%% TeX-auto-save: t
%%% reftex-plug-into-AUCTeX: t
%%% mode: latex
%%% mode: flyspell
%%% mode: reftex
%%% TeX-master: "../thesis"
%%% End:


\section{Semantics}
\label{sec:semantics}

Next we define the semantics of XSPARQL by reusing the semantics of \ac{SQL} and SPARQL. 
%
We start by defining how we ensure that the semantics of \ac{SQL} and SPARQL queries respects any existing XSPARQL
variable bindings.
%
In \cref{sec:chang-xquerys-semant}, we present the extensions to the \ac{W3C} XQuery's
semantics~\cite{DraperFankhauserFernandez:2010aa}, namely the new types we use, an extension to the normalisation rules
of XQuery \ForClause{s}, and necessary additional environment components.
%
\cref{sec:sparqlforclause-sem} presents the semantics of the newly introduced expressions: \SparqlForClause,
\ConstructClause, and \SQLForClause, based on XQuery's formal semantics~\cite{DraperFankhauserFernandez:2010aa}, by
defining normalisation, static type and dynamic evaluation rules for each of the new expressions.



\subsection{XSPARQL Types}
\label{sec:xsparql-types}
%
We extend the \acl{XDM} (described in \cref{sec:xpath-data-model}) with the following new types to accommodate
for \ac{SQL} and SPARQL specific parts of XSPARQL:
%
\begin{enumerate}[(1),noitemsep]
\item\label{enum:t0} the \type{SQLTerm} is an extension of \qname{xs}{anyAtomicType} (as presented in
  \cref{sec:mapping-xml-types});
\item\label{enum:t1} the \type{RDFTerm} type further consists of the subtypes \type{uri}, \type{bnode} and
  \type{literal} and is used as the type of SPARQL variables;
\item\label{enum:t2} the \type{PatternSolution} type consists of a sequence of pairs \seq{\grammarRule{variableName},
  \type{RDFTerm}}, representing \ac{SQL} or SPARQL variable bindings;
\item\label{enum:t3} the \type{RDFGraph} is the type resulting from the evaluation of \CONSTRUCT expressions; and
\item\label{enum:t4} the \type{RDFDataset} is the type used for representing \ac{RDF} datasets, which is further
  constituted by one default \type{RDFGraph} and a sequence of named graphs (\type{RDFNamedGraph}).
\end{enumerate}
%
\cref{verb:types} presents the formal definition of \ref{enum:t0}--\ref{enum:t4} following the notation for \ac{XML}
Schema datatypes (presented in \cref{sec:xml-schema}).
%
The \type{RDFTerm} type is used to represent RDF terms (composed of \acp{URI}, blank nodes or literals).  The type of
SPARQL variables is represented by the \type{Binding} type, that consists of the variable name and the RDF term that is
assigned to it.
%
Finally, sequences of SPARQL variable bindings are represented by the type \type{PatternSolution}.  

\begin{figure}
  \centering
\begin{tabular}{|cc|}
\hline
\begin{minipage}{.45\linewidth}
\begin{Verbatim}[framesep=2mm,frame=none,fontsize=\scriptsize,commandchars=\\\(\)]

define type \textbf(URI-reference) restricts xs:anyURI;

define type \textbf(Literal) extends xs:string {
      attribute datatype of type URI-reference?, 
      attribute lang of type xml:lang? };

define type \textbf(RDFTerm) { 
      element uri of type URI-reference |
      element bnode of type xs:string |
      element literal of type Literal };

define type \textbf(SPARQLBinding) extends RDFTerm { 
      attribute name of type xs:string };

define type \textbf(SPARQLResult) { 
      element binding of type SPARQLBinding* };

define type \textbf(SQLTerm) extends xs:anyAtomicType ;

define type \textbf(SQLBinding) extends SQLTerm { 
      attribute name of type xs:string };

define type \textbf(SQLResult) { 
      element binding of type SQLBinding* };

define type \textbf(SQLAttribute) extends xs:string ;

  \end{Verbatim}
\end{minipage}
&
\begin{minipage}{.45\linewidth}
\begin{Verbatim}[frame=none,fontsize=\scriptsize,commandchars=\\\(\)]

define type \textbf(PatternSolution) { 
      element result of type SPARQLResult |
      element result of type SQLResult };

define type \textbf(RDFGraph) { 
      element triple of type RDFTriple* };

define type \textbf(RDFTriple) { 
      element subject of type RDFTerm,
      element predicate of type RDFTerm,
      element object of type RDFTerm };

define type \textbf(RDFDataset) { 
      element defaultGraph of type RDFGraph,
      element namedGraphs of type RDFNamedGraphs };

define type \textbf(RDFNamedGraphs) { 
      element namedGraph of type RDFNamedGraph* };

define type \textbf(RDFNamedGraph) { 
      attribute name of type xs:string,
      element graph of type RDFGraph };
  \end{Verbatim}
\end{minipage}%
\\
\hline
\end{tabular}
\caption{XSPARQL Type Definitions}
\label{verb:types}
\end{figure}


Similarly for \ac{SQL} results, sequences of \ac{SQL} variable bindings are also represented by the type
\type{PatternSolution}.
%
Analogously, we define the types \type{SQLResult} and \type{SQLBinding} for representing \ac{SQL} results.  The
\type{SQLBinding} type is defined as an extension of \qname{xs}{anyAtomicType}, and we follow the mapping from \ac{SQL}
types into \ac{XML} types presented in \cref{fig:SQL2XML}.


The \type{RDFGraph} type corresponds to a sequence of \type{RDF\-Triple}{\type{s}}, which are in turn a complex type
composed of \type{sub\-ject}, \type{predicate} and \type{object}.
%
The \type{RDFDataset} type is defined as an \type{RDFGraph} that is considered the default graph and a sequence of
\type{RDFNamed\-Graph}{\type{s}} represented by the \type{name} of the graph and the corresponding \type{RDFGraph}.




\subsubsection{Translating SQL and SPARQL Solutions into the \type{PatternSolution} Type}
\label{sec:transl-sql-sparql}

The next definition presents the translation between a SPARQL solution sequence and a sequence of \type{SPARQLResult}
type elements that we implement in XSPARQL.
%
This serialisation of SPARQL results mimics the SPARQL Query Results XML Format~\cite{BeckettBroekstra:2008aa}, defined
by the XML Schema available at {\url{http://www.w3.org/2007/SPARQL/result.xsd}}.
%
\begin{definition}[Serialisation of Solution Sequences]
  \label{lem:solution2types}
  Given a SPARQL solution sequence $\Omega = (\mu_1, \dotsc, \mu_n)$ a serialisation of $\Omega$ into a sequence of
  \type{PatternSolution} is defined as follows:
  \begin{itemize}[noitemsep]
  \item $\funcCall{serialise}{\Omega} \Rightarrow \funcCall{serialise}{\mu_1}, \ldots, \funcCall{serialise}{\mu_n}$
  \item $\funcCall{serialise}{\mu} \Rightarrow 
    \stt{<result>} 
    \{\forall x \in \dom{\mu}, \funcCall{serialise}{\mu, x}\}
    \stt{</result>}
    $
  \item $\funcCall{serialise}{\mu, x} \Rightarrow  
    \stt{<binding name="}x\stt{">}
    \{\funcCall{term}{\funcCall{\mu}{x}}\}
    \stt{</binding>}
    $,
    %
    where \funcCall{term}{\mu(x)} is 
    \begin{itemize}[noitemsep]
    \item $\stt{<uri>}\{\mu(x)\}\stt{</uri>} \hfill \textrm{if}~\mu(x) \in \AU$
    \item $\stt{<bnode>}\{\mu(x)\}\stt{</bnode>} \hfill\textrm{if}~\mu(x) \in \AB$
    \item $\stt{<literal>}\{\mu(x)\}\stt{</literal>}\hfill\textrm{if}~\mu(x) \in \AL$
    \end{itemize}

  \end{itemize}
  %
\end{definition}
%
\noindent Following the definition of the \funcName{serialise} function, in evaluation rules, we will refer to sequences
of elements of type \type{PatternSolution} as \omg{}{} and to elements of type \type{SPARQLResult} as \sm{}{}.

For the representation of \ac{SQL} results we follow a similar approach:
%
\begin{definition}[Serialisation of SQL Relation Instances]
  \label{lem:relation2types}
  % 
  The serialisation of a relation instance~$I = \seq{I_1, \ldots, I_n}$ of relation~$R$ with~$\funcCall{sort}{R} = U$,
  into \type{PatternSolution} is:
  % 
  \begin{itemize}[noitemsep]
  \item $\funcCall{serialise}{I} \Rightarrow \funcCall{serialise}{I_1}, \ldots, \funcCall{serialise}{I_n}$
  \item $\funcCall{serialise}{I_i} \Rightarrow
    \stt{<result>} \{\forall x \in U, \funcCall{serialise}{I_i, x}\} \stt{</result>}
    $
  \item $\funcCall{serialise}{I_i, x} \Rightarrow  
    \stt{<binding name="}x\stt{">} \{ \funcCall{sql2xml}{\funcCall{I_i}{x}} \} \stt{</binding>}
    $.
  \end{itemize}
  %
\end{definition}



\subsubsection{Serialisation into SQL and SPARQL Representations}
\label{sec:sql-sparql-serialisation}

%
The following definitions present the \funcName{SQLTerm} and \funcName{RDFTerm} functions that, when applied to an
\ac{XSD} datatype, return their representation in \ac{SQL} or SPARQL syntax, respectively.  We first present the
serialisation into \ac{SQL}:
%
\begin{definition}[SQL representation]
  %
  \label{def:sqlTerm}
  %
  Let~$C$ be an expression context with static environment~$T_C = \funcCall{\stat}{C}$ and dynamic environment~$D_C =
  \funcCall{\dyn}{C}$, and~$x \in \funcCall{dom}{T_C.varType}$ an XSPARQL variable name.
  %
  The \ac{SQL} representation of~$x$ according to~$C$, denoted~$\funcCall{SQLTerm_C}{x}$ is:
  %
  \begin{itemize}[noitemsep]
  \item~$\funcCall{data}{D_C.\funcCall{\ecomp{varValue}}{x}}$ \hfill if $T_C.\funcCall{\ecomp{varType}}{x} =
    (\type{SQLTerm} \textrm{ or } \type{SQLAttribute} \textrm{ or } \type{RDFTerm}  \textrm{ or } \type{node()})$; and
  \item~$\funcCall{xml2sql}{D_C.\funcCall{\ecomp{varValue}}{x}}$ \hfill otherwise, 
  \end{itemize}
  % 
  where \funcName{xml2sql} is the value conversion function presented in \cref{sec:mapping-xml-types}.
  %
\end{definition}
%
Similarly, we next present the serialisation of SPARQL terms:
%
\begin{definition}[\funcName{RDFTerm}]
  % 
  \label{def:sparqlTerm}
  %
  Let~$C$ be an expression context with static environment~$T_C = \funcCall{\stat}{C}$ and dynamic environment~$D_C =
  \funcCall{\dyn}{C}$, and~$x \in \funcCall{dom}{T_C.varType}$ an XSPARQL variable name.
  %
  The \ac{RDF} representation of~$x$ according to~$C$, denoted~$\funcCall{RDFTerm_C}{x}$ is:
  % 
  \begin{itemize}[noitemsep]
  \item~$D_C.\funcCall{\ecomp{varValue}}{x} \hfill \textrm{if } T_C.\funcCall{\ecomp{varType}}{x} = \type{RDFTerm}$,
  \item~$\texttt{"}D_C.\funcCall{\ecomp{varValue}}{x}\texttt{"} \hfill\textrm{if } T_C.\funcCall{\ecomp{varType}}{x} = \type{xsd:string}$,
  \item~$\dt{\funcCall{D_C.\ecomp{varValue}}{\ensuremath{x}}}{\qname{rdf}{XMLLiteral}} \hfill \textrm{if }
    T_C.\funcCall{\ecomp{varType}}{x} = \type{element()}$,
  \item~$\texttt{"}data(D_C.\funcCall{\ecomp{varValue}}{x})\texttt{"} \hfill \textrm{if } T_C.\funcCall{\ecomp{varType}}{x} =
    (\type{attribute()} \textrm{ or } \type{SQLTerm} \textrm{ or } \type{SQLAttribute})$, and
  \item~$\texttt{"}D_C.\funcCall{\ecomp{varValue}}{x}\texttt{"\^{}\^{}}T_C.\funcCall{\ecomp{varType}}{x}$ \hfill
    otherwise.
  \end{itemize}
\end{definition}

%%% Local Variables:
%%% fill-column: 120
%%% TeX-master: t
%%% TeX-PDF-mode: t
%%% TeX-debug-bad-boxes: t
%%% TeX-parse-self: t
%%% TeX-auto-save: t
%%% reftex-plug-into-AUCTeX: t
%%% mode: latex
%%% mode: flyspell
%%% mode: reftex
%%% TeX-master: "../thesis"
%%% End:



\subsection{XSPARQL Semantics for Querying Relational and RDF data}
\label{sec:query-heter-data}

We now define the semantics of \SQLForClause{s} and \SparqlForClause{s} by relying on the evaluation semantics of their
original query languages, namely \ac{SQL} (presented in \cref{sec:sql}) and SPARQL (presented in
\cref{sec:sparql-preliminaries}).
%
The approach we take is to rely on the translation of each language into their respective algebra expressions and
further combine these algebra expressions with any existing XSPARQL variable bindings.
%
Since XSPARQL is based on the semantics of XQuery, variable bindings are stored in the \ecomp{varValue} environment
component of the dynamic environment (\cf~\cref{sec:xquery}), that maps variable names to their value.
%
Next we present how we interpret these variable mappings as a relation and as a solution sequence, thus allowing to
combine the results of \ac{SQL} and SPARQL queries with the existing variable bindings.


\subsubsection{Querying Relational Data}
\label{sec:extending-sql}


In order to reuse the semantics of \ac{SQL} for defining the semantics of XSPARQL \SQLForClause{s} we transform the
\ecomp{varValue} component of the dynamic environment in which the \SQLForClause is executed into a relation (which we
call the \emph{XSPARQL instance relation}).
%
The following definition presents this translation:
%
\begin{definition}[XSPARQL instance relation]
  \label{def:xml2sqlterm}
  Let the set of relation names (\AR) be defined as in \cref{sec:relational-model}, and let~$C$ be an expression
  context.
  %
  The \emph{XSPARQL instance relation} of~$C$ is a relation instance named \lit{$xir_C$}, where $xir_C$ is a reserved
  relation name, \ie~$xir_C \not\in \AR$, and~$\funcCall{sort}{xir_C} = \dom{\funcCall{\dyn}{C}.\ecomp{varValue}}$.
  %
  For each mapping~$v_{i} \rightarrow x_{i} \in \funcCall{\dyn}{C}.\ecomp{varValue}$, the value of~$xir_C$ for
  attribute~$v_i$, denoted~$\funcCall{xir_C}{v_{i}}$, is defined as:
  %
  \begin{itemize}[noitemsep]
  \item if~$x_{i} = \seq{}$ is an empty sequence then~$\funcCall{xir_C}{x_{i}} = \NULL$;
  \item if~$x_{i} = \seq{e_{1}, \dotsc, e_{n}}$ is a sequence, then~$\funcCall{xir_C}{x_{i}} =
    \funcCall{fn{:}concat}{\funcCall{SQLTerm_C}{e_{1}}, \dotsb, \funcCall{SQLTerm_C}{e_{n}}}$.\footnote{Since the values
      of any relation attribute must be atomic, in the case of a variable being assigned to an XQuery sequence we assume
      the concatenation of each element of the sequence.}
    %
  \end{itemize}
  %
   For a \synt{SQLWhereClause}~$S$, we call the XSPARQL instance relation of the expression context in which~$S$ is
  executed the \emph{XSPARQL instance relation of~$S$}.
\end{definition}
%

%
Another necessary step to enable the reuse of \ac{SQL} evaluation semantics is to convert our extended syntax (that
allows for \var{}-prefixed variable names) into valid \ac{SQL} syntax: each \synt{WhereSpec} in a \SQLForClause that
contains an XSPARQL variable is removed from the normalised \ac{SQL} query (by replacing it with \lit{true}) and is
stored for a later evaluation by the XSPARQL semantics.  For this we rely on the following normalisation function:
%
\begin{definition}[\ac{SQL} Representation of \synt{SQLWhereClauses}]
  %
  \label{def:normalisation-where}
  %
  Let~$S = \lit{where}~\synt{WhereSpecList}$ be a \synt{SQLWhereClause}.  The normalisation of~$S$,
  $\funcCall{normaliseSQL}{S} = \lit{where}~\funcCall{normaliseSQL}{\synt{WhereSpecList}}$, where
  \funcCall{normaliseSQL}{\synt{WhereSpecList}} is defined as:
  % 
  \begin{itemize}
  \item if \synt{WhereSpecList} is of form~$\lit{(}\ \synt{WhereSpecList}_{1}\ \synt{Op}\ \synt{WhereSpecList}_2\
    \lit{)}$ then
    %
    \[
    \lit{(}\ \funcCall{normaliseSQL}{\synt{WhereSpecList}_1}\ \synt{Op}\ \funcCall{normaliseSQL}{\synt{WhereSpecList}_2}\ \lit{)}
    \]
  \item if~\synt{WhereSpecList} is of form~$Attr_1~\synt{Op}~Attr_2$ then~$\funcCall{normaliseSQL}{Attr_1~Op~Attr_2}$
    is:
    \[
    \begin{dcases*}
      \lit{true}        & if~$Attr_1$  or~$Attr_2$ is an XSPARQL variable \\
      Attr_1~Op~Attr_2  & otherwise.
    \end{dcases*}
    \]
  \end{itemize}
  %
  Furthermore we denote the set of \synt{WhereSpec} of~$S$ in which an attribute is an XSPARQL variable as
  \funcCall{whereSpecVars}{S}.
  % 
\end{definition}
%
\noindent
%
The normalisation of complete \SQLForClause{s} consists also of the normalisation of the syntactical elements
\synt{AttrSpecList} and \synt{TableSelector} presented in \cref{sec:syntax-rdb}.
%
In the normalisation of~\synt{AttrSpecList} we remove any existing \synt{AttrNameSpec} component, since they reflect
only the name of the corresponding XSPARQL variable.
%
However, the normalisation of the \synt{TableSelector} can only be performed during the dynamic evaluation of the
XSPARQL query since any variables present in the \synt{TableSelector} must be evaluated to determine the corresponding
relation name.
%
With the restriction of performing the substitution at evaluation time, we can reuse the standard translation of a
\ac{SQL} query into relational algebra as presented in \cref{sec:sql-semantics}.

Next we present how XSPARQL combines the results of a \ac{SQL} query with an XSPARQL instance mapping.
%
For this we rely on the standard relational selection~($\sigma$) and cross-product~($\times$) algebra operators
presented in \cref{sec:querying-rdb} and on the~$xir_C$ relation instance from \cref{def:xml2sqlterm}.
%
Firstly, we present the construction of the relational algebra select expression that, based on the provided
\SQLForClause~$S$ and the XSPARQL instance mapping of~$S$, makes the connection between the results of the \ac{SQL}
query and the existing XSPARQL variable bindings:
%
\begin{definition}[XSPARQL $\sigma$ expression]
  \label{def:xsparql-join}
  %
  Let~$S$ be a \SQLForClause with expression context~$C$ and~$V = \funcCall{whereSpecVars}{S}$ the attribute
  specifications that contain XSPARQL variables in~$S$.
  %
  The \emph{XSPARQL $\sigma$ expression} of~$S$, denoted~$\funcCall{\sigma_{xs}}{S}$, is a relational algebra~$\sigma$
  expression that, for each $Att_1~Op~Att_2 \in V$ is~$\funcCall{trans}{Att_1}~Op~\funcCall{trans}{Att_2}$,
  where~$\funcCall{trans}{Att}$ is defined as:
  % 
  \begin{itemize}[noitemsep]
  \item $Attr$ \hfill if~$Attr$ is not an XSPARQL variable;
  \item if~$Attr = \lit{\$}AttrName$ is an XSPARQL variable then 
    %
    \[
    \funcCall{trans}{Attr} = \begin{dcases*}
        \dyn.\funcCall{\ecomp{varValue}}{AttrName} & if~\stat.\funcCall{\ecomp{varType}}{Attr} = \type{SQLAttribute}\\
        \textrm{`}xir_C.AttrName\textrm{'} & otherwise.
      \end{dcases*}
    \]
  \end{itemize}
\end{definition}
%
\noindent
%
This definition creates a relational algebra expression from the extended XSPARQL \SQLForClause syntax, which can then
be used to further restrict the results of the normalised \ac{SQL} expression.
%
Based on these definitions we can introduce the translation of \SQLForClause{s} into relational algebra.
%
\begin{definition}[XSPARQL relational algebra expression]
  \label{def:xsparql-sql-answers}
  %
  Let~$Q$ be a \SQLForClause,~$Q' = \funcCall{normaliseSQL}{Q}$ the \ac{SQL} rewriting of~$Q$,~$E =
  \funcCall{\sigma_{xs}}{S}$ the XSPARQL~$\sigma$ expression of~$S$, and~$\RASQL{Q'}$ the relational algebra expression
  obtained from the standard \ac{SQL} translation into relational algebra.
  %
  The \emph{XSPARQL relational algebra expression} of~$Q$, denoted~$\RAXSP{Q}$, combines the relational algebra
  expression of the \ac{SQL} query and restricts its results to the existing bindings for XSPARQL variables as follows:
  %
  \[
  \funcCall{\sigma_{E}}{\RASQL{Q'} \times xir_C} \enspace .
  \]
  %
\end{definition}
%
The following example illustrates the translation of XSPARQL \SQLForClause{s} into XSPARQL relational algebra
expressions.
%
\begin{figure}[t]
  \subfloat[Value Matching]{\label{ex:sql-for-ex1}
    \begin{minipage}{.5\linewidth}
      \lstinputlisting{0-data+queries/sql-for-clause-ex1.xsparql}
    \end{minipage}
  }
  % 
  \subfloat[Attribute Matching]{\label{ex:sql-for-ex2}
    \begin{minipage}{.5\linewidth}
      \lstinputlisting{0-data+queries/sql-for-clause-ex2.xsparql}
    \end{minipage}
  }
  \caption{XSPARQL \SQLForClause examples}%
  \label{fig:sql-for-ex}%
\end{figure}
%
\begin{example}[Translation of \SQLForClause{s} into Relational Algebra]
  %
  \cref{fig:sql-for-ex} presents two XSPARQL queries including \SQLForClause{s}.
  %
  The query in \cref{ex:sql-for-ex1} illustrates the syntax for querying values of a relation.  First the
  normalisation function drops the restriction in line~3, which is incorporated into the relational algebra~$\sigma$
  expression:
  % 
  \[
  \funcCall{\sigma_{band.bandId = xir_C.x}}{\funcCall{\sigma_{band.bandName = 'Nightwish'}}{band} \times xir_C} \enspace ,
  \]
  where~$\funcCall{sort}{xir_C} = \set{x}$ and~$\funcCall{xir_C}{x} = 1$.

  \medskip

  On the other hand, the query in \cref{ex:sql-for-ex2} shows how to match attribute names.
  %
  The query in this figure is converted into the following relational algebra expression:
  % 
  \[
  \funcCall{\sigma_{band.bandName = 'Nightwish'}}{\funcCall{\sigma_{band.bandId = 1}}{band} \times xir_C} \enspace ,
  \]
  %
  where~$\funcCall{sort}{xir_C} = \set{x}$ and~$\funcCall{xir_C}{x} = \textrm{`bandName'}$.  
  %
\end{example}







%%% Local Variables:
%%% fill-column: 120
%%% TeX-master: t
%%% TeX-PDF-mode: t
%%% TeX-debug-bad-boxes: t
%%% TeX-parse-self: t
%%% TeX-auto-save: t
%%% reftex-plug-into-AUCTeX: t
%%% mode: latex
%%% mode: flyspell
%%% mode: reftex
%%% TeX-master: "../thesis"
%%% End:




\subsubsection{Querying RDF Data}
\label{sec:bf-extended-bgp}
%
For querying \ac{RDF} data, we extend the notion of SPARQL \ac{BGP} (\cref{def:bgp-matching}) in order to provide SPARQL
with the variable bindings from XQuery.
%
For this we interpret the XQuery \ecomp{varValue} dynamic environment component as a set of bindings in the spirit of
SPARQL solution mappings (as presented in \cref{def:solution-mapping}).
%
Along these lines, we will regard the \ecomp{varValue} component of the dynamic environment in which a SPARQL graph
pattern~$P$ is executed as the basis for the \emph{XSPARQL instance mapping} of~$P$.  The transformation from the
\dyn.\ecomp{varValue} into the XSPARQL instance mapping is defined next:
%
\begin{definition}[XSPARQL instance mapping]
  \label{def:xml2rdfterm}
  Let~$C$ be an expression context, and furthermore let~$D_C = \funcCall{\dyn}{C}.\ecomp{varValue}$ and~$T_C =
  \funcCall{\stat}{C}.\ecomp{varType}$ the \ecomp{varValue} component of the dynamic environment of~$C$ and be the
  \ecomp{varType} component of the static environment of~$C$, respectively.
  %
  The \emph{XSPARQL instance mapping}~$\mu_C$ is a solution mapping where, for each mapping~$v_{i} \rightarrow x_{i} \in
  D_C$,~$x_{i}$ is converted into an instance of type \type{RDFTerm} or an \ac{RDF} \emph{Collection} according to the
  following conditions:
  %
  \begin{itemize}[noitemsep]
  \item if~$\funcCall{D_C}{v_{i}} = \seq{}$ and~$\funcCall{T_C}{v_{i}} = \type{RDFTerm}$ or~$\funcCall{T_C}{v_{i}} =
    \type{SQLTerm}$ then~$\funcCall{\mu_C}{\funcCall{D_C}{v_{i}}}$ is undefined;
  \item if~$\funcCall{D_C}{v_{i}} = \seq{}$ and~$\funcCall{T_C}{v_{i}} \neq \type{RDFTerm}$ and~$\funcCall{T_C}{v_{i}}
    \neq \type{SQLTerm}$ then~$\funcCall{\mu_C}{\funcCall{D_C}{v_{i}}} = \seq{}$ is an empty \ac{RDF} Collection;
  \item if~$\funcCall{D_C}{v_{i}}$ is a singleton sequence then~$\funcCall{\mu_C}{\funcCall{D_C}{v_{i}}} =
    \funcCall{RDFTerm}{\funcCall{D_C}{v_{i}}}$;
  \item if~$\funcCall{D_C}{v_{i}} = \seq{e_{1}, \dotsc, e_{n}}$,~$n > 1$, is a sequence
    then~$\funcCall{\mu_C}{\funcCall{D_C}{v_{i}}} = \seq{\funcCall{RDFTerm}{e_{1}} \dotsb \funcCall{RDFTerm}{e_{n}}}$ to
    be read as an \ac{RDF} Collection in Turtle notation (\cf~\cref{sec:turtle}).
  \end{itemize}
  % 
  For a graph pattern~$P$, we call the XSPARQL instance mapping of the expression context in which~$P$ is executed the
  \emph{XSPARQL instance mapping of~$P$}.
\end{definition}
%
Next we define the notion of XSPARQL \ac{BGP} matching based on the semantics of SPARQL \ac{BGP} matching presented in
\cref{sec:sparql-semantics}.
%
\begin{definition}[Extended solution mapping]\label{def:extended-solution-mapping}
  Let~$C$ be an expression context. An \emph{extended solution mapping} of a graph pattern~$P$ in~$C$ is a solution
  mapping \emph{compatible} with the \emph{XSPARQL instance mapping} of~$C$.
\end{definition}
% 
\noindent Accordingly, XSPARQL \ac{BGP} matching is defined analogously to the SPARQL \ac{BGP} matching with the
exception that we consider only extended solution mappings:
%
\begin{definition}[XSPARQL \ac{BGP} matching]\label{def:xsparql-bgp-matching}
  Let~$P$ be a \ac{BGP} with expression context~$C$, and~$G$ be an \ac{RDF} graph. We say that~$\mu$ is a
  \emph{solution} for~$P$ with respect to active graph~$G$, if there exists an extended solution mapping~$\mu'$ of~$C$
  such that~$\funcCall{\mu'}{P}$ is a subgraph of~$G$ and~$\mu$ is the restriction of~$\mu'$ to the variables
  in~$\vars{P}$.
\end{definition}
%
\noindent This definition quasi injects the variable bindings inherited from XQuery into SPARQL patterns occurring
within XSPARQL.  By considering \emph{extended solution mappings} the bindings returned for a \ac{BGP}~$P$ will not only
match the input graph~$G$ but also respect any bindings for variables in the dynamic environment.
%
We can extend the XSPARQL \ac{BGP} matching to generic graph patterns by following the SPARQL evaluation semantics
(presented in \cref{sec:sparql-semantics}).
%
Considering a graph pattern~$P$ with XSPARQL instance mapping~$\mu_C$, we denote by~$\evalXS{D}{P}{\mu_C}$ the
evaluation of~$P$ over dataset~$D$ following XSPARQL \ac{BGP} matching.
 
\subsubsection{Matching Blank Nodes in Nested Queries}
\label{sec:bgpmatch}
%
Although in XSPARQL, similar to SPARQL, we are not considering blank nodes in the semantics definitions of graph
patterns, in the case of nested \SparqlForClause{s} XSPARQL instance mappings may in fact contain assignments of
variables to blank nodes, injected from the outer \SparqlForClause into the inner \SparqlForClause.
%
\begin{query}
  \lstinputlisting[language=XSPARQL]{0-data+queries/nested-bnodes.xsparql}%
  \caption{Nested XSPARQL query}%
  \label{qr:nested-bnodes}%
\end{query}
%
\begin{example}[Blank node injection in XSPARQL nested queries]
  %
  For example, in \cref{qr:nested-bnodes}, blank nodes bound in the outer \SparqlForClause
  (\crefrange{qr:nested-bnodes_out-s}{qr:nested-bnodes_out-e}) to the variable~\var{song} will be injected into
  the inner \SparqlForClause expression (\crefrange{qr:nested-bnodes_in-s}{qr:nested-bnodes_in-e}).
  %
  If we would consider both \SparqlForClause{s} as distinct SPARQL queries, the blank nodes in the inner
  \SparqlForClause would be matched as variables.
\end{example}
%
However in XSPARQL, we want to enable coreference within nested queries over the same dataset and thus such injected
blank nodes should be matched like constants against the blank nodes present in the input \ac{RDF} data (rather than
being treated as variables).
%
To ensure this behaviour, we introduce the notion of \emph{active dataset} (similar to the concept of active graph in
SPARQL), where nested queries over the same active dataset keep the same the scoping graphs
(\cf~\cref{sec:sparql-semantics}).
%
Any \SparqlForClause with an \emph{explicit} \DatasetClause causes the \emph{active dataset} to change, \ie~new scoping
graphs (with fresh blank nodes) for each graph within it are created.  On the other hand, if no \DatasetClause is
present in a nested \SparqlForClause (implicit dataset), the active dataset remains
unchanged.
%
To ensure this behaviour in the dynamic evaluation we have to introduce a new dynamic environment component called
\ecomp{activeDataset}, that will be used to evaluate \SparqlWhereClause{s}.  Initially, this component is empty (or set
to a system default) and is changed by a \DatasetClause appearing in a \SparqlForClause{}, as defined in the next
section.



%%% Local Variables:
%%% fill-column: 120
%%% TeX-master: t
%%% TeX-PDF-mode: t
%%% TeX-debug-bad-boxes: t
%%% TeX-parse-self: t
%%% TeX-auto-save: t
%%% reftex-plug-into-AUCTeX: t
%%% mode: latex
%%% mode: flyspell
%%% mode: reftex
%%% TeX-master: "../thesis"
%%% End:




%%% Local Variables:
%%% fill-column: 120
%%% TeX-master: t
%%% TeX-PDF-mode: t
%%% TeX-debug-bad-boxes: t
%%% TeX-parse-self: t
%%% TeX-auto-save: t
%%% reftex-plug-into-AUCTeX: t
%%% mode: latex
%%% mode: flyspell
%%% mode: reftex
%%% TeX-master: "../thesis"
%%% End:




\subsection{Extensions to the XQuery Semantics}
\label{sec:chang-xquerys-semant}

In order to define the XSPARQL semantics according to XQuery's semantics we need to introduce new environment components
and extend the dynamic evaluation rules of XQuery \ForClause{s} to populate these new components.
%
We also introduce the functions that we will use in the dynamic evaluation rules presented in the next section.


\subsubsection{New Environment Components}
\label{sec:new-envir-comp}

For the definition of the XSPARQL semantics we add the following components to the dynamic environment:
%
\begin{itemize}[(i),noitemsep]
\item \ecomp{activeDataset}; and
\item \ecomp{globalPosition}.
\end{itemize}

The \dyn.\ecomp{activeDataset} is used to store the dataset over which \SparqlForClause{s} are evaluated in order to be
accessible when a nested \SparqlForClause without a \DatasetClause is specified.

The other introduced environment component, \dyn.\ecomp{globalPosition}, stores all the positions in the tuple streams.
%
The standard XQuery dynamic evaluation rules can only access the position of the current tuple stream, however, in order
to generate distinct blank node labels for each \ConstructClause, we need to guarantee that the labels are also distinct
in case of nested queries.
%
To ensure this, we store not only the position in the current tuple stream but also the positions of all previous ones.


Both environment components are populated in the dynamic evaluation rules introduced in
\cref{sec:new-semantics-rules}.  For the \dyn.\ecomp{globalPosition} we also need to adapt the evaluation rules
of XQuery \ForClause{s} to correctly populate this component.  These updated rules are presented next.


% using globalPosition
\subsubsection{XQuery \FOR Dynamic Evaluation}
\label{sec:norm-for}
%
In order to correctly generate blank node identifiers in \ConstructClause{s}, XSPARQL relies on the
\dyn.\ecomp{globalPosition} environment component to store the positions.
%
As such, we adapt the XQuery \FOR dynamic evaluation rules, presented in \cref{sec:xquery}, to populate the
\dyn.\ecomp{globalPosition} component and also make sure that the newly introduced XSPARQL \SQLForClause{s} and
\SparqlForClause{s} populate this component.  The case of these newly XSPARQL expressions is detailed later in
\cref{sec:new-semantics-rules}.


We show here only the adapted rule for \ForClause{s} with position variables and without type declaration.  The rules
that handle \FOR expressions without position variables and possibly containing type declarations are adapted
analogously, adding the \dyn.\ecomp{globalPosition} premisses to the rules presented
in~\citet[Section~4.8.2]{DraperFankhauserFernandez:2010aa}.
%
\begin{dynamicrule}
    % 
    \AxiomC{$\dyn.\ecomp{globalPosition} = \seq{ \envElem{Pos}{1}, \cdots, \envElem{Pos}{m} } $}
    %
    \UnaryInfC{$\dynEnv{\envElem{Expr}{1} \Rightarrow \envElem{Item}{1}, \dotsb, \envElem{Item}{n}}$}
    %
    \UnaryInfC{$\statEnv \envElem{VarName}{}~\textbf{of var expands to}~\envElem{Variable}{}$}
    %
    \UnaryInfC{$\statEnv \envElem{VarName}{pos}~\textbf{of var expands to}~\envElem{Variable}{pos}$}
    %
    \UnaryInfC{$\begin{array}{r@{\hspace{-1pt}}l}%
        \dyn & \envExtend{globalPosition}{\seq{ \envElem{Pos}{1}, \cdots, \envElem{Pos}{m}, 1 }} \\
        & \envExtend{varValue}{\begin{array}{l}
            \envElem{Variable}{} \Rightarrow \envElem{Item}{1};\\
            \envElem{Variable}{pos} \Rightarrow 1
          \end{array}}
      \end{array}~\proofs \envElem{Expr}{2} \Rightarrow \envElem{Value}{1}$}
    %
    \UnaryInfC{$\vdots$}
    %
    \UnaryInfC{$\begin{array}{r@{\hspace{-1pt}}l}%
        \dyn & \envExtend{globalPosition}{\seq{ \envElem{Pos}{1}, \cdots, \envElem{Pos}{m}, n }} \\
        &\envExtend{varValue}{\begin{array}{l}
            \envElem{Variable}{} \Rightarrow \envElem{Item}{n};\\
            \envElem{Variable}{pos} \Rightarrow n 
          \end{array}}
      \end{array}~\proofs \envElem{Expr}{2} \Rightarrow \envElem{Value}{n}$}
    %
    \singleLine
    %
    \UnaryInfC{$\dynEnv{
        \begin{array}{l}
          \FOR~\envElem{\var{VarName}}{}~\keyword{at}~\envElem{\var{VarName}}{pos}~\keyword{in}~\envElem{\grammarRule{Expr}}{1}\\
          \keyword{return}~\envElem{\grammarRule{Expr}}{2}
        \end{array} \Rightarrow \envElem{Value}{1}, \dotsb, \envElem{Value}{n}
      }$}
  \label{eq:dyn-ForClause-adapt}
\end{dynamicrule}%
%


%%% Local Variables:
%%% fill-column: 120
%%% TeX-master: t
%%% TeX-PDF-mode: t
%%% TeX-debug-bad-boxes: t
%%% TeX-parse-self: t
%%% TeX-auto-save: t
%%% reftex-plug-into-AUCTeX: t
%%% mode: latex
%%% mode: flyspell
%%% mode: reftex
%%% TeX-master: "../thesis"
%%% End:




\subsubsection{New Formal Semantics Functions}
\label{sec:form-semant-funct}

Next we will introduce the new XQuery formal semantics functions that we use in the static and dynamic evaluation rules
presented in the next section.  These functions are specified in an informal style, in a similar fashion to formal
semantics functions presented in~\citet[Section~7.1]{DraperFankhauserFernandez:2010aa} and the XQuery~1.0 and XPath~2.0
Functions and Operators specifications~\cite{MalhotraMeltonWalsh:2010aa}.  For each function, we present its signature,
consisting of the function name, the function parameters, and the return type, and include a textual description of the
semantics of the function.

The first introduced functions, \funcName{fs{:}sql} and \funcName{fs{:}sparql}, represent the extended \ac{SQL} and
SPARQL querying facilities implemented in XSPARQL (described in \cref{sec:query-heter-data}).
%
We further introduce two auxiliary functions~$\funcName{fs{:}value}$,~$\funcName{fs{:}dataset}$,
and~$\funcName{fs{:}evalCT}$.  These functions are used to access the value of a variable in a \type{PatternSolution},
to determine the dataset over which a \SparqlForClause is evaluated, and to evaluate a \CONSTRUCT query, \ie~a
\ConstructTemplate, respectively.
%


\paragraph{\funcName{\bf fs{:}sql}}
%
This function is responsible for executing the extended XSPARQL \ac{SQL} querying presented in
\cref{sec:extending-sql}.
%
In our semantics this function also implements the normalisation of $\grammarRule{SQLWhereClauses}$ (presented in
\cref{def:normalisation-where}) by receiving two parameters:~$\grammarRule{RelationList}$ and
$\grammarRule{SQLWhereClause}$ representing the list of relations involved in the query and the \ac{SQL} \emph{pattern}
to be executed, respectively.
%
The static type signature of this function is defined as:
%
{\small
\begin{verbatim}
fs:sql($SparqlWhere as xs:string)
  as PatternSolution*
\end{verbatim}
}%
%
The replacement of variables in \texttt{SQLWhereClauses} represented by \cref{def:normalisation-where} (that this
function implements), produces a valid \ac{SQL} query, that can be evaluated directly by the relational engine.
%
The results of this query are then translated into an instance of \type{PatternSolution} (according to
\cref{lem:relation2types}).


\paragraph{\funcName{\bf fs{:}sparql}.}
\label{sec:fs:sparql}
%
The \funcName{fs{:}sparql} function corresponds to the implementation of the \funcName{\e{eval}{xs}{}} function, that
evaluates SPARQL graph patterns and implements the extended notion of BGP Matching presented
in~\cref{def:xsparql-bgp-matching}.
%
The static type signature of this function is defined as:
%
{\small
\begin{verbatim}
fs:sparql($dataset as RDFDataset, $SparqlWhere as xs:string, $solutionModifiers as xs:string)
  as PatternSolution*
\end{verbatim}
}%
% 
\noindent The result of this function consists of a solution sequence, which can be translated directly into an XQuery
sequence of elements of type~\type{PatternSolution} by applying the~$\funcName{serialise}$ function
(\cf~\cref{lem:solution2types}).





\paragraph{\funcName{\bf fs{:}value}.}
\label{sec:fs:value}
%
The $\funcCall{fs{:}value}{PS, var}$ function returns the value of the specified variable $var$ in a
\type{Pattern\-Solution} specified by~$PS$.  If $var$ is not present in $PS$, the empty sequence is returned. 
%
The static type signature of this function is:
%
{\small
\begin{verbatim}
fs:value($ps as PatternSolution, $variable as xs:string)
  as (RDFTerm | SQLTerm)?
\end{verbatim}
}%
%
\noindent This function returns the respective \type{Binding} for the variable, which is an element of type
\type{SQLTerm} or \type{RDFTerm}, depending on whether the pattern solution was the result of a \SQLForClause or a
\SparqlForClause.


\paragraph{\funcName{\bf fs{:}dataset}.}
\label{sec:fs:dataset}
%
The $\funcCall{fs{:}dataset}{\DatasetClause}$ auxiliary function returns an element of type \type{RDFData\-set} based on
the evaluation of its argument.  
%
This conversion is performed according to the SPARQL semantics presented in \cref{sec:sparql-preliminaries}.  The result
of this function is stored (by dynamic evaluation rules) in the newly introduced \ecomp{activeDataset} dynamic
environment component and can be retrieved when a \SparqlForClause without an explicit \DatasetClause is found.
%
The static type signature of this function is:
%
\begin{center}
\begin{minipage}{0.9\linewidth}
{\small\centering
\begin{verbatim}
fs:dataset($datasetClause as xs:string)
  as RDFDataset
\end{verbatim}
}
\end{minipage}
\end{center}
%

\paragraph{\funcName{\bf fs{:}evalCT}.}
\label{sec:fs:evalCT}
%
The $\funcName{fs{:}evalCT}$ function ensures the created RDF graph is valid and rewrites any blank nodes inside of
\ConstructTemplate{s} to comply with the SPARQL semantics (as described in \cref{sec:bgpmatch}).  The auxiliary
$\funcName{fs{:}validTriple}$ function checks if each triple is valid according to the RDF semantics and is defined by
rules~\eqref{validTriple1} and~\eqref{validTriple2} presented in the next section.
%
The \funcName{fs{:}evalCT} function is further detailed in the following section by presenting specific rules that
ensure the generated \ac{RDF} graph is valid and to guarantee the generation of new blank node labels for each pattern
solution.
%
The static type signatures of these functions are defined as:
%
{\small
\begin{verbatim}
fs:evalCT($template as RDFTerm*)
  as RDFGraph

fs:validTriple($subject as RDFTerm, $predicate as RDFTerm,  $object as RDFTerm)
  as RDFTriple?
\end{verbatim}
}
%
\noindent The \funcName{fs{:}evalCT} function, and hence \CONSTRUCT expressions, return elements of type
\type{RDFGraph}, thus allowing the result of \CONSTRUCT expressions to be used in a \DatasetClause of a subsequent
\SparqlForClause.

\subsection{Semantics Rules for XSPARQL Expressions}
\label{sec:new-semantics-rules}

We are now ready to present the normalisation, static, and dynamic evaluation rules for the newly defined XSPARQL
expressions.
%
As presented in \cref{sec:syntax}, XQuery and SPARQL namespace declarations can be used interchangeably in the
prolog of an XSPARQL query and thus we start by presenting the translation of the query prolog into XQuery namespace
declarations via normalisations rules.
%
We then present the necessary normalisation, static, and dynamic evaluation rules for \SQLForClause{s},
\SparqlForClause{s}, and \ConstructClause{s}.
%
Please note that, since the variables included in \SQLForClause{s} and \SparqlForClause{s} are not allowed to contain a
namespace prefix, we omit the rules handling the namespace expansion for the respective variables.


\subsubsection{Query Prolog Normalisation}
\label{sec:query-prol-norm}

%
In order to follow the XQuery semantics, we convert any SPARQL syntax prefix declaration into XQuery namespace
declarations by the following normalisation rules: %
%
\begin{normalisationrule}
  \mapping{%
    \sem{\PREFIX~\grammarRule{NCName} \keyword{:}~\keyword{<}\grammarRule{IRI}\keyword{>}}{Expr}%
  }{%
    \sem{\keyword{declare~namespace}~\grammarRule{NCName} \keyword{ = "} \grammarRule{IRI} \keyword{" ;}}{Expr}%
  }
  \label{eq:2}
\end{normalisationrule}%
%
The empty \PREFIX declaration is converted into the default namespace for \ac{XML} elements: 
%
\begin{normalisationrule}
  \mapping{%
    \sem{\PREFIX ~ \keyword{:}~\keyword{<}\grammarRule{IRI}\keyword{>}}{Expr}%
  }{%
    \sem{\keyword{declare~default~element~namespace} \keyword{ = "} \grammarRule{IRI} \keyword{" ;}}{Expr}%
  }
  \label{eq:3}
\end{normalisationrule}%
%
Furthermore the SPARQL \keyword{base} declaration is considered equivalent to the XQuery~\keyword{base-uri} declaration:
%
\begin{normalisationrule}
  \mapping{%
    \sem{%
      \keyword{base} ~ \keyword{<}\grammarRule{IRI}\keyword{>}}{Expr}%
  }{%
    \sem{\keyword{declare~base{-}uri "}\grammarRule{IRI} \keyword{" ;}}{Expr}%
  }
  \label{eq:4}
\end{normalisationrule}%




\subsubsection{\SQLForClause}
\label{sec:semantics-rdb}

In this section we define the semantics of the newly introduced \SQLForClause by means of the normalisation rules,
static type analysis rules, and dynamic evaluation rules.

\paragraph{Normalisation rules.}
%
Let us start by presenting the normalisation rule that handles the syntactic shortcut~\lit{\FOR *}.
%
\begin{normalisationrule}%
  \mapping{%
    \sem{%
      \FOR~*~\envElem{RelationList}{}~\grammarRule{SQLWhereClause}~\grammarRule{ReturnClause}%
    }{\grammarRule{Expr}}%
  }{%
    \sema{%
      \FOR~\sem{%
        \envElem{RelationList}{}~\envElem{SQLWhereClause}{}
      }{attrs}~\envElem{RelationList}{}\\
      \grammarRule{SQLWhereClause}~\grammarRule{ReturnClause}%
    }{Expr}%
  }%
  \label{for_star-rdb}%
\end{normalisationrule}%
% 
The normalisation rule $\sem{\cdot}{attrs}$ returns a comma separated list of variables representing all the attributes
from each relation from $\envElem{RelationList}{}$.  As described in \cref{sec:syntax-rdb}, these generated
variables are of the form: $\var{\grammarRule{relationName}.\grammarRule{attributeName}}$.
%
Furthermore, the next normalisation rule guarantees that each variable in a \SQLForClause contains a variable alias:
%
\begin{normalisationrule}
  \mapping{%
    \sema{%
      \FOR~\envElem{AttrSpec}{1}, \dotsb, \envElem{AttrSpec}{n} \\
      \grammarRule{RelationList}~\grammarRule{SQLWhereClause}\\
      \grammarRule{ReturnClause}%
    }{\grammarRule{Expr}}
  }{%
    \begin{array}{l}
      \FOR~\sem{\envElem{AttrSpec}{1}}{Alias}, \dotsb, \sem{\envElem{AttrSpec}{n}}{Alias} \\
      \grammarRule{RelationList}~\grammarRule{SQLWhereClause}\\
      \sem{\grammarRule{ReturnClause}}{\grammarRule{Expr}}
    \end{array}}%
  \label{posvar-sql-normalisation1}%
\end{normalisationrule}%
%
A new normalisation rule~$\sem{\cdot}{\envElem{Alias}{}}$ takes care of introducing the variable alias when necessary,
where the variable alias will be the same as the attribute specification.
%
{\small
  \[
  \sem{\envElem{AttrSpec}{}}{\envElem{Alias}{}} == \envElem{AttrSpec}{}~\keyword{as}~\envElem{\var{AttrSpec}}{} \enspace
  .
\]}%
%
In case a variable alias is already present it is reused: 
%
{\small
  \[
  \sem{\envElem{AttrSpec}{}~\keyword{as}~\envElem{\var{VarRef}}{}}{\envElem{Alias}{}} ==
  \envElem{AttrSpec}{}~\keyword{as}~\envElem{\var{VarRef}}{} \enspace .
  \]}%


\paragraph{Static type analysis.}
\label{sec:static-type-analysis-rdb}

The following static type rule defines the type of each variable in an \SQLForClause as \type{SQLTerm} and infers the
static type of whole expression.
%
%
This rule, based on the static environment~$\stat$, creates a new environment with the added information that each of
the variables in the \SQLForClause ($\envElem{\var{Var}}{1} \dots \envElem{\var{Var}}{n}$) is of type
$\type{\qname{xs}{anySimpleType}}$.
%
Given this new extended environment the type of $\envElem{ReturnExpr}{}$ can be inferred to be $\envElem{Type}{}$,
making the type of the overall \SQLForClause a sequence of elements of inferred type $\envElem{Type}{}$.
%
\begin{staticrule}
  \begin{prooftree}
    \def\ScoreOverhang{1pt}%
    \def\extraVskip{1pt}%
    \alwaysNoLine%
    %
    \AxiomC{$\stat\envExtend{varType}{
          \begin{array}{l}
            \envElem{Var}{1} \Rightarrow \type{SQLTerm};\\
            \dots;\\
            \envElem{Var}{n} \Rightarrow \type{SQLTerm}
          \end{array}
        } 
      \proofs \grammarRule{ReturnExpr} \colon \envElem{Type}{}
      $}
    %
    \singleLine
    %
    \UnaryInfC{$\statEnv{
        \begin{array}{l}
          \FOR~\envElem{AttrSpec}{1}~\keyword{as}~\envElem{\var{Var}}{1}, \dotsb, \envElem{AttrSpec}{n} \texttt{ as } \envElem{\var{Var}}{n}\\
          \grammarRule{RelationList}~\grammarRule{SQLWhereClause}~
          \RETURN~\grammarRule{ReturnExpr} 
        \end{array} : \envElem{Type}{}*
      }$}
  \end{prooftree}
  \label{eq:static-type-rdb}
\end{staticrule}%



\paragraph{Dynamic Evaluation.}
\label{sec:dynamic-evaluation-rdb}

The dynamic evaluation rules for \SQLForClause{s} ensures that the return expression~($\grammarRule{ReturnExpr}$) is
executed for each \type{SQLResult} that is returned by the evaluation of the \ac{SQL} expression.
%
If the evaluation of the SQL expression does not yield any solutions, \ie~evaluates to an empty sequence, then the
overall result will also be the empty sequence:
%
\begin{dynamicrule}
  \AxiomC{$\dynEnv \funcCall{fs{:}sql}{\envElem{RelationList}{}, \grammarRule{SQLWhereClause}} \Rightarrow \seq{}$}
  % 
  \singleLine
  % 
  \UnaryInfC{$\dynEnv{\begin{array}{l}
        \FOR~\envElem{\var{Var}}{1}~\envElem{OptVarAlias}{1},  \dots, \envElem{\var{Var}}{n}~\envElem{OptVarAlias}{n}\\
        \grammarRule{RelationList}~\grammarRule{SQLWhereClause}~
        \RETURN~\grammarRule{ReturnExpr}
      \end{array}} \Rightarrow \seq{}
    $}%
  % 
\end{dynamicrule}%
%
Otherwise, for each solution, the respective value in the pattern solution is accessed and assigned to the respective
variable name in the \dyn.\ecomp{varValue} component.  The results of evaluating~$\grammarRule{ReturnExpr}$ in this
extended environment are then collected into the final sequence.  Please note that this rule also populates the
\dyn.\ecomp{globalPosition} environment.
%
\begin{dynamicrule}
  % 
  \AxiomC{$\dyn.\ecomp{globalPosition} = \seq{ \envElem{Pos}{1}, \cdots, \envElem{Pos}{j} } $}
  % 
  \UnaryInfC{$\dynEnv{\funcCall{fs{:}sql}{\envElem{RelationList}{}, \grammarRule{SQLWhereClause}} \Rightarrow \envElem{SR}{1}, \dots, \envElem{SR}{m}}$}
  % 
  \UnaryInfC{$        
    \begin{array}{r@{\hspace{-1pt}}l}
      \dyn & \envExtend{globalPosition}{\seq{ \envElem{Pos}{1}, \cdots, \envElem{Pos}{j}, 1 }} \\
      & \envExtend{varValue}{
        \begin{array}{l}
          \envElem{Var}{1} \Rightarrow \funcCall{fs{:}value}{\envElem{SR}{1}, \envElem{Var}{1}};\\
          \dots;\\
          \envElem{Var}{n} \Rightarrow \funcCall{fs{:}value}{\envElem{SR}{1},\envElem{Var}{n}}
        \end{array}
      } \proofs \grammarRule{ReturnExpr} \Rightarrow \envElem{Value}{1}
    \end{array}
    $}
  % 
  \UnaryInfC{$\vdots$}
  % 
  \UnaryInfC{$
    \begin{array}{r@{\hspace{-1pt}}l}
      \dyn & \envExtend{globalPosition}{\seq{ \envElem{Pos}{1}, \cdots, \envElem{Pos}{j}, m }} \\
      &  \envExtend{varValue}{
        \begin{array}{l}
          \envElem{Var}{1} \Rightarrow \funcCall{fs{:}value}{\envElem{SR}{m}, \envElem{Var}{1}};\\
          \dots;\\
          \envElem{Var}{n} \Rightarrow \funcCall{fs{:}value}{\envElem{SR}{m},\envElem{Var}{n}} \\
        \end{array}} \proofs \grammarRule{ReturnExpr} \Rightarrow \envElem{Value}{m}
    \end{array}
    $}
  % 
  \singleLine
  % 
  \UnaryInfC{$\dynEnv{\begin{array}{l}%
        \FOR~\envElem{AttrSpec}{1} \texttt{ as } \envElem{\var{Var}}{1} \dots \envElem{AttrSpec}{n} \texttt{ as } \envElem{\var{Var}}{n}\\
        \grammarRule{RelationList}~\grammarRule{SQLWhereClause}~
        \RETURN~\grammarRule{ReturnExpr} 
      \end{array}  \Rightarrow \envElem{Value}{1}, \dots, \envElem{Value}{m}
    }$}
  % 
\end{dynamicrule}%



\subsubsection{\SparqlForClause}
\label{sec:sparqlforclause-sem}
%
The semantics of the \SparqlForClause expression (\cref{fig:xsparql-flwor}) is defined by the following normalisation
rules, static type analysis rules and dynamic evaluation rules.
%
Again, we start by presenting the normalisation rules for \SparqlForClause{s} with implicit variable selection (by means
of ``\FOR~*''), which are translated into explicitly stated variables:
%
\begin{normalisationrule}
  \mapping{%
    \sema{%
      \FOR~\keyword{*}~\OptDatasetClause~\SparqlWhereClause\\
      \SolutionModifier~\ReturnExpr }{ \grammarRule{Expr} }%
  }%
  {%
    \sema{%
      \FOR~\sem{\SparqlWhereClause}{vars}\\
      \OptDatasetClause~\SparqlWhereClause\\
      \SolutionModifier~\ReturnExpr }{Expr}%
  }%
  \label{for_star}%
\end{normalisationrule}%
%
The normalisation rule $\sem{\grammarRule{WhereClause}}{vars}$ determines all statically \emph{unbound variables}
present in the \SparqlWhereClause, \ie~returns a whitespace separated list of all variables in the \SparqlWhereClause
that are not present in the $\stat{}.\ecomp{varType}$ environment component.


\paragraph{Static type analysis.}
\label{sec:static-type-analysis}
%
The following static rule takes care of defining the types of variables present in a \FOR expression as \type{RDFTerm}
and infers the static type of the \SparqlForClause expression:\footnote{Similar to the XQuery Core
  \grammarRule{OptPositionalVar}, the \grammarRule{OptDatasetClause} covers both cases when a \SparqlForClause contains
  (or does not contain) a \DatasetClause.}
%
\begin{staticrule}
  \begin{prooftree}
    \def\ScoreOverhang{1pt}%
    \def\extraVskip{1pt}%
    \alwaysNoLine%
    %
    \AxiomC{$\stat\envExtend{varType}{\begin{array}{l}
          \envElem{Var}{1} \Rightarrow \type{RDFTerm};\\
            \dotsb; \\
            \envElem{Var}{n} \Rightarrow \type{RDFTerm}  
          \end{array}}~\proofs  \grammarRule{ExprSingle} \colon \envElem{Type}{}
      $}
    %
    \singleLine
    %
    \UnaryInfC{$\statEnv{
        \begin{array}{l}
          \FOR~\envElem{\var{Var}}{1} \dotsb \envElem{\var{Var}}{n}~\grammarRule{OptDatasetClause}\\
          \SparqlWhereClause~\SolutionModifier ~
          \RETURN~\grammarRule{ExprSingle}  \colon \envElem{Type}{}*
        \end{array}
      }$}
  \end{prooftree}%
  \label{eq:static-type}
\end{staticrule}%



\paragraph{Dynamic Evaluation.}
\label{sec:dynamic-evaluation}


We can now define the dynamic evaluation rules for the \SparqlForClause expression.  Intuitively these rules state that
the return expression \grammarRule{ExprSingle} will be executed for each \type{Pattern\-Solution} that is returned from
the evaluation of the \funcName{fs{:}sparql} function. The following two dynamic rules specify the evaluation of the
\SparqlForClause with an explicit \DatasetClause.  These rules use the \funcName{fs{:}dataset} function to parse the
\DatasetClause into an element of type \type{RDFDataset}, which will be stored in the \dyn.\ecomp{activeDataset}
component: If the evaluation of the \funcName{fs{:}sparql} function does not yield any solutions, \ie~evaluates to an
empty sequence, the overall result will also be the empty sequence:
%
\begin{dynamicrule}
  % 
  \AxiomC{$\dynEnv{\funcCall{fs{:}dataset}{\DatasetClause} \Rightarrow \envElem{Dataset}{}}$}
  % 
  \UnaryInfC{$\dynEnv{%
      \funcCall{fs{:}sparql}{%
        \begin{array}{l}
          \envElem{Dataset}{},\SparqlWhereClause,\\
          \SolutionModifier
        \end{array}
      }%
    } \Rightarrow \seq{}$}
  % 
  \singleLine
  % 
  \UnaryInfC{$\dynEnv{\begin{array}{l}
        \FOR~\envElem{\var{Var}}{1}\dotsb\envElem{\var{Var}}{n}~\DatasetClause\\
        \SparqlWhereClause~\SolutionModifier\\
        \ReturnExpr
      \end{array} \Rightarrow \seq{} }$}
  \label{dyn:empty-sparqlfor}
\end{dynamicrule}%
%
Otherwise, \grammarRule{ExprSingle} is evaluated for each solution in the results of the SPARQL query:
%
\begin{dynamicrule}
    %
    \AxiomC{$\dyn.\ecomp{globalPosition} = \seq{ \envElem{Pos}{1}, \cdots, \envElem{Pos}{j} } $}
    % 
    \UnaryInfC{$\dynEnv{\funcCall{fs{:}dataset}{\DatasetClause} \Rightarrow \envElem{Dataset}{}}$}
    % 
    \UnaryInfC{$\begin{array}{r@{\hspace{-1pt}}l}%
        \dyn &~\proofs \funcCall{fs{:}sparql}{%
          \begin{array}{l}%
            \envElem{Dataset}{}, \SparqlWhereClause,\\
            \SolutionModifier
            \end{array}
          } \Rightarrow \sm{1}{}, \dots, \sm{m}{}
      \end{array}
      $}
    %
    \UnaryInfC{$\begin{array}{r@{\hspace{-1pt}}l}%
        \dyn & \envExtend{globalPosition}{\seq{ \envElem{Pos}{1}, \cdots, \envElem{Pos}{j}, 1 }} ~\envExtend{activeDataset}{\grammarRule{Dataset}}\\
        &\envExtend{varValue}{%
          \begin{array}{l}%
            \envElem{Var}{1} \Rightarrow \funcCall{fs{:}value}{\sm{1}{},\envElem{Var}{1}};\\
            \dotsc;\\
            \envElem{Var}{n} \Rightarrow \funcCall{fs{:}value}{\sm{1}{},\envElem{Var}{n}}
          \end{array}%
        }~\proofs \grammarRule{ExprSingle} \Rightarrow
        \envElem{Value}{1}
      \end{array}$}
    %
    \UnaryInfC{$\vdots$}
    %
    \UnaryInfC{$\begin{array}{r@{\hspace{-1pt}}l}
        \dyn & \envExtend{globalPosition}{\seq{ \envElem{Pos}{1}, \cdots, \envElem{Pos}{j}, m }} ~\envExtend{activeDataset}{\grammarRule{Dataset}}\\
        &\envExtend{varValue}{\begin{array}{l}
            \envElem{Var}{1} \Rightarrow \funcCall{fs{:}value}{\sm{m}{},\envElem{Var}{1}}; \\
            \dotsc; \\
            \envElem{Var}{n} \Rightarrow \funcCall{fs{:}value}{\sm{m}{},\envElem{Var}{n}} 
          \end{array}%
        }~\proofs \grammarRule{ExprSingle} \Rightarrow \envElem{Value}{m}
      \end{array}$}
    %
    \singleLine
    %
    \UnaryInfC{$\dynEnv{\begin{array}{l}
          \FOR~\envElem{\var{Var}}{1}\dotsb\envElem{\var{Var}}{n}~\DatasetClause\\
          \SparqlWhereClause~\SolutionModifier\\
          \RETURN~\grammarRule{ExprSingle}
        \end{array}} \Rightarrow \envElem{Value}{1}, \dots, \envElem{Value}{m}$}
  \label{eq:sparqlForClause}
\end{dynamicrule}%
%
This rule ensures that the \ecomp{activeDataset} component of the dynamic environment is updated to reflect the explicit
\DatasetClause of the \SparqlForClause and that the \ecomp{globalPosition} environment contains all the positions in the
previous tuple streams.

The rule that handles the \SparqlForClause without an explicit \DatasetClause is presented next.  These rules are very
similar, with the exception that in following rules, the dataset over which the \SparqlForClause is evaluated is read
from the \dyn.\ecomp{activeDataset} component.
%
\begin{dynamicrule}
    % 
    \AxiomC{$\dyn.\ecomp{globalPosition} = \seq{ \envElem{Pos}{1}, \cdots, \envElem{Pos}{j} } $}
    %
    \UnaryInfC{$\dyn.\ecomp{activeDataset} \Rightarrow \envElem{Dataset}{}$}
    % 
    \UnaryInfC{$\dynEnv{%
        \funcCall{fs{:}sparql}{%
          \begin{array}{l}
            \envElem{Dataset}{},\SparqlWhereClause,\\
            \SolutionModifier
          \end{array}
        } \Rightarrow \sm{1}{}, \dots, \sm{m}{}%
      } $}
    % 
    \UnaryInfC{$\begin{array}{r@{\hspace{-1pt}}l}
        \dyn & \envExtend{globalPosition}{\seq{ \envElem{Pos}{1}, \cdots, \envElem{Pos}{j}, 1 }}\\
        &\envExtend{varValue}{\begin{array}{l}
            \envElem{Var}{1} \Rightarrow \funcCall{fs{:}value}{\sm{1}{},\envElem{Var}{1}}; \\
            \dotsc; \\
            \envElem{Var}{n} \Rightarrow \funcCall{fs{:}value}{\sm{1}{},\envElem{Var}{n}} 
          \end{array}%
        }~\proofs \grammarRule{ExprSingle} \Rightarrow \envElem{Value}{1}
      \end{array}$}
    % 
    \UnaryInfC{$\vdots$}
    % 
    \UnaryInfC{$\begin{array}{r@{\hspace{-1pt}}l}
        \dyn & \envExtend{globalPosition}{\seq{ \envElem{Pos}{1}, \cdots, \envElem{Pos}{j}, m }}\\
        &\envExtend{varValue}{%
        \begin{array}{l}
          \envElem{Var}{1} \Rightarrow \funcCall{fs{:}value}{\sm{m}{}, \envElem{Var}{1}};\\
          \dots;\\
          \envElem{Var}{n} \Rightarrow \funcCall{fs{:}value}{\sm{m}{},\envElem{Var}{n}} 
        \end{array}}~\proofs \grammarRule{ExprSingle} \Rightarrow \envElem{Value}{m}
    \end{array}
    $}
    % 
    \singleLine
    % 
    \UnaryInfC{$\dynEnv{
        \begin{array}{l}
          \FOR~\envElem{\var{Var}}{1}\dotsb\envElem{\var{Var}}{n}\\
          \SparqlWhereClause~\SolutionModifier\\
          \ReturnExpr 
        \end{array}}  \Rightarrow \envElem{Value}{1}, \dots, \envElem{Value}{m}$}
\label{xsparql.new.fssparql}
\end{dynamicrule}%
%
Analogously to the \SparqlForClause with an explicit dataset (Rule~\ref{dyn:empty-sparqlfor}), whenever the
$\funcName{fs{:}sparql}$ function evaluates to an empty sequence, the result will also be an empty sequence.



\subsubsection{\ConstructClause}
\label{sec:constructsem}
%
XSPARQL normalises \ConstructClause{s} into standard XQuery \RETURN expressions with the necessary mechanisms for
validation of the returned \ac{RDF} graph and as such, we define the semantics of \ConstructClause{s}
(\cref{fig:xsparql-flwor}) by means of normalisation rules.
%
One valid syntax for XSPARQL is a SPARQL stand-alone \CONSTRUCT query (as described in \cref{sec:syntax}).  These
queries are normalised into \CONSTRUCT queries with a surrounding \SparqlForClause by the following rule:
%
\begin{normalisationrule}
  \mapping{%
    \sema{%
      \CONSTRUCT~\ConstructTemplate\\
      \DatasetClause~\SparqlWhereClause\\
      \SolutionModifier%
    }{Expr}%
  }{%
    \sema{%
      \FOR~*~\DatasetClause\\
      \SparqlWhereClause~\SolutionModifier\\
      \CONSTRUCT~\ConstructTemplate%
    }{Expr}}
\label{eq:construct2forclause}
\end{normalisationrule}%
%
The recursive call to \sem{\cdot}{Expr} ensures that the resulting query will be further rewritten according to
normalisation Rule~\eqref{for_star} presented above, in order to explicitly state the variables present in the
\SparqlWhereClause.
%

Similar to the normalisation rule for stand-alone \textit{ReturnClauses} presented in~\citet[Section
4.8.1]{DraperFankhauserFernandez:2010aa}, the following normalisation rule transforms \CONSTRUCT clauses into XQuery
\ReturnClause{s}.
%
\begin{normalisationrule}%
  \mapping{%
    \sem{\mathtt{construct}~\ConstructTemplate}{Expr}%
  }{%
    \RETURN~\funcCall{fs{:}evalCT}{\sem{\ConstructTemplate}{normCT}}%
  }
  \label{eq:construct2forclause2}
\end{normalisationrule}%
%
In the following we assume that \ConstructTemplate is a simple \character{.} separated list of \grammarRule{Subject},
\grammarRule{Predicate} and \grammarRule{Object}. The~$\sem{\cdot}{normCT}$ rule transforms any Turtle shortcut notation
used in \ConstructTemplate to these simple lists.
%
As an example of this rule, we present the rule for normalising Turtle~\character{;} abbreviations (previously described
in \cref{sec:turtle}):
%
\begin{normalisationrule}%
  \mapping{%
    \sem{\grammarRule{Subject}~\envElem{Pred}{1}~\envElem{Obj}{1};~\dotsb;
      ~\envElem{Pred}{n}~\envElem{Obj}{n}~.}{normCT}%
  } {%
    \grammarRule{Subject}~\envElem{Pred}{1}~\envElem{Obj}{1}~.~\dotsb~
    \grammarRule{Subject}~\envElem{Pred}{n}~\envElem{Obj}{n}~.%
  }
  \label{eq:flatten-triples}
\end{normalisationrule}%
%
The normalisation rules for the other Turtle shortcuts that are allowed in the SPARQL \ConstructTemplate syntax are
similar to this one and are not presented here. 

Since anonymous blank nodes can be written in numerous ways in Turtle, the~$\sem{\cdot}{normCT}$ normalisation rule also
transforms each anonymous blank node into a labelled blank node where the identifier/label is distinct from any other
blank node labels present in the \ConstructTemplate.  This label will then be used by the skolemisation function to
generate the distinct blank node label for each position in the tuple stream.


In more detail, the \funcName{fs{:}evalCT} function checks the constructed RDF graph for validity (according to the
conditions described in \cref{sec:sparql-syntax}), filtering out any non-valid RDF triples where \emph{subjects} are
literals or \emph{predicates} are literals or blank nodes.  This is illustrated by the following dynamic evaluation
rules.
%
\begin{dynamicrule}
    %
    \AxiomC{$\dynEnv{\funcCall{fs{:}validTriple}{\mathit{Subj_1}, \mathit{Pred_1}, \mathit{Obj_1}}} \Rightarrow \envElem{Triple}{1}$}
    %
    \UnaryInfC{$\vdots$}
    %
    \UnaryInfC{$\dynEnv{\funcCall{fs{:}validTriple}{\envElem{Subj}{n}, \envElem{Pred}{n}, \envElem{Obj}{n}}} \Rightarrow \envElem{Triple}{n}$}
    %
    \singleLine
    %
    \UnaryInfC{$\begin{array}{r@{\hspace{-1pt}}l}
        \dyn &~\proofs \funcCall{fs{:}evalCT}{\begin{array}{c}
            Subj_1~Pred_1~Obj_1\\
            \dots\\
            Subj_n~Pred_n~Obj_n
          \end{array}}~\Rightarrow  \envElem{Triple}{1}, ~ \dotsb, ~ \envElem{Triple}{n}
      \end{array}
      $}
\end{dynamicrule}%
%
The following dynamic evaluation rule for the $\funcName{fs{:}validTriple}$ function checks, relying on the
\funcName{fs{:}bnode} function defined below, if a triple is valid according to the RDF semantics.
%
\begin{dynamicrule}
    % 
    \AxiomC{$\dynEnv{\funcCall{fs{:}bnode}{\grammarRule{Subject}} \Rightarrow \envElem{ValS}{}}$}
    % 
    \UnaryInfC{$\statEnv{\envElem{ValS}{}\ \mathbf{matches}\ (\type{uri} \mid \type{bnode})}$}
    % 
    \UnaryInfC{$\dynEnv{\grammarRule{Predicate} \Rightarrow \envElem{ValP}{}}$}
    % 
    \UnaryInfC{$\statEnv{\envElem{ValP}{}\ \mathbf{matches}\ \type{uri}}$}
    % 
    \UnaryInfC{$\dynEnv{\funcCall{fs{:}bnode}{\envElem{Object}{}} \Rightarrow \envElem{ValO}{}}$}
    % 
    \UnaryInfC{$\dynEnv{\envElem{ValO}{}\ \mathbf{matches}\ (\type{uri} \mid \type{bnode} \mid \type{literal})}$}
    % 
    \singleLine
    % 
    \UnaryInfC{$\begin{array}{r@{\hspace{-1pt}}l} \dyn &~\proofs
        \funcCall{fs{:}validTriple}{\begin{array}{l}
            \grammarRule{Subject},\\
            \grammarRule{Predicate},\\
            \grammarRule{Object}
          \end{array}}~\Rightarrow \begin{array}{l} 
          \keyword{element}~\stt{triple}~\keyword{of}~\keyword{type}~\stt{RDFTriple}~\{ \\
          \qquad\keyword{element}~\stt{subject}~\keyword{of}~\keyword{type}~\stt{RDFTerm}~\{ \envElem{ValS}{} \}\\
          \qquad\keyword{element}~\stt{predicate}~\keyword{of}~\keyword{type}~\stt{RDFTerm}~\{ \envElem{ValP}{} \}\\
          \qquad\keyword{element}~\stt{object}~\keyword{of}~\keyword{type}~\stt{RDFTerm}~\{ \envElem{ValO}{} \}\\
          \}\!\!
        \end{array}\\
        \end{array}
    $}%
\label{validTriple1}
\end{dynamicrule}%
%
In case any of the subject, predicate or object do not match an allowed type, the empty sequence is
returned. Effectively this suppresses any invalid RDF triples from the output graph.
%
\begin{dynamicrule}
  % 
    \AxiomC{$\dynEnv{\funcCall{fs{:}bnode}{\grammarRule{Subject}} \Rightarrow \mathit{ValueS}}$}
    \UnaryInfC{$\dynEnv{\grammarRule{Predicate} \Rightarrow \envElem{ValueP}{}}$}
    \UnaryInfC{$\dynEnv{\funcCall{fs{:}bnode}{\envElem{Object}{}} \Rightarrow \envElem{ValueO}{}}$}
    \UnaryInfC{$\dynEnv{\textbf{not}\left(\begin{array}{@{}l@{}}
              \envElem{ValueS}{}\ \mathbf{matches}\ \left(\type{uri} \mid \type{bnode}\right)\ \textbf{and}\\
              \envElem{ValueP}{}\ \mathbf{matches}\ \type{uri}\ \textbf{and}\\
              \envElem{ValueO}{}\ \mathbf{matches}~\left(\begin{array}{l}\type{uri} \mid \type{bnode} \mid  \type{literal}\end{array}\right)
            \end{array}\right)}$}
    \singleLine
    \UnaryInfC{$\dynEnv{\funcCall{fs{:}validTriple}{\grammarRule{Subject},\grammarRule{Predicate},\grammarRule{Object}}} \Rightarrow \left(\right)$}
\label{validTriple2}
\end{dynamicrule}%

\paragraph{Blank Node Skolemisation.}
%
In order to comply with the SPARQL \CONSTRUCT semantics, all blank nodes inside a \ConstructTemplate need
to be \emph{skolemised}, \ie~for each solution a new distinct blank node identifier needs to be generated.
%
Since we keep all the positions in the tuple streams, we can rely on the blank node label and these position values to
generate a unique blank node label (represented by the~$\funcName{fs{:}skolemConstant}$ function).
%
This skolemisation of blank nodes is performed by the \funcName{fs{:}bnode} function.  If the argument of this function
is of type \type{bnode} the skolemised label is calculated.  
%
\begin{dynamicrule}%
    % 
    \AxiomC{$\dynEnv{\envElem{ValueR}{}\ \mathbf{matches}\ \type{bnode}}$}
    % 
    \UnaryInfC{$\dyn.\ecomp{globalPosition} = \seq{ \envElem{PosValue}{1}, \dotsb, \envElem{PosValue}{n} }$}
    % 
    \UnaryInfC{$\begin{array}{r@{\hspace{-1pt}}l}
        \dyn &~\proofs \funcCall{fs{:}skolemConstant}{
          \begin{array}{l}ValueR, \\
            \envElem{PosValue}{1},\\
            \dotsc,\\
            \envElem{PosValue}{n}
          \end{array}}~\Rightarrow \envElem{ValueRS}{}\\
      \end{array}$}
    % 
    \singleLine
    % 
    \UnaryInfC{$\dyn~\proofs\funcCall{fs{:}bnode}{\envElem{ValueR}{}}~\Rightarrow~\keyword{element}~\stt{bnode}~\keyword{of}~\keyword{type}~\stt{xs:string}~\{ \envElem{ValueRS}{} \}$}
  \label{eq:bnode-skolem}
\end{dynamicrule}%
%
Otherwise, \funcName{fs{:}bnode} returns its argument unchanged:
%
\begin{dynamicrule}
  %
  \AxiomC{$\dynEnv{\envElem{Value}{}\ \mathbf{matches}\ (\type{uri} \mid \type{literal})}$}
  %
  \singleLine
  % 
  \UnaryInfC{$\dynEnv{\funcCall{fs{:}bnode}{\envElem{Value}{}} \Rightarrow \envElem{Value}{}}$}
  \label{eq:bnode-skolem-default}
\end{dynamicrule}%



%%% Local Variables:
%%% fill-column: 120
%%% TeX-master: t
%%% TeX-PDF-mode: t
%%% TeX-debug-bad-boxes: t
%%% TeX-parse-self: t
%%% TeX-auto-save: t
%%% reftex-plug-into-AUCTeX: t
%%% mode: latex
%%% mode: flyspell
%%% mode: reftex
%%% TeX-master: "../thesis"
%%% End:


\section{Semantic Correspondence between XSPARQL, SQL, XQuery, and SPARQL}
\label{sec:correspondence}
%
Since XSPARQL syntactically extends XQuery, and also any SPARQL \CONSTRUCT query is syntactically valid in XSPARQL,
these queries are considered semantically equivalent to the semantics in their base languages.
%
Regarding \ac{SQL} and SPARQL \SELECT expressions, we can show that their results remain the same under XSPARQL extended
semantics.
%
The next propositions formally establish this intuitive correspondence.

\begin{proposition}
\label{prop:xsparql}
  XSPARQL is a conservative extension of XQ\-uery.
\end{proposition}
%
\begin{proof}
  %
  The additional rules introduced in \cref{sec:semantics} do not modify the semantics of any native XQuery: the
  XSPARQL semantics -- expressed in terms of normalisation rules, static typing rules and dynamic evaluation rules --
  strictly extends the native semantics of XQuery.
  %
  In the semantics definition we also define new environment components, namely $\stat.\ecomp{globalPosition}$ and
  $\dyn.\textrm{activeDataset}$, which are not used in the XQuery semantics and thus do not interfere with query
  evaluation.
  %
  The only rules that use these newly created environments are the evaluation rules of \SparqlForClause{s}
  (\dyn.\ecomp{activeDataset}) and the dynamic evaluation rule~\eqref{eq:bnode-skolem} (\dyn.\ecomp{globalPosition}),
  which generates Skolem-identifiers for blank nodes in \CONSTRUCT parts.
  % 
  However, all these rules only apply to XSPARQL queries, which fall outside the native XQuery fragment, whereas the
  semantics of native XQuery queries remains untouched and independent of the extra environment components in XSPARQL.
  %
\end{proof}


We can also show that the answers of an XSPARQL \SQLForClause without any previously bound XSPARQL variables are the
same as the answers of the normalised query under \ac{SQL} semantics:
%
\begin{lemma}
  \label{lemma:xsparql-relation-instance}
  Let~$S$ be a \SQLForClause,~$xir_C$ the XSPARQL instance relation of~$S$, and~$S' = \funcCall{normaliseSQL}{S}$ the
  \ac{SQL} normalised query of~$S$. 
  %
  Furthermore, let~$R_1 = \RAXSP{S}$ and~$R_2 = \RASQL{S'}$, where~$\funcCall{sort}{R_2} = U$ be the relation instances
  resulting from evaluating~$S$ according to the XSPARQL semantics and the \ac{SQL} semantics, respectively.
  %
  If~$S$ does not contain any XSPARQL variables, \ie~$\vars{S} = \emptyset$, then~$R_1[U] =
  R_2$.
\end{lemma}
%
\begin{proof}
  %
  Following \cref{def:xsparql-sql-answers} we have that the answers of~$S$ under XSPARQL semantics are given
  by~$\funcCall{\sigma_{E}}{R_1 \times xir_C}$, where~$E$ is the XSPARQL select expression of~$S$.  
  % 
  Since~$\vars{S} = \emptyset$, according to \cref{def:xsparql-join},~$E$ will also be empty and we can
  simplify the expression that produces~$R_1$ to~$R_2 \times xir_C$.
  % 
  According to the definition of XSPARQL instance relation (\cref{def:xml2sqlterm})~$xir_C$ has
  cardinality~$1$ and thus the cross product does not change the cardinality of $R_2$, simply extending each solution
  in~$R_2$ with the attributes from the~$xir_C$ relation.  
  % 
  Since the cardinality of~$R_1$ and~$R_2$ is the same and the~$\times$ operation does not change any existing
  attributes of~$R_1$, we have that~$R_1[U] = R_2$.
\end{proof}

Similarly for SPARQL, we show the equivalence between SPARQL \ac{BGP} Matching~\cite[Section
12.3.1]{PrudhommeauxSeaborne:2008aa} and XSPARQL \ac{BGP} Matching (presented in \cref{sec:bf-extended-bgp}).
%
Based on this, we can then prove the equivalence between the XSPARQL and SPARQL semantics for \CONSTRUCT queries.
%
\begin{lemma}
  \label{lemma:xsparql-instance-mapping}
  Given a graph pattern~$P$, a dataset~$D$ and~$\mu_C$ the XSPARQL instance mapping of~$P$. Furthermore, let $\Omega_1 =
  \evalXS{D}{P}{\mu_C}$ and $\Omega_2 = \evalS{D}{P}$ be solution mappings.  If $\vars{P} \cap \dom{\mu_C} = \emptyset$,
  then $\Omega_1 = \Omega_2 \bowtie \set{\mu_C}$.
\end{lemma}
%
\begin{proof}
  %
  The XSPARQL BGP matching, $\evalXS{D}{P}{\mu_C}$, extends SPARQL's BGP matching, $\evalS{D}{P}$, by defining that the
  solutions of the BGP are the ones \emph{compatible} with the \emph{XSPARQL instance mapping} $\mu_C$.
  %
  Since the evaluation of graph patterns (such as \UNION, \OPTIONAL, \GRAPH and \FILTER) remains unchanged from the
  SPARQL semantics let us focus on the evaluation of a BGP $P$. 
  %
  If there are no shared values between the graph pattern and the XSPARQL instance mapping, as is the case
  when~$\vars{P} \cap \dom{\mu_C} = \emptyset$, then each solution $\mu \in \Omega_2$ returned by the SPARQL BGP
  evaluation semantics is trivially compatible with $\mu_C$ and the result of the XSPARQL BGP matching is $\mu \cup
  \mu_C$.  Extending this result to all solution mappings in $\Omega_2$, we obtain that $\Omega_1 = \Omega_2 \bowtie
  \set{\mu_C}$.
  % 
\end{proof}


\noindent
%
Finally, for SPARQL \CONSTRUCT queries we can state the following:
%
\begin{proposition}
\label{prop:xsparqlsparql}
  XSPARQL is a conservative extension of SP\-ARQL \CONSTRUCT queries.
\end{proposition}
%
\begin{proof}
  % 
  For XSPARQL queries consisting of a standalone SPARQL \CONSTRUCT query, there cannot exist any previous bindings for
  variables in XSPARQL and thus the XSPARQL instance mapping $\mu_C$ over which the \CONSTRUCT query will be executed is
  empty.  Let $P$ represent the graph pattern of the \CONSTRUCT query and $D$ the dataset, since~$\mu_C$ is empty,
  trivially there are no shared variables between $\mu_C$ and $P$.  Thus, following
  \cref{lemma:xsparql-instance-mapping} the bindings for XSPARQL BGP matching (say $\Omega_1$) are the same
  bindings as SPARQL BGP matching ($\Omega_2$), since $\Omega_1 = \Omega_2 \cup \{\emptyset\}$ and hence $\Omega_1 =
  \Omega_2$.
  % 
  Furthermore the formal semantics function $\funcName{fs{:}evalTemplate}$ returns an RDF graph satisfying all the
  conditions of \cref{def:sparql-construct}:
  \begin{inparaenum}[(i)]
  \item ignoring invalid RDF triples -- \cref{def:sparql-construct-1} -- is guaranteed by
    Rules~\eqref{validTriple1} and~\eqref{validTriple2}; and
  \item the generation of distinct blank nodes for each solution sequence -- \cref{def:sparql-construct-2} -- is
    enforced by the blank node skolemisation rules, Rules~\eqref{eq:bnode-skolem} and~\eqref{eq:bnode-skolem-default}.
  \end{inparaenum}
  %
\end{proof}





%%% Local Variables:
%%% fill-column: 120
%%% TeX-master: "../thesis" 
%%% TeX-PDF-mode: t
%%% TeX-debug-bad-boxes: t
%%% TeX-parse-self: t
%%% TeX-auto-save: t
%%% reftex-plug-into-AUCTeX: t
%%% End:



\section{Consuming JSON Data}
\label{sec:xsparql-json}

Due to the similarity between \ac{JSON} and \ac{XML}, in XSPARQL we incorporate \ac{JSON} data by translating the
\ac{JSON} objects into \ac{XML} data.
%
Furthermore \ac{JSON} does not specify a query language (this representation format is meant to be incorporated directly
into the JavaScript scripting language).
%
As presented in \cref{sec:json}, \ac{XML} is more flexible than \ac{JSON} and it is possible to convert \ac{JSON} into
\ac{XML} but not so easy in the opposite direction.  


This translation of \ac{JSON} to \ac{XML} enables access to the \ac{JSON} data using standard \ac{XPath}.  The following
definition presents the translation we use in XSPARQL.
% 
\begin{definition}[Translation from JSON to XML]
  % 
  \label{def:json2xml}
  %
  Let~$J$ be a \ac{JSON} object.  The translation of~$J$ to \ac{XML}, denoted \funcCall{translateXML}{J}, is an \ac{XML}
  document \xml{jsonObject}{\funcCall{translateMembers}{J}}, where \funcCall{translateMembers}{J} is defined as follows:
  % 
  \begin{itemize}[noitemsep]
  \item if~$J$ is an empty \ac{JSON} object or empty array, then \seq{};
  \item if~$J$ is a \ac{JSON} object, then for each~$K_i: V_i \in J$, \xml{K_i}{\funcCall{translateMembers}{V_i}};
  \item if~$J$ is a \ac{JSON} array, then for each~$E_i \in J$, \newline ~~~~~~\xml{arrayElement}{\funcCall{translateMembers}{E_i}};
  \item otherwise~$J$.
  \end{itemize}
\end{definition}
% 
\noindent For example the \ac{JSON} from \cref{fig:bands-json} translated into \ac{XML} according to
\cref{def:json2xml} is presented in \cref{fig:bands-json-xml}.
%
\begin{data}[t]
  \centering
  \lstinputlisting[language=XML,basicstyle=\small\ttfamily]{0-data+queries/usecaseData-json.xml}
  \caption{XML representation of JSON data}
  \label{fig:bands-json-xml}
\end{data}


\subsection*{Querying the XML representation of JSON}
\label{sec:query-jsonxml-repr}

\begin{query}[t]
  \centering
  \lstinputlisting[language=XSPARQL,basicstyle=\small\ttfamily]{0-data+queries/xsparql-json.xsparql}
  \caption{Querying JSON using XSPARQL}
  \label{qr:xsparql-json-xml}
\end{query}
%
\ac{JSON} data can be manipulated directly in JavaScript, where accessing members of objects can be done using the
\lit{.} separator, while accessing array elements is done using the standard bracket notation:~\lit{[} and~\lit{]}.
%
For example, if the \ac{JSON} object in \cref{fig:bands-json} is assigned to a JavaScript variable named \lit{b}, we
can access the member \lit{bands} by using \lit{b.bands} and accessing the second member of the Nightwish band can be
done with \lit{b.bands.Nightwish.members[1]}.\footnote{Please note that in JavaScript the first element of an array is
  at position 0, while the first element of XPath sequences is 1.}
%
In XSPARQL, querying the \ac{XML} representation of \ac{JSON} data can be done using an XPath expression, where,
assuming \funcCall{translateXML}{b} is assigned to an XSPARQL variable~\var{b}:
%
\begin{enumerate}[(i),noitemsep]
\item accessing members of an object can is done using the \lit{child} XPath axis, for example to access the
  representation of member \lit{bands} we write \lit{\var{b}/bands}; and
\item accessing specific elements of an array can be done using XPath predicates, \eg~to access the second member of the
  Nightwish band can be done with \lit{\var{b}/bands/Nightwish/members/*[2]}.\footnote{We can also use
    \lit{arrayElement} instead of \lit{*} in the XPath expression.}
\end{enumerate}
%
\begin{example}[Querying JSON using XSPARQL]
  %
  \cref{qr:xsparql-json-xml} presents the XSPARQL query that returns all members of the \stringValue{Nightwish}
  band from the (translated) \ac{JSON} \cref{fig:bands-json}.
  %
  In an XSPARQL query the transformation from \ac{JSON} into \ac{XML} is implemented using the
  \funcName{\qname{xsparql}{json\text{-}doc}} function (as shown in \cref{qr:xsparql-json-xml-line1} of
  \cref{qr:xsparql-json-xml}).
\end{example}
%
The implementation of the translation in \cref{def:json2xml} currently translates the complete \ac{JSON}
provided as input.  One possible optimisation for this implementation is to make it aware of the \ac{XPath} expression
and perform a selective translation of the input \ac{JSON} data.



%%% Local Variables:
%%% mode: latex
%%% mode: flyspell
%%% mode: reftex
%%% TeX-master: "../thesis"
%%% End:



\section{Processing RDB2RDF Mappings in XSPARQL}
\label{sec:examples-rdb2rdf}

The W3C RDB2RDF Working Group (WG) is currently in the process of defining a standard language to translate a relational
database into \ac{RDF}.  
%
The WG has defined 2 documents: the \emph{Direct Mapping (DM)}~\cite{ArenasPrudhommeauxSequeda:2011aa} specifies the
process of translating a relational database into \ac{RDF} in an automated manner, and the \emph{R2RML} language
definition~\cite{DasSundaraCyganiak:2011aa} corresponds to a user specified translation (in Turtle syntax) of the input
relational database.  The direct mapping provides a generic representation of the relational database while the R2RML
provides more fine-tuned control over the produced \ac{RDF}.

Next we start by giving an overview of the RDB2RDF Direct Mapping, the R2RML language, and then provide an algorithm for
the implementation of R2RML in XSPARQL.

\subsection{Direct Mapping}
\label{sec:direct-mapping}

The aim of the DM is to provide an \emph{off-the-shelf} translation of relational databases into \ac{RDF}, \ie~a
transformation that requires minimal user input.  
%
This translation follows already existing approaches, implemented by several conversion tools, and relies on creating
the output \ac{RDF} graph by assigning a unique identifier to each tuple in a relation from the input database.  This
identifier is created based on the relation name and the values for any existing primary keys and is then used as the
subject of each \ac{RDF} triple generated from the specific tuple.\footnote{In case a relation does have any primary
  keys a distinct blank node is used as an identifier for each tuple.}
%
Attributes names are used to generate a \ac{URI} that is used as a predicate, while the object consists of the value for
the specific attribute.
%

For processing DM in XSPARQL we need to have access to the underlying relational schema.  For this we rely on a custom
function that returns an \ac{XML} representation of the relational schema and, based on this representation, the DM
implementation is similar to the R2RML mappings, where we can use \SQLForClause{s} to access the relational database and
generate the target \ac{RDF} graph.
%
In the rest of this section we will focus on R2RML mappings and describe in more detail the XSPARQL query used to
implement such transformations.


\subsection{The R2RML mapping language}
\label{sec:r2rml-mapp-lang}

The R2RML mapping is itself an \ac{RDF} graph consisting of several \stt{TriplesMap}, that specify how to map a
\emph{logical table} in the input relational database into \ac{RDF}.  The \emph{logical table} can correspond to a
table, a view in the database, or the result of a SQL query to be executed over the input relational
database.\footnote{Arbitrary SQL queries can be executed in XSPARQL via an implementation-defined XQuery function and
  were included only to cater for this feature of R2RML.}


Each \stt{TriplesMap} consists of one \stt{SubjectMap} and possibly multiple \stt{Predicate\-ObjectMaps}.  Each row in
the logical table produces a single \emph{subject} in the target \ac{RDF}, which is specified by the \stt{SubjectMap}.
The multiple \stt{PredicateObjectMaps} each specify how to generate a \emph{predicate} and possibly several
\emph{objects} (by means of \stt{PredicateMaps} and \stt{ObjectMaps}, respectively) that are related to the generated
\emph{subject}.

\begin{figure}[t]
  \centering
  \lstinputlisting[escapeinside={\#@}{\^^M}]{0-data+queries/bands-rdb2rdf.ttl}
  \caption{RDB2RDF mapping for tables ``band'' and ``person''}
  \label{fig:rdb2rdf-band}
\end{figure}

Furthermore, each \stt{SubjectMap}, \stt{PredicateMap}, and \stt{ObjectMap} can specify how the \ac{RDF} term is created
by using different \ac{RDF} predicates.  For instance, using the \texttt{column} predicate for the mapping rule (\eg~the
predicate of the \stt{ObjectMap} on \cref{fig:rdb2rdf-band-column} of \cref{fig:rdb2rdf-band}) indicates that
the \ac{RDF} object should be generated based on the value of the column in the input database.
%
Another example is the \texttt{template} predicate, which specifies how terms are generated by using a template that
will be instantiated with values from the logical table, \eg~the \stt{subjectMap} from \cref{fig:rdb2rdf-band-template}
of \cref{fig:rdb2rdf-band}, states that the generated subject should be of the format
%
\begin{verbatim}
      http://example.com/band/{bandId}
\end{verbatim}
%
where \stt{\{bandId\}} is to be replaced by the value of the \stringValue{bandId} attribute in the specific tuple.
%
The other predicate used in the example from \cref{fig:rdb2rdf-band} is \qname{rr}{predicate}
(\cref{fig:rdb2rdf-band-predicate}), which states that the predicate of the generated triples should be
\qname{foaf}{name}.

Finally, foreign keys can be specified using \stt{RefObjectMap} and by indicating the \stt{TriplesMap} that represents
the foreign logical table and possibly a \stt{joinCondition} that specifies how to merge the two relations, as shown in
\crefrange{fig:rdb2rdf-band-parentTriplesMap}{fig:rdb2rdf-band-joinCondition} of \cref{fig:rdb2rdf-band}.

An R2RML mapping produces an \ac{RDF} dataset with all the generated triples belonging to the default graph unless
otherwise stated.
%
To cater for the possibility of creating triples in a named graph, we extend XSPARQL's generation of \ac{RDF} graphs in
Turtle format to generate an N-Quads representation~\cite{CyganiakHarthHogan:2009aa} of the \ac{RDF} data.
%
This extension is also used and is further expanded in \cref{cha:usecase}.


\subsection{R2RML Implementation in XSPARQL}
\label{sec:r2rml-xsparql}


\SetKw{From}{from}
\SetKw{Row}{row}
\SetKw{SPARQLForClause}{for}
\SetKw{kwLet}{let}
\SetKw{In}{in}
\SetKw{Assign}{\textbf{:=}}
\SetKwBlock{Where}{where}{}
\SetKwBlock{Construct}{construct}{}
\SetKwBlock{Return}{return}{}
\SetKwFunction{createTerm}{createTerm}
\SetKwFunction{createSubject}{createSubject}
\SetKwFunction{createLiteral}{createLiteral}
\SetKwFunction{createURI}{createURI}
\SetKwFunction{Value}{value}
\SetKwFunction{getLogicalTable}{getLogicalTable}
\SetKwFunction{createPO}{createPO}
%
\begin{algorithm}[t]
  \caption{\fc{rdb2rdf}{\var{m}} }
  \label{algo:rdb2rdf}
  \DontPrintSemicolon
  \SetNlSty{textrm}{}{}
  \KwIn{RDB2RDF mapping~$\var{m}$ (represented as RDF)}
  \KwResult{RDF Graph}
  \kwLet \var{mapSk} \Assign skolemise(\var{m}) \; \nllabel{skolem}
  \SPARQLForClause * \From{\var{mapSk}}\; \nllabel{loop-TriplesMap-start}
  \Where{ \var{map} rdf:type TriplesMap;  rr:logicalTable \var{table}; rr:subjectMap \var{s} . \;}
  \Return{
    \For{\Row \var{tableRow} \In \getLogicalTable{\var{table}}}{\nllabel{getTable}
      \kwLet \var{subject} \Assign \createSubject{\var{mapSk}, \var{tableRow}, \var{s} } \; \nllabel{createSubject}
      \createPO{\var{mapSk}, \var{tableRow}, \var{subject}, \var{map}} \nllabel{createPO}
    }
  }\nllabel{loop-TriplesMap-end}
\end{algorithm}

In this section we present an algorithm that implements the R2RML transformation in XSPARQL.
%
This transformation is implemented as an XSPARQL query but, for readability purposes, is summarised in
\cref{algo:rdb2rdf}.  
%
In this algorithm we rely on multiple queries to the R2RML input mapping file and since the R2RML representation may use
blank nodes for describing the mapping, we start by \emph{skolemising} blank nodes in the input \ac{RDF} graph, \ie~any
blank nodes used in the R2RML mapping are substituted with newly generated URIs that are distinct from any other URI in
the graph. This transformation allows us to use these newly generated URIs to merge data across different queries and is
represented in the algorithm by the~$\mathit{skolemise}$ function (line~\ref{skolem}).

The \SparqlForClause on lines~\ref{loop-TriplesMap-start}--\ref{loop-TriplesMap-end} iterates over all the
\stt{TriplesMaps} present in the mapping file and, for each of these \stt{TriplesMaps}, retrieves the specified data
from the input relational database.
%
This access to the (logical) table of the relational database is represented by the \SQLForClause on
line~\ref{getTable}, which instantiates \var{row} for each result row that the corresponding \ac{SQL} query returns (as
described in \cref{sec:syntax}).  The function \stt{getLogicalTable} is responsible for processing the different
available forms of specifying the input relation in R2RML.
%
In line~\ref{createSubject} we generate the \emph{subject} that is shared by all the triples derived from the same row
of the relation and pass it to the \stt{createPO} auxiliary function (line~\ref{createPO}) that takes care of generating
the predicate-object pairs.


\begin{algorithm}[t]
  \caption{\fc{createPO}{\var{mapSk}, \var{row}, \var{subject}, \var{map}} }
  \label{algo:createPO}
  \DontPrintSemicolon
  \SetNlSty{textrm}{}{}
  \KwIn{skolemised RDB2RDF mapping \var{mapSk}, Database data \var{row}, generated RDF term \var{subject}, input RDF term \var{po}}
  \KwResult{RDF Graph}
  \SPARQLForClause * \From{\var{mapSk}} \; \nllabel{pred-obj1}
  \Where{ \var{map} rr:predicateObjectMap [ rr:predicateMap \var{p};  rr:objectMap \var{o} ] \;} \nllabel{pred-obj2}
  \Return{
    \kwLet \var{predicate} \Assign \createTerm{\var{mapSk}, \var{row}, \var{p}} \; \nllabel{create-pred}
    \kwLet \var{object} \Assign \createTerm{\var{mapSk}, \var{row}, \var{o}} \; \nllabel{create-obj}
    \Construct{ \var{subject} \var{predicate} \var{object}}
  }
\end{algorithm}
%
The \textit{createPO} function described in \cref{algo:createPO} retrieves all the \stt{predicateMap} and
\stt{objectMaps} associated with the \stt{TriplesMap} we are processing (lines~\ref{pred-obj1}-\ref{pred-obj2}), creates
the respective \emph{predicate} (line~\ref{create-pred}) and \emph{object} (line~\ref{create-obj}) and then generates an
\ac{RDF} triple using the XSPARQL built-in \CONSTRUCT expression.  The \CONSTRUCT expression automatically takes care of
discarding any non-valid \ac{RDF} triples.



\begin{algorithm}[t]
  \caption{\fc{createTerm}{\var{mapSk}, \var{row}, \var{spec}} }
  \label{algo:createTerm}
  \DontPrintSemicolon
  \SetNlSty{textrm}{}{}
  \KwIn{skolemised RDB2RDF mapping \var{mapSk}, Database data \var{row}, RDF term specification \var{spec}}
  \KwResult{RDF Term}
  \SPARQLForClause * \From{\var{mapSk}}\; \nllabel{spec-type1}
  \Where{ \var{spec} \var{specType} \var{specValue} \;} \nllabel{spec-type2}
  \Return{\nllabel{return1}
    \uIf{\var{specType} == rr:predicate}{createURI(\var{specValue})}
    \uElseIf{\var{specType} == rr:column}{\createLiteral{\Value{\var{row}, \var{specValue}}}}
    \lElse{\dots}\nllabel{return2}
  }
\end{algorithm}
%
The auxiliary function \stt{createTerm} is partially presented in \cref{algo:createTerm}: given a specific database
table \emph{\$row}, this function produces an \ac{RDF} term according to the specification given in the RDB2RDF mapping.
The \SparqlForClause from lines~\ref{spec-type1}-\ref{spec-type2} takes care of querying the RDB2RDF mapping to
determine the type of the \ac{RDF} term to be produced. Finally, the \textbf{return} clause
(lines~\ref{return1}-\ref{return2}) presents the process of creating \ac{RDF} terms for the \qname{rr}{predicate} and
\qname{rr}{column} types of specifications.  The \stt{createURI} and \stt{createLiteral} functions used in this
algorithm are XSPARQL built-in functions that behave as constructors for \acp{URI} and literals, respectively.
%
The \stt{value} function returns the value associated with a column name in an \stt{SQLResult} (similar to the formal
semantics function presented in \cref{sec:fs:value}).
%
The missing RDB2RDF specifications are similar to the presented ones possibly requiring some extra processing, \eg~the
\qname{rr}{template} specification needs to be parsed to extract the column names from the template and then access
their values of the current row.



\begin{data}[t]
  \centering
  \lstinputlisting{0-data+queries/bands-rdb2rdf-output.ttl}
  \caption{Output of algorithm rdb2rdf (\cref{algo:rdb2rdf})}
  \label{fig:rdb2rdf-example-output}
\end{data}
%
\cref{fig:rdb2rdf-example-output} presents the \ac{RDF} graph resulting from the applying
\cref{algo:rdb2rdf} to the RDB2RDF mapping presented in \cref{fig:rdb2rdf-band}.






%%% Local Variables:
%%% mode: latex
%%% mode: flyspell
%%% mode: reftex
%%% TeX-master: "../thesis"
%%% End:



\section{Related Work}
\label{sec:relworks}

\begin{table}[t]
\caption{Overview of Related Work}
\label{fig:related-work-overview}
\centering
{\small
\begin{tabular}{r|>{\centering\arraybackslash}m{.63cm}>{\centering\arraybackslash}m{.66cm}>{\centering\arraybackslash}m{.63cm}|c|cc|c}
  \toprule
                                               &         \multicolumn{3}{c|}{Input Format}    & Target   &   \multicolumn{2}{c|}{Query Language}       & Ontology\\
  System                                       & {\footnotesize RDB} & {\footnotesize XML} & {\footnotesize RDF}        & Model    &   Surface      &  Target            & Generation\\
  \midrule
  Gloze                                        &      & \yes &      & RDF & --- & ---  & partial \\
  \citet{DroopFlarerGroppe:2008aa}             &      & \yes &      & RDF & SPARQL+XSLT & SPARQL & \no \\
  \citet{VrandecicDenglerRudolph:2005aa}       &      &      & \yes & --- & --- & SPARQL &  \no \\
  \citet{BohringAuer:2005aa}                   &      & \yes &      & RDF & --- & --- &  \yes \\
  \citet{RodriguesRosaCardoso:2008aa}          &      & \yes &      & RDF &--- & --- &  \yes \\
  \citet{FischerFlorescuKaufmann:2011aa}       & \yes & \yes & \yes & --- &--- & XQuery &  \no \\
  \citet{Walsh:2003aa}                         &      & \yes & \yes & --- &--- & --- &  \no \\
  \midrule
  \citet{BerruetaLabraHerman:2008aa}           &      & \yes & \yes & RDF & XSLT+SPARQL & --- &  \yes \\
  \citet{BikakisGioldasisTsinaraki:2009aa}     &      & \yes & \yes & --- &  SPARQL & XQuery & \yes \\
  \citet{GroppeGroppeLinnemann:2008aa}         &      & \yes & \yes & XML & SPARQL & XQuery &  \yes \\
  MarkLogic Server                             &      & \yes & \yes & XML &SPARQL & XQuery &  \yes \\ 
  \citet{CorbyKefi-KhelifCherfi:2009aa}        & \yes & \yes & \yes &  --- & SPARQL & --- & \no \\
  RDB2RDF                                      & \yes &      & \yes & RDF &--- & --- &  \no \\
  XSPARQL                                      & \yes & \yes & \yes & \begin{minipage}{.85cm}XML/ RDF\end{minipage} & XSPARQL & XQuery &  \no \\
  \bottomrule
\end{tabular}%
}
\end{table}
%
Several proposals for integrating data from relational databases, \ac{XML}, and \ac{RDF} were presented before. 
%
On one hand, converting between relational databases and \ac{XML} has been long studied, either by the integration of
\ac{SQL} and \ac{XML}~\cite{EisenbergMelton:2001aa,EisenbergMelton:2004aa} or the specification of the representation of
database instances in \ac{XML}.\footnote{\url{http://www.w3.org/XML/RDB.html}, retrieved on 2012/07/17.}  
%
In practice, most relational database management systems include a datatype for storing \ac{XML} data while other works
focus on the implementation of the XQuery language over a relational database
backend~\cite{GrustSakrTeubner:2004aa,GrustRittingerTeubner:2008aa}.
%
As such, this
section focuses on the integration of \ac{XML} and \ac{RDF} or relational databases and \ac{RDF} data.

To begin, an analysis of tools for converting between relational databases and \ac{RDF} is presented
by~\citet{GrayGrayOunis:2009aa}, which also aims at studying the expressivity of SPARQL to represent scientific queries,
namely in the astronomy domain.
%
Although, as stated by the authors, data and queries were mostly numeric and thus biased towards relational data and
SQL, the comparison gives a good overview of how the tested tools perform in comparison to relational databases.  
%
Some of the conclusions indicate that these tools are still not able to compete with relational databases in terms of
performance and that SPARQL is also not yet expressive enough to pose the necessary queries.


\citet{Patel-SchneiderSimeon:2003aa} present a proposal to integrate the semantics of \ac{XML} and \ac{RDF} by defining
a model-theory that encapsulates both the \ac{XDM} and \ac{RDF} data models.
%
This proposal has not been applied in practice and most of the existing proposals to merge \ac{XML} and \ac{RDF} rely on
translating the data from different formats and/or translating the queries from different languages.
%
With this in mind, we divided the proposals into the following categories:
\begin{enumerate}[label=(\arabic*),noitemsep]
\item \label{item:rel-1} \textbf{Normalised Representations:} include proposals that suggest using a normalised format for representing
  \ac{RDF} in \ac{XML}.  Although similar to the next proposals, in these systems the translation can usually be
  automated and they do not address querying, simply reusing standardised languages.
\item \label{item:rel-2} \textbf{Translation of data:} these tools aim at integrating the heterogeneous data by translating between
  different formats, usually relying on user predefined mappings.
\item \label{item:rel-3} \textbf{Integration of query languages:} this category of approaches (where XSPARQL is also included) considers
  the integration and/or expansion of query languages to allow querying different formats without requiring the
  translation of data from the original formats.
\end{enumerate}
%
\cref{fig:related-work-overview} presents an overview of systems considered in~\ref{item:rel-2} and~\ref{item:rel-3}.
This table classifies the different systems according to whether they support input from relational databases (RDB),
\ac{XML}, or \ac{RDF}.  The target model indicates, if there exists a data translation step, what is the format used for
the integrated representation.  The Surface and Target query languages state, if available, the language in which the
system accepts queries and if they are translated into a different query language.  Finally, the Ontology Generation
column specifies if the system generates an ontology description based an input \ac{XML} Schema or relational database
structure.
%
We next give a short description of some of the tools and proposals available grouped by the presented categories.
%
\subsection*{Normalised Representations}
\label{sec:normalised-representations}
%
The following proposals specify a normalised syntax for \ac{RDF} in \ac{XML}.
%
The TriX format~\cite{CarrollStickler:2004aa} consists of an alternative normalised serialisation for \ac{RDF} in \ac{XML}, with the aim of
being compatible with standard \ac{XML} tools.  
%
In this serialisation, each \ac{RDF} triple is represented as a \texttt{triple} \ac{XML} element with three children
elements representing the subject, predicate, and object of the triple.
%
It uses \ac{XSLT} as an extensibility mechanism, allowing syntactic extensions to be specified and macros to be defined.

Also in 2003, \emph{TreeHugger}\footnote{\scriptsize\url{http://rdfweb.org/people/damian/treehugger/index.html},
  retrieved on 2012/07/17.}  defines abstraction functions (implemented as extensions of the Saxon XQuery engine) that
enable the navigation of an \ac{RDF} graph structure in both \ac{XSLT} and XQuery.
%
This navigation is specified using XPath-like expressions that specify the \ac{RDF} class and property that users want
to query, which are in turn translated into SPARQL queries.

Well known parsers for \ac{RDF}, such as the Redland \ac{RDF} Libraries\footnote{\url{http://librdf.org/}, retrieved on
  2012/07/17.} also provide canonical formats of RDF/XML.
%
\emph{R3X}\footnote{\url{http://wasab.dk/morten/blog/archives/2004/05/30/transforming-rdfxml-with-xslt}, retrieved on
  2012/07/17.} takes this representation one step further by grouping the canonical RDF/XML output of the Redland parser
and grouping the triples by subject.  The aim of this grouping by subject is to make the canonical format easier to
process with \ac{XSLT}.
%
Very similar to R3X, Grit\footnote{\url{http://code.google.com/p/oort/wiki/Grit}, retrieved on 2012/07/17.} also defines
a normalised format of RDF/XML where triples are grouped by their subject to facilitate processing in \ac{XSLT} and
improve the triple access evaluation times.  Furthermore Grit normalises \acp{URI} to make lookups easier in
\ac{XSLT}. 


\subsection*{Data Translation}
\label{sec:data-translation}
%
We now present the proposals that rely on a user-specified normalised format for \ac{RDF}.
%
Gloze~\cite{Battle:2006aa} aims at interpreting an \ac{XML} document as \ac{RDF} data based on the \ac{XML} Schema
definition.
%
\ac{XML} elements and attributes are mapped to \ac{RDF} object or datatype properties, depending on wether they are
described as complex or simple types in the XML Schema (complex types are mapped to object properties and simple types
are mapped to datatype properties). 

\citet{DroopFlarerGroppe:2008aa} translate the \ac{XML} document into RDF, annotating it with necessary information to
answer XPath queries, namely the ordering, axes relations between XML elements, and attributes of elements.
%
The authors then propose to integrate XPath queries into SPARQL as subqueries in \acp{BGP}, where the result of the
subexpressions is assigned to SPARQL variables.
%
These XPath subexpressions are in turn translated into SPARQL queries that, using the introduced annotations, allow the
preservation of the semantics of the original queries and ordering of solutions.

\citet{DeursenPoppeMartens:2008aa} presents an approach for the transformation between \ac{XML} and \ac{RDF} by
specifying mappings between an XML Schema and an \ac{OWL} ontology.
%
The authors introduce a language for the mapping specification, relying on XPath expressions for selecting the \ac{XML}
elements, and defining with \ac{OWL} classes the elements are mapped to.  The target \ac{RDF} data is generated by
processing these input mappings.

\citet{VrandecicDenglerRudolph:2005aa} suggests using a normalised form of RDF/XML by specifying a restricted form of
\acp{DTD} that generate normalised \ac{XML} format and again relying on standard \ac{XML} processing tools for
subsequent transformations.  The provided \ac{DTD} is used to generate SPARQL queries that access the \ac{RDF} data and
the system then relies on post-processing of the SPARQL query results to generate the desired output.  The use of
\acp{DTD} and automatic generation of SPARQL queries allows to leverage the existing \ac{XML} users that are not
familiar with \ac{RDF} technologies.


Also catering for \ac{SQL} queries, \citet{FischerFlorescuKaufmann:2011aa} present a translation of both \ac{SQL} and
SPARQL queries into XQuery.  Again the translation of SPARQL to XQuery operates on a normalised form of RDF/XML and thus
a data translation step is required.  A similar approach is taken for translating relational data into \ac{XML} and then
rewriting \ac{SQL} to XQuery.  In this paper the authors do not present an extended syntax language for the combination
of data in the different formats and rather rely on the translation of data into \ac{XML}.


\emph{RDF Twig}~\cite{Walsh:2003aa} suggests \ac{XSLT} extension functions that provide views on the sub-trees of an RDF
graph. 
%
The main idea of \ac{RDF} Twig is that while RDF/XML is hard to navigate using XPath, a subtree of an \ac{RDF} graph can
be serialised in more useful forms that facilitate navigation.  As such the authors provide \ac{XSLT} extension
functions that create different views of parts of the input \ac{RDF}.



Several other approaches aim at automatically translating an \ac{XML} Schema into an equivalent \ac{OWL}
ontology~\cite{BohringAuer:2005aa,RodriguesRosaCardoso:2008aa}, focusing on mapping \ac{XML} elements to \ac{OWL}
classes and properties.
%
However in XSPARQL, we are focusing on translation and integration of instance data, rather than aiming to provide a
semantic interpretation for \ac{XML} data.


While not catering for the integration of \ac{XML}, several other approaches focus on mapping relational data to
\ac{RDF}.
%
For instance, D2R Server~\cite{Bizer:2003aa} and D2R Map or Triplify~\cite{AuerDietzoldLehmann:2009aa} enable the
conversion between RDB data and \ac{RDF}.
%
Large commercial database companies are also providing solutions for \ac{RDF} triple stores, such as
Oracle~\cite{DasSrinivasan:2009aa} and Virtuoso~\cite{ErlingMikhailov:2007aa}.  Most of these projects assume a fixed
translation schema where, for instance, database tables are translated into \ac{RDFS} classes and table attributes are
represented as properties.



\subsection*{Language Integration}
\label{sec:language-integration}
%
In this category of proposals we include the systems that consider the integration and/or expansion of query languages
that allow the querying different formats without requiring the translation of data from the original formats.

\citet{BerruetaLabraHerman:2008aa} presents a framework that facilitates SPARQL queries to be performed from \ac{XSLT}:
XSLT+SPARQL.  It adds functions to \ac{XSLT} that provide the ability to query SPARQL endpoints and uses standard
\ac{XSLT} to process the SPARQL \ac{XML} results format.
%
Similar to our current implementation, this relies on a clear separation between the SPARQL query and \ac{XSLT} parts of the
query.



Some proposals suggest compiling a SPARQL query to \ac{XSLT} or XQuery. \citet{BikakisGioldasisTsinaraki:2009aa}
translate each SPARQL query into an XQuery using a mapping from \ac{OWL} to \ac{XML} Schema.  The translation from
SPARQL to XQuery is guided by the provided mapping (which can be automatically generated by a separate system) and thus
allows the use of the SPARQL query language to access legacy \ac{XML} data without the need to perform data translation.

Similarly~\citet{GroppeGroppeLinnemann:2008aa} proposes to embed SPARQL into \ac{XSLT} or XQuery, by presenting
extensions to these languages that enable to query \ac{RDF} data.  In this proposal each SPARQL query is also translated
into an equivalent XQuery.
%
This language is very close to the XSPARQL language but however it requires converting the \ac{RDF} data to \ac{XML}
according to a predefined schema.
%
Assuming the queried dataset is available beforehand, this translation introduces an overhead to the query and, in case
the dataset is not available for example due to being stored behind a SPARQL endpoint, such translation is not possible.
%
In \cref{cha:optimisations}, we present some benchmark comparisons between an implementation of this language
(provided to us by the authors) and our implementation of the XSPARQL language.

Ding and Buxton presented another approach to translate SPARQL into XQuery at the 2011 Semantic Technology
Conference.\footnote{\url{http://semtech2011.semanticweb.com/sessionPop.cfm?confid=62&proposalid=4015}, retrieved on
  2012/07/17.} 
%
This rewriting generates XQuery specifically tailored for the Marklogic Server \ac{XML} database
engine,\footnote{\url{http://www.marklogic.com/products-and-services/marklogic-5/}, retrieved on 2012/07/17.} which
incorporates \ac{RDF} triples by using an internal \ac{XML} representation.


Part of the CORESE Semantic Web framework,\footnote{\url{http://wimmics.inria.fr/corese}, retrieved on 2012/07/17.}
\citet{CorbyKefi-KhelifCherfi:2009aa} provides extensions of SPARQL to process SQL, XPath, and \ac{XSLT} in SPARQL queries.  
%
The authors also define an \ac{XSLT} extension function that allows to evaluate SPARQL queries and integrate the query
result into the \ac{XSLT} processing.
%
The implementation of these extensions is based on the CORESE framework, which employs caching mechanisms for the input
\ac{XML} and \ac{RDF} documents.
%
This approach is again similar to XSPARQL however the choice here was to extend SPARQL and \ac{XSLT}, opposed to XSPARQL's extension
of XQuery.  


The Saxon XQuery engine (which we are using in our implementation) provides extension functions that allow to execute
\ac{SQL} queries and represent the results of the query as \ac{XML}, easily incorporating them into the XQuery or
\ac{XSLT} query.
%
This feature follows a similar implementation as XSPARQL but however does not provide the extend syntax as XSPARQL.  The
extension function executes a \ac{SQL} query, although the functionality of injecting variable values provided by
XSPARQL can be done, this task is left in charge of the query writer.

The nSPARQL query language~\cite{PerezArenasGutierrez:2008aa} proposes to extend SPARQL with navigational capabilities
using nested regular expressions.  With this addition, the language is sufficiently expressive to capture the semantics
of \ac{RDFS}.  In addition to this, it introduces a number of graph navigation operators and adds the ability to selectively
traverse the graph.  This work is different than our current proposed approach for XSPARQL, but one of the possibilities
for extending XSPARQL is to enable it to perform XQuery enriched SPARQL queries.



%%% Local Variables:
%%% fill-column: 120
%%% TeX-master: t
%%% TeX-PDF-mode: t
%%% TeX-debug-bad-boxes: t
%%% TeX-parse-self: t
%%% TeX-auto-save: t
%%% reftex-plug-into-AUCTeX: t
%%% mode: latex
%%% mode: flyspell
%%% mode: reftex
%%% TeX-master: "../thesis"
%%% End:



\section{Conclusion}
\label{sec:xsparql-conclusions}

This chapter described the novel query language that we defined to tackle the integration of heterogeneous sources.  We
presented the syntax and semantics of the language, which are based on the syntax and semantics of the XQuery language.
% 
XSPARQL relies on the semantics of the other languages, \ac{SQL} and SPARQL for querying the relational and \ac{RDF}
data and we also presented equivalences between the execution of queries in the different languages.

This query language forms the basis for a possible solution for the presented data integration scenario.
%
In the next chapter we present our implementation of XSPARQL and tackle the problem of defining optimisations for the
XSPARQL language, in an attempt to lower the query evaluation times for more complex queries, while the issue of
representing meta-information in \ac{RDF} is addressed in \cref{cha:anql}.


%%% Local Variables:
%%% fill-column: 120
%%% TeX-master: t
%%% TeX-PDF-mode: t
%%% TeX-debug-bad-boxes: t
%%% TeX-parse-self: t
%%% TeX-auto-save: t
%%% reftex-plug-into-AUCTeX: t
%%% mode: latex
%%% mode: flyspell
%%% mode: reftex
%%% TeX-master: "../thesis"
%%% End:
