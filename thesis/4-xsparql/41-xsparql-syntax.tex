\section{Syntax}
\label{sec:syntax}



\begin{figure}[t]
  \centering
  { \scriptsize
    \setlength{\extrarowheight}{2pt}
    \begin{tabular}{llll}
      \cline{2-2}
      \multicolumn{1}{l|}{Prolog:} & \DECLARE \emph{prefix}=\texttt{"}\emph{namespace-URI}\texttt{"}&\multicolumn{1}{|c}{\multirow{2}{*}{\textbf{or}}}&\\
      \cline{2-2}
      \multicolumn{1}{l|}{} & \PREFIX \emph{prefix}: \texttt{<}\emph{namespace-URI}\texttt{>}&\multicolumn{1}{|c}{}&\\
      \cline{2-2}\\
      \cline{2-2}
      \multicolumn{1}{l|}{Body:} & \FOR \emph{var} \IN \emph{\XSPARQLFLWORExpr expression}&\multicolumn{1}{|c}{}&\multirow{4}{*}{\ForClause}\\
      \multicolumn{1}{l|}{} & \LET \emph{var} := \emph{\XSPARQLFLWORExpr expression}&\multicolumn{1}{|c}{}&\\
      \multicolumn{1}{l|}{}& \WHERE \emph{\XSPARQLFLWORExpr expression} &\multicolumn{1}{|c}{}&\\
      \multicolumn{1}{l|}{}& \ORDERBY \emph{\XSPARQLFLWORExpr expression}&\multicolumn{1}{|c}{\multirow{2}{*}{\textbf{or}}}&\\
      \cline{2-2}
      \multicolumn{1}{l|}{}& \FOR \emph{SelectSpec} &\multicolumn{1}{|c}{}&\multirow{3}{*}{\SQLForClause}\\
      \multicolumn{1}{l|}{}& \FROM \emph{RelationList} &\multicolumn{1}{|c}{}&\\
      \multicolumn{1}{l|}{}& \WHERE \emph{WhereSpecList } &\multicolumn{1}{|c}{\multirow{2}{*}{\textbf{or}}}&\\
      \cline{2-2}
      \multicolumn{1}{l|}{}& \FOR \emph{varlist} &\multicolumn{1}{|c}{}&\multirow{6}{*}{\SparqlForClause}\\
      \multicolumn{1}{l|}{}& \FROM\,/\,\FROMNAMED~\DatasetClause &\multicolumn{1}{|c}{}&\\
      \multicolumn{1}{l|}{}& \WHERE \{\emph{ pattern } \}&\multicolumn{1}{|c}{}&\\
      \multicolumn{1}{l|}{}& \ORDERBY \emph{expression}&\multicolumn{1}{|c}{}&\\
      \multicolumn{1}{l|}{}& \LIMIT \emph{integer $> 0$} &\multicolumn{1}{|c}{}&\\
      \multicolumn{1}{l|}{}& \OFFSET \emph{integer $> 0$}&\multicolumn{1}{|c}{}&\\
      \cline{2-2}\\
      \cline{2-2}
      \multicolumn{1}{c|}{Head:}& \CONSTRUCT \{ \emph{ConstructTemplate (with nested  \XSPARQLFLWORExpr expressions)} \} &\multicolumn{1}{|c}{\multirow{2}{*}{\textbf{or}}}&\multirow{1}{*}{\ConstructClause}\\
      \cline{2-2}
      \multicolumn{1}{c|}{}& \RETURN \emph{XML+ nested  \XSPARQLFLWORExpr expressions}&\multicolumn{1}{|c}{}&\multirow{1}{*}{\ReturnClause}\\
      \cline{2-2}
    \end{tabular}}
  \caption{Schematic view of XSPARQL}
  \label{fig:XSPARQL}
\end{figure}


%%% Local Variables:
%%% fill-column: 120
%%% TeX-master: t
%%% TeX-PDF-mode: t
%%% TeX-debug-bad-boxes: t
%%% TeX-parse-self: t
%%% TeX-auto-save: t
%%% reftex-plug-into-AUCTeX: t
%%% mode: latex
%%% mode: flyspell
%%% mode: reftex
%%% TeX-master: "../thesis"
%%% End:

%
Conceptually, XSPARQL is a merge of XQuery, SPARQL \CONSTRUCT and \SELECT queries, and \ac{SQL} \SELECT queries, as
presented schematically in \cref{fig:XSPARQL}.
%
This re-use of different query languages allows us to benefit from their facilities for retrieving data in the different
models, while also allowing us to use Turtle-like syntax for constructing RDF graphs (inherited from the SPARQL
language).
%
Since XSPARQL is based on XQuery, we allow any native XQuery expression and further extend XQuery's syntax with the following
expressions:
%
\begin{enumerate}[label=(\roman*),noitemsep]
\item XQuery and SPARQL namespace declarations in the Prolog may be used interchangeably;
\item in the Body, we allow the existing XQuery \ForClause{s} and also
  SPARQL \SELECT queries (\SparqlForClause) and
  SQL \SELECT queries (\SQLForClause); and

\item in addition to XQuery's native \ReturnClause, in the head we allow RDF graphs to be created directly using
  \CONSTRUCT templates (\ConstructClause).
\end{enumerate}
%
In XSPARQL we also allow different forms of nesting:
%
\begin{inparaenum}[(i)]
\item \LET assignments can contain the result of subqueries that construct \ac{RDF} graphs, and the assigned variable
  can later be used in SPARQL-style \FROM clauses, or
\item nesting can also be used for value construction within SPARQL-style \CONSTRUCT templates.
\end{inparaenum}
%
Since the new \SQLForClause and \SparqlForClause expressions stand at the same level as XQuery's \FOR and \LET
expressions, such clauses are allowed to start new \synt{\XSPARQLFLWORExpr} expressions and may also occur inside deeply
nested XSPARQL queries.
%
The main difference between these new expressions and \ac{SQL} and SPARQL \SELECT expressions is that while the latter
expressions return bindings for variables (as described in \cref{sec:sql,sec:sparql-preliminaries}), the new expressions
follow an approach similar to XQuery's \ForClause by adding new variables to the scope of query and as such we choose a
syntax also inspired by the XQuery \ForClause.


An overview of the grammar productions for these newly introduced expressions (\SQLForClause, \SparqlForClause, and
\ConstructClause) is presented in \cref{fig:xsparql-flwor}.  Notably, when compared to the XQuery grammar, we
introduced a new production (\synt{\XSPARQLFLWORExpr}) that changes the XQuery \FLWORExpr to include the new
expressions.
%
\setlength{\grammarindent}{90pt}%

\begin{figure}[t]
  \centering
  \begin{boxedminipage}{0.935\linewidth}
    {\scriptsize
      \vspace{-\grammarparsep}
      \begin{grammar}
        <\textbf{\XSPARQLFLWORExpr}>       ::= ( <FLWOExpr> | <\SQLForClause> | <\SparqlForClause> ) \\
        ( <\ReturnClause> | <\ConstructClause> )
        
        <FLWOExpr>        ::=  ( <\ForClause> | <LetClause> )+ <XQWhereClause>? <OrderByClause>? 
        
        <\ReturnClause>     ::=  `return' <ExprSingle> 
        
        <\textbf{\SQLForClause}>    ::=  `for' <SelectSpec> <RelationList> <SQLWhereClause>?
        
        <\textbf{\SparqlForClause}>  ::=  `for' ( <VarName>+ | `*' ) <DatasetClause>? <WhereClause>? <SolutionModifier>
        
        <\textbf{\ConstructClause}>  ::=  `construct' <ConstructTemplate'>
      \end{grammar}
    }
    \end{boxedminipage}
  \caption{\XSPARQLFLWORExpr syntax overview}
  \label{fig:xsparql-flwor}
\end{figure}


We next look at the syntax of each newly introduced expression in more detail while presenting some XSPARQL query
examples that allow us to perform the lifting and lowering tasks in a straightforward fashion.


\subsection{\SparqlForClause}
\label{sec:sparqlforclause}
%
{\scriptsize 
  \begin{grammar}
    <\textbf{\SparqlForClause}> ::= `for' (<VarRef>+ | `*') <DatasetClause>? <WhereClause>? <SolutionModifier>
  \end{grammar}
  \vspace{-.5\grammarparsep}
}

\noindent The newly introduced \SparqlForClause is similar to an XQuery \FOR expression that returns a sequence of
SPARQL results.
%
In this grammar production, the \SparqlWhereClause and \SolutionModifier correspond to rules~[13] and~[14] from the
SPARQL grammar, respectively \cf~\citet[Appendix~A.8]{PrudhommeauxSeaborne:2008aa}.
%
Similar to SPARQL's and \ac{SQL}'s \lit{\SELECT *} shortcut, we allow to write \lit{\FOR *} in place of \lit{\FOR
  [\emph{\it list of all unbound variables appearing in the \grammarRule{WhereClause}}]} in \SparqlForClause{s}, which
effectively avoids listing the distinguished variables of the query.

We also extended the rules for the SPARQL \stt{SourceSelector} grammar expression (rule~[12] of the SPARQL grammar) in
order to allow graphs in a dataset to be specified by a variable:

{\scriptsize
  \begin{grammar}
    <SourceSelector>   ::=  <IRIref> | <VarRef>
  \end{grammar}
}


The variables used here must contain an \ac{RDF} graph, resulting from a \ConstructClause (as described in the next
section and further detailed in \cref{sec:semantics}).




Regarding the syntax for variables in XQuery and SPARQL, we restrict the use of SPARQL \lit{?}-prefixed variables and
allow only \lit{\var{}}-prefixed variables that are compatible with XQuery's variable specifications.
%
On the other hand, as mentioned in \cref{sec:xquery}, XQuery also allows to specify variables as QNames, allowing the
disambiguation of variables based on their namespace.  However, since such variable names are not allowed in SPARQL we
further assume that only unprefixed variables are shared between the XQuery and SPARQL expressions of XSPARQL.

\begin{query}[t]
  \lstinputlisting[language=XSPARQL]{0-data+queries/lowering-bands.xsparql}%
  \caption{Lowering using XSPARQL}%
  \label{qr:lowering_xsparql}%
\end{query}
%
The lowering transformation can also be rewritten using XSPARQL.  These are the kind of transformations that present
extra problems for the \ac{XSLT} and XQuery languages and where we can see the advantages of using XSPARQL.  By using the
introduced \SparqlForClause{s} for accessing the \ac{RDF} graph, XSPARQL avoids handling \ac{RDF} as \ac{XML} data,
along with all the encapsulated issues.
%
\begin{example}[Lowering RDF data with XSPARQL]
  % 
  The lowering XSPARQL query for our running example is shown in \cref{qr:lowering_xsparql}. Here we can note the
  inclusion of \SparqlForClause{s}, for instance in \cref{qr:lowering_xsparql_bands_start}, to retrieve all the
  bands (\qname{mo}{Band}) contained in the \ac{RDF} data.
  %
  Furthermore, nested \SparqlForClause{s} can be used for further processing of the input data: the \SparqlForClause in
  \crefrange{qr:lowering_xsparql_members_start}{qr:lowering_xsparql_members_end} is responsible for retrieving all
  the members of the respective band, where the \litVar{band} variable is instantiated with the band identifier
  currently being processed.
  % 
  A similar structure is repeated for converting the corresponding albums of the band
  (\crefrange{qr:lowering_xsparql_albums_start}{qr:lowering_xsparql_albums_end}) and songs of each album
  (\crefrange{qr:lowering_xsparql_songs_start}{qr:lowering_xsparql_songs_end}).
  % 
\end{example}




\subsection{\ConstructClause}
\label{sec:constructclause-syntax}
%
{\scriptsize 
  \begin{grammar}
    <\textbf{\ConstructClause}> ::= `construct' <ConstructTemplate>
  \end{grammar}
  \vspace{-.5\grammarparsep}%
}
%
\noindent The \ConstructClause allows XSPARQL to produce \ac{RDF} graphs and, by following SPARQL's restrictions on the
generated \ac{RDF} triples (\cf~\cref{sec:sparql-syntax}), we also ensure that the resulting graph is valid
\ac{RDF}.
%
The XSPARQL \synt{\ConstructTemplate} expression is defined in the same way as the production \synt{\ConstructTemplate}
in SPARQL~\cite{PrudhommeauxSeaborne:2008aa}, but we additionally allow nested \synt{\XSPARQLFLWORExpr} expressions in
subject, predicate, and object positions.
%
\noindent We allow three types of nested expressions, identified by the shortcuts
\lit{\textbraceleft}~\synt{\XSPARQLFLWORExpr}~\lit{\textbraceright},
\lit{<\textbraceleft}~\synt{\XSPARQLFLWORExpr}~\lit{\textbraceright>}, and
\lit{\_:\textbraceleft}~\synt{\XSPARQLFLWORExpr}~\lit{\textbraceright} that construct \emph{literals}, \emph{URIs}, and
\emph{blank nodes}, respectively.
%
This syntax is used during static analysis to correctly determine the type of each element:~\type{lit\-eral},
\type{uri}, and \type{bnode} (\cf~\cref{sec:xsparql-types} below).

Additionally, we also allow SPARQL-style \ConstructClause{s} to appear before the body part of queries, and as such
XSPARQL becomes a syntactic superset of native SPARQL \CONSTRUCT queries (with the minor exception being the restriction
on \lit{?}-prefixed variables).


\begin{query}
  \lstinputlisting[language=XSPARQL]{0-data+queries/lifting-bands.xsparql}%
  \caption{Lifting in XSPARQL}%
  \label{qr:lifting_xs}%
\end{query}
%
The following lifting query shows the use of the \ConstructClause expression.
%
\begin{example}[Lifting XML data with XSPARQL]
  \cref{qr:lifting_xq} can be reformulated into its slightly more concise XSPARQL version in \cref{qr:lifting_xs}.
  %
  This query behaves in a similar way to \cref{qr:lifting_xq}, creating the \ac{RDF} triples for each entity in the
  input \ac{XML} data.  The difference is that we are using nested SPARQL-like \CONSTRUCT clauses for creating the
  \ac{RDF} triples (\cf~\crefrange{lift-xs-construct-bands-start}{lift-xs-bands-end}).
  %
  In \cref{lift-xs-songs-end} we use the different XSPARQL shortcuts, in this case to create \acp{URI} and literals.
  %
  The result of this query is also guaranteed to be valid \ac{RDF} as explained in \cref{sec:constructsem}.
\end{example}


\subsection{\SQLForClause}
\label{sec:syntax-rdb}
%
{\scriptsize 
  \begin{grammar}
    <\textbf{\SQLForClause}>    ::=  `for' <SelectSpec> <RelationList> <SQLWhereClause>?
  \end{grammar}
  \vspace{-.5\grammarparsep}%
}


\noindent The \SQLForClause element represents an \ac{SQL} \SELECT query that can be evaluated against the underlying
database.  Similar to XQuery's \FOR clause, the \SQLForClause expression represents the results of the execution of a
SQL query and exposes the result values to other subsequent expressions in the query.
%
\noindent The additional \SQLForClause syntax rules are presented next, where \synt{VarRef} corresponds to an XSPARQL
variable (\lit{\var{}}-prefixed), \synt{TableAlias} represents a string used as an alternative name for the relation,
and \synt{Constant} represents an integer or string:

{\scriptsize
  \begin{grammar}
    <SelectSpec>      ::=  <AttrSpecList> | `*'  | `row' <VarRef>

    <AttrSpecList>    ::=  <AttrSpec> <AttrNameSpec>? (`,' <AttrSpec> <AttrNameSpec>?)*

    <AttrSpec>        ::=       <attrName> 
                           \alt <relationName> `.' <attrName> 
                           \alt <VarRef> 

    <AttrNameSpec>    ::=  `as' <VarRef>

    <RelationList>    ::=  `from' <RelationSelector> (`,' <RelationSelector>)*

    <RelationSelector> ::=      <RelationName> (`as' <RelationAlias>)? 
                           \alt <VarRef> (`as' <RelationAlias>)?

    <SQLWhereClause>  ::=  `where' <WhereSpecList>

    <WhereSpecList>   ::=  `(' <WhereSpecList> <BooleanOp> <WhereSpecList> `)' 
                            \alt <WhereSpec> 

    <WhereSpec>       ::=        <WhereAttrSpec> <ComparisonOp> <WhereAttrSpec> 
                            \alt <WhereAttrSpec> <ComparisonOp> <Constant>
                            \alt <Constant> <ComparisonOp> <WhereAttrSpec>

    <WhereAttrSpec>   ::=        <AttrSpec> 
                            \alt `\textbraceleft' <VarRef> `\textbraceright'

    <BooleanOp>       ::=  `and' | `or'

    <ComparisonOp>    ::=  `=' | `!=' | `!=' | `<' | `<=' | `>' | `=>'
  \end{grammar}
}%
%
When comparing the XQuery and \ac{SQL} languages we find a syntactical mismatch between the representation
of variables: while \ac{SQL} considers the relation names specified in \synt{RelationSelector} as variables (as
described in \cref{sec:sql}), XQuery assumes \lit{\$}-prefixed variable names.
%
XSPARQL provides ways of overcoming this mismatch, allowing to specify variable names for the results of an
\SQLForClause, by:
%
\begin{enumerate}[label=(\roman*), noitemsep]
\item explicitly specifying a variable name for each attribute -- represented by the syntax rule \synt{AttrNameSpec},
  where \synt{VarRef} is the variable name to which the attribute value is assigned: \eg~\lit{for bands.bandId as
    \$bandId};
\item \label{forStar} implicitly by omitting the variable name or using \lit{\FOR *}; and
\item \label{forRow} using the \lit{row} keyword instantiates the specified variable with each \emph{result row} the
  query produces.
\end{enumerate}
%
For~\ref{forStar}, each attribute in the result set is assigned a variable name automatically with the same name as the
attribute name, of the format: \litVar{relationName.attributeName}.
%
\begin{example}[Variable Name Generation]
  %
  Consider the relational schema presented in \cref{ex:bands-schema}.
  %
  If we specify a \SQLForClause in the form of \lit{for * from person}, the variable names that will be available for
  the query will be \litVar{person.personId}, \litVar{person.personName}, and \litVar{person.bandId}.
  % 
\end{example}

If the relation attributes are not known beforehand, \eg~if the relation is specified as a variable, it is not possible
to generate the variable names as described in~\ref{forStar}.
%
In this case, we can use \lit{row \$r} in place of the variable names specification, and at execution time, \lit{\$r}
will be instantiated with an \ac{XML} representation containing all the attributes in the queried relations.
%
It is then possible to access all the attributes or to retrieve (if known) a specific attribute.
%
This form of selecting attributes is necessary for processing RDB2RDF mappings (presented in \cref{sec:r2rml-xsparql})
since the queried relations and attributes are read from a user-specified \ac{RDF} graph and thus the attributes of the
relation cannot be determined during syntactical analysis of the query.

In \ac{SQL}, \WHERE clauses indicate specific values of an attribute to be matched or that the value of two attributes
must be the same.  When we introduce the extended XSPARQL syntax (which allows to use \var{}-prefixed variables) we need
a way to specify if the variable represents an attribute name or an attribute value.  We make this distinction in the
syntax of XSPARQL: a \var{}-prefixed variable represents an attribute value, in case we want a variable to represent an
attribute name of a relation we use the \lit{\textbraceleft} \synt{VarRef} \lit{\textbraceright} syntax.  Further
details on how XSPARQL handles this distinction are presented in \cref{sec:semantics}.

In a similar fashion to the lifting query from \ac{XML} data (\cref{qr:lifting_xs}), we can use \SQLForClause{s} to
access relational data and convert it to \ac{RDF}, as presented in the following example.
%
\begin{query}[t]
  \centering
  \lstinputlisting{0-data+queries/xsparql-rdb-bands.xsparql}%
  \caption{Lifting from relational database}%
  \label{qr:xsparql-db-example}%
\end{query}
%
\begin{example}[Lifting Relational data with XSPARQL]
  \label{ex:xsparql-rdb}
  %
  \cref{qr:xsparql-db-example} shows an XSPARQL query that performs the lifting task over the relational schema
  described in \cref{ex:bands-schema}.
  %
  In this query we are using the primary key (generated identifier) of each relation for generating the blank node label
  of each entity (\cf~\cref{qr:xsparql-db-band-end}).
  %
  The rest of the query consists of creating the respective \ac{RDF} triples for the other relations: person
  (\crefrange{qr:xsparql-db-person-start}{qr:xsparql-db-person-end}), album
  (\crefrange{qr:xsparql-db-album-start}{qr:xsparql-db-album-end}), and song
  (\crefrange{qr:xsparql-db-song-start}{qr:xsparql-db-song-end}).
\end{example}






%%% Local Variables:
%%% fill-column: 120
%%% TeX-master: t
%%% TeX-PDF-mode: t
%%% TeX-debug-bad-boxes: t
%%% TeX-parse-self: t
%%% TeX-auto-save: t
%%% reftex-plug-into-AUCTeX: t
%%% mode: latex
%%% mode: flyspell
%%% mode: reftex
%%% TeX-master: "../thesis"
%%% End:
