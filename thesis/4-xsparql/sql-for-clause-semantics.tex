\subsubsection{Querying Relational Data}
\label{sec:extending-sql}


In order to reuse the semantics of \ac{SQL} for defining the semantics of XSPARQL \SQLForClause{s} we transform the
\ecomp{varValue} component of the dynamic environment in which the \SQLForClause is executed into a relation (which we
call the \emph{XSPARQL instance relation}).
%
The following definition presents this translation:
%
\begin{definition}[XSPARQL instance relation]
  \label{def:xml2sqlterm}
  Let the set of relation names (\AR) be defined as in \cref{sec:relational-model}, and let~$C$ be an expression
  context.
  %
  The \emph{XSPARQL instance relation} of~$C$ is a relation instance named \lit{$xir_C$}, where $xir_C$ is a reserved
  relation name, \ie~$xir_C \not\in \AR$, and~$\funcCall{sort}{xir_C} = \dom{\funcCall{\dyn}{C}.\ecomp{varValue}}$.
  %
  For each mapping~$v_{i} \rightarrow x_{i} \in \funcCall{\dyn}{C}.\ecomp{varValue}$, the value of~$xir_C$ for
  attribute~$v_i$, denoted~$\funcCall{xir_C}{v_{i}}$, is defined as:
  %
  \begin{itemize}[noitemsep]
  \item if~$x_{i} = \seq{}$ is an empty sequence then~$\funcCall{xir_C}{x_{i}} = \NULL$;
  \item if~$x_{i} = \seq{e_{1}, \dotsc, e_{n}}$ is a sequence, then~$\funcCall{xir_C}{x_{i}} =
    \funcCall{fn{:}concat}{\funcCall{SQLTerm_C}{e_{1}}, \dotsb, \funcCall{SQLTerm_C}{e_{n}}}$.\footnote{Since the values
      of any relation attribute must be atomic, in the case of a variable being assigned to an XQuery sequence we assume
      the concatenation of each element of the sequence.}
    %
  \end{itemize}
  %
   For a \synt{SQLWhereClause}~$S$, we call the XSPARQL instance relation of the expression context in which~$S$ is
  executed the \emph{XSPARQL instance relation of~$S$}.
\end{definition}
%

%
Another necessary step to enable the reuse of \ac{SQL} evaluation semantics is to convert our extended syntax (that
allows for \var{}-prefixed variable names) into valid \ac{SQL} syntax: each \synt{WhereSpec} in a \SQLForClause that
contains an XSPARQL variable is removed from the normalised \ac{SQL} query (by replacing it with \lit{true}) and is
stored for a later evaluation by the XSPARQL semantics.  For this we rely on the following normalisation function:
%
\begin{definition}[\ac{SQL} Representation of \synt{SQLWhereClauses}]
  %
  \label{def:normalisation-where}
  %
  Let~$S = \lit{where}~\synt{WhereSpecList}$ be a \synt{SQLWhereClause}.  The normalisation of~$S$,
  $\funcCall{normaliseSQL}{S} = \lit{where}~\funcCall{normaliseSQL}{\synt{WhereSpecList}}$, where
  \funcCall{normaliseSQL}{\synt{WhereSpecList}} is defined as:
  % 
  \begin{itemize}
  \item if \synt{WhereSpecList} is of form~$\lit{(}\ \synt{WhereSpecList}_{1}\ \synt{Op}\ \synt{WhereSpecList}_2\
    \lit{)}$ then
    %
    \[
    \lit{(}\ \funcCall{normaliseSQL}{\synt{WhereSpecList}_1}\ \synt{Op}\ \funcCall{normaliseSQL}{\synt{WhereSpecList}_2}\ \lit{)}
    \]
  \item if~\synt{WhereSpecList} is of form~$Attr_1~\synt{Op}~Attr_2$ then~$\funcCall{normaliseSQL}{Attr_1~Op~Attr_2}$
    is:
    \[
    \begin{dcases*}
      \lit{true}        & if~$Attr_1$  or~$Attr_2$ is an XSPARQL variable \\
      Attr_1~Op~Attr_2  & otherwise.
    \end{dcases*}
    \]
  \end{itemize}
  %
  Furthermore we denote the set of \synt{WhereSpec} of~$S$ in which an attribute is an XSPARQL variable as
  \funcCall{whereSpecVars}{S}.
  % 
\end{definition}
%
\noindent
%
The normalisation of complete \SQLForClause{s} consists also of the normalisation of the syntactical elements
\synt{AttrSpecList} and \synt{TableSelector} presented in \cref{sec:syntax-rdb}.
%
In the normalisation of~\synt{AttrSpecList} we remove any existing \synt{AttrNameSpec} component, since they reflect
only the name of the corresponding XSPARQL variable.
%
However, the normalisation of the \synt{TableSelector} can only be performed during the dynamic evaluation of the
XSPARQL query since any variables present in the \synt{TableSelector} must be evaluated to determine the corresponding
relation name.
%
With the restriction of performing the substitution at evaluation time, we can reuse the standard translation of a
\ac{SQL} query into relational algebra as presented in \cref{sec:sql-semantics}.

Next we present how XSPARQL combines the results of a \ac{SQL} query with an XSPARQL instance mapping.
%
For this we rely on the standard relational selection~($\sigma$) and cross-product~($\times$) algebra operators
presented in \cref{sec:querying-rdb} and on the~$xir_C$ relation instance from \cref{def:xml2sqlterm}.
%
Firstly, we present the construction of the relational algebra select expression that, based on the provided
\SQLForClause~$S$ and the XSPARQL instance mapping of~$S$, makes the connection between the results of the \ac{SQL}
query and the existing XSPARQL variable bindings:
%
\begin{definition}[XSPARQL $\sigma$ expression]
  \label{def:xsparql-join}
  %
  Let~$S$ be a \SQLForClause with expression context~$C$ and~$V = \funcCall{whereSpecVars}{S}$ the attribute
  specifications that contain XSPARQL variables in~$S$.
  %
  The \emph{XSPARQL $\sigma$ expression} of~$S$, denoted~$\funcCall{\sigma_{xs}}{S}$, is a relational algebra~$\sigma$
  expression that, for each $Att_1~Op~Att_2 \in V$ is~$\funcCall{trans}{Att_1}~Op~\funcCall{trans}{Att_2}$,
  where~$\funcCall{trans}{Att}$ is defined as:
  % 
  \begin{itemize}[noitemsep]
  \item $Attr$ \hfill if~$Attr$ is not an XSPARQL variable;
  \item if~$Attr = \lit{\$}AttrName$ is an XSPARQL variable then 
    %
    \[
    \funcCall{trans}{Attr} = \begin{dcases*}
        \dyn.\funcCall{\ecomp{varValue}}{AttrName} & if~\stat.\funcCall{\ecomp{varType}}{Attr} = \type{SQLAttribute}\\
        \textrm{`}xir_C.AttrName\textrm{'} & otherwise.
      \end{dcases*}
    \]
  \end{itemize}
\end{definition}
%
\noindent
%
This definition creates a relational algebra expression from the extended XSPARQL \SQLForClause syntax, which can then
be used to further restrict the results of the normalised \ac{SQL} expression.
%
Based on these definitions we can introduce the translation of \SQLForClause{s} into relational algebra.
%
\begin{definition}[XSPARQL relational algebra expression]
  \label{def:xsparql-sql-answers}
  %
  Let~$Q$ be a \SQLForClause,~$Q' = \funcCall{normaliseSQL}{Q}$ the \ac{SQL} rewriting of~$Q$,~$E =
  \funcCall{\sigma_{xs}}{S}$ the XSPARQL~$\sigma$ expression of~$S$, and~$\RASQL{Q'}$ the relational algebra expression
  obtained from the standard \ac{SQL} translation into relational algebra.
  %
  The \emph{XSPARQL relational algebra expression} of~$Q$, denoted~$\RAXSP{Q}$, combines the relational algebra
  expression of the \ac{SQL} query and restricts its results to the existing bindings for XSPARQL variables as follows:
  %
  \[
  \funcCall{\sigma_{E}}{\RASQL{Q'} \times xir_C} \enspace .
  \]
  %
\end{definition}
%
The following example illustrates the translation of XSPARQL \SQLForClause{s} into XSPARQL relational algebra
expressions.
%
\begin{figure}[t]
  \subfloat[Value Matching]{\label{ex:sql-for-ex1}
    \begin{minipage}{.5\linewidth}
      \lstinputlisting{0-data+queries/sql-for-clause-ex1.xsparql}
    \end{minipage}
  }
  % 
  \subfloat[Attribute Matching]{\label{ex:sql-for-ex2}
    \begin{minipage}{.5\linewidth}
      \lstinputlisting{0-data+queries/sql-for-clause-ex2.xsparql}
    \end{minipage}
  }
  \caption{XSPARQL \SQLForClause examples}%
  \label{fig:sql-for-ex}%
\end{figure}
%
\begin{example}[Translation of \SQLForClause{s} into Relational Algebra]
  %
  \cref{fig:sql-for-ex} presents two XSPARQL queries including \SQLForClause{s}.
  %
  The query in \cref{ex:sql-for-ex1} illustrates the syntax for querying values of a relation.  First the
  normalisation function drops the restriction in line~3, which is incorporated into the relational algebra~$\sigma$
  expression:
  % 
  \[
  \funcCall{\sigma_{band.bandId = xir_C.x}}{\funcCall{\sigma_{band.bandName = 'Nightwish'}}{band} \times xir_C} \enspace ,
  \]
  where~$\funcCall{sort}{xir_C} = \set{x}$ and~$\funcCall{xir_C}{x} = 1$.

  \medskip

  On the other hand, the query in \cref{ex:sql-for-ex2} shows how to match attribute names.
  %
  The query in this figure is converted into the following relational algebra expression:
  % 
  \[
  \funcCall{\sigma_{band.bandName = 'Nightwish'}}{\funcCall{\sigma_{band.bandId = 1}}{band} \times xir_C} \enspace ,
  \]
  %
  where~$\funcCall{sort}{xir_C} = \set{x}$ and~$\funcCall{xir_C}{x} = \textrm{`bandName'}$.  
  %
\end{example}







%%% Local Variables:
%%% fill-column: 120
%%% TeX-master: t
%%% TeX-PDF-mode: t
%%% TeX-debug-bad-boxes: t
%%% TeX-parse-self: t
%%% TeX-auto-save: t
%%% reftex-plug-into-AUCTeX: t
%%% mode: latex
%%% mode: flyspell
%%% mode: reftex
%%% TeX-master: "../thesis"
%%% End:
