
\subsection{XSPARQL Types}
\label{sec:xsparql-types}
%
We extend the \acl{XDM} (described in \cref{sec:xpath-data-model}) with the following new types to accommodate
for \ac{SQL} and SPARQL specific parts of XSPARQL:
%
\begin{enumerate}[(1),noitemsep]
\item\label{enum:t0} the \type{SQLTerm} is an extension of \qname{xs}{anyAtomicType} (as presented in
  \cref{sec:mapping-xml-types});
\item\label{enum:t1} the \type{RDFTerm} type further consists of the subtypes \type{uri}, \type{bnode} and
  \type{literal} and is used as the type of SPARQL variables;
\item\label{enum:t2} the \type{PatternSolution} type consists of a sequence of pairs \seq{\grammarRule{variableName},
  \type{RDFTerm}}, representing \ac{SQL} or SPARQL variable bindings;
\item\label{enum:t3} the \type{RDFGraph} is the type resulting from the evaluation of \CONSTRUCT expressions; and
\item\label{enum:t4} the \type{RDFDataset} is the type used for representing \ac{RDF} datasets, which is further
  constituted by one default \type{RDFGraph} and a sequence of named graphs (\type{RDFNamedGraph}).
\end{enumerate}
%
\cref{verb:types} presents the formal definition of \ref{enum:t0}--\ref{enum:t4} following the notation for \ac{XML}
Schema datatypes (presented in \cref{sec:xml-schema}).
%
The \type{RDFTerm} type is used to represent RDF terms (composed of \acp{URI}, blank nodes or literals).  The type of
SPARQL variables is represented by the \type{Binding} type, that consists of the variable name and the RDF term that is
assigned to it.
%
Finally, sequences of SPARQL variable bindings are represented by the type \type{PatternSolution}.  

\begin{figure}
  \centering
\begin{tabular}{|cc|}
\hline
\begin{minipage}{.45\linewidth}
\begin{Verbatim}[framesep=2mm,frame=none,fontsize=\scriptsize,commandchars=\\\(\)]

define type \textbf(URI-reference) restricts xs:anyURI;

define type \textbf(Literal) extends xs:string {
      attribute datatype of type URI-reference?, 
      attribute lang of type xml:lang? };

define type \textbf(RDFTerm) { 
      element uri of type URI-reference |
      element bnode of type xs:string |
      element literal of type Literal };

define type \textbf(SPARQLBinding) extends RDFTerm { 
      attribute name of type xs:string };

define type \textbf(SPARQLResult) { 
      element binding of type SPARQLBinding* };

define type \textbf(SQLTerm) extends xs:anyAtomicType ;

define type \textbf(SQLBinding) extends SQLTerm { 
      attribute name of type xs:string };

define type \textbf(SQLResult) { 
      element binding of type SQLBinding* };

define type \textbf(SQLAttribute) extends xs:string ;

  \end{Verbatim}
\end{minipage}
&
\begin{minipage}{.45\linewidth}
\begin{Verbatim}[frame=none,fontsize=\scriptsize,commandchars=\\\(\)]

define type \textbf(PatternSolution) { 
      element result of type SPARQLResult |
      element result of type SQLResult };

define type \textbf(RDFGraph) { 
      element triple of type RDFTriple* };

define type \textbf(RDFTriple) { 
      element subject of type RDFTerm,
      element predicate of type RDFTerm,
      element object of type RDFTerm };

define type \textbf(RDFDataset) { 
      element defaultGraph of type RDFGraph,
      element namedGraphs of type RDFNamedGraphs };

define type \textbf(RDFNamedGraphs) { 
      element namedGraph of type RDFNamedGraph* };

define type \textbf(RDFNamedGraph) { 
      attribute name of type xs:string,
      element graph of type RDFGraph };
  \end{Verbatim}
\end{minipage}%
\\
\hline
\end{tabular}
\caption{XSPARQL Type Definitions}
\label{verb:types}
\end{figure}


Similarly for \ac{SQL} results, sequences of \ac{SQL} variable bindings are also represented by the type
\type{PatternSolution}.
%
Analogously, we define the types \type{SQLResult} and \type{SQLBinding} for representing \ac{SQL} results.  The
\type{SQLBinding} type is defined as an extension of \qname{xs}{anyAtomicType}, and we follow the mapping from \ac{SQL}
types into \ac{XML} types presented in \cref{fig:SQL2XML}.


The \type{RDFGraph} type corresponds to a sequence of \type{RDF\-Triple}{\type{s}}, which are in turn a complex type
composed of \type{sub\-ject}, \type{predicate} and \type{object}.
%
The \type{RDFDataset} type is defined as an \type{RDFGraph} that is considered the default graph and a sequence of
\type{RDFNamed\-Graph}{\type{s}} represented by the \type{name} of the graph and the corresponding \type{RDFGraph}.




\subsubsection{Translating SQL and SPARQL Solutions into the \type{PatternSolution} Type}
\label{sec:transl-sql-sparql}

The next definition presents the translation between a SPARQL solution sequence and a sequence of \type{SPARQLResult}
type elements that we implement in XSPARQL.
%
This serialisation of SPARQL results mimics the SPARQL Query Results XML Format~\cite{BeckettBroekstra:2008aa}, defined
by the XML Schema available at {\url{http://www.w3.org/2007/SPARQL/result.xsd}}.
%
\begin{definition}[Serialisation of Solution Sequences]
  \label{lem:solution2types}
  Given a SPARQL solution sequence $\Omega = (\mu_1, \dotsc, \mu_n)$ a serialisation of $\Omega$ into a sequence of
  \type{PatternSolution} is defined as follows:
  \begin{itemize}[noitemsep]
  \item $\funcCall{serialise}{\Omega} \Rightarrow \funcCall{serialise}{\mu_1}, \ldots, \funcCall{serialise}{\mu_n}$
  \item $\funcCall{serialise}{\mu} \Rightarrow 
    \stt{<result>} 
    \{\forall x \in \dom{\mu}, \funcCall{serialise}{\mu, x}\}
    \stt{</result>}
    $
  \item $\funcCall{serialise}{\mu, x} \Rightarrow  
    \stt{<binding name="}x\stt{">}
    \{\funcCall{term}{\funcCall{\mu}{x}}\}
    \stt{</binding>}
    $,
    %
    where \funcCall{term}{\mu(x)} is 
    \begin{itemize}[noitemsep]
    \item $\stt{<uri>}\{\mu(x)\}\stt{</uri>} \hfill \textrm{if}~\mu(x) \in \AU$
    \item $\stt{<bnode>}\{\mu(x)\}\stt{</bnode>} \hfill\textrm{if}~\mu(x) \in \AB$
    \item $\stt{<literal>}\{\mu(x)\}\stt{</literal>}\hfill\textrm{if}~\mu(x) \in \AL$
    \end{itemize}

  \end{itemize}
  %
\end{definition}
%
\noindent Following the definition of the \funcName{serialise} function, in evaluation rules, we will refer to sequences
of elements of type \type{PatternSolution} as \omg{}{} and to elements of type \type{SPARQLResult} as \sm{}{}.

For the representation of \ac{SQL} results we follow a similar approach:
%
\begin{definition}[Serialisation of SQL Relation Instances]
  \label{lem:relation2types}
  % 
  The serialisation of a relation instance~$I = \seq{I_1, \ldots, I_n}$ of relation~$R$ with~$\funcCall{sort}{R} = U$,
  into \type{PatternSolution} is:
  % 
  \begin{itemize}[noitemsep]
  \item $\funcCall{serialise}{I} \Rightarrow \funcCall{serialise}{I_1}, \ldots, \funcCall{serialise}{I_n}$
  \item $\funcCall{serialise}{I_i} \Rightarrow
    \stt{<result>} \{\forall x \in U, \funcCall{serialise}{I_i, x}\} \stt{</result>}
    $
  \item $\funcCall{serialise}{I_i, x} \Rightarrow  
    \stt{<binding name="}x\stt{">} \{ \funcCall{sql2xml}{\funcCall{I_i}{x}} \} \stt{</binding>}
    $.
  \end{itemize}
  %
\end{definition}



\subsubsection{Serialisation into SQL and SPARQL Representations}
\label{sec:sql-sparql-serialisation}

%
The following definitions present the \funcName{SQLTerm} and \funcName{RDFTerm} functions that, when applied to an
\ac{XSD} datatype, return their representation in \ac{SQL} or SPARQL syntax, respectively.  We first present the
serialisation into \ac{SQL}:
%
\begin{definition}[SQL representation]
  %
  \label{def:sqlTerm}
  %
  Let~$C$ be an expression context with static environment~$T_C = \funcCall{\stat}{C}$ and dynamic environment~$D_C =
  \funcCall{\dyn}{C}$, and~$x \in \funcCall{dom}{T_C.varType}$ an XSPARQL variable name.
  %
  The \ac{SQL} representation of~$x$ according to~$C$, denoted~$\funcCall{SQLTerm_C}{x}$ is:
  %
  \begin{itemize}[noitemsep]
  \item~$\funcCall{data}{D_C.\funcCall{\ecomp{varValue}}{x}}$ \hfill if $T_C.\funcCall{\ecomp{varType}}{x} =
    (\type{SQLTerm} \textrm{ or } \type{SQLAttribute} \textrm{ or } \type{RDFTerm}  \textrm{ or } \type{node()})$; and
  \item~$\funcCall{xml2sql}{D_C.\funcCall{\ecomp{varValue}}{x}}$ \hfill otherwise, 
  \end{itemize}
  % 
  where \funcName{xml2sql} is the value conversion function presented in \cref{sec:mapping-xml-types}.
  %
\end{definition}
%
Similarly, we next present the serialisation of SPARQL terms:
%
\begin{definition}[\funcName{RDFTerm}]
  % 
  \label{def:sparqlTerm}
  %
  Let~$C$ be an expression context with static environment~$T_C = \funcCall{\stat}{C}$ and dynamic environment~$D_C =
  \funcCall{\dyn}{C}$, and~$x \in \funcCall{dom}{T_C.varType}$ an XSPARQL variable name.
  %
  The \ac{RDF} representation of~$x$ according to~$C$, denoted~$\funcCall{RDFTerm_C}{x}$ is:
  % 
  \begin{itemize}[noitemsep]
  \item~$D_C.\funcCall{\ecomp{varValue}}{x} \hfill \textrm{if } T_C.\funcCall{\ecomp{varType}}{x} = \type{RDFTerm}$,
  \item~$\texttt{"}D_C.\funcCall{\ecomp{varValue}}{x}\texttt{"} \hfill\textrm{if } T_C.\funcCall{\ecomp{varType}}{x} = \type{xsd:string}$,
  \item~$\dt{\funcCall{D_C.\ecomp{varValue}}{\ensuremath{x}}}{\qname{rdf}{XMLLiteral}} \hfill \textrm{if }
    T_C.\funcCall{\ecomp{varType}}{x} = \type{element()}$,
  \item~$\texttt{"}data(D_C.\funcCall{\ecomp{varValue}}{x})\texttt{"} \hfill \textrm{if } T_C.\funcCall{\ecomp{varType}}{x} =
    (\type{attribute()} \textrm{ or } \type{SQLTerm} \textrm{ or } \type{SQLAttribute})$, and
  \item~$\texttt{"}D_C.\funcCall{\ecomp{varValue}}{x}\texttt{"\^{}\^{}}T_C.\funcCall{\ecomp{varType}}{x}$ \hfill
    otherwise.
  \end{itemize}
\end{definition}

%%% Local Variables:
%%% fill-column: 120
%%% TeX-master: t
%%% TeX-PDF-mode: t
%%% TeX-debug-bad-boxes: t
%%% TeX-parse-self: t
%%% TeX-auto-save: t
%%% reftex-plug-into-AUCTeX: t
%%% mode: latex
%%% mode: flyspell
%%% mode: reftex
%%% TeX-master: "../thesis"
%%% End:



\subsection{XSPARQL Semantics for Querying Relational and RDF data}
\label{sec:query-heter-data}

We now define the semantics of \SQLForClause{s} and \SparqlForClause{s} by relying on the evaluation semantics of their
original query languages, namely \ac{SQL} (presented in \cref{sec:sql}) and SPARQL (presented in
\cref{sec:sparql-preliminaries}).
%
The approach we take is to rely on the translation of each language into their respective algebra expressions and
further combine these algebra expressions with any existing XSPARQL variable bindings.
%
Since XSPARQL is based on the semantics of XQuery, variable bindings are stored in the \ecomp{varValue} environment
component of the dynamic environment (\cf~\cref{sec:xquery}), that maps variable names to their value.
%
Next we present how we interpret these variable mappings as a relation and as a solution sequence, thus allowing to
combine the results of \ac{SQL} and SPARQL queries with the existing variable bindings.


\subsubsection{Querying Relational Data}
\label{sec:extending-sql}


In order to reuse the semantics of \ac{SQL} for defining the semantics of XSPARQL \SQLForClause{s} we transform the
\ecomp{varValue} component of the dynamic environment in which the \SQLForClause is executed into a relation (which we
call the \emph{XSPARQL instance relation}).
%
The following definition presents this translation:
%
\begin{definition}[XSPARQL instance relation]
  \label{def:xml2sqlterm}
  Let the set of relation names (\AR) be defined as in \cref{sec:relational-model}, and let~$C$ be an expression
  context.
  %
  The \emph{XSPARQL instance relation} of~$C$ is a relation instance named \lit{$xir_C$}, where $xir_C$ is a reserved
  relation name, \ie~$xir_C \not\in \AR$, and~$\funcCall{sort}{xir_C} = \dom{\funcCall{\dyn}{C}.\ecomp{varValue}}$.
  %
  For each mapping~$v_{i} \rightarrow x_{i} \in \funcCall{\dyn}{C}.\ecomp{varValue}$, the value of~$xir_C$ for
  attribute~$v_i$, denoted~$\funcCall{xir_C}{v_{i}}$, is defined as:
  %
  \begin{itemize}[noitemsep]
  \item if~$x_{i} = \seq{}$ is an empty sequence then~$\funcCall{xir_C}{x_{i}} = \NULL$;
  \item if~$x_{i} = \seq{e_{1}, \dotsc, e_{n}}$ is a sequence, then~$\funcCall{xir_C}{x_{i}} =
    \funcCall{fn{:}concat}{\funcCall{SQLTerm_C}{e_{1}}, \dotsb, \funcCall{SQLTerm_C}{e_{n}}}$.\footnote{Since the values
      of any relation attribute must be atomic, in the case of a variable being assigned to an XQuery sequence we assume
      the concatenation of each element of the sequence.}
    %
  \end{itemize}
  %
   For a \synt{SQLWhereClause}~$S$, we call the XSPARQL instance relation of the expression context in which~$S$ is
  executed the \emph{XSPARQL instance relation of~$S$}.
\end{definition}
%

%
Another necessary step to enable the reuse of \ac{SQL} evaluation semantics is to convert our extended syntax (that
allows for \var{}-prefixed variable names) into valid \ac{SQL} syntax: each \synt{WhereSpec} in a \SQLForClause that
contains an XSPARQL variable is removed from the normalised \ac{SQL} query (by replacing it with \lit{true}) and is
stored for a later evaluation by the XSPARQL semantics.  For this we rely on the following normalisation function:
%
\begin{definition}[\ac{SQL} Representation of \synt{SQLWhereClauses}]
  %
  \label{def:normalisation-where}
  %
  Let~$S = \lit{where}~\synt{WhereSpecList}$ be a \synt{SQLWhereClause}.  The normalisation of~$S$,
  $\funcCall{normaliseSQL}{S} = \lit{where}~\funcCall{normaliseSQL}{\synt{WhereSpecList}}$, where
  \funcCall{normaliseSQL}{\synt{WhereSpecList}} is defined as:
  % 
  \begin{itemize}
  \item if \synt{WhereSpecList} is of form~$\lit{(}\ \synt{WhereSpecList}_{1}\ \synt{Op}\ \synt{WhereSpecList}_2\
    \lit{)}$ then
    %
    \[
    \lit{(}\ \funcCall{normaliseSQL}{\synt{WhereSpecList}_1}\ \synt{Op}\ \funcCall{normaliseSQL}{\synt{WhereSpecList}_2}\ \lit{)}
    \]
  \item if~\synt{WhereSpecList} is of form~$Attr_1~\synt{Op}~Attr_2$ then~$\funcCall{normaliseSQL}{Attr_1~Op~Attr_2}$
    is:
    \[
    \begin{dcases*}
      \lit{true}        & if~$Attr_1$  or~$Attr_2$ is an XSPARQL variable \\
      Attr_1~Op~Attr_2  & otherwise.
    \end{dcases*}
    \]
  \end{itemize}
  %
  Furthermore we denote the set of \synt{WhereSpec} of~$S$ in which an attribute is an XSPARQL variable as
  \funcCall{whereSpecVars}{S}.
  % 
\end{definition}
%
\noindent
%
The normalisation of complete \SQLForClause{s} consists also of the normalisation of the syntactical elements
\synt{AttrSpecList} and \synt{TableSelector} presented in \cref{sec:syntax-rdb}.
%
In the normalisation of~\synt{AttrSpecList} we remove any existing \synt{AttrNameSpec} component, since they reflect
only the name of the corresponding XSPARQL variable.
%
However, the normalisation of the \synt{TableSelector} can only be performed during the dynamic evaluation of the
XSPARQL query since any variables present in the \synt{TableSelector} must be evaluated to determine the corresponding
relation name.
%
With the restriction of performing the substitution at evaluation time, we can reuse the standard translation of a
\ac{SQL} query into relational algebra as presented in \cref{sec:sql-semantics}.

Next we present how XSPARQL combines the results of a \ac{SQL} query with an XSPARQL instance mapping.
%
For this we rely on the standard relational selection~($\sigma$) and cross-product~($\times$) algebra operators
presented in \cref{sec:querying-rdb} and on the~$xir_C$ relation instance from \cref{def:xml2sqlterm}.
%
Firstly, we present the construction of the relational algebra select expression that, based on the provided
\SQLForClause~$S$ and the XSPARQL instance mapping of~$S$, makes the connection between the results of the \ac{SQL}
query and the existing XSPARQL variable bindings:
%
\begin{definition}[XSPARQL $\sigma$ expression]
  \label{def:xsparql-join}
  %
  Let~$S$ be a \SQLForClause with expression context~$C$ and~$V = \funcCall{whereSpecVars}{S}$ the attribute
  specifications that contain XSPARQL variables in~$S$.
  %
  The \emph{XSPARQL $\sigma$ expression} of~$S$, denoted~$\funcCall{\sigma_{xs}}{S}$, is a relational algebra~$\sigma$
  expression that, for each $Att_1~Op~Att_2 \in V$ is~$\funcCall{trans}{Att_1}~Op~\funcCall{trans}{Att_2}$,
  where~$\funcCall{trans}{Att}$ is defined as:
  % 
  \begin{itemize}[noitemsep]
  \item $Attr$ \hfill if~$Attr$ is not an XSPARQL variable;
  \item if~$Attr = \lit{\$}AttrName$ is an XSPARQL variable then 
    %
    \[
    \funcCall{trans}{Attr} = \begin{dcases*}
        \dyn.\funcCall{\ecomp{varValue}}{AttrName} & if~\stat.\funcCall{\ecomp{varType}}{Attr} = \type{SQLAttribute}\\
        \textrm{`}xir_C.AttrName\textrm{'} & otherwise.
      \end{dcases*}
    \]
  \end{itemize}
\end{definition}
%
\noindent
%
This definition creates a relational algebra expression from the extended XSPARQL \SQLForClause syntax, which can then
be used to further restrict the results of the normalised \ac{SQL} expression.
%
Based on these definitions we can introduce the translation of \SQLForClause{s} into relational algebra.
%
\begin{definition}[XSPARQL relational algebra expression]
  \label{def:xsparql-sql-answers}
  %
  Let~$Q$ be a \SQLForClause,~$Q' = \funcCall{normaliseSQL}{Q}$ the \ac{SQL} rewriting of~$Q$,~$E =
  \funcCall{\sigma_{xs}}{S}$ the XSPARQL~$\sigma$ expression of~$S$, and~$\RASQL{Q'}$ the relational algebra expression
  obtained from the standard \ac{SQL} translation into relational algebra.
  %
  The \emph{XSPARQL relational algebra expression} of~$Q$, denoted~$\RAXSP{Q}$, combines the relational algebra
  expression of the \ac{SQL} query and restricts its results to the existing bindings for XSPARQL variables as follows:
  %
  \[
  \funcCall{\sigma_{E}}{\RASQL{Q'} \times xir_C} \enspace .
  \]
  %
\end{definition}
%
The following example illustrates the translation of XSPARQL \SQLForClause{s} into XSPARQL relational algebra
expressions.
%
\begin{figure}[t]
  \subfloat[Value Matching]{\label{ex:sql-for-ex1}
    \begin{minipage}{.5\linewidth}
      \lstinputlisting{0-data+queries/sql-for-clause-ex1.xsparql}
    \end{minipage}
  }
  % 
  \subfloat[Attribute Matching]{\label{ex:sql-for-ex2}
    \begin{minipage}{.5\linewidth}
      \lstinputlisting{0-data+queries/sql-for-clause-ex2.xsparql}
    \end{minipage}
  }
  \caption{XSPARQL \SQLForClause examples}%
  \label{fig:sql-for-ex}%
\end{figure}
%
\begin{example}[Translation of \SQLForClause{s} into Relational Algebra]
  %
  \cref{fig:sql-for-ex} presents two XSPARQL queries including \SQLForClause{s}.
  %
  The query in \cref{ex:sql-for-ex1} illustrates the syntax for querying values of a relation.  First the
  normalisation function drops the restriction in line~3, which is incorporated into the relational algebra~$\sigma$
  expression:
  % 
  \[
  \funcCall{\sigma_{band.bandId = xir_C.x}}{\funcCall{\sigma_{band.bandName = 'Nightwish'}}{band} \times xir_C} \enspace ,
  \]
  where~$\funcCall{sort}{xir_C} = \set{x}$ and~$\funcCall{xir_C}{x} = 1$.

  \medskip

  On the other hand, the query in \cref{ex:sql-for-ex2} shows how to match attribute names.
  %
  The query in this figure is converted into the following relational algebra expression:
  % 
  \[
  \funcCall{\sigma_{band.bandName = 'Nightwish'}}{\funcCall{\sigma_{band.bandId = 1}}{band} \times xir_C} \enspace ,
  \]
  %
  where~$\funcCall{sort}{xir_C} = \set{x}$ and~$\funcCall{xir_C}{x} = \textrm{`bandName'}$.  
  %
\end{example}







%%% Local Variables:
%%% fill-column: 120
%%% TeX-master: t
%%% TeX-PDF-mode: t
%%% TeX-debug-bad-boxes: t
%%% TeX-parse-self: t
%%% TeX-auto-save: t
%%% reftex-plug-into-AUCTeX: t
%%% mode: latex
%%% mode: flyspell
%%% mode: reftex
%%% TeX-master: "../thesis"
%%% End:




\subsubsection{Querying RDF Data}
\label{sec:bf-extended-bgp}
%
For querying \ac{RDF} data, we extend the notion of SPARQL \ac{BGP} (\cref{def:bgp-matching}) in order to provide SPARQL
with the variable bindings from XQuery.
%
For this we interpret the XQuery \ecomp{varValue} dynamic environment component as a set of bindings in the spirit of
SPARQL solution mappings (as presented in \cref{def:solution-mapping}).
%
Along these lines, we will regard the \ecomp{varValue} component of the dynamic environment in which a SPARQL graph
pattern~$P$ is executed as the basis for the \emph{XSPARQL instance mapping} of~$P$.  The transformation from the
\dyn.\ecomp{varValue} into the XSPARQL instance mapping is defined next:
%
\begin{definition}[XSPARQL instance mapping]
  \label{def:xml2rdfterm}
  Let~$C$ be an expression context, and furthermore let~$D_C = \funcCall{\dyn}{C}.\ecomp{varValue}$ and~$T_C =
  \funcCall{\stat}{C}.\ecomp{varType}$ the \ecomp{varValue} component of the dynamic environment of~$C$ and be the
  \ecomp{varType} component of the static environment of~$C$, respectively.
  %
  The \emph{XSPARQL instance mapping}~$\mu_C$ is a solution mapping where, for each mapping~$v_{i} \rightarrow x_{i} \in
  D_C$,~$x_{i}$ is converted into an instance of type \type{RDFTerm} or an \ac{RDF} \emph{Collection} according to the
  following conditions:
  %
  \begin{itemize}[noitemsep]
  \item if~$\funcCall{D_C}{v_{i}} = \seq{}$ and~$\funcCall{T_C}{v_{i}} = \type{RDFTerm}$ or~$\funcCall{T_C}{v_{i}} =
    \type{SQLTerm}$ then~$\funcCall{\mu_C}{\funcCall{D_C}{v_{i}}}$ is undefined;
  \item if~$\funcCall{D_C}{v_{i}} = \seq{}$ and~$\funcCall{T_C}{v_{i}} \neq \type{RDFTerm}$ and~$\funcCall{T_C}{v_{i}}
    \neq \type{SQLTerm}$ then~$\funcCall{\mu_C}{\funcCall{D_C}{v_{i}}} = \seq{}$ is an empty \ac{RDF} Collection;
  \item if~$\funcCall{D_C}{v_{i}}$ is a singleton sequence then~$\funcCall{\mu_C}{\funcCall{D_C}{v_{i}}} =
    \funcCall{RDFTerm}{\funcCall{D_C}{v_{i}}}$;
  \item if~$\funcCall{D_C}{v_{i}} = \seq{e_{1}, \dotsc, e_{n}}$,~$n > 1$, is a sequence
    then~$\funcCall{\mu_C}{\funcCall{D_C}{v_{i}}} = \seq{\funcCall{RDFTerm}{e_{1}} \dotsb \funcCall{RDFTerm}{e_{n}}}$ to
    be read as an \ac{RDF} Collection in Turtle notation (\cf~\cref{sec:turtle}).
  \end{itemize}
  % 
  For a graph pattern~$P$, we call the XSPARQL instance mapping of the expression context in which~$P$ is executed the
  \emph{XSPARQL instance mapping of~$P$}.
\end{definition}
%
Next we define the notion of XSPARQL \ac{BGP} matching based on the semantics of SPARQL \ac{BGP} matching presented in
\cref{sec:sparql-semantics}.
%
\begin{definition}[Extended solution mapping]\label{def:extended-solution-mapping}
  Let~$C$ be an expression context. An \emph{extended solution mapping} of a graph pattern~$P$ in~$C$ is a solution
  mapping \emph{compatible} with the \emph{XSPARQL instance mapping} of~$C$.
\end{definition}
% 
\noindent Accordingly, XSPARQL \ac{BGP} matching is defined analogously to the SPARQL \ac{BGP} matching with the
exception that we consider only extended solution mappings:
%
\begin{definition}[XSPARQL \ac{BGP} matching]\label{def:xsparql-bgp-matching}
  Let~$P$ be a \ac{BGP} with expression context~$C$, and~$G$ be an \ac{RDF} graph. We say that~$\mu$ is a
  \emph{solution} for~$P$ with respect to active graph~$G$, if there exists an extended solution mapping~$\mu'$ of~$C$
  such that~$\funcCall{\mu'}{P}$ is a subgraph of~$G$ and~$\mu$ is the restriction of~$\mu'$ to the variables
  in~$\vars{P}$.
\end{definition}
%
\noindent This definition quasi injects the variable bindings inherited from XQuery into SPARQL patterns occurring
within XSPARQL.  By considering \emph{extended solution mappings} the bindings returned for a \ac{BGP}~$P$ will not only
match the input graph~$G$ but also respect any bindings for variables in the dynamic environment.
%
We can extend the XSPARQL \ac{BGP} matching to generic graph patterns by following the SPARQL evaluation semantics
(presented in \cref{sec:sparql-semantics}).
%
Considering a graph pattern~$P$ with XSPARQL instance mapping~$\mu_C$, we denote by~$\evalXS{D}{P}{\mu_C}$ the
evaluation of~$P$ over dataset~$D$ following XSPARQL \ac{BGP} matching.
 
\subsubsection{Matching Blank Nodes in Nested Queries}
\label{sec:bgpmatch}
%
Although in XSPARQL, similar to SPARQL, we are not considering blank nodes in the semantics definitions of graph
patterns, in the case of nested \SparqlForClause{s} XSPARQL instance mappings may in fact contain assignments of
variables to blank nodes, injected from the outer \SparqlForClause into the inner \SparqlForClause.
%
\begin{query}
  \lstinputlisting[language=XSPARQL]{0-data+queries/nested-bnodes.xsparql}%
  \caption{Nested XSPARQL query}%
  \label{qr:nested-bnodes}%
\end{query}
%
\begin{example}[Blank node injection in XSPARQL nested queries]
  %
  For example, in \cref{qr:nested-bnodes}, blank nodes bound in the outer \SparqlForClause
  (\crefrange{qr:nested-bnodes_out-s}{qr:nested-bnodes_out-e}) to the variable~\var{song} will be injected into
  the inner \SparqlForClause expression (\crefrange{qr:nested-bnodes_in-s}{qr:nested-bnodes_in-e}).
  %
  If we would consider both \SparqlForClause{s} as distinct SPARQL queries, the blank nodes in the inner
  \SparqlForClause would be matched as variables.
\end{example}
%
However in XSPARQL, we want to enable coreference within nested queries over the same dataset and thus such injected
blank nodes should be matched like constants against the blank nodes present in the input \ac{RDF} data (rather than
being treated as variables).
%
To ensure this behaviour, we introduce the notion of \emph{active dataset} (similar to the concept of active graph in
SPARQL), where nested queries over the same active dataset keep the same the scoping graphs
(\cf~\cref{sec:sparql-semantics}).
%
Any \SparqlForClause with an \emph{explicit} \DatasetClause causes the \emph{active dataset} to change, \ie~new scoping
graphs (with fresh blank nodes) for each graph within it are created.  On the other hand, if no \DatasetClause is
present in a nested \SparqlForClause (implicit dataset), the active dataset remains
unchanged.
%
To ensure this behaviour in the dynamic evaluation we have to introduce a new dynamic environment component called
\ecomp{activeDataset}, that will be used to evaluate \SparqlWhereClause{s}.  Initially, this component is empty (or set
to a system default) and is changed by a \DatasetClause appearing in a \SparqlForClause{}, as defined in the next
section.



%%% Local Variables:
%%% fill-column: 120
%%% TeX-master: t
%%% TeX-PDF-mode: t
%%% TeX-debug-bad-boxes: t
%%% TeX-parse-self: t
%%% TeX-auto-save: t
%%% reftex-plug-into-AUCTeX: t
%%% mode: latex
%%% mode: flyspell
%%% mode: reftex
%%% TeX-master: "../thesis"
%%% End:




%%% Local Variables:
%%% fill-column: 120
%%% TeX-master: t
%%% TeX-PDF-mode: t
%%% TeX-debug-bad-boxes: t
%%% TeX-parse-self: t
%%% TeX-auto-save: t
%%% reftex-plug-into-AUCTeX: t
%%% mode: latex
%%% mode: flyspell
%%% mode: reftex
%%% TeX-master: "../thesis"
%%% End:
